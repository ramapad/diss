
\section{Background}
\label{sec:related}

% \subsection{Goal}

% My goal is to measure Internet reliability 

\subsection{Definitions}

The goal of this work is to provide broad, longitudinal, and accurate measurements of
Internet reliability across ISPs, media-types, and geographic
locations in a variety of circumstances. Such measurements can help
users choose from their available Internet options and can inform ISPs
about potential problems in their networks. To achieve this goal, I
propose the following thesis and define terms in the thesis as follows:

\emph{It is possible to remotely and accurately detect substantial outages
  experienced by any device with a stable public IP address that typically
  responds to active probes and use these outages to compare
  reliability across ISPs, media-types and geographical areas.}


\begin{itemize}

\item {\emph{Device with a stable public IP address}: This is a device
    connected to the Internet, like a
home-router, to which an ISP has assigned a public IP address such
that the
assignment is either static, or dynamic in a manner that allows the
duration of dynamic assignment to be estimated.}

\item {\emph{Substantial outage}: I define a substantial outage to be an event where a device
    with an Internet connection is unable to send or receive any
    packets for at least 10 minutes.}

\item {\emph{Accuracy of outage detection}: An outage detection technique is accurate when it
correctly identifies every substantial outage event experienced by an Internet-connected-device, along with its
duration. There are no time-periods when the address
experiences a substantial outage but it goes undetected (false
negatives). Similarly, there are no time-periods classified as
outages when the destination address is able to receive packets from the
Internet (false positives).}

\item {\emph{Reliability}: I define two measures of reliability: one is the raw count of outage
events over measured time and the other is the proportion of total measured time
detected as outage events.} % When estimating a particular ISP's reliability, I
% only use the subset of outages that solely affected that ISP's
% addresses.}
% When estimating the reliability of a geographical area, I
% consider the subset of outages that affected only that geographical
% area.

\end{itemize}

\subsection{Related work}

The architects of the Internet predicted that network outages could
occur, and designed the Internet to have the ability to route around
outages~\cite{clark-darpa}. As predicted, a variety of factors cause
outages in the Internet, including optical fiber
cuts~\cite{fiber-cuts}, routing and infrastructure
failures~\cite{backbone-failures-1999, ratulbgp}, and
hurricanes~\cite{pingin}.

%TODO: Cite ratulbgp somewhere, network-black-holes, feamster:sigmetrics:failures

Large Internet outages that can affect packets from
thousands of Internet hosts have received attention from the research
community~\cite{censorship-outages, trinocular, hubble, paxson-e2e,
  hubble, netdiagnoser, lifeguard,
  poiroot, phillipa-outages-mailing-list,
  california-fault-lines, delayed-routing-convergence, consensus-routing,
  routing-e2e-path-perf, voip-bgp-convergence}. Outages occurring in the Internet's core can cause Internet
path failures; researchers have
investigated transient Internet path failures caused by route
changes~\cite{delayed-routing-convergence, consensus-routing,
  routing-e2e-path-perf, voip-bgp-convergence} and longer path failures caused by
infrastructure device outages~\cite{paxson-e2e,
  hubble, netdiagnoser, lifeguard,
  poiroot, phillipa-outages-mailing-list,
  california-fault-lines}. Other studies detect outages at the
country-level~\cite{censorship-outages} and at the network prefix
level~\cite{trinocular, hubble}.

 % Industry provides some options to
% study failures but they either focus solely on websites that are
% down~\cite{downdetector, outageanalyzer, isitdownrightnow, downforeveryoneorjustme}, or offer services to monitor
% large customer networks~\cite{thousandeyes}. 
However, outages that affect individual users have received comparatively less
attention~\cite{pingin, samknows, ripe-atlas-website,
  grover2013peeking}. In the rest of this
section, I classify these efforts to detect outages into on-premises
outage detection techniques, and remote probing-based outage detection
techniques, and
discuss their approaches and challenges in detail.

% Why do I care only about complete outages and not partial ones?
% Because they are easy to define! :P
% Because they are more likely to be last-mile link. Aha! Yes, so a
% complete outage means we will have the ability to isolate the fault

% We define a link to experience an outage when it experiences peformance
% degradation resulting in unusually high loss and/or delay. When a link
% experiences delay but no loss, we define that link to be \emph{sleepy}
% and we refer to the event as a \emph{sleep}. When a link experiences
% loss but no delay, we define that link to be \emph{lossy} and we refer
% to the event as a loss. When a link experiences complete loss, i.e.,
% all packets on that link are dropped, we define that link to be
% \emph{out} and we refer to the event as an \emph{outage}. Note that by
% definition, every \emph{outage} event is also a \emph{loss}
% event. When a link experiences delay and loss, we define that link to
% be \emph{sleepy-lossy} and we refer to the event as a
% \emph{sleep-loss}. When we speak of a single probe being lost/delayed,
% we will refer to it a \emph{probe-loss} /\emph{probe-sleep}.



% Several studies have tried to detect outages. 

% Find which ones study outages using passive techniques?

% Find which ones study outages at not the last-mile

% Find which ones study outages at the last-mile but using dedicated
% hardware (Ark, RIPE Atlas)

\subsection{On-premises outage detection techniques}
% The redundancy present in the core of the Internet is
% mostly absent in residential networks, owing to the high cost of
% deploying redundant last mile links. Even multihomed last mile links for business connectivity
% often share the same upstream hardware, representing a single point of 
% failure. Residential link failures directly impact end-users and as a result,
% are of interest to service providers, policy makers, and the end-users themselves.

% Most residential end-users today lack the means to understand the
% reliability of their Internet connectivity over time, and of comparing
% reliability across competing ISPs. They have to rely upon speedtest
% tools which can offer estimates of connectivity over a few seconds but
% not over longer timescales. 
Recognizing the need for long term
measurements of residential Internet performance, policymakers such as
the FCC from the U.S., and Ofcom from the U.K. have deployed the SamKnows
hardware platform~\cite{samknows} inside residences to measure residential
Internet connections continuously by performing active and passive
measurements and reporting their results to users, ISPs, and
policy makers. RIPE NCC, the European RIR, has pioneered the RIPE Atlas~\cite{atlas} project and
Sundaresan et al. the BISmark project~\cite{bismark-main-bib}, which also study
user connectivity using dedicated hardware measurement devices on user premises. Another
on-premises technique is the DIMES project~\cite{netdimes}, which
deploys measurement software on user machines; however, this approach
is not well suited to detecting outages since the DIMES software is
often installed on laptops~\cite{dhcp-dimes}.

% To
% offset some of the scalability costs, on-premises outage detection systems can also be software-based 
While some of these costs can be offset by utilizing measurements from
deployed software on user systems~\cite{netdimes, dhcp-dimes, Dasu:NSDI2013} or using a combination of hardware
and software measurements~\cite{IMC2014-Broadband-bischof}, deploying software widely remains
challenging. Separating user behavior, such as turning off their laptops, from
Internet outage events presents additional challenges for these techniques~\cite{dhcp-dimes}.


% In essence,
% they are also probe-based outage detection techniques, in that the
% absence of any probes indicates an outage; however, they are
% on-premises and not remote.
Hardware-based approaches can offer accurate reports about
Internet connectivity since the hardware devices are designed to make
measurements continuously as long as they are powered. These techniques have the
ability to perform a range of other measurements such as DNS anycast
tests that can identify which instance of a root-server is closest,
and even passive measurement of the websites that users
access. However, these approaches are fundamentally limited in scale
since their hardware is expensive, distributing the hardware to users
is time consuming, and convincing users to keep their hardware running
is challenging. For example, the RIPE Atlas project, which began in
2010 and has been continuously expanding all across the world, has fewer than 10,000 probes that are currently making measurements, out
of more than 15,000 distributed probes.
 
\subsection{Probing-based remote outage detection techniques}

% TODO: Talk about Zmap. It cannot do adaptive probing, being
% stateless, and hence cannot be used for individual outage detection.

Probing-based remote outage detection techniques can detect
connectivity problems remotely through active probing from servers
that we control. Though this approach will prevent us from running
certain types of measurements, such as DNS anycast tests, it can measure
Internet connectivity for individual users at scale. However,
existing techniques can infer false outages in some scenarios as I
illustrate next.

Probing-based remote outage detection techniques study connectivity problems by
sending active probes (such as ping's echo requests) and use probe
responses to infer connectivity problems. For example, an
echo-response from the end-host indicates that its network connection
is working. If a previously responsive destination ceases to respond
to probes, current techniques infer that the destination could be
experiencing connectivity problems. Thunderping~\cite{pingin},
Trinocular~\cite{trinocular}, and Zmap~\cite{durumeric2013zmap}, have
used this technique to detect outages, albeit at different scales:
Zmap detected outages that affected several US states during Hurricane
Sandy, Trinocular detects outages that affect /24 prefixes, and
Thunderping detects outages at the individual IP address level.

While probing-based outage detection techniques can scale to probing hundreds
of thousands of addresses, they cannot scale indefinitely. Very high
probe volume can cause our traffic to be viewed as malicious and can
result in probes getting blacklisted and in abuse
reports~\cite{durumeric2013zmap}. Further, high probe volume increases
the state that needs to be stored by the prober; though probing
schemes like Zmap have circumvented this problem by not storing
state~\cite{durumeric2013zmap}, adaptive retransmission to confirm a
suspected outage requires the storage of state.

% Trinocular is an outage detection system that employs active
% probes to detect outages for entire /24 prefixes. It uses historical
% measurements to 

% Thunderping~\cite{pingin}, detects last-mile
% link outages for individual residential links during times of predicted severe weather
% conditions using pings, and correlates outages with weather
% conditions. The US National Weather service issues severe weather alerts for areas that are
% likely to experience conditions of severe weather; Thunderping uses
% these alerts to select geographic areas to study. Using
% Maxmind, a popular geolocation service, it then finds IP addresses in
% these geographic areas and pings these addresses from multiplee
% PlanetLab vantage points before, during, and after the weather
% event. We use the results of these pings to infer outages
% and correlate them with observed weather conditions to measure
% the effect of weather upon residential Internet connectivity.


% \begin{figure}[tb]
% % \centering
% \begin{center}
% \includegraphics[width=3in]{figs/pingin_real_deal_v13}
% \end{center}
% \caption{\label{fig:thunderping} Thunderping detects outages in
%   last-mile links during times of predicted severe weather. It uses
%   weather alerts from NOAA to find areas that are likely to be
%   affected by severe weather. It then pings IP addresses in those
%   areas before, during and after the weather alerts from multiple
%   PlanetLab vantage points and uses the
%   results to infer outages.}
% \end{figure}

% \subsubsection{Improving the accuracy of remote probing based outage
%   detection techniques}

% Can measure reliability inaccurately
\subsubsection{Probing-based remote outage detection techniques can
be inaccurate}

% I define a probed destination address to undergo an ``outage'' event
% when the address is unable to send or receive any Internet packets. The ideal outage detection technique should be capable of
% identifying every outage event, along with its
% duration. There should be no time-periods when the destination address
% experiences an outage but the outage is undetected (false
% negatives). Also, there should be no time-periods classified as
% outages when the destination address is able to receive packets from the
% Internet (false positives).

Probing-based remote outage detection techniques can infer false
negative and false positive outages as a consequence of their foundational
assumption: that a response to an active probe indicates a working path to the probed
IP address and that lack of response is indicative of
failure. False negatives can occur when the probe rate
to a destination address is low, so that very short outages
experienced by the address go undetected. For example, with
Thunderping's probing scheme of sending a ping every 11 minutes from
each of its vantage points to a destination address~\cite{pingin}, it is possible that an outage lasting
shorter than 11 minutes is not observed by each vantage
point. Increasing the probe rate can limit the maximum duration
of these false negative outages; remote probing based outage detection
techniques must tradeoff the rate
with which they probe a given destination and the duration of the
longest outage that they can fail to detect. 

While false negative outages can be controlled by probing faster,
false positive outages pose a potentially larger accuracy problem. Current
techniques can make false positive inferences about
outages in the following scenarios:

\begin{itemize}

\item{\bf{Confusing delay with loss:}}
 Traditionally, active probe based approaches time out probes after a few seconds. Thunderping~\cite{pingin} and Trinocular~\cite{trinocular}
time out probes after a few seconds. Responses that arrive after the
timeout will be reported as lost. When this happens, existing
techniques would infer loss though the responses are in fact merely
delayed. 

% TODO: Raise whether extreme delay is an outage.

\item{\bf{Making false inferences about outages due to dynamic
      addressing:}}

    Consider an IP address that was previously responsive. If the host
    to which that IP address was assigned changed its IP address as a
    result of dynamic addressing, and if the probed IP
    address is not reassigned to any host, then echo responses will
    cease to arrive. Existing techniques would thus infer false
    probe-loss and consequently, false
    outages. Consider an alternate scenario where the probed IP address has an
    outage. Suppose that at some point during the outage, the IP
    address is reassigned to some other end-host which responds to
    probes. Existing techniques would infer that the arrival of
    responses signals the end of the outage and would infer that the
    outage ended prematurely.

% \item{\bf{Some outages can falsely lower inferred reliability}}
% When analyzing ISP-level or media-type-level reliability, our
% reliability inferences for an ISP should only be based upon outages that
% affected solely that ISP's customers. However, power outages and
% undersea cable cuts can result in outages to many ISPs'
% customers. Also, users sometimes choose to voluntarily shut down their home Internet
% equipment~\cite{grover2013peeking}. If a probing-based remote outage detection technique happens to
% measure an address during these times, probes sent to that address
% will cease to arrive, leading to the inference of an outage. When
% measuring ISP-level or media-type-level reliability, these outages
% must be filtered.

\end{itemize}

I address how the accuracy of probing-based remote outage detection
techniques can be improved by mitigating these listed causes of false
positive outages in Section~\ref{sec:timeouts} and Section~\ref{sec:addr_change}.
% the rest of the proposal.
 % false probe-loss inference due to early timeout in
% Section~\ref{sec:timeouts} and false probe-loss inference due to IP
% address change in Section~\ref{sec:addr_change}.

