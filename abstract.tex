
\begin{abstract}

% We use the Internet without know that the Internet is actually super reliable. It's based more upon: hey, the Internet seems reliable. 

% The Internet is used today for communication

% Detection of Internet outages, which are potentially rare events, demands broad and longitudinal measurements of users' Internet connections.
% Internet reliability is increasingly important as the applications we use increasingly depend upon the Internet.
% Internet reliability is increasingly important as a variety of services that we use migrate to the Internet.
% Internet reliability is increasingly important as the applications that we depend upon migrate to the Internet.
Internet reliability is increasingly important as a variety of services that we use migrate to the Internet. Yet, we lack authoritative measures of last-mile Internet reliability. The first step towards measuring last-mile reliability is to detect Internet outage events experienced by users. Since Internet outages are rare events, detecting them requires broad and longitudinal measurements; however, such measurements of Internet reliability at the individual user level are challenging to obtain accurately and at scale. The second step is to use detected outages to reason about Internet reliability across different dimensions such as ISPs, media-types, and geographical areas.

Probing-based remote outage detection techniques can scale but their accuracy is questionable. These techniques detect Internet outages across time as well as across the IPv4 address space by sending active probes, such as pings and traceroutes, to users' IP addresses and use probe responses to infer Internet connectivity. However, they can infer false outages since their foundational assumption can sometimes be invalid: that the lack of response to an active probe is indicative of failure. In this dissertation, I defend the following thesis: \emph{It is possible to remotely and accurately detect substantial outages experienced by any device with a stable public IP address that typically responds to active probes and use these outages to compare reliability across ISPs, media-types and geographical areas}.

In the first part of the dissertation, I address the inaccuracy of probing-based techniques' detected outages and show how to use probe responses to correctly detect outages. I illustrate two potential scenarios where the foundational assumption of probing-based techniques is invalid. In the first scenario, responses are delayed beyond the prober's timeout, leading these techniques to infer packet-loss instead of delay. In the second scenario, these techniques can falsely infer packet-loss when the address they are probing gets dynamically reassigned. I examine how commonly delayed responses and dynamic reassignment occur across ISPs to quantify the inaccuracy of these techniques. I show how outages can be inferred correctly even in the presence of dynamic reassignment using complementary datasets that can track devices across the address space.

In the second part of the dissertation, I use two orthogonal approaches to demonstrate how to use individual outages detected by probing-based techniques to assess Internet reliability. Individual outages are not direct measures of reliability: they can occur independently because users disable equipment or be observed falsely due to dynamic address renumbering. In the first approach, I use the insight that the statistical change in outage rate in different challenging environemnts (e.g. thunderstorm) can quantitatively expose actual outage “inflation”. I show how to study the effect of challenging environments upon the reliability of a group of addresses, such as the addresses in a particular geographical area, by analyzing the \emph{inflation} in outage rate for that group during its presence. In the second approach, my insight is that simultaneous outages of related addresses could share a common underlying cause. With this insight, I develop and evaluate an approach to segregate outages into categories that suggest their cause: across ISPs likely power, within an ISP likely the network itself.

This dissertation's contributions will help achieve comprehensive measurements of Internet reliability that can be used to identify vulnerable networks and their challenges, inform which enhancements can help networks improve reliability, and evaluate the efficacy of deployed enhancements over time.

% comprehensive datasets of Internet reliability that can aid in improving reliability by targeting problem regions or adapting strategies that have proven to be successful
\end{abstract}
