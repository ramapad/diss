
%TODO: Find out what's different about ns-proposal used in reliability proposal
% \documentclass[letterpaper,10pt]{article}
%% \documentclass[10pt,onecolumn]{article}
% \documentclass[10pt,onecolumn]{ns-article}
\newcommand{\mydriver}{pdflatex}
% \documentclass[12pt,\mydriver]{thesis-2}
\documentclass[12pt,\mydriver]{thesis}
%% \usepackage[letterpaper, margin=1.25in]{geometry} % For giving to neil

% To avoid widows
\clubpenalty=10000 
\widowpenalty = 10000

\PassOptionsToPackage{hyphens}{url}
\usepackage[hyphens]{url}
\usepackage{hyperref}
\hypersetup{breaklinks=true}
\urlstyle{same}
\usepackage{balance}

\usepackage{titlesec}
   \titleformat{\chapter}
      {\large \bf}{Chapter \thechapter:}{1em}{}
      % {\normalfont\large}{Chapter \thechapter:}{1em}{}

% TODO: I seem to have used most of these from the reliability
% proposal. Find out what the role of each command is.
\usepackage{ifpdf,floatrow} 

% nspring 10.35pt palatino
\renewcommand*{\rmdefault}{ppl} % Roman default
\usepackage{fix-cm}
% \renewcommand{\normalsize}{\fontsize{10.35pt}{12.5pt}\selectfont}
% \renewcommand{\normalsize}{\fontsize{10pt}{12.5pt}\selectfont}
% endspring

%% ns - replacement for times that lacks stupidity.
% \usepackage{mathptmx}
% \usepackage[scaled=.90]{helvet}
% \usepackage{courier}
% 
% \usepackage[override]{cmtt} % make tt font tighter / less ugly

% \usepackage{makeidx}  % allows for indexgeneration

\usepackage[square,comma,numbers,sort&compress]{natbib}

\usepackage{color}
\usepackage[table]{xcolor}
\definecolor{orange}{RGB}{255,127,0}
\newcommand{\rama}[1]{{\color{red}[\todo{rama: #1}]}}
\newcommand{\ns}[1]{{\color{green}[ns: #1]}}
\newcommand{\etal}{et~al.\xspace}
\providecommand{\ie}{\emph{i.e.,} }
\providecommand{\eg}{\emph{e.g.,} }
\providecommand{\cf}{\emph{cf.,} }
\providecommand{\vs}{\emph{vs.} }
\providecommand{\etc}{\emph{etc.}}   
\providecommand{\ione}{\emph{(i)} }
\providecommand{\itwo}{\emph{(ii)} }
\providecommand{\ithree}{\emph{(iii)} }
\providecommand{\ifour}{\emph{(iv)} }
\providecommand{\ifive}{\emph{(v)} }

% \newcommand{\ignore}[1]{}



\usepackage{enumitem}
\setlist[itemize]{leftmargin=*}
% \usepackage{slashbox}
\usepackage{graphicx}
\usepackage{amsmath,amssymb}
\usepackage{verbatimbox}
\usepackage{boxedminipage}
\usepackage{multirow}
% \usepackage{subfigure}
\usepackage[labelformat=simple]{subcaption}
% \usepackage{subfig}
\usepackage{xspace}
\usepackage[]{pdfpages}

\usepackage[utf8x]{inputenc}
\usepackage{ucs}

% Was purportedly a fix for getting the Bibliography to appear in the
% Table fo Contents. However, it threw an error. So I went with the
% other fix: \addcontentsline{toc}{section}{Bibliography}
% \usepackage[nottoc,numbib]{tocbibind}

\newcounter{FileStack}
\let\OrigInput\input
\newcommand{\ninput}[1]{%
  \stepcounter{FileStack}
  \expandafter\let
  \csname NameStack\theFileStack\endcsname
  \ThisFile
  \def\ThisFile{#1}%
  \OrigInput{#1}%
  \expandafter\let\expandafter
  \ThisFile
  \csname NameStack\theFileStack\endcsname
  \addtocounter{FileStack}{-1}%
}

%boxed,vlined,,linesnumbered,commentsnumbered
\usepackage[vlined]{algorithm2e}
\providecommand{\SetAlgoLined}{\SetLine}
\providecommand{\DontPrintSemicolon}{\dontprintsemicolon}



% Rk: The following commands came from the NSF reliability proposal
% \setlength{\textwidth}{6.5in}
% \setlength{\textheight}{9in}
% \setlength{\topmargin}{0in}
% \setlength{\headheight}{0in}
% \setlength\columnsep{.30in}
% \setlength{\headsep}{0in}
% \setlength{\oddsidemargin}{0pt}
% \setlength{\evensidemargin}{0pt}

% Rk: The following commands are from mainthesis.tex from UMD's
% official style guide. Revisit and fix to get tables.
\newcommand{\tbsp}{\rule{0pt}{18pt}} %used to get a vertical distance after \hline
\renewcommand{\baselinestretch}{2}
\setlength{\textwidth}{5.9in}
\setlength{\textheight}{9in}
\setlength{\topmargin}{-.50in}
%\setlength{\topmargin}{0in}    %use this setting if the printer makes the the top margin 1/2 inch instead of 1 inch
\setlength{\oddsidemargin}{.55in}
\setlength{\parindent}{.4in}
\pagestyle{empty}

% \setlength{\textfloatsep}{.1in plus 0.05in} % This command reduces
% space between a fig's caption and the following text. May perhaps be
% useful for recovering space for a document with a
% word-limit. It negatively affects the diss.
% \setlength{\itemsep}{-10pt}

\captionsetup{font=small,labelfont={bf}}