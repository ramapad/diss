
% \chapter{Analyzing the effect of external factors upon Internet
%   reliability}
\chapter{Analyzing weather's effect on Internet Reliability}
\label{cpt:weather}

One aspect of measuring Internet reliability is to determine if the occurrence of certain events adversely affects Internet connectivity. Consider the occurrence of adverse weather conditions for instance: prior work has shown that Internet outages occur more frequently during times of precipitation~\cite{pingin}. However, this work was preliminary in nature and was performed over a short duration (three months). Further, it treated every instance of a previously responsive address failing to respond to pings as an outage, ignoring the effects of dynamic addressing.

In this section, I discuss an approach to quantify the effect of external factors, such as the occurrence of various weather conditions, upon Internet connectivity of residential addresses using measurements from the Thunderping probing system~\cite{pingin}. 

\subsection{Thunderping measurement system}

Thunderping probes addresses during times of severe weather. 

\subsection{Find the inflation in dropout rate}

The key insight here is that determining the inflation in \emph{dropout} rate when event(s) occur captures the increased likelihood of the \emph{outage} rate. 

% \begin{figure*}[t]
% %
% \begin{subfigure}[t]{0.47\linewidth}
% \centering
% \includegraphics[width=\linewidth]{figs/frate_by_timeofweek_jan11todec17_scatter}
% \caption{
% \label{fig:frate_lts_timeofweek}
% Dropout probability has significant diurnal variation.
% }
% \end{subfigure}
% %
% \hfill
% %
% \begin{subfigure}[t]{0.47\linewidth}
% \centering
% \includegraphics[width=\linewidth]{figs/addresshoursbywtyp_by_timeofweek_scatter}
% \caption{
% \label{fig:weather_timeofweek}
% Different weather conditions are prominent at different times.
% }
% \end{subfigure}
% %
% \caption{
%  Weather does not occur most often during hours of the week when there are an inflated number of dropouts. 
% }
% \end{figure*}



 