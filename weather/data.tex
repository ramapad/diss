\section{Data Set}

This section describes the data we collected and its initial
processing.
%
We start with a definition of the partially interpreted data
we seek: ``dropouts,'' where an address fails to respond in
the context of otherwise ``responsive hours'' of an address.
%
Dropouts and \emph{disruptions} from Chapter~\ref{cpt:corrfails} are synonymous.
%
We next briefly review the ThunderPing data probing system and
present brief statistics about the raw active probing data.
% 
Then, we review the weather data, particularly how and
where it is collected and how we handle hurricanes.
%
We conclude by describing the benefits (and limitations) of this
data for our study of weather-related effects.

\subsection{Dropouts, Defined} %{{{

A dropout happens when the address attached to a residential link transitions from
being responsive to pings from multiple vantage points, to being unresponsive
from all of the vantage points.
% 
Specifically, we define a residential link ``dropout'' as an hour when at least three
vantage points pinging a host and receiving replies suddenly experience 11 minutes
(an entire probing interval) where they do not receive a 
reply before a five second timeout.
%
This dropout occurs within a ``responsive address hour,'' a
continuous observation of an IP address in known weather
conditions.  A responsive hour may or may not include a
dropout, and the ratio of dropouts to responsive hours is a
measure of outage likelihood.  Responsive hours add: two
addresses both observed in the same hour or one address
observed for two hours in the same conditions are equivalent.

Our selection of three vantage points is based on prior work's selection
of three vantage points to observe outages~\cite{trinocular}.
%
Our selection of a five second timeout for ping responses is based on our
prior work that observed that most ping replies to residential
hosts are received within five seconds~\cite{timeouts}.
%
Our selection of one hour as the time period for a dropout is based on the fact
that the weather data we collected consists of hourly reports.
%
Considering at most one dropout per probed address per hour will diminish the
number of observed dropouts from individual links, if they should alternate
between responsive and unresponsive states: there can be at most one per hour,
not five (due to the 11 minute probing interval).

% To contrast with a dropout, a ``responsive hour'' is an hour
% in which weather is experienced, but the address does not
% experience a dropout.
% We define responsive hour later in Sec 2.4. And we do it
%   correctly there!
% nspring - I think it needs to be here, corrected if necessary.

Observing a dropout is a sign that a residential link may (but may not) have
experienced an outage:
%
\emph{Dropouts are a superset of outages.}
%
Dropouts can also occur if the device re-attaches to the
network with a new address after only momentary
disconnection, typically through re-association of a PPP
session for a DSL modem, but potentially through
administrative renumbering of prefixes. For our purposes, we
expect these events to occur independent of weather, such
that the two events can be studied separately.
%
% Restated, when a residential link experiences a prolonged outage, a prolonged
% dropout will also occur.
%
%Other network changes can also lead to dropouts.
%
%IP renumbering is the most common source of dropouts that are not outages. 
%IP renumbering is when residential providers change the IP addresses assigned
%to their customers either via forced renumbering~\cite{forced-renumbering},
%DHCP lease expiration~\cite{dhcp-lease}, or PPPoE sessions
%%expiring~\cite{pppoe}.  For our purposes, we expect these events (address
%reassignment or premise equipment reboot) to occur independent
%of weather, such that the two events can be studied separately.
%
We confirm that dropouts during typical maintenance intervals are
unlikely to correlate with weather in
Section~\ref{subsec:independence_of_maintenance}.
% , and we
% revisit DHCP in Section~\ref{sec:recovery}.

%
% \todo{insert that our approach for dealing with this is statistical?}

In short, by observing dropouts, we will be able to observe how residential
links behave during weather, at the scale necessary to make quantitative
conclusions about weather's effect on residential links in the U.S.
%
% We must also be careful because we observe the effects of
% other network changes that do not correspond with outages.
%}}}

% \subsection{How did we observe if there were dropouts on residential links during weather?} %{{{

\subsection{Dropouts, collected}

We briefly summarize the methodology of ``ThunderPing'': our probing
system that has been running for seven years.
%
More details about ThunderPing can be found in our preliminary work in
%ThunderPing was described in our preliminary work in
IMC~2011~\cite{pingin}.
%, details can be found there.
%
% nspring - I know I wrote this, but I'm not feeling it any more.
% surgical strike preferred.
%Although our initial probing methodology was sufficient for data collection, our
%analysis methods were inadequate in that they did not consider the background
%rate of failures and they were too ambitious in assuming outage duration could
%be understood for all link types.  We describe our revised method in 
%Section~\ref{sec:method}.

The ThunderPing probing methodology is as follows:
%
For every forecast of severe weather provided by the US National Weather
Service, ThunderPing pings a sample of 100 residential hosts from each
provider in the affected region.  The affected region is specified by FIPS code, which roughly
corresponds to counties in the U.S.
%
The probing starts up to six hours before the forecast event, continues during the
event, and terminates six hours after the event, regardless of whether the weather
materializes.
 
The residential hosts ThunderPing pings during each weather event are selected
from a master list of residential hosts classified by provider (reverse DNS
name) and geographic location (FIPS code).  We classify link type by
provider, when the provider implies a well-defined link type; (typically rural) providers that
use a variety of media types to provide connectivity are included under ``All'' link
types with the rest, but are not classified further.  We determine location using a 
MaxMind database from the same year for choosing which addresses to probe, but from the same
month for analysis. Although there are errors in both classifications, a location error
would be expected to cause an underestimate of the effect of weather by placing a host
not in the forecast region falsely into the area of weather effect.
 
ThunderPing sends pings to each of these hosts from up to 10 geographically
dispersed \planetlab vantage points every 11 minutes.  This interval is due to~\cite{imc08-heidemann}.
When a \planetlab node fails, we replace it, but if the number of working vantage points
drops below three, we discard observations at that time as untrustworthy.
%
% Having more than the three vantage points needed to detect dropouts allows us
% to continue probing uninterrupted even in the inevitable case that a
% \planetlab node fails.
% %
When there are at least three, we require
that all active vantage points do not have a response in order to label the event as a dropout.
  
ThunderPing retransmits failed ICMP requests: when a vantage point sees a
lack of ping response it retries that ping with an exponential backoff up to 10
times within the 11~minute probing interval.  Therefore, a dropout will typically require
at least 30 failed ICMP requests.

ThunderPing has been running for seven years, and has
collected 76 billion responsive pings to 8.7 million
residential addresses.
 
\subsection{Weather, classified}

To quantify the effect of weather on dropouts, we needed to determine
what weather residential links were exposed to when a dropout did or did not
occur.

The US National Weather Service (NWS) operates a 
network of 900 automated ``ASOS'' weather stations.
%
These weather stations are typically located at airports.
%
The NWS weather stations record hourly observations of 24 weather
variables in METAR format and make those available~\cite{metar-ftp}.
% nspring: I feel like "we downloaded it" is implied.
%
%For all of the geographic locations that we probed, across all of the time
%periods that we probed, we downloaded METAR data
%from~\cite{metar-ftp}.
%
 
There are two types of weather information: categories that
account for the common precipitation types (e.g.,
thunderstorm, hail, snow) and continuous variables (e.g.,
wind speed, precipitation quantity).

We annotate each responsive address hour for an address with the
corresponding weather information associated with the geographically
closest weather station to that address. Doing so allows us to find
the number of responsive hours and dropout address hours in specific
weather conditions. 

\paragraph{Hurricanes are special}
%
Severe events are among the most important failure events for us to study how the
Internet is affected, as the Internet is increasingly relied on as the primary
mode of communication in an emergency~\cite{emergency-voip-fcc}.
However, severe events have the potential to overwhelm the typical and
obscure interesting observations.

% hurricane_times = [
%    ['October 7 2017 00:00:00', 'October 10 2017 00:00:00'], # Hurricane Nate                  
%    ['September 9 2017 00:00:00', 'September 13 2017 00:00:00'], # Hurricane Irma              
%    ['August 24 2017 00:00:00', 'September 1 2017 00:00:00'], # Hurricane Harvey                
%    ['October 6 2016 00:00:00', 'October 9 2016 00:00:00'], # Hurricane Matthew                
%    ['September 1 2016 00:00:00', 'September 3 2016 00:00:00'], # Hurricane Hermine            
%    ['July 3 2014 00:00:00', 'July 5 2014 00:00:00'], # Hurricane Arthur, don't seem to have an\
% y data                                                                                          
%    ['October 28 2012 00:00:00', 'November 2 2012 00:00:00'], # Hurricane Sandy                
%    ['August 25 2012 00:00:00', 'August 31 2012 00:00:00'], # Hurricane Isaac                  
%    ['August 26 2011 00:00:00', 'August 30 2011 00:00:00'], # Hurricane Irene, don't seem to ha\
% ve any data                                                                                    
% ]

The following hurricanes made US landfall during our
measurement:
Irene (Aug 26--30, 2011),
Isaac (Aug 25--31, 2012),
Sandy (Oct 28--Nov 1, 2012),
Arthur (Jul 3--5, 2014),
Hermine (Sept 1--3, 2016),
Matthew (Oct 6--9, 2016),
Harvey (Oct 6--9, 2017),
Irma (Sept 9--13, 2017), and
Nate (Oct 7--10, 2017)~\cite{noaa-sotc}.
Hurricanes manifest as a combination of weather features and
are so pronounced that their contribution to thunderstorm or
rain outages would be disproportionate.\footnote{It is
  disappointing to realize the irony that the most
  significant weather events are also the least
  surprising.}  We thus omit them from categorical weather
classification (e.g., Figure~\ref{fig:inflatedfrate_by_wtyp_ci_whiskers}).
However, we consider data from Hurricane events
when studying continuous variables (inches of rain and wind speed,
for example, where these extremes are clearly distinguishable).
%
Collectively, these hurricane times account for less than 3\% of
%
%All these hurricane times put together account for less than 3\% of
responsive address hours and 4\% of dropout hours.

\begin{table}
  \hspace{-0.1in}\tiny
\ninput{weather/fragments/summary_table_vert}
  \caption[Summary of data set for large ISPs
	classified by link type]{\label{tbl:summary}Summary of data set for large ISPs
	classified by link type.  ``All'' comprises data from ISPs not
	included in this sample.
	(For this table, we count D.C.~as a state.)
	}
\end{table}



\subsection{Data, Summarized} % Comparing with our IMC 2011 Study}

% \todo{ alter this to cover table 1 and mention the old data in comparison. } 

This data set comprises observations from January 2011 to December 2017,
though only 1467 days included sufficiently many operating vantage points to
classify a responsive address hour.

We show per-ISP highlights in Table~\ref{tbl:summary}.  We
observe major providers such as Comcast, Qwest, and ViaSat
in all fifty states (and DC).  Of the 1.77 Billion
responsive address hours from Table~\ref{tbl:summary}, 139M
(8\%) were hours where responsive addresses experienced
rain, 66M (4\%) snow, and 19M (1\%) thunderstorm.

Contrasted to our preliminary study~\cite{pingin}, this
covers nearly 22 times the duration (compared to 66 days),
and includes roughly 60 times as many dropout events (likely
because those days were in spring and early summer).%
% sentiment in next section :
% is collected over enough time that we can draw
% conclusions with confidence about the increase in dropouts correlated with
% many different types of weather in different geographic locations.
%
\footnote{In the public reviews of the IMC 2011 paper, all of the reviewers stated that
they wished the dataset was more comprehensive so conclusions could be made
about the effects of weather on residential links.}
%
% The dataset in this work covers many seasons several times and contains a 
% severe weather events.
%
%
% At 1467 days, the data in this work spans a period of time 22
% times longer than our preliminary study in 2011.
%

%}}}

\subsection{Why this data?} %{{{

Others have studied outages and collected broad IP
responsiveness data.  Here we describe the benefits
of our data, addressing its limitations in Section~\ref{sec:datasux}.

Our data provides a view on outages of individual addresses,
including isolated outages of ``customer premises''
equipment or singly-connected links that are most exposed.
We rely on statistics to identify a significant change in likelihood of
failure, rather than rely upon large outages of infrastructure
common to a larger aggregate prefix to signify significance.
Every residential link is wired with its
own infrastructure: every residence can have different equipment installed
in different ways and has its own resident network administrator.
As a concrete example, we expect to observe
the effect of water infiltration in the network interface
device (the demarcation point connecting premise phone
wiring to the provider).  (We discuss the flip side of this
coin below.)

Our data is of a scale large enough to compare link
types, providers, geography, and across time.  Seven years
of data make it feasible to observe multiple instances of
both severe and common weather events.  Rare events include
a fair number of tornadoes and virtually unique events such
as snow in Louisiana.  Many observations of similar weather
increase the confidence in our dropout probability estimates,
making it possible to split the data and identify the differences
between, for example, heavy and light rain on wireless ISPs in
Kentucky.  The sampling approach---providing data for each 
provider in an area---ensures that even less-used network links
and providers are well-represented, permitting a comparison with
satellites and wireless ISPs that might be poorly represented
in end host measurement probes~\cite{samknows, sundaresan-sigcomm11,
  ripe-atlas} or when using provider-specific
data~\cite{conext10-jin,cisco-transponder,alpha-transponder}.

Our data includes data from times not subject to interesting
weather: the method probes before and after forecast weather
alerts.  ``Typical'' weather occurs particularly when the
forecast does not materialize or the forecast is for a
long-term event (e.g., summer fire warnings).  With these
measurements, we can establish a baseline for the rate of
dropouts in common weather conditions.  Probing after the
weather also permits measuring recovery time as we wait for
previously responsive addresses to return.

Our data is not sensitive to link failures elsewhere in the
Internet or to PlanetLab vantage point failures.
Restated, with multiple vantage points, catastrophic Internet link outages,
such as the fiber cuts during the ``Baltimore tunnel fire'' in
2001~\cite{bmore-tunnel-fire} will only be considered as an outage if all
vantage points are unable to communicate with the host over the residential
link.  As described above, without three active vantage points,
we make no decision about address responsiveness.

\subsection{Dataset Limitations} % What are the limitations of our data set?} %{{{
\label{sec:datasux}

% \paragraph{Probing is biased to links and time periods where there are weather
% forecasts}
%
The essence of ThunderPing is to selectively probe only when
there is a weather alert forecast for an area, which biases the data
toward time periods where there is some atypical weather present.
Obviously, regions that experience temperate weather are 
unlikely to be represented, and we thus do not attempt to quantify
what fraction of all residential network outages are caused by weather.
More subtly, during the interval around
forecast severe weather, the weather conditions may not be
ideal: our estimate of the background dropout rate is likely inflated 
by proximity to potentially severe weather, thus causing us to underestimate
the quantitative effect of that weather.

% \todo{Aaron- did I capture what you wanted (prior text commented below)?}

% Since ThunderPing only probes when there is a weather forecast for an area, we
% may not probe some residential links that received few or no forecasted weather
% over the course of our seven year study.
% %
% Also, by focusing only on links that are expected to experience weather, we may
% have bias toward hosts that will dropout if our weather assignment is
% incorrect. 
% %
% The result of this is that our baseline number of dropouts may be inflated.

Our approach relies upon active probing to gain breadth
across hundreds of providers, but there are limits to this
breadth: providers may administratively filter ICMP requests
and home routers may decline to respond.  We assume that
providers and end hosts that filter are no more or less
vulnerable to weather and that these features do not affect
our conclusions.

Our data set does not identify the cause of an individual
dropout. Our analysis seeks to correlate observations of
dropouts with weather events under the expectation that a
change in probability of outage is related to the weather.
Should a user turn off equipment nightly, this is
independent of weather and will not not be a factor; should a user
unplug equipment when lightning is nearby, such would
contribute to the probability of dropouts in thunderstorm.
Residential Internet infrastructure is also explicitly reliant
on residential electrical power, and we do not isolate power
failures.  We expect network service outages to be more common
than power outages, for power outages to occur only in the most
severe of weather conditions, and for power outages not
to correlate with link type.
% In part, this is why we omit hurricanes: prior
% studies of network outages in hurricanes were likely to be
% measurements of power loss.~\cite{zmap}

Finally, AT\&T, one of the largest DSL and fiber providers in the US does
not assign reverse DNS names to their residential customers. As such, they
were not included in our master list of residential links that we probe
with ThunderPing.  
%}}}
