\section{Quantifying weather dropouts}
\label{sec:method}

In this section we describe our methodology for quantifying how much weather
correlates with a change in the probability of residential link dropouts.
%
The goal is to find a metric that can measure how the likelihood of a dropout
\emph{increases} (or not) during weather.

The challenge in computing this metric is, dropouts are uncommon.
%
This makes it difficult to demonstrate that there is a statistically
significant increase in dropouts during weather.
%
Even if there is an increase, it may simply be due to the fact that certain
types of weather tend to occur during time periods that are commonly used for
network maintenance or IP renumbering.

By leveraging statistics developed for epidemiology, we overcome the first
challenge and find statistical significance.
%
By carefully inspecting our data set, we verify that no type of weather that we
study correlates with diurnal variations dropout probability.

We will now describe our metric, inflated probability of dropout, then we will
verify that conclusions we draw from this metric will not be skewed due to
time-correlated network state changes.

\subsection{Metric: Inflated probability of dropout} %{{{
\label{sec:hazard}

\subsubsection*{Challenge: Dropouts are uncommon}
%
Correlating dropouts with weather is challenging to do with statistical
significance because \emph{dropouts are rare events}
%
\footnote{...and they should be. If dropouts were common, residential links
would be unusable.}.
%
On average, we observed a link dropping out only once every $2-30$ days that we
were actively pinging and receiving responses from the host residential link,
depending on the link type.
%
The inverse of the average dropout rate per link type---including the average across
\textit{all} link types---is as follows:

\vspace{0.2cm}
\noindent
\begin{small}
	\begin{tabular}{@{~~}r@{~}c@{~}|@{~}c@{~}|@{~}c@{~}|@{~}c@{~}|@{~}c@{~}|@{~}c@{~}}
  \textbf{Link type:}    & Fiber & Cable & WISP & DSL & Sat & \textit{All} \\
  \textbf{Days b/w dropouts:} &   30  &   16  &    8 &   9 &   2 & \textit{8}   \\
\end{tabular}
\end{small}
\vspace{0.2cm}

Given how rarely dropouts occur, we now describe a metric that accounts for
this phenomenon, and even provides a way to determine statistical significance
so that we can ensure that our analysis does not involve too much slicing and
dicing of the few dropouts that occur.

% We would like our metric to capture failures occurring from these severe events
% and do not want to ignore outliers; after all, if thousands of hosts failed
% during an hour because of a hurricane, then those failures are indeed caused by
% weather. Thus, we aggregate failures from all hours of the week into a single
% bin and subtract the resulting P_noweather(Dropout) from P_weather(Dropout).

\subsubsection*{Our Approach: Hazard Rate}
%
Fortunately, there is a well-established set of techniques from the
field of epidemiology that permit statistical significance over rare
events.
%
Epidemiology---the study of the occurrence and determinants of
disease---faces similar challenges when analyzing mortality: deaths
(``failures'') are rare, and subjects (``links'') can be exposed to
their disease (``weather condition'') for different amounts of time
until the time of death (``dropout'').
%
Here, we describe the techniques we borrow from
biostatistics~\cite{biostats-book} to address these concerns.
%
Throughout our study, we will consider different groups of ``subjects'':
link types, geographic regions, and combinations thereof.

Like in epidemiological studies, we focus on estimating the \emph{hazard rate}
(sometimes referred to as the instantaneous death rate).
%
In essence, what a hazard rate gives us is the expected number of deaths per
unit time.
%
More concretely, for a given hazard rate $\lambda$, the probability of death
over a short duration of time $t$ is $\lambda \cdot t$.


The first challenge in estimating hazard rates is that different subjects may
be observed over different periods of time: in our study, hosts that remain
responsive can naturally be observed for longer periods of time than those that
drop out.
%
Throughout an observation period, we track the amount of time $O_i$ that we
observe each host $i = 1, \ldots, n$, and we also count the total number of
dropouts, $F$. 
%
An \emph{unbiased} estimate of the hazard rate $\hat{\lambda}$ can be obtained
as follows~\cite[Chapter 15.4]{biostats-book}:

\begin{equation}
\hat{\lambda} = \frac{F}{\sum_{i=1}^n O_i}
\label{eq:hazard}
\end{equation}

We exclude any bin of data if it does not have enough samples to permit
computing confidence intervals.
%
We adhere to the following rule of
thumb~\cite[Chapter~6]{biostats-book}: we accept a bin with $n$ samples
and estimated hazard rate $\hat{\lambda}$ only if 

\begin{equation}
	n \geq 20 ~~~\mbox{and}~~~ n\hat{\lambda}(1-\hat{\lambda}) \geq 10
	\label{eq:samples}
\end{equation}

\noindent
%
When these conditions hold, we can calculate 95\% confidence intervals
over the estimated hazard rate as
follows~\cite[Chapter~6.3]{biostats-book}:

\begin{equation}
	\hat{\lambda} \pm 1.96 \cdot
	\sqrt{\frac{\hat{\lambda}(1-\hat{\lambda})}{n}}
	\label{eq:hazard-conf}
\end{equation}

%% When we have sufficiently many samples, we can treat failures as a
%% Poisson process, and may therefore compute the standard error of an
%% estimated hazard rate with~\cite[Chapter~15.3]{biostats-book}:
%% 
%% \begin{equation}
%% \mbox{SE}(\hat{\lambda}) = \frac{\sqrt{F}}{\sum_{i=1}^n O_i}
%% \label{eq:hazard-se}
%% \end{equation}
%% 
%% \noindent
%% %
%% The 95\% confidence intervals over the estimated hazard rate can
%% therefore be computed as
%% 
%% \begin{equation}
	%% \hat{\lambda} \pm 1.96 \cdot \mbox{SE}(\hat{\lambda})
	%% \label{eq:hazard-conf}
%% \end{equation}


The above calculations yield the hazard rate along with its confidence
intervals; what remains is to \emph{compare} two hazard rates, for
instance, the overall hazard rate for a given link type and the hazard
rate for that link type specifically in the presence of snow.
%
Two estimated hazard rates $\widehat{\lambda_1}$ and
$\widehat{\lambda_2}$ can be compared by simply subtracting
them~\cite{biostats-book}.
%
Fortunately, with sufficiently many samples, the confidence intervals
over the difference of two hazard rates is given by the addition of the
confidence intervals over the original hazard rates.\footnote{This
follows from the fact that $\mbox{var}(\lambda_1 - \lambda_2)$ is
approximately $\mbox{var}(\lambda_1) + \mbox{var}(\lambda_2)$ when
Eq.~(\ref{eq:samples}) holds.}


To summarize: in the results throughout this paper, we compute hazard
rates using Eq.~(\ref{eq:hazard}), discard any bins that do not satisfy
Eq.~(\ref{eq:samples}), and compute confidence intervals using
Eq.~(\ref{eq:hazard-conf}).
%
When presenting our results, we multiply the hazard rate by a short
time interval, one hour, to estimate the hourly probability of a
dropout.


%}}}

\subsection{Verifying the metric will not be skewed by common dropouts correlating with weather} %{{{
\label{subsec:independence_of_maintenance}

\subsubsection*{Challenge: Dropout probability in weather may be inflated due
to diurnal events}
%
Observing an increase in dropout probability during weather periods compared to
non-weather periods may be skewed by common network state changes that tend to
occur during certain types of weather events.
%
This is a significant problem because it is more likely that a dropout would be
caused by everyday changes in network state such as nightly maintenance
periods, IP renumbering, and customers powering off their links at night,
rather than weather-induced outages.

\subsubsection*{Our Approach: Verify that weather events do not positively correlate with common dropout periods}

\begin{figure*}[t]
%
\begin{subfigure}[t]{0.47\linewidth}
\centering
\includegraphics[width=\linewidth]{weather/figs/frate_by_timeofweek_jan11todec17_scatter}
\caption{
\label{fig:frate_lts_timeofweek}
Dropout probability has significant diurnal variation.
}
\end{subfigure}
%
\hfill
%
\begin{subfigure}[t]{0.47\linewidth}
\centering
\includegraphics[width=\linewidth]{weather/figs/addresshoursbywtyp_by_timeofweek_scatter}
\caption{
\label{fig:weather_timeofweek}
Different weather conditions are prominent at different times.
}
\end{subfigure}
%
\caption[Weather does not occur most often during hours of the week when there are an inflated number of dropouts]{
 Weather does not occur most often during hours of the week when there are an inflated number of dropouts. 
}
\end{figure*}

The first question we must answer is: Are there any hours of the week that have
a significantly higher probability of dropout than other hours of the week?
%
To answer this question, we evaluate the probability of dropouts in each hour
of the week in the following manner:
%
for each hour of the week, we counted the number of dropouts (recall that
dropouts only occur at most once per hour per link) across all links observed
during that hour,
%
then we divided that by the number of hours which the link was responsive.
%
We did this for each link type separately, as some link types may be more
likely to be renumbered.
%
For example, in prior work we discovered that European DSL links have their IP
addresses reassigned every night at 2:00 AM UTC~\cite{addrchange-reasons}.
%
Also, some link types may require maintenance more often than others.
%

The results are shown in Figure~\ref{fig:frate_lts_timeofweek}.
%
As expected, the hourly probability of dropouts significantly varies in a
diurnal pattern over the course of each week.
%
Prior work suggests that ISPs are more likely to perform
administrative maintenance during weekday night
hours~\cite{Beverly_et_al_2015, comarela2013studying}; we speculate
that the increased dropout probability during weekday night hours could be due to administrative
maintenance.
%
The highest probability of dropouts for every link type rises in the evening
and peaks near midnight Eastern Standard Time (indicated with vertical dotted
black lines),
%
after which it drops off significantly until the early hours of the morning.
 
Given that we observe a diurnal variation in hourly likelihood of dropouts, and
the fact that weather conditions also have a known diurnal pattern of
occurrence~\cite{diurnal-weather}, the next question we must answer is: Does
hourly weather occurrence positively correlate with dropout probability?
%
To answer this question, we count the total number of responsive hours that we
observed in each hour of the week for each weather condition.

The results are shown in Figure~\ref{fig:weather_timeofweek}.
%
As expected, most weather conditions, possibly except for snow, have a diurnal
pattern in their occurrence.
%
Fortunately, none of the weather conditions have a positive correlation with
the hourly probability of dropouts.
%
There is however a negative correlation with cold weather: the coldest point of
the night is also when the lowest hourly probably of dropouts occurred.
%
This negative correlation will not have an effect on the quantified failure
rate, as dropouts are less likely to occur during the hours when it is cold
than during other hours.

\subsubsection*{Baseline probability of dropout depends on link type}

The investigation into probability of dropout for each link type also provides
additional justification for the selection of a metric that is based on the
\emph{increase} in failure probability due to weather.
%
The dropout probability significantly different for each link type, with Fiber
being the lowest and Satellite being the highest
(Figure~\ref{fig:frate_lts_timeofweek}).
%
With this metric, the baseline failure rate will be removed from all link
types; including the diurnal variations in dropout probability.

\subsection{Summary}

We selected a hazard rate-based metric that enables us to study the
statistically significant \emph{increase} in dropout probability during weather
events.
%
Then we verified that this metric will not be skewed by nightly spikes in
dropout probability because they do not correlate with occurrence of weather
conditions.
%}}}
