\section{Introduction}

Wather-related damage to vital infrastructure
can lead to significant economic harm.
%
Yet, %  in the Internet era,
little is known about the economic impact of
weather-induced outages on the most pervasive infrastructure that people use to
access the Internet: residential last-mile links.
%
For massive last-mile outages, telcos are required by U.S.~policy~\cite{cfr49}
to report the outage to the FCC.
%
However, the minimum reporting threshold is high: the outage must be at
least 30 minutes in duration, and it must have affected tens of thousands of
customers~\cite{cfr49}.
%
Researchers have also studied widespread link failures in the Internet,
like undersea cable cuts~\cite{chan-pam11, bilski-ccis09}, natural
disasters~\cite{heidemann-sandy}, and backbone router
failures~\cite{iannaccone-imw02}. 

In practice, most weather events are much more localized and not severe enough
to generate such a large outage.
%
For decades, this everyday weather has been known to lead to to smaller scale
outages of telecom infrastructure.
%
For example, early telephone and cable television engineering documents
describe how to avoid moisture in wires because it impedes signal
propagation~\cite{aiee09-jewett, toct66-smith}.  Also, rain attenuates
satellite signals above 10~GHz~\cite{ieee75-hogg}.  Finally, point-to-point
wireless links can experience multipath fading due to objects moving in the
wind~\cite{ieeecm01-bolcskei}. In short, residential links are vulnerable to
everyday weather because residential equipment and wiring are often installed
outdoors: wind can blow trees onto overhead wires, heat can cause equipment to
fail, and rain can seep into underground equipment cabinets.
 
Surprisingly, for these everyday weather conditions, there are no public
statistics on the frequency or magnitude of the outages they induce (directly
and indirectly).
%
This could be a problem for Internet-based companies because they do not know
how many customers they are losing to nature, and for regulators because they
do not know how significant the problem is, and which conditions and geographic
areas deserve their attention.
%
In this work we resolve this issue: we provide the first comprehensive study
that identifies the correlation between everyday weather and residential
Internet last-mile outages.
%
Specifically, we quantify the absolute increase in the number of outages
observed during weather, when compared to non-weather periods.

%To make this information useful to the intended audience of this work (i.e.,
%policymakers, web companies, and operators), 

Quantifying the relationship between occurrences of weather and an increase in
outages cannot be answered with a short term %, broadly collected
study.
%
The data set needs to be longitudinal because weather is
\emph{seasonal}---certain weather conditions only happen at certain times of
year---and because some weather events are \emph{rare} enough that providers in a 
specific location may not be adequately prepared.
%
Targeted probing is needed because weather is \emph{localized}: at any time
only specific geographic locations are exposed to weather conditions.
%
Broad observation of outages of several links will capture correlated outages
of several hosts, such as the work by Heidemann et al.~\cite{imc08-heidemann,
trinocular}, but it will not reveal failures of individual links as may be
the case for weather.
%
Although some systems can obtain detailed measurements at residential
gateways~\cite{ripe-atlas,sundaresan-sigcomm11}, the limited deployment of
these measurement systems make them inadequate for studying the scale needed to
observe many different weather conditions, multiple times, in different
geographic areas.
%
Therefore, we performed a seven year longitudinal study with targeted
measurements of residential links surviving weather events.

In 2011, we introduced a measurement system for this task called
ThunderPing~\cite{schulman-imc11}.
%
For the past seven years ThunderPing has been following forecasts of weather in
the U.S. and pinging a sample of 100 hosts from each last-mile provider in the
area for six hours before, during, and six hours after the forecasted weather
event.
%
The focus of our initial paper on ThunderPing was its probing methodology, but
it also included a preliminary study that looked at 66 days of data.
%
Given how limited the data set was, we were unable to draw statistically 
significant conclusions and we saw only one season, summer, of one
year. % ---therefore we could not, with any confidence, quantify the increase in
% outages during weather events.
%
We also did not have enough data to explore variations in effect of
weather by geography,
%
nor could we explore if the likelihood of failure varies
with continuous weather conditions (e.g., wind speed),
 
In this paper, the time totaled across all responsive links
exposed to different weather events is in the \textbf{centuries}.
%
For example,
we have observed a total of \emph{100 centuries} of DSL links exposed to cold
weather.
%
This large data set enabled us to address all of these significant limitations
of our prior preliminary study.
 
There is a challenge with quantifying how weather correlates with
outages: outages are relatively uncommon events, and thus every outage
is a significant event.
%
%This means every single outage is a significant event.
%
This is compounded by the fact that we wish to analyze subsets of our
data to focus on, say, particular link types or locations.
%% Yet, we need to slice and dice the data to isolate different properties
%% of residential links that may be related to their failure likelihood:
%% namely their link type and location.
%
With so few outages observed compared to the time that links are responsive,
it is difficult to determine if different weather causes a statistically significant
increase in outages.
%
To address this issue, we borrow statistical tools from epidemiology
that enable us to reason about the \emph{inflation} in dropout
probability, and to establish statistical significance to our results,
even though failures happen at relatively low rates.
%
We detail this approach in Section~\ref{sec:hazard}, as we believe it
to be of general use to the community.


%% known as
%% \emph{hazard rate}.
%% %
%% This tool enables us to establish statistical significance of our results, even
%% though there are relatively few failure events.
%% %
%% This metric enables us to measure the \emph{increase} in probability of a failure
%% that correlates with a specific type, or intensity, of weather.
%% 
%% We expect that this metric may be useful for web and cloud service operators
%% because they can use it to quantify how many of the outages that they observe
%% are likely due to weather alone, and thus estimate how much revenue they are
%% losing to weather outages.
%% %
%% It may be useful for regulators because they can determine if the added number of
%% failures from weather is significant enough to warrant new policy focused on
%% addressing weather outages, such as advocating for undergrounding
%% infrastructure as it has a demonstrated improvement on robustness to
%% weather~\cite{undergrounding}.

Another challenge is this metric could be artificially inflated by
weather conditions coinciding with daily network state changes such as
maintenance or renumbering~\cite{addrchange-reasons}.
%
We verify that weather does not appear to be positively correlated with
peak diurnal failure periods.
% Rk: I did not want recovery in the dissertation, it's
% tangential. Besides, we will eventually do a better job using the NS data.
%  and to do recovery time analysis we
% select networks and outage durations unlikely to suffer from address
% renumbering, allowing us to ensure that we are probing the same
% residential link before and after the outage.

\paragraph{Observations and Contributions}
%
We present a dataset spanning seven years, all weather conditions, and
76~billion responsive pings to 8.7~million hosts throughout the U.S. 
%
We apply techniques from epidemiology to attribute statistically
significant rates of dropout to individual weather conditions.
%
Our key findings span four broad areas of analysis:

\begin{itemize}

\item \textbf{Link type variations} (\S\ref{sec:inflations}): Different link
types experience weather in highly varying ways.  For instance, compared to
wired link types (cable, DSL, fiber), wireless link types (WISPs and satellite)
experience greater increases in dropout rates during rainy conditions and high
temperatures, but often \emph{decreases} in dropout rates in snow and cold
temperatures.

\item \textbf{Geographic variations} (\S\ref{sec:geography}): Different
geographic regions can be affected to varying degrees.  For instance,
Midwestern U.S.~states are more prone to failures in thunderstorms and rain
than coastal states.  Southern states are more prone to failures in snow than
other states.

\item \textbf{Continuous variable analysis} (\S\ref{sec:continuous}): Most link
types have highly nonlinear dropout rates with respect to changes in
temperature, wind speed, and precipitation.  For temperature, dropout rates are
typically non-monotonic; satellite links drop out more in moderate temperatures
than low or high temperatures.

% \item \textbf{Recovery times} (\S\ref{sec:recovery}): The amount of time to
% recover from a dropout varies significantly across different weather
% conditions.  Thunderstorms take the longest for cable links to recover from.

\end{itemize}

Our findings have ramifications on how network outage detection and analysis
should be performed; limiting measurements to any particular geographic region,
link type, or time of year can introduce statistically significant bias.
%
We believe our results also have implications for network administrators and
policy-makers; an increased use of satellite links in the Midwestern U.S.~has
resulted in those states' increased dropout rates in rainy weather.
%
We will be making all of our data and code publicly available.

% New contributions =============================================================

% Can quantify weather because we have more data

% Tons more data allowed us to compute confidence intervals

% Double checking if we have more failures due to maintenance periods and what not

% Failure duration 

% Double checking if failure duration with NetSession data

% Each of the PLOTS, DUH!

% Extra text {{{

% Among our findings, we show that temperature has a
% surprising relationship with link reliability: extreme
% temperature correlates with failures of typical link types,
% while satellite links and dialup links see high failure at
% modest temperatures.  Wind degrades reliability proportional
% to the square of wind speed, and the amount of hourly
% precipitation, in contrast, has a less pronounced effect on
% reliability.  However, the type of precipitation has a
% substantial effect---freezing rain, hail, and thunderstorm
% often double the probability of seeing a failure, depending 
% on link type.

% 
% A location can be exposed to the same weather patterns year over year, then one
% year a severe storm such as Hurricane Sandy in New York City in 2012.
% %
% This is important for our study because it will reveal how resilient last-mile
% Internet links are to failures when an anomalous weather event occurs that
% providers are not prepared for.
% 
% Therefore, the seven years worth of data that we collected for this study
% provides an opportunity to quantify the effect of weather on a particular
% location.
%

% Unlike the measurements needed to understand
% these failures, the collecting measurements needed to understand residential link failures
% requires a particularly broad set of hosts. Weather 
% varies by location and time, and how residential hosts fail in weather 
% depends on location, link type, and Internet service provider.

% Evaluation of weather-related failures in this work presented in
% Section~\ref{sec:failurerate} are backed with more pings which enable
% statistical comparisons between failures in different weather conditions, and
% observations of continuous weather conditions such as wind speed and
% temperature.

% For the continuous weather measurements, we observe clear
% relationships between the probability of failure depending on the residential
% link type~\todo{nums}. For the discrete weather measurements, we observe~\todo{nums}.

% With both breadth and depth, the observations of residential link
% failures can also be correlated with other possible causes: We study weather as
% a likely cause of outages to gain confidence that outages can be detected
% remotely; this confidence enables future comparisons of reliability measures
% across providers and countries.


% Weather can cause fixed \lastmile transmission media to
% fail~\cite{schulman-imc11}. 
% 
% Also there are so many different ISPs that it is difficult to get a global
% picture of the weather-related failures.
% 
% We know a lot about how loss of vital infrastructure affects business from the supplier side.
% %
% We do not however know what the costs are of having customers disconnected due to weather.
% In this work we deliver on that promise
% 
% 
% Unlike other vital infrastructure, residential Internet depends on the operation
% of other vital infrastructure to function.
% %
% Also, it has a lower barrier for a user to shut it down on their own, in
% particular to avoid damage of their equipment during storms.
% %
% If it does turn out that users are doing this, this is still valuable knowledge
% for operators as those customers will not be able to access their networks.
% 
% 
% (1) Categorical
%  - overall
%  - geographic (rama finds the worst link type/geography combination)
%}}}

