

\subsubsection{Proposed work: Set timeouts based upon
  destination addresses}

The wide variation in observed latencies for IP addresses around the
world indicate that probers should set timeout values
based upon the addresses that they are probing. Even a 3s
timeout may suffice for 90\% of addresses in the ISI survey since 90\% of addresses respond
within 3s for 99\% of the pings sent to them. My proposed work is to find expected latency values
associated with the IP addresses that need to be probed, and to set
their timeouts accordingly.
 
I propose to find expected latencies for any IP address on the
Internet by analyzing historical and current ping data, available from
the Zmap project~\cite{censys-icmp}. Zmap has continued to perform
their scans of the IPv4 Internet, averaging one scan per week since
April 2015. For each IP address that has consistently responded to
pings, I expect to have roughly 100 samples. I will calculate expected
latencies for all addresses using their own latencies weighted by the
number of observed samples and will also include latency samples of
other ``related'' addresses. Related addresses can be addresses
belonging to the same /24 network, addresses belonging to the same
ISP, addresses sharing the same last-hop router, addresses from the
same dynamically addressed pools etc; I describe related addresses in
more detail in Section~\ref{sec:last_mile}.

Once I have determined the expected latencies for all IP addresses
that respond to pings in the IPv4 Internet, the next task is to
determine appropriate per-address timeouts based upon the destinations
that need to be probed. Given any address to probe, I will modify the probing scheme to
set timeouts that are just high enough as to capture almost all
responses (say 99.9\%) from that address. Setting adaptive timeouts this way will achieve the
twin goals of capturing most responses while also keeping the state
required at the prober low.