

\section{Conclusion and Discussion}

Researchers use tools like ping to detect network outages,
but generally guessed at the timeout after which a ping
should be declared ``failed'' and an outage suspected.  The
choice of timeout can affect the accuracy and timeliness of
outage detection: if too small, the outage detector may
falsely assert an outage in the presence of congestion; if
too large, the outage detector may not pass the outage along
quickly for confirmation or diagnosis.

We investigated the latencies of responses in the ISI survey
dataset to determine a good timeout, considering the
distributions of latencies on a per-destination basis.
Foremost, latencies are higher than we expected, based on
conventional wisdom, and appear to have been increasing.
We show that these high latencies are not an artifact of
measurement choices such as using ICMP or the particular
vantage points or probing schemes used, although different data sets vary somewhat.
We show that high latencies are not caused by links with a substantial
base timeout, such as satellite links.  
% 
% 
% We analyzed the latencies of echo responses, including
% unmatched responses in the ISI survey data set, to determine
% what timeout would capture different fractions of
% responses from different fractions of remote IP addresses.
% We showed that high latencies are not an artifact of ISI
% data. % , and in fact, it may underestimate the extent of large latencies by sampling.
% We showed that high latencies are not an artifact of using ICMP
% echo requests: prioritization that might favor data-carrying TCP or UDP
% protocols does not have this substantial of an effect.
% We showed that high latencies have been increasing over the
% last few years by comparing different surveys taken at different times.
% We showed that satellite links are not particularly blameworthy
% in requiring high timeout values.
Finally, we showed that in many instances, the initial communication to
cellular wireless devices is largely to blame for high latency measures.
Similar spikes that may be consistent with handoff also dissipate over time,
to more conventional latencies that support application traffic.
With this data, researchers should be able to reason about
what to expect in terms of false outage detection for a
given timeout and how to design probing methods to account for
these behaviors.

% Our initial hypothesis was that it would be a simple matter
% to confirm that widely used timeout values would be adequate
% for studying outages, or failing that, that one or two
% additional seconds would be enough.
As memory capacity and performance becomes less of a
limiting factor, we believe that the lesson of this work is
to design network measurement software to approach outage
detection using a method comparable to that of TCP: send
another probe after 3 seconds, but continue listening for a
response to earlier probes, at least for a duration based, at
least in part, on the error rates implied by
Table~\ref{tbl:grand_2015}. 
% We plan
% to use 60 seconds when we need a timeout, and avoid timeouts
% otherwise.

% a per-ISP
% and per-region basis, since there can be wide variation in
% latencies. This work provides timeout values that would capture most
% pings (say 99\%) per-ISP and per-region. 

% The wide variation in observed latencies for IP addresses around the
% world indicate that probers should set timeout values
% based upon the addresses that they are probing. Even a 3s
% timeout may suffice for 90\% of addresses in the ISI survey since 90\% of addresses respond
% within 3s for 99\% of the pings sent to them. 

When investigating historical outage measurement data collected by
probing based techniques, the timeouts used by the technique must be
compared with timeouts that would have captured almost all responses
for the addresses pinged by the technique. For example, Thunderping
probes addresses only in U.S. networks. For these addresses, both the ISI
and Zmap datasets showed that more than 99\% of ping responses arrived
within the 5s timeout used by Thunderping. The probability of false
outage inferences due to high latency is small. However, if the Thunderping
technique had been used to ping addresses in South American cellular
ISPs, there would be a significantly higher probability of detecting
false outages, since the 5s timeout would have missed many delayed responses.

% In cases where a timeout is 


% My proposed work is to find expected latency values
% associated with the IP addresses that need to be probed, and to set
% their timeouts accordingly.
 
% I propose to find expected latencies for any IP address on the
% Internet by analyzing historical and current ping data, available from
% the Zmap project~\cite{censys-icmp}. Zmap has continued to perform
% their scans of the IPv4 Internet, averaging one scan per week since
% April 2015. For each IP address that has consistently responded to
% pings, I expect to have roughly 100 samples. I will calculate expected
% latencies for all addresses using their own latencies weighted by the
% number of observed samples and will also include latency samples of
% other ``related'' addresses. Related addresses can be addresses
% belonging to the same /24 network, addresses belonging to the same
% ISP, addresses sharing the same last-hop router, addresses from the
% same dynamically addressed pools etc; I describe related addresses in
% more detail in Section~\ref{sec:last_mile}.

% Once I have determined the expected latencies for all IP addresses
% that respond to pings in the IPv4 Internet, the next task is to
% determine appropriate per-address timeouts based upon the destinations
% that need to be probed. Given any address to probe, I will modify the probing scheme to
% set timeouts that are just high enough as to capture almost all
% responses (say 99.9\%) from that address. Setting adaptive timeouts this way will achieve the
% twin goals of capturing most responses while also keeping the state
% required at the prober low.