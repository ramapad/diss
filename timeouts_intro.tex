
% \section{Understanding false probe-loss inference due to early
%   timeout}
\chapter{Mitigating false inferences due to early timeout}
\label{cpt:timeouts}

In this chapter, I begin by describing how probe responses delayed beyond
timeouts used by current probing-based techniques can lead to false
probe-loss inferences, and thereby to false outage inferences. 

Next, I describe work with colleagues that measured how frequently responses
to active probes are delayed beyond timeouts set by existing
approaches. We began by studying ping latencies from Internet-wide surveys~\cite{census-survey} conducted by ISI,
including 9.64 billion ICMP Echo Responses from 4 million different IP
addresses in 2015, and identified addresses that are particularly likely
to be subject to high delay.  We then \emph{verified} the high latencies
by repeating measurements using other probing techniques, comparing the
statistics of various surveys, and investigating high-latency
behavior of ICMP compared to UDP and TCP.  Finally, we
explained these distributions by isolating satellite links,
considering sequences of latencies at a higher sampling rate,
and classifying a complete sample of the Internet address
space through a modified Zmap client. These results are reproduced
from our published work~\cite{timeouts}.

Using these results, I discuss how probing-based outage
detection techniques can mitigate false outage inferences caused by
delayed responses.


