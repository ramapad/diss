
% \section{Understanding false probe-loss inference due to early
%   timeout}
\chapter{Mitigating false inferences due to early timeout}
\label{cpt:timeouts}

In this chapter, I describe how probe responses delayed beyond
timeouts used by current probing-based techniques can lead to false
probe-loss inferences, and thereby to false outage inferences. 

Next, I describe work with colleagues that measured how frequently responses
to active probes are delayed beyond timeouts set by existing
approaches. We began by studying ping latencies from Internet-wide surveys~\cite{census-survey} conducted by ISI,
including 9.64 billion ICMP Echo Responses from 4 million different IP
addresses in 2015, and identified addresses that are particularly likely
to be subject to high delay.  We then \emph{verified} the high latencies
by repeating measurements using other probing techniques, comparing the
statistics of various surveys, and investigating high-latency
behavior of ICMP compared to UDP and TCP.  Finally, we
explained these distributions by isolating satellite links,
considering sequences of latencies at a higher sampling rate,
and classifying a complete sample of the Internet address
space through a modified Zmap client. These results are reproduced
from our published work~\cite{timeouts}.

Using these results, I discuss how probing-based outage
detection techniques can mitigate false outage inferences caused by
delayed responses.



However, are these high latencies spread randomly across all addresses
in the Internet? Or instead, are some addresses particularly likely to
experience high latencies?
% The former would be the case if core
% routers in the Internet experience congestion, which could potentially
% delay packets for a wide swath of the Internet's addresses. The latter
% would happen if the cause of the high latencies is something to do
% with the last-mile link, so that a few addresses experience higher delay
% owing to some aspect of their last-mile link.
To find how high latencies are distributed across the Internet, we
investigated which Autonomous Systems' addresses are particularly
likely to have high latencies. For this analysis, we used three Zmap
scans conducted in 2015 to identify high latency addresses, conducted
on May 22, Jun 21 and Jul 9. These scans were conducted at different
times of the day, on different days of the week and in different
months. For each of these Zmap scans, we used Maxmind to find the ASN
and geographic location for every address that responded.

\newcommand{\hdrbar}[1]{\multicolumn{1}{c|}{\textbf{#1}}}
\begin{table*}[t]%
  \begin{center}%
  \begin{small}%
  \begin{tabular}{ll|rrr|rrr|rrr}
  % \begin{tabular}{r|rrr|rrr|rrr|rrr}
  % \begin{tabular}{rl|r|rr}
    % & & \multicolumn{3}{c|}{\textbf{April 2015}} &
    & & 
    \multicolumn{3}{c|}{\textbf{May 2015}} &
    \multicolumn{3}{c|}{\textbf{June 2015}} &
    \multicolumn{3}{c}{\textbf{July 2015}} \\ 
    % \hdr{ASN} & \hdr{$>$1s} & \%  & \hdr{Rank} &
    % \hline 
    \hdr{ASN} & \hdrbar{Owner} & 
    \hdr{$>$1s} & \%  & \hdrbar{Rank} &
    \hdr{$>$1s} & \%  & \hdrbar{Rank} &
    \hdr{$>$1s} & \% & \hdr{Rank} \\
    % \hdr{$>$1s} & \% & \hdr{Rank} \\
    \hline 
    26599 & TELEFONICA BRASIL & 
    % 26599 &
    % 2,941,446 & 79.275 & 1 & 
    3.56M & 80.4 & 1 & 
    3.87M & 77.5 & 1 &
    4.20M & 77.0 & 1\Tstrut \\

    26615 & Tim Celular S.A. &
    1.35M & 74.5 & 3 &
    1.42M & 71.5 & 2 &
    1.72M & 71.6 & 2 \\

    45609 & Bharti Airtel Ltd. &
    1.46M & 76.6 & 2 &
    1.21M & 81.0 & 3 &
    1.03M & 79.2 & 3 \\

    22394 & Cellco Partnership &
    0.55M & 73.4 & 8 &
    0.58M & 73.5 & 4 &
    0.63M & 72.7 & 4 \\

    1257 & TELE2 &
    0.67M & 69.5 & 5 &
    0.42M & 65.5 & 9 &
    0.58M & 67.4 & 5 \\

    27831 & Colombia Movil &
    0.53M & 68.8 & 9 &
    0.54M & 64.3 & 5 & 
    0.53M & 62.8 & 6 \\

    6306 & VENEZOLAN &
    0.69M & 77.3 & 4 &
    0.41M & 76.4 & 10 &
    0.40M & 75.7 & 10 \\

    9829 & National Internet Backbone &
    0.57M & 27.6 & 7 &
    0.43M & 30.9 & 7 &
    0.43M & 29.5 & 9 \\

    4134 & Chinanet &
    0.60M & 1.5 & 6 &
    0.38M & 0.9 & 11 &
    0.34M & 0.9 & 11 \\

    35819 & Etihad Etisalat (Mobily) & 
    0.42M & 54.0 & 10 &
    0.43M & 54.5 & 6 &    
    0.45M & 55.8 & 8 \\
  \end{tabular}
  \end{small}
  \end{center}
  \caption{\label{tbl:zmap_asns} Autonomous Systems sorted by the
    addresses summed across three Zmap scans for addresses that observed
    RTTs greater than 1s. The table shows for each AS: the number and
    percentage of addresses with RTT greater than 1s and the rank in that scan.}
\end{table*}


Inspecting the Autonomous Systems and countries of addresses with high latencies
reveals that a majority of them belong to cellular ASes in South
America and Asia, as shown in Table~\ref{tbl:zmap_asns}. AS26599
(TELEFONICA BRASIL), a cellular AS in Brazil, has the most addresses
with latencies exceeding 1s---more than double that of the next
largest AS in each of the scans. The next two ASes, AS45609 (Bharti
Airtel Ltd.), and AS26615 (Tim Celular), are also cellular, and so are
5 of the remaining 7 ASes in the top 10 ASes with the most addresses
with latencies exceeding a second. Also notable is that more than 70\%
of all responding addresses in these ASes had latencies exceeding a
second. 

% Include the next para if you really want to talk about the
% experiments that appear to confirm that we're reaching cellular devices.
% We conducted additional experiments upon some addresses that
% were particularly likely to have high latencies from the ISI dataset,
% and confirmed that 

While the results from the ISI and Zmap datasets reveal that high
latencies exceeding typical timeouts occur in the Internet, they also
show that these latencies are not uniformly distributed across all
addresses. This observation lies at the root of my proposed work to
set timeouts for probe-based remote outage detection systems.

\subsubsection{Proposed work: Set timeouts based upon
  destination addresses}

The wide variation in observed latencies for IP addresses around the
world indicate that probers should set timeout values
based upon the addresses that they are probing. Even a 3s
timeout may suffice for 90\% of addresses in the ISI survey since 90\% of addresses respond
within 3s for 99\% of the pings sent to them. My proposed work is to find expected latency values
associated with the IP addresses that need to be probed, and to set
their timeouts accordingly.
 
I propose to find expected latencies for any IP address on the
Internet by analyzing historical and current ping data, available from
the Zmap project~\cite{censys-icmp}. Zmap has continued to perform
their scans of the IPv4 Internet, averaging one scan per week since
April 2015. For each IP address that has consistently responded to
pings, I expect to have roughly 100 samples. I will calculate expected
latencies for all addresses using their own latencies weighted by the
number of observed samples and will also include latency samples of
other ``related'' addresses. Related addresses can be addresses
belonging to the same /24 network, addresses belonging to the same
ISP, addresses sharing the same last-hop router, addresses from the
same dynamically addressed pools etc; I describe related addresses in
more detail in Section~\ref{sec:last_mile}.

Once I have determined the expected latencies for all IP addresses
that respond to pings in the IPv4 Internet, the next task is to
determine appropriate per-address timeouts based upon the destinations
that need to be probed. Given any address to probe, I will modify the probing scheme to
set timeouts that are just high enough as to capture almost all
responses (say 99.9\%) from that address. Setting adaptive timeouts this way will achieve the
twin goals of capturing most responses while also keeping the state
required at the prober low.