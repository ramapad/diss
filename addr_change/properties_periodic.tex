
\subsection{ISPs that renumber periodically}
\label{sec:periodic_asns}

% \newcommand{\hdr}[1]{\multicolumn{1}{c|}{\textbf{#1}}}
\newcommand{\ehdr}[1]{\multicolumn{1}{c}{\textbf{#1}}}
\begin{table*}[t]%
  \begin{center}%
  \begin{tiny}%
  \begin{tabular}{ccc|r|rr|r<{\%}r<{\%}|r<{\%}r<{\%}}
    \ehdr{AS} & \ehdr{ASN} & \hdr{Country} & \hdr{$d$} &
    \ehdr{N} & \hdr{$f^p_d > 0.25$} & \ehdr{$f^p_d > 0.5$} & \hdr{$f^p_d
      > 0.75$} & \ehdr{$MAX \leq d$}
     & \ehdr{Harmonic}\\
\hline
\multicolumn{3}{c|}{All}&  24&   2272&  193&   88.6&   68.4&   43.5&   89.6\\
\multicolumn{3}{c|}{All}& 168&   2272&  123&   74.0&   13.8&   94.3&   98.4\\
Orange         &  3215&France    & 168&    122&  111&     77&     14&     98&     99\\
DTAG           &  3320&Germany   &  24&     63&   51&     96&     86&     78&     98\\
Telefonica DE 2&  6805&Germany   &  24&     17&   15&     93&     80&     27&     93\\
Telefonica DE 1& 13184&Germany   &  24&     14&   14&     93&     86&     21&    100\\
PJSC Rostelecom&  8997&Russia    &  24&     22&   13&    100&     69&     23&    100\\
BT             &  2856&U.K.      & 337&     67&   13&     15&      0&     38&     62\\
Proximus       &  5432&Belgium   &  36&     41&   12&     83&      8&      0&     83\\
A1 Telekom     &  8447&Austria   &  24&     12&   11&    100&     91&     73&    100\\
Vodafone GmbH  &  3209&Germany   &  24&     21&    9&     78&     11&      0&     89\\
Hrvatski       &  5391&Croatia   &  24&      7&    7&    100&    100&     43&     86\\
ISKON          & 13046&Croatia   &  24&      6&    6&     83&     33&      0&    100\\
ANTEL          &  6057&Uruguay   &  12&      6&    6&    100&    100&     33&    100\\
Global Village Telecom & 18881&Brazil    &  48&      6&    6&    100&     67&      0&     17\\
Mauritius Telecom& 23889&Mauritius &  24&      6&    5&    100&     20&     20&    100\\
JSC Kazakhtelecom&  9198&Kazakhstan&  24&     15&    5&     80&     80&     60&     80\\
Orange Polska  &  5617&Poland    &  22&     10&    5&    100&    100&     60&     80\\
VIPnet         & 31012&Croatia   &  92&      7&    4&     75&      0&     75&     75\\
Proximus       &  5432&Belgium   &  24&     41&    4&     50&     25&      0&     75\\
Digi Tavkozlesi& 20845&Hungary   & 168&      4&    4&    100&     25&    100&    100\\
Orange Polska  &  5617&Poland    &  24&     10&    4&    100&    100&     50&    100\\
Free SAS       & 12322&France    &  24&     12&    3&    100&     67&      0&     67\\
SONATEL-AS     &  8346&Europe    &  24&      7&    3&     33&     33&     33&     33\\
Net by Net     & 12714&Russia    &  47&      7&    3&    100&    100&     67&    100\\
   \end{tabular}
  \end{tiny}
  \end{center}
  \caption[ASes that renumber periodically]{\label{tbl:periodic_asns} Autonomous
    systems that had at least three probes with a total time fraction
    for duration $d$ (in hours) greater than 0.25. $f^p_d > 0.25$ shows
    the number of probes that had a total time fraction at $d$ greater than 0.25;
    $f^p_d > 0.50$ and $f^p_d > 0.75$ show the percentage of those probes
    that had fractions greater than 0.5 and 0.75 for the same duration.
    $MAX \leq d$ shows the percentage of probes whose maximum duration was no greater than
    $d$. ``Harmonic'' represents the percentage of probes that, if not renumbered after $d$, are renumbered after some multiple of $d$ hours.
    The ASes are sorted in decreasing order of $f^p_d > 0.25$.}
\end{table*}

In this section, we look specifically at ISPs that renumber
periodically to infer the period over which they renumber,
the fraction of the ISPs' probes which periodically renumber, how
reliably the renumbering occurs at the end of the period, and whether the
renumbering is synchronized across probes.  We classify a
probe as ``periodic'' when its total time fraction at
some duration $d$ exceeds 0.25.  We set the threshold to 0.25
because we expect a probe whose address is reassigned
periodically to sometimes have a shorter duration, say, due
to a reboot, and sometimes have a longer duration, say, by
receiving the same address again. 

We consider autonomous systems having at least five probes with an
address change of which at least three probes are
periodic, and provide an overview of their renumbering
period and behavior in Table~\ref{tbl:periodic_asns}.
The periodic duration $d$ is shown in hours; 24 hour durations are
typical.  Renumbering in this table is primarily a feature of
central Europe, with some in Russia, Kazakhstan, Mauritius, and South America.
We describe the rest of the columns in the next subsections.

\subsubsection{What fraction of probes is periodic?}

Even for ISPs such as Orange and DTAG which have total
time fraction at period $d$ in excess of 0.5, not all address durations equal
$d$; some durations are shorter and others longer as seen in Figure~\ref{fig:top_asns_all_durs}. One possible
explanation is that only a few
probes in these ISPs were periodically renumbered while others were
not. Alternately, periodic probes sometimes have address
durations not equal to $d$. We find that it is usually a combination
of both factors that lead to non-periodic durations in these
ISPs, although the extent to which each is responsible varies by ISP.

In Table~\ref{tbl:periodic_asns}, the $N$ column shows
the number of probes with at least one address change in the
dataset.  The next column, $f^p_d > 0.25$, shows the number
of periodic probes---those having a time fraction of
more than 0.25 at duration $d$. In some ISPs, only a small fraction of
probes are periodically renumbered. For example, only a fifth
of the probes in BT were periodic with a 2-week period, partially explaining why
the total time fraction at 2-weeks for BT in
Figure~\ref{fig:top_asns_all_durs} is only 0.13. 

The subsequent columns, $f^p_d > 0.5$ and $f^p_d > 0.75$
show what percentage of the periodic probes are persistently
so, where the total time fraction at duration $d$ is more
than half or three quarters.  We show percentages rather
than raw counts in these columns to simplify the comparison,
given that these providers have different sizes.  A high percentage
indicates that most of the periodic probes (with $f^p_d > 0.25$),
are strongly so ($f^p_d > 0.75$).  A low percentage indicates
that probes may either be reassigned early (due to outages) or late
(due to inconsistent reassignment). We can see that only 15\% of the
periodic probes in
BT had $f^p_d > 0.5$  and none had $f^p_d > 0.75$, providing further
explanation for why the total time fraction at 2-weeks for BT is low.

Other ISPs have a much larger fraction of their
probes that are periodic: more than 80\% of
probes in Orange, DTAG, Telefonica Germany, A1 Telekom, Hrvat\-ski, ISK\-ON,
ANT\-EL, Global Village Telecom, Mauritius Telekom, 
Orange Polska, and Digi Tavkozlesi are periodic. 
For each of these ISPs, more than 75\% of probes are persistently
periodic, having $f^p_d > 0.5$.
For DTAG, Telefonica, A1 Telekom, Hrvatski, ANTEL, and Orange Polska, 
more than 75\% of probes have $f^p_d > 0.75$.
Notable is Orange Polska, which has four of its ten probes 
periodic at 24 hours, and five more probes periodic at 22
hours, but 100\% of them have a time fraction at their respective
durations greater than 0.75. 

Probes in these ISPs typically have
address durations capped at $d$. Address durations can sometimes be shorter---potentially due to outages or
reboot/reconnect events as we show in Section~\ref{sec:outages}---but
can occasionally be larger than $d$ as well. We study these next.

\subsubsection{Why are some address durations longer than the period?}

Some address durations exceed the typical period, $d$, for an ISP.  
We
would like to determine whether this is a behavior limited to a few
probes in the ISP (potentially caused by unusually designed CPE devices), or if the longer-than-typical durations are
spread across probes.  

How many periodic probes have an address duration longer
than $d$?  We expected that no address duration for such
probes would exceed the periodic duration $d$.  That
is, if the ISP was renumbering a probe on a schedule, then
some additional renumbering would be possible due to other
reasons, but the probe would never keep its address longer
than $d$.  It turns out that this expectation is not the
case.  The column $MAX \leq d$ shows the percentage of the
periodic probes that had their maximum address duration less than
$d$ (to capture only those durations that clearly exceeded
$d$, we adjusted $d$ to be $d$ $+$ $5\%$ for this column).  Across all periodic
probes, 94\% of those that appear to be on a one-week renumbering schedule 
did not have an address duration longer than one week; only 44\% of those
that appeared to be on a one-day renumbering schedule had all durations limited
by twenty-four hours.

This fraction seemed surprisingly low.  Why would so many
probes show daily renumbering, even reporting a total time
fraction of 0.75, when the probe might also keep its address longer?  We considered two possible explanations that
would have the same symptoms: that a periodic renumbering
was skipped or that the same address was (perhaps by random
chance) assigned again.  In these cases, rather than see an
address change after 24 hours, we might see one at 48 or
even 72 hours.  We term such address changes
``Harmonics'', and consider what fraction
of the time all address changes are at or before $d$ (as
expected), or occur at a multiple of $d$.  The percentage of
probes that match this loosened definition (a superset of those in
$MAX \leq d$) appears in the last column of Table~\ref{tbl:periodic_asns}.
Most periodic probes from all ISPs
except Global Village Telecom and SONATEL-AS have maximum durations of this kind.  

% Let's force the table to appear here
\clearpage

\subsubsection{Are changes synchronized?}

\begin{figure}[th]
 \centering
    \includegraphics[width=3in]{addr_change/figs/weekly_3215_a_periodicrenums_per_h24_connlogs_bar}
  \caption[Periodic address changes in Orange]{\label{fig:3215_renums_per_h24}Periodic address changes
    in Orange appear more evenly distributed among the hours of the
    day.}
\end{figure}

\begin{figure}[th]
  \centering
    \includegraphics[width=3in]{addr_change/figs/daily_3320_a_periodicrenums_per_h24_connlogs_bar}
  \caption[Periodic address changes in Deutsche Telekom]{\label{fig:3320_renums_per_h24}Periodic address changes
    are more likely in some hours for Deutsche Telekom.}
\end{figure}

We imagine two broad strategies for daily renumbering:
either leaving each customer on an independent, free-running
clock that resets after 24 hours, or synchronizing all
address changes to an off-peak time when few would be
interrupted.  Both seem reasonable strategies: independent
clocks seem simple to implement, synchronized address
changes seem more likely to shuffle addresses since many
addresses are made available during the synchronized
interval.  However, if one were to blacklist addresses for
misbehavior, knowing which strategy is in use would help to
choose for how long to keep the blacklist entry.  We expect
that plotting the time of day at which addresses change for
each ISP will expose whether the renumbering is synchronized.

For Orange and DTAG, the two ISPs with the most periodic probes, we
choose the hour of the day in which every address duration that had
duration $d$ ended and show these in Figure~\ref{fig:3215_renums_per_h24}
and Figure~\ref{fig:3320_renums_per_h24}. For Orange, periodic address
changes are not concentrated during any specific hours of
the day. However, DTAG assigns periodic
durations more often during some hours of the day. In private
correspondence with a large European ISP, we learned that many CPE devices come with an option to choose the time
at which they should disconnect and reconnect to receive a new
address, as a privacy
feature. Figure~\ref{fig:3320_renums_per_h24} supports this
deployment scenario, observing almost three quarters of all periodic
address changes between hours 24 to 6 (in GMT). However, some
CPEs do not have this feature because a quarter of the periodic
address changes happen at other hours of the day.

