
% \begin{sloppypar}
% \section{Outage-caused address changes}
% \end{sloppypar}
\section{Outage-caused address \\ changes}
\label{sec:outages}

In Section~\ref{sec:periodic_asns}, we saw that even probes from ISPs
that renumber periodically
often have durations shorter than the typical period. In this section,
we study another potential cause of address change: outages occurring
at the CPE (customer premises equipment), due to loss of power or network connectivity. Here, we quantify how frequently and for which probes an
outage event at the CPE device appears to
cause the reassignment of its IP address. If an outage event occurs
at approximately the same time as an address change, we assume that the
outage caused the address change. If an outage event occurs
distant in time from an address change, then we assume that the outage
did not cause an address change. 

There are three versions of RIPE Atlas probes: v1,v2, and v3. More
than 75\% of probes are v3, although the distribution of versions within
individual ISPs varies. We find network outage events
on all versions of probes since network outages are by definition caused when a
probe was up and reporting measurements.
However, finding power outage
events is confounded by the presence of potential false positives and
negatives. We address these in detail next and describe our approach
for filtering falsely inferred power outages.

\subsection{Filtering falsely inferred power outages}

The SOS-uptime data (Section~\ref{sec:sos}) allows us to
determine when the \emph{probe rebooted}. Ideally, however, we would like to know
\emph{when the CPE rebooted}.  Fortunately, probe
reboots are often representative of CPE reboots due to a
combination of how the RIPE NCC suggests that probes be
installed~\cite{cpepowersprobe} and expected fate sharing of co-located
devices powered together, as we describe next.  

The RIPE Atlas probe gets power from USB; because of this
design, the probe can be powered by the USB port on the CPE and
will be power-cycled whenever the CPE reboots. When the probe
is plugged into the CPE, or both together are power-cycled,
a probe reboot indicates that the CPE also rebooted. 
These represent the 
typical cases that are useful for the  analysis of power outage related address changes.  The
potential error scenarios are as follows. When the CPE
alone is rebooted but the probe is not, we would not observe a
power outage, leading to a false negative. 
When the probe alone is rebooted but the CPE is not, we
would detect a power outage, leading to a false positive.
Although we expect probe reboots to be rare, a specific scenario in which they occur is when the probe receives a 
firmware upgrade. We discuss how to remove probe reboots due to firmware upgrades below in Section~\ref{sec:firmware}.

Older probe hardware (v1,v2) can also confound our inference of power
outages, because these probes may reboot when they create new TCP
connections, since they are vulnerable to memory fragmentation
~\cite{atlas-fragmentation}. Address changes create new TCP
connections and could induce such reboots, so for our power
outage analysis we discard data from these older probes.


\subsection{Removing reboots caused by firmware updates}
\label{sec:firmware}

The RIPE Atlas servers push firmware updates to probes
simultaneously. When a probe's TCP connection to the central controller
breaks, the probe will reboot and install the firmware update.
Our goal is to filter reboots that were associated with a firmware update,
since these reboots occur \emph{as a result of} a dropped connection rather
than as a cause.  
Figure~\ref{fig:global_rebooted_idxs_per_d} shows the number
of unique probes that rebooted on each day of 2015. We observe five
periods during the year when probes experienced more than twice as many reboots
as the median for at least two consecutive days.

\begin{figure*}[tb]
  % \centering
  \begin{center}
    \includegraphics[width=0.99\textwidth]{addr_change/figs/global_rebooted_idxs_per_d_connlogs_yearly_bar}
  \end{center}
  \caption{\label{fig:global_rebooted_idxs_per_d} Number of unique
    probes that rebooted on each day of the year.  Days with exceptionally
    many reboots follow the distribution of firmware updates.  We indicate
    days where updates seem to have been distributed with 
    diamonds along the x-axis.}
\end{figure*}
For each of these periods, we found the first day
corresponding to the spike, and identify that day as when
the firmware update was distributed.  Some dates (April 14,
July 6, October 5), agree precisely with documented
RIPE Atlas firmware and UI
updates~\cite{atlas-firmware-updates}. Other dates are close---we
observe March 23 instead of March 28, and January 25 instead
of January 14---but nevertheless show the same spike in reboots.
We then discard the first reboot for each probe that occurred
after the firmware update.

\subsection{Most outages result in an address change for some ASes}

We found network and power outage events and associated them with inter-connection gaps as
described in Section~\ref{sec:dataset}. If the connection log entries
on either side of the inter-connection gap used different
addresses, we infer that the event caused an address change and call
the address change an \emph{Address change with network outage},
\emph{Address change with power outage}, and \emph{Address change with
no-outage}, depending upon the event.

For each individual probe, we consider the conditional
probability of an address change given a detected
outage. $P(ac|nw)$
represents the conditional probability that an address change occurred
given a network outage and $P(ac|pw)$ represents the same for a power outage.
We estimate this probability using the
fraction of outages occurring contemporaneously with an address change (out of the
total number of outages).  We show the distribution of
these probabilities by probe to estimate whether the group
of probes (by geography or ISP) is dominated by those that
always or seldom change addresses on an outage.

\begin{figure}[tb]
  % \centering
  \begin{center}
    \includegraphics[width=3in]{addr_change/figs/top_asns_frac_norenums_over_totalnos_cdf}
  \end{center}
  \caption{\label{fig:top_asns_frac_norenums_over_totalnos}
Distribution of $P(ac|nw)$ per probe for the ASes with the most probes
that had at least one address change. Probes in DTAG, Orange, and BT, are far more likely to change addresses upon a
    network outage than probes in Verizon and LGI.}
\end{figure}

\begin{figure}[tb]
  % \centering
  \begin{center}
    \includegraphics[width=3in]{addr_change/figs/top_asns_frac_porenums_over_totalpos_cdf}
  \end{center}
  \caption{\label{fig:top_asns_frac_porenums_over_totalpos}
%
Distribution of $P(ac|pw)$ per probe for probes running version 3. As
with network outages, probes in DTAG and Orange are more likely
to change addresses upon power outage than probes in Verizon and LGI.}
\end{figure}

We find that the likelihood of address change upon an outage event
differs across
ASes. Figure~\ref{fig:top_asns_frac_norenums_over_totalnos} shows the
CDF of $P(ac|nw)$ for the five ASes that host the most probes with at
least one address change and at least three network outage events. We
find that probes in ASes that periodically renumber---Orange, DTAG,
and BT---have high $P(ac|nw)$ compared to probes from ASes that do not
periodically renumber, LGI and Verizon. Around half of the probes in
both Orange and DTAG had $P(ac|nw)$ equal to 1: every network outage
was accompanied by an address change!

Figure~\ref{fig:top_asns_frac_porenums_over_totalpos} shows $P(ac|pw)$
for these ASes. Recall that we discarded probes with versions 1 and 2
due to their potential to reboot as a result of an address change,
thus we have fewer samples. The AS-level behavior for power outages is
similar to network outages. DTAG and Orange tend to renumber
frequently upon power outages; half of the probes in Orange and 40\%
of the probes in DTAG have $P(ac|pw)$ equal to 1. Verizon and LGI do
not renumber frequently upon power outages; only about half of their
probes had an address change even once upon an outage. Since the
likelihood of an address change upon an outage can also depend upon
the duration of the outage, we investigate the distribution of outage
durations and the likelihood of address changes for different outage
durations in Section~\ref{sec:outage_durs}.

\begin{table*}[t]%
  \begin{center}%
  \begin{tiny}%
  \begin{tabular}{ccc|r|r<{\%}r<{\%}|r<{\%}r<{\%}}
    \ehdr{AS} & \ehdr{ASN} & \hdr{Country} & \hdr{N} & \ehdr{$P(ac|nw) > 0.8$} &
    \hdr{$P(ac|nw) = 1$}  & \ehdr{$P(ac|pw) > 0.8$} & \ehdr{$P(ac|pw) = 1$}\\
    \hline
\multicolumn{3}{c|}{All}                     &  1113  &  29.1  &  16.9  &  28.3  &  14.6\\
Orange                   &  3215  &  France   &  84    &  79  &  54  &  77  &  50\\
Telecom Italia           &  3269  &  Italy    &  28    &  71  &  50  &  57  &  21\\
BT                       &  2856  &  U.K.     &  22    &  64  &  55  &  50  &  14\\
Proximus                 &  5432  &  Belgium  &  20    &  70  &  45  &  60  &  30\\
DTAG                     &  3320  &  Germany  &  19    &  58  &  47  &  47  &  42\\
Vodafone GmbH            &  3209  &  Germany  &  12    &  83  &  75  &  58  &  42\\
Wind Telecomunicazioni   &  1267  &  Italy    &  12    &  67  &  42  &  83  &  42\\
SFR                      &  15557  & France   &  16    &  38  &  25  &  50  &  6\\
ISKON                    &  13046  & Croatia  &  6     &  100  &  50  &  83  &  67\\
PJSC Rostelecom          &  8997  &  Russia   &  7     &  71  &  29  &  57  &  14\\
   \end{tabular}
  \end{tiny}
  \end{center}
  \caption{\label{tbl:outages}Probes likely to change
    addresses upon network outages are also likely to change addresses
    upon power outages. The table shows autonomous
    systems with at least five
    probes whose conditional probability of address change upon
    network outage was greater than 0.8. The N column
    shows the number of probes with at least three
    network outages and at least three power outages. $P(ac|nw) > 0.8$ and
    $P(ac|nw) = 1$ show the percentage of N for which the conditional probability of
    address change upon network outage was greater than 0.8 and equal
    to 1 respectively, and $P(ac|pw) > 0.8$,
    $P(ac|pw) = 1$ show the same for power outages.}
\end{table*}

Since the ASes in
Figure~\ref{fig:top_asns_frac_norenums_over_totalnos} and
Figure~\ref{fig:top_asns_frac_porenums_over_totalpos} exhibit such
disparate behavior, we considered if some ASes are particularly likely
to renumber upon outages. To investigate this, we found the set of
probes with at least three network and power outages. We then found
probes with $P(ac|nw)$ of 0.8 or more and show ASes with 5 or more
such probes in Table~\ref{tbl:outages}.

First, we observe strong geographic correlation; all these ISPs are in
Europe. Second, we observe that $P(ac|pw)$ is also high; $P(ac|nw) >
0.8$ and $P(ac|pw) > 0.8$ are similar for all these ISPs (although
$P(ac|pw) = 1$ tends to be lower because our power outage detection
technique is more prone to false positives). This suggests that both
types of outages are likely to cause address changes. Third, we find
that 7 of the 10 ISPs also appeared in Table~\ref{tbl:periodic_asns}.
Maier et al.~\cite{maier2009dominant} studied the logs from an urban
area of a major European ISP that used Radius to assign addresses:
neither CPE nor Radius servers remember addresses.  The behavior of
these ISPs that nearly always renumber is consistent with the behavior
of the large DSL provider in that study. Private communication with a
large European ISP whose probes consistently had an address change
upon outage confirmed that they use PPPoE and Radius to assign
addresses for their DSL lines. We expect that this property can be
used as evidence in inferring a device's link type.


\subsection{Is there a relationship between outage duration and
  address changes?}
\label{sec:outage_durs}
\begin{figure}[t]
  \includegraphics[width=1.5in]{figs/6830_combined_merged_bar}~~~
  \includegraphics[width=1.5in]{figs/3215_combined_merged_bar}

  \caption{\label{fig:outagedurs} The likelihood of an address change (renumbering)
    given network or power outages of different durations in LGI (left)
    and Orange (right).  The top graph is a histogram; the
    complete bar represents the number of outages observed
    across all probes in that AS.  The lightly-shaded bar
    extends for those outages that also saw an address
    change.  The lower graph shows the same data as a
    percentage.  Although relatively few outages
    lasted longer than a day, the majority of these were
    coincident with an address change in both ISPs. However,
    Orange (right) changed addresses even on the shortest
    outages.}
\end{figure}

Dynamic addresses assigned using DHCP should typically retain
their addresses as long as they continue to renew their lease half-way
into the lease duration as the standard
recommends~\cite{rfc2131}. However, an outage could
prevent them from renewing their lease. Depending upon the address
churn at the time, the address they had previously been assigned may
be reassigned to another device.  In this way, an outage
longer than half a lease duration could potentially cause an address
change.

To investigate this, we analyzed the conditional probability of an
address change given the occurrence of network or power outages of different durations
for probes from LGI (AS 6830) and
Orange (AS 3215) in Figure~\ref{fig:outagedurs}. For network outages,
we considered outages from all versions of probes while for power
outages, we only considered outages from probes running v3.
We chose these ISPs
due to their difference in address change behavior upon the occurrence
of outages as seen in
Figure~\ref{fig:top_asns_frac_norenums_over_totalnos} and
Figure~\ref{fig:top_asns_frac_porenums_over_totalpos}.

The behavior upon outages for the two ISPs is strikingly
different. LGI's behavior appears consistent with what we would 
expect for dynamic addresses assigned using DHCP: fewer than
3\% of outages of up to an hour resulted in an address
change.  More than 25\% of outage
durations that lasted at least twelve hours resulted in an address
change. This behavior is consistent with a DHCP lease duration on the
order of a few hours.  Not every outage longer than twelve
hours resulted in an address change, consistent with DHCP 
behavior when a client returns after an expired lease and the
previously assigned address
is still available.

For Orange, we found that even very short outages resulted in
address changes. 91\% of outages that lasted less than five
minutes resulted in an address change, and for every outage duration
longer than five minutes and shorter than three hours, more than 75\% occurred with an
address change. For outages between three hours to three days
long, the percentage of address changes was closer to 50\%, suggesting
the presence of some CPE devices that do not renumber upon every outage. However, as the
outage duration increases beyond 3 days, almost every outage results
in an address change.

Private communication with a large European ISP confirmed that this
behavior is expected for PPPoE based DSL lines in that ISP: any
reboot/reconnect event will result in the assignment of a new address
from the ISP's dynamic address pool. Since outages of such short
durations can result in an address change, a simple reboot of the CPE
(resulting in a power outage), or unplugging and replugging the
network cable (resulting in a network outage), can change the dynamic
address assigned to the end-user. That end-users can change their
dynamically assigned address has implications for researchers and
operators who use IP addresses to identify end-hosts, particularly
when IP addresses are being used to blacklist malicious actors.

