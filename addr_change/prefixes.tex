
\section{Does a user's dynamic address prefix change?}

\label{sec:prefixes}

% \begin{table*}[th]
%   \small
%   \centering
%   \begin{tabular}{ccc|rr<{\%}|rr<{\%}}
%     \ehdr{AS} & \ehdr{ASN} & \hdr{Country}  & \multicolumn{2}{c|}{\textbf{Diff /16}} & \multicolumn{2}{c}{\textbf{Diff BGP}} \\
%     \hline
%     % 17262 & 2014-12-30 17:38:37 & 2015-01-01 22:30:44 &
%     % 83.115.35.156 \\
% \multicolumn{3}{c|}{All}&79,430&      47.7&     81,571&      48.9\\
% Orange         &3215&France    &      6,961&        67&      7,016&        68\\
% LGI            &6830&many      &       168&        55&       171&        56\\
% BT             &2856&U.K.      &      2,685&        68&      1,736&        44\\
% DTAG           &3320&Germany   &      5,391&        28&      4,706&        24\\
% Verizon        &701&U.S.      &       241&        23&       241&        23\\
% Proximus       &5432&Belgium   &      2,331&        53&      2,152&        49\\
% Telecom Italia &3269&Italy     &      4,412&        88&      4,281&        85\\
% Ziggo          &9143&Netherlands&        22&        43&        18&        35\\
% Virgin Media   &5089&U.K.      &        49&        89&        46&        84\\
% Kabel DE       &31334&Germany   &       108&        56&       115&        60\\
% \multicolumn{3}{c|}{Other} &     24,240&        37&     32,925&        50\\
%   \end{tabular}
%     \caption{Number of address changes across prefixes. Diff
%     /16 shows the number of address changes where the previous address
%     and the next address belonged to different /16 prefixes. Diff BGP show the number of address changes where the previous address
%     and the next address belonged to different BGP prefixes. The \%
%     column shows the percentage of total address changes for that
%     autonomous system.}
%     \label{tbl:prefs}
% \end{table*}

\begin{table*}[th]
  \small
  \centering
  \begin{tabular}{ccc|rr<{\%}|rr<{\%}|rr<{\%}}
    \ehdr{AS} & \ehdr{ASN} & \hdr{Country}  &
    \multicolumn{2}{c|}{\textbf{Diff BGP}} &
    \multicolumn{2}{c|}{\textbf{Diff /16}} &
    \multicolumn{2}{c}{\textbf{Diff /8}} \\
    \hline
\multicolumn{3}{c|}{All}&81,571&      48.9&     79,430&      47.7&     55,835&      33.5\\
Orange         &3215&France    &      7,016&        68&      6,961&        67&      5,513&        53\\
LGI            &6830&many      &       171&        56&       168&        55&       136&        45\\
BT             &2856&U.K.      &      1,736&        44&      2,685&        68&      1,735&        44\\
DTAG           &3320&Germany   &      4,706&        24&      5,391&        28&      4,610&        24\\
Verizon        &701&U.S.      &       241&        23&       241&        23&       209&        20\\
Comcast        &7922&U.S.      &        76&        37&        74&        36&        63&        31\\
Proximus       &5432&Belgium   &      2,152&        49&      2,331&        53&      1,983&        45\\
Telecom Italia &3269&Italy     &      4,281&        85&      4,412&        88&      2,374&        47\\
Ziggo          &9143&Netherlands&        18&        35&        22&        43&        16&        31\\
Virgin Media   &5089&U.K.      &        46&        84&        49&        89&        39&        71\\
  \end{tabular}
    \caption{Number of address changes across prefixes.  Diff BGP shows the number of address changes where the previous address
    and the next address belonged to different BGP prefixes. Diff
    /16 shows the number of address changes where the previous address
    and the next address belonged to different /16 prefixes and Diff
    /8 shows the number of address changes where the previous address
    and the next address belonged to different /8 prefixes. The \%
    column shows the percentage of total address changes for that
    autonomous system.}
    \label{tbl:prefs}
\end{table*}

It is tempting to expect that a new address, when
reassigned, will typically be drawn from nearby addresses,
say, from the same enclosing /24 prefix.  If such an
assumption were true, it would allow blacklisting of the
enclosing prefix of a malicious host, if it were thought
that the malicious host could cause its address to change
via reboot or by waiting a day.  However, we find that such
locality of addresses is rare and address changes typically
span prefixes.

We examined whether the dynamic address assigment also varies the
enclosing prefix, defined three ways. 
For each instance of address change that
we observed, we found the BGP prefix of the previous address and the
new address using CAIDA's IP-to-AS
dataset~\cite{caidapfx2as}, as described in
Section~\ref{sec:dataset}. We also extracted the /16 and /8 prefixes from the
previous and new addresses.  We then compared how often the prefix of
the previous address differed from the prefix of the new address.  
Table~\ref{tbl:prefs} presents the results for the overall AS-level dataset with 2,272
probes and for the ten ASes with the most probes that
had at least one address change.

ISPs varied prefixes even for consecutive addresses
assigned to the same customer; nearly half of the 166,644 total address
changes we observed also changed BGP prefixes. Unlike periodicity and renumbering upon
outages, assigning addresses out of different prefixes appears to be a
common behavior for ISPs. For the ten ASes in
Table~\ref{tbl:prefs}, Verizon and DTAG had the lowest percentage of
address changes across prefixes, but even for these ASes, almost a
quarter of all address changes were across /16s and a fifth of all
address changes were across /8s. Thus, it is not just
the dynamic addresses that change; their prefixes change too. When a
malicious actor receives a new address, even blacklisting the entire
enclosing /8 prefix of the old address would fail to prevent access
for a third of
the address changes we observed.