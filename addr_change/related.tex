\section{Related Work}

Previous work
studied the performance of DHCP in small campus
networks~\cite{dhcpwatch, dhcp-gatech} and
settings where smartphone usage is
widespread~\cite{dhcp-smartphones} and developed techniques to
reduce network address utilization and DHCP broadcast traffic. The goal of those studies was
to improve the performance of DHCP by tuning configuration.

Conceptually, so long as there is some uniquely identifying feature
that remains constant across a host's address change,
it is possible to track IP address changes over time for that host.
Several studies have used this broad
method~\cite{udmap,census-survey, zmap-dhcp,
maier2009dominant, dhcp-gatech, peering-shroud, dhcp-dimes, uav6}. 
UDmap~\cite{udmap} studied
dynamic address properties using Hotmail user login
traces where the user's login serves as the identifying
feature. Casado et al.~\cite{peering-shroud} tracked clients using HTTP cookies when
clients access a CDN. 
Other studies~\cite{maier2009dominant,census-survey} used continuous
responsiveness of an address itself as the identifying feature, assuming
that an address that responds continuously belongs to the same user and that
when an address stops responding to pings, it has been reassigned.

While we share the same goal as these studies, our approach diverges
in that we are interested in the events associated with an address
change.
Previous studies lacked
access to end-host information that could reveal the cause of an
address change. One exception, Maier et al.~\cite{maier2009dominant},
used access to the Radius server of a European DSL provider from one
urban area to
identify why DSL sessions terminated, and noted that the DSL
provider often limited Radius session length to 24 hours in that area.
We extend this result to several ISPs in countries from Europe, Asia,
and South America, and identify other typical session length limits.
%
Argon et al.~\cite{dhcp-dimes} used periodic measurements
from end-hosts in the DIMES infrastructure~\cite{netdimes}.
DIMES software installed on an end-user computer is
different from RIPE Atlas hardware probes primarily in that it reports back
only every 30-60 minutes (as opposed to RIPE Atlas's 3 minutes),
the agent can be installed on
laptops that move (as opposed to RIPE Atlas probes that could
move, but do not), the hosts running DIMES are often
powered down (resulting in limited uptime), and DIMES hosts
appear to have static IP addresses more often (they reported 60\% had
only one address).  Nevertheless,
Argon et al. observed that some small ISPs exhibited 
address alternation with a 24 hour periodicity. In IPv6, the RFC for
privacy extensions for stateless address autoconfiguration recommends
that IPv6 addresses be changed every 24 hours~\cite{rfc4941} and
empirical results by Plonka and Berger found that more than 90\% of
client IPv6 addresses were ephemeral~\cite{akamai-v6addr-usage}. We
showed that 24 hour defaults are not uncommon in IPv4 as well.

These studies relied on relatively uncontrolled observations of
the address assigned to a device or user, both in terms of whether
the devices are active, whether the users connect using multiple devices,
and how frequently samples are provided.  As a consequence, 
the dynamic IP address churn rates reported by these studies
vary.  While UDmap reported that over 30\% of IP addresses have
inter-user durations of 1--3 days~\cite{udmap},  Heidemann et
al. reported that 90\% of IP addresses were occupied for less than a
day~\cite{census-survey}.  Maier et al.~\cite{maier2009dominant} reported that a
major European ISP had per-user median durations of just 20 minutes during
their study in 2009 (we did not observe this duration in 2015).
We believe that the perspective of a device using the dynamically assigned
network is necessary for understanding the reasons behind the address change
and for getting precise information about the duration that any
address is held. Further, since RIPE Atlas probes provide continuous,
longitudinal measurements enabling the inference of successive
addresses assigned to a CPE device, we perform the first
analysis of dynamic prefixes from which devices are assigned
successive addresses.

