
\section{Periodic address changes}
\label{sec:periodic}

ISPs can assign dynamic addresses for as long as they wish.
In DHCP, long leases simplify administration, while short
leases can be more efficient in reclaiming unused addresses.
DHCP leases, however, are meant to be renewable by devices
that are still active.  In this section, we look at periodic
address reassignment: instances where a device changes
address periodically, despite actively using the address. Periodic
reassignment is atypical for devices using DHCP since a
device that is continuously renewing its lease should continue to
keep its current address~\cite{rfc2131}.

\subsection{Metric to detect periodic address durations}

If ISPs intentionally renumber after specific durations, we would
expect those address durations to be prominent in a distribution
of all address durations belonging to that ISP. We initially
considered studying distributions
of raw address durations, similar to the analyses by
Maier et al.~\cite{maier2009dominant} and
Moura et al.~\cite{zmap-dhcp}, but found that short address-durations
were overrepresented. For example, in Table~\ref{tbl:sample},
inspecting the cumulative distribution of address
durations
would suggest that only half the addresses (3 of 6) were assigned for 24
hours. However, 
when trying to reason about the expected duration that an address will
continue to be assigned to the CPE, we would like to know the fraction
of total time that each duration accounted for. For example, in
Table~\ref{tbl:sample}, the CPE was assigned
24 hour long addresses for roughly three-quarters of the total
measured time. This latter notion is more useful to find whether an
ISP is using periodic durations consistently, since the modes at
intervals on the scale of days will be more visible. 

To capture this notion we
define a metric, the \emph{total time fraction}. 
For a given probe and an address duration $d$,
we define the total time fraction for $d$ as the fraction of time
spent by the probe in durations of length $d$. We compute the total time fraction for a given probe and a duration
$d$ by obtaining the total address
time for the probe, and computing the fraction of the total
address time that was accounted for by address durations of 
length $d$. For a probe $p$, if $n(d)$ is the number of times the probe had an address duration
$d$ and $D$ is an array containing all address durations that were assigned to
the probe, the total time fraction for the address duration $d$ is
given by:

$f^p_d =  d \times n(d) / \Sigma(D)$

We use a similar procedure for computing the total time fraction
considering all probes in an ISP, country, or continent. We believe that  
the total time fraction offers a better representation of 
the probability that an address was assigned for a certain
duration than a simple inspection of the address durations. 

\subsection{Periodic address changes by geography}
\label{sec:periodic-geography}

\begin{figure}[tb]
  \begin{center}
    \includegraphics[width=3in]{addr_change/figs/conts_a_all_ip_durs_connlogs_wtd_cdf}
  \end{center}
  \caption[Cumulative distribution of total time fraction by continent]{\label{fig:conts_all_durs} 
    Cumulative distribution of total time fraction by continent. 
    Modes (vertical segments
    in the CDF) indicate periodic renumbering.  Addresses in North America
    are relatively long lived and free of periodic renumbering.}
\end{figure}

We begin by inspecting how address durations vary across continents.
We expected that address scarcity might affect address durations,
leading to longer durations in North America and shorter durations in
Asia.  We use RIPE Atlas's probe database to find the country to which
each probe belongs. Next, we aggregate the address durations of probes
by their respective countries and subsequently, to their continents.
Figure~\ref{fig:conts_all_durs} shows the cumulative distribution of
the total time fraction for each continent, i.e., the y-axis shows the
fraction of total address duration accounted for by durations less
than the x-axis value.
The number in
parentheses in the legend for each continent shows the total
 address duration for that continent in years ($\Sigma(D)$).

In Europe, Asia, Africa, and South America, address durations exhibit
well-defined modes, mostly at intervals that are multiples of 24
hours. The most common mode is exactly at 24 hours: the total time
fraction for European addresses at 24 hours is 0.16, African addresses
is also 0.16, and Asian addresses is 0.07.  One week address durations
are also common in Europe, with the total time fraction at 1 week
equaling 0.08.  South American addresses exhibit multiple modes: their
total time fraction is 0.11 at 12 hours, 0.07 at 28 hours, 0.09 at 48
hours, and 0.03 at 192 hours (8 days).

The curves for North America and Oceania do not have well-defined
modes, suggesting that ISPs in these continents do not periodically
change addresses. Further, North American probes typically retain
their dynamic addresses for much longer durations than other
continents; North American addresses spent more than half of the total
time in address durations longer than 50 days. This suggests that IP
addresses can be used as end-host identifiers in North America for
several weeks.

\begin{comment}

\subsubsection{Which countries are most likely to have periodic address changes?}

\begin{figure}[tb]
  % \centering
  \begin{center}
    \includegraphics[width=3in]{addr_change/figs/top_periodic_ctrys_a_all_ip_durs_connlogs_wtd_cdf}
  \end{center}
  \caption{\label{fig:top_ctrys_periodic_all_durs}
    Cumulative distribution of total time fraction per country for the
    countries that had the most periodic addresses. The use of 24 hour
    durations could be geographically correlated, with German,
    Austrian and Polish addresses spending large proportions of their
    total time in this 24 hour durations.
  }
\end{figure}

Next, we examine which countries were most likely to have dynamic
addresses with periodic durations. Among all countries for which we
had at least 10 years of cumulative address durations, we found the
countries which had the highest proportion of addresses that appeared
periodic. To determine if addresses appeared periodic, we found
address durations that were within an hour of each other, summed
them, and found what proportion of total duration this sum
constituted. For these countries, we show the 
cumulative distribution of total time fractions per duration
aggregated per country in Figure~\ref{fig:top_ctrys_periodic_all_durs}.

German and Austrian address durations have total time fractions
exceeding 0.5 at 24 hours, and Polish address durations observe a
smaller mode of 0.17 at 24 hours. 24 hour durations are not limited to
Central Europe, however; Kazakhstan also observes a total time
fraction of 0.42 at 24 hours. French addresses have a total time
fraction of 0.03 at 24 hours but observe a much larger mode of 0.47 at
168 hours (1 week).

\end{comment}

\subsection{Periodic address changes by AS}
\label{sec:periodic-as}
\begin{figure}[tb]
  % \centering
  \begin{center}
    \includegraphics[width=3in]{addr_change/figs/top_asns_a_all_ip_durs_connlogs_wtd_cdf}
  \end{center}
  \caption[Cumulative distribution of total time fraction by AS]{\label{fig:top_asns_all_durs}    
    Cumulative distribution of total time fractions for 
    ASes with most RIPE Atlas probes
    that yielded at least one address duration. Probes from Orange and
    DTAG spent more than half of their total duration in periodic
    durations of 1 week and 1 day respectively. BT also showed evidence
    of periodic renumbering with a mode at two weeks. On the other
    hand, LGI and Verizon have no modes at any durations, and spent
    most of their total time in durations that were weeks long.
  }
\end{figure}

 We next considered whether the configuration decision to
renumber periodically was uniform across an AS, or could
reflect some other feature. For example, periodic
renumbering could be a result of an unexpected
cron job on the RIPE Atlas probe or a faulty DHCP client that
could not renew. Periodic renumbering could be due to government
regulations in countries, perhaps as a privacy measure. It could also
simply reflect ISP policy, perhaps to hinder users from running web servers as anecdotal
evidence suggests~\cite{forcedseparation}.  % If it were concentrated in some ASes and
% absent in others, that would convince us that the behavior is
% a feature, not a bug.
Investigating AS-level behavior can inform whether the periodic renumbering
behavior is concentrated in some ASes and absent in others, shedding light on its potential cause.


\subsubsection{Is periodic renumbering prevalent across all ISPs?}

We first investigate the ASes with the largest deployment of RIPE
Atlas probes where we detected at least two instances of address
changes. Recall that we only obtain an address duration when the
address began and ended during the interval we studied, so that a
minimum of two address changes are necessary for a probe to yield an
address duration.  Figure~\ref{fig:top_asns_all_durs} shows the
cumulative distribution of total time fractions
for the five autonomous systems
with the most probes that yielded address durations. In this figure, Orange, an ISP from France, appears to change addresses
after a duration of 168 hours (1 week): 55\% of its total
address duration was a week long. The German ISP, Deutsche Telekom AG (DTAG)
reassigns addresses after 24 hours: 76\% of the total address duration
lies in that mode. British
Telecom (BT) has a mode at 336 hours (2 weeks) with 13\% of its
total duration being in 2 week intervals. We study these ASes further
in Section~\ref{sec:periodic_asns}.

The other two ISPs do not exhibit any evidence of periodic
renumbering. Liberty Global, an ISP to which probes spread
across Europe belong, does not appear to change addresses periodically
and neither does Verizon (US). Among these ASes, Verizon
has the longest address durations.

Since periodic renumbering behavior is widespread in some ISPs and
non-existent in others, we conclude that the cause of periodic renumbering is likely ISP
policy. 

\subsubsection{Is periodic renumbering geographically correlated?}
\label{sec:germany}

\begin{figure}[tb]
  % \centering
  \begin{center}
    \includegraphics[width=3in]{addr_change/figs/DE_asns_a_all_ip_durs_connlogs_wtd_cdf}
  \end{center}
  \caption[Cumulative distribution of total time fraction for German ASes]{\label{fig:DE_asns_all_durs}
    Cumulative distribution of total time fractions for ASes in Germany.
        Many
      German ISPs appear to change addresses every 24 hours. However,
      some ISPs have more stable addresses.
  }
\end{figure}


Next, we investigate how the periodic renumbering behavior of ISPs
correlates with the country in which they operate. 
Germany has more than a hundred RIPE Atlas probes deployed across
several ISPs, thus we study their
address durations in Figure~\ref{fig:DE_asns_all_durs} for ISPs
with probes that contributed at least 3 years of total time. 
Many ISPs in Germany change addresses every 24 hours: 77\% of the duration in DTAG
(AS 3320), 76\% in Telefonica1 (AS 6805), 74\% in Telefonica2 (AS
13184), and 29\% in Vodafone (AS 3209), is 24 hours. We observe
that the 'other' ISPs also have a mode at 24 hours, suggesting that
German ISPs are particularly likely to renumber every 24
hours. However, this behavior is not universal: Kabel
Deutschland (AS 31334) and
Kabel BW (AS29562) do not exhibit a mode at 24 hours; instead, more than 90\% of their
total address duration was spent in durations longer than two weeks. 

These results suggest that periodic renumbering behavior can exhibit
some geographic correlation, but is likely
largely caused by ISP policy. 

Private communication with a large European ISP confirmed that the ISP renumbers every 24
hours, since the ISP considers this scheme to be more 'privacy secure' although
there is no government regulation that forces this feature. The ISP
also reported that it uses PPPoE instead of DHCP for its DSL
lines (which accounted for the vast majority of its customers). Since
periodic behavior would be atypical of DHCP but consistent with PPP
techniques for address assignment, we speculate that periodic
renumbering is a property of ISPs that use PPP.

