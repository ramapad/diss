
% The main position I will take is that it's possible to confirm outages. However, recovery analysis is also enabled.

\section{Using complementary datasets that provide IDs to confirm outages}

\label{sec:complementary-ns}

The results from the RIPE Atlas measurement study allow the
identification of networks with stable addresses. However, they also
show the existence of networks where dynamic addressing is common. In
this section, I show how to use a complementary dataset to confirm
outages detected in networks where dynamic addressing can occur.


Recall that Section~\ref{sec:addrchange-false-inferences} had described the different ways probing-based techniques can make false inferences about outages when dynamic addressing occurs. If the address being probed is withdrawn from its home router and not reassigned to any other device, probe responses will cease to arrive and a false outage will be inferred. If the home router experiences an outage causing its address to cease to respond to probes, and before the outage ends, the address is reassigned to some other device which responds to probes, probing-based techniques will infer the occurrence of the outage correctly but will falsely conclude that the outage ended before it did.

When a responsive address being probed is withdrawn from its host or
when the host it belongs to experiences an Internet outage, a
probing-based technique will observe that responses cease to arrive. I
define this event to be a dropout. Formally: \emph{A dropout happens
when the address attached to a residential link transitions from being
responsive to pings from multiple vantage points, to being
unresponsive from all of the vantage points.} An observed dropout can
either be due to an outage or dynamic reassignment.

 % Since dynamic reassignment and Internet outages will manifest similarly to probing-based techniques, I distinguish between Internet outages and \emph{dropouts}. 

My key insight is that a complementary dataset which can yield some
sort of an unchanging identifier (an ID) uniquely associated with the device
can provide information about whether the device's address
changed. For instance, consider the probe-ID field provided by RIPE
Atlas, which uniquely identifies a device. If the address associated
with the device before and after the dropout is the same, it is proof
that dynamic address reassignment did not occur. The only way that the
address before and after the dropout can be identical and yet for
dynamic reassignment to have occurred, is if the device's address
changed to a new one and then changed back to the original
address. However, Section~\ref{sec:prefixes} showed that subsequent
addresses are often assigned from entirely different prefixes; thus, the
probability that a subsequent address is exactly the address that was
assigned before is small. Since dynamic reassignment is highly
unlikely to have occurred, we can infer that an outage occurred. 

The ID-based approach provides two benefits:

\begin{itemize}

\item{It can offer confirmation of the occurrence of an outage.}

\item{It allows the estimation of outage recovery durations for the instances where an outage is confirmed.}

\end{itemize}


\subsection{CDN dataset provides IDs}

I measure how often addresses remained the same before and after a dropout using a dataset
of CDN software logs that contain a timestamp, unique identifier of
the software installation on the client machine and the public source
IP address visible to the CDN.  The CDN offers a service to content owners
whereby end users can elect to install software that will improve the
performance the client experiences when accessing the content through
the CDN. The CDN records logs collected from its software
installations on users' desktops and laptop machines. Each logline
contains (among other fields) the timestamp at whch the logline was
created, the unique identifier of the software installation on the
machine (the ID), and the public IP address seen by the CDN's
infrastructure at this time. Loglines in the CDN software dataset are
dependent on user activity, and therefore, their frequency varies. % In March 2017, there
% were over a billion loglines from close to 20M unique IDs and 66M
% unique IP addresses. 

 % The dataset comprises $\approx$ 70 million IP addresses in March of 2017.

\subsection{Confirming outages detected by Thunderping}

For dropouts detected by the Thunderping system~\cite{pingin}, I
measure how frequently the complementary dataset confirmed that the
dropout was an outage. I used all dropout events detected in three
years (2015, 2016, 2017) in this analysis and compare against the CDN
dataset during the same period.

To determine whether the address associated with a home router remained the same before and after a detected dropout, I
first collect all entries where the address that experienced
the dropout is present in the log up to one week before the
start time; this applies to only about one percent of the
dropouts.  The matched address is $ip_p$ (for previous), and
I refer to the next address after the dropout $ip_n$ (for
next).  There are three categories of comparison that I
show in Table~\ref{tbl:lt_vs_outages_fp}:
% TODO: Fix the substantial outage part
\begin{enumerate}
\item When $ip_p = ip_n$, there was no apparent
  reassignment, which suggests that an outage occurred and that an inferred outage duration
  is correct.
% \item When $ip_p \not= ip_n$, and the observation of $ip_n$
%   was before 11 minutes elapsed, a \emph{substantial outage} as defined earlier couldn't have occurred. Thus, the inferred outage duration is incorrect.
\item When $ip_p \not= ip_n$, and the observation of $ip_n$
  was before $ip_p$ became responsive again, the address was
  reassigned and the inferred outage duration is incorrect. An outage may or may not have occurred.
\item When $ip_p \not= ip_n$, and the observation of $ip_n$
  was after $ip_p$ became responsive again, the address was
  reassigned but the address change may be independent. Again, an outage may or may not have occurred.
\end{enumerate}

\begin{table*}[th]
  \centering
  \hspace{-0.04in}\begin{tabular}{@{}c|rrrrr@{}}
    % & & \textbf{Correct outage} & \textbf{Outage} & \textbf{False outage} & \\
    % \textbf{Link type} & \textbf{ID available} & \textbf{duration} &
    % \textbf{false positive} & \textbf{duration} & \textbf{Unknown}\\
    % & & \textbf{Correct outage} & \textbf{Outage} & \textbf{False outage} & \\
    \multicolumn{1}{c}{} & \multicolumn{1}{c}{} & & \multicolumn{2}{c}{$ip_p \not= ip_n$} \\
    \centering{\textbf{Link Type}} &
    \centering{$ip_p$ \textbf{present}} &
    \multicolumn{1}{c}{$ip_p = ip_n$} & 
    \multicolumn{1}{c}{\textbf{during}} & 
    \multicolumn{1}{c}{\textbf{after}}  \\
    \hline
ALL       &      84837 (0.7\%) &     50973 (60.1\%) &       4765 (5.6\%) &      29047 (34.3\%) & \\
CABLE      &      21455 (1.1\%) &      18860 (88.0\%) &        354 (1.7\%) &       2221 (10.4\%)\\
DSL        &      25061 (0.9\%) &       7761 (31.0\%) &       2857 (11.4\%) &      14422 (57.6\%)\\
FIBER      &       1516 (1.0\%) &        853 (56.3\%) &         60 (4.0\%) &        603 (39.8\%)\\
WISP       &       7381 (1.1\%) &       6013 (81.5\%) &        177 (2.4\%) &       1191 (16.1\%)\\
SAT        &       9600 (0.4\%) &       6939 (72.3\%) &        241 (2.5\%) &       2412 (25.1\%)\\

% All       &      84,837 (0.7\%) &     50,973 (60.1\%) &       1,181 (1.4\%) &       3,584 (4.2\%) &      29,047 (34.3\%) \\
% Cable      &      21,455 (1.1\%) &      18,860 (88.0\%) &         35 (0.2\%) &        319 (1.5\%) &       2221 (10.4\%)\\
% DSL        &      25,061 (0.9\%) &       7,761 (31.0\%) &        869 (3.5\%) &       1,988 (7.9\%) &      14,422 (57.6\%)\\
% Fiber      &       1,516 (1.0\%) &        853 (56.3\%) &          3 (0.2\%) &         57 (3.8\%) &        603 (39.8\%)\\
% WISP       &       7,381 (1.1\%) &       6,013 (81.5\%) &         61 (0.8\%) &        116 (1.6\%) &       1,191 (16.1\%)\\
% Sat        &       9,600 (0.4\%) &       6,939 (72.3\%) &          9 (0.1\%) &        232 (2.4\%) &       2,412 (25.1\%)\\

    \end{tabular}
  \caption{\label{tbl:lt_vs_outages_fp} Recovery analysis by link type: outage durations are safe when $ip_p=ip_n$.}
\end{table*}

 Table~\ref{tbl:lt_vs_outages_fp} shows that 60\% of Thunderping's detected dropouts when considering all linktypes are \emph{not} accompanied by address changes; thus the majority of dropouts are outages. Additionally, the table shows that nearly all dropouts for addresses with cable connections are outages, corroborating the results from RIPE Atlas which suggested that cable addresses tend to be stable. 

For DSL addresses, 31\% of dropouts were confirmed. Without the complementary dataset, all of these dropouts were suspect, since prior results showed that DSL addresses tend to be renumbered frequently. However, through the use of a dataset that provides IDs, I am able to confirm outages even in networks where addresses are not stable.

We next try to determine whether longer apparent outages
correlate with address changes.  If short outages typically
have no address change, we can at least characterize short
outage durations.  However, if all dropouts lead to address
changes on recovery, the time until an address starts
responding again is more a function of address reuse than of
recovery.  Figure~\ref{fig:outdur_vs_addrchange} shows the
results for each of the media types in our study.  This uses
the same data as in Table~\ref{tbl:lt_vs_outages_fp}.  In
Figure~\ref{fig:outdur_vs_addrchange}, the top graphs
represent the raw histograms of apparent outage duration,
though only the distribution of dark bars (where the address
is unchanged) should be taken as a distribution of true
outage duration.  The bottom graph represents the fraction
of outages having an address change or no address change.
At a high level, graphs with more dark are media types or
durations that are more likely to be true outages rather
than address renumbering.  For WISP and Cable, the bulk of
the outages at most 3 hours long have very little
renumbering and outage durations can be estimated well.  For
DSL, even short apparent outages are often accompanied by
address changes, meaning that outage duration should not be
estimated based on responsiveness alone (to do so would
require additional data from clients).  We observe few Fiber
outages, but the time-dependence is more pronounced.


% \newlength{\fairsubfigwidth}
% \setlength{\fairsubfigwidth}{0.19\textwidth}
% \newlength{\commonfigheight}
% \setlength{\commonfigheight}{2.4in}

% \begin{figure*}[t]

% \begin{subfigure}[t]{0.32\linewidth}
% \centering
% \includegraphics[width=\linewidth]{addr_change/figs/outdurvsaddrchange_hist_CABLE}
% \caption{
% \label{fig:outdur_vs_addrchange_cable}
% Cable}
% \end{subfigure}
% %
% \hfill
% %
% \begin{subfigure}[t]{0.32\linewidth}
% \centering
% \includegraphics[width=\linewidth]{addr_change/figs/outdurvsaddrchange_hist_DSL}
% \caption{
% \label{fig:outdur_vs_addrchange_dsl}
% DSL}
% \end{subfigure}
% %
% \hfill
% %
% \begin{subfigure}[t]{0.32\linewidth}
% \centering
% \includegraphics[width=\linewidth]{addr_change/figs/outdurvsaddrchange_hist_FIBER}
% \caption{
% \label{fig:outdur_vs_addrchange_fiber}
% Fiber}
% \end{subfigure}
% %
% \hfill
% %
% \end{figure*}

\begin{figure*}[t]

\begin{subfigure}[t]{0.19\linewidth}
\centering
\includegraphics[width=\linewidth]{addr_change/figs/outdurvsaddrchange_hist_CABLE}
\caption{
\label{fig:outdur_vs_addrchange_cable}
Cable}
\end{subfigure}
%
%\hfill
%
\begin{subfigure}[t]{0.19\linewidth}
\centering
\includegraphics[width=\linewidth]{addr_change/figs/outdurvsaddrchange_hist_DSL}
\caption{
\label{fig:outdur_vs_addrchange_dsl}
DSL}
\end{subfigure}
%
%\hfill
%
\begin{subfigure}[t]{0.19\linewidth}
\centering
\includegraphics[width=\linewidth]{addr_change/figs/outdurvsaddrchange_hist_FIBER}
\caption{
\label{fig:outdur_vs_addrchange_fiber}
Fiber}
\end{subfigure}
%
%\hfill
%
\begin{subfigure}[t]{0.19\linewidth}
\centering
\includegraphics[width=\linewidth]{addr_change/figs/outdurvsaddrchange_hist_WISP}
\caption{
\label{fig:outdur_vs_addrchange_wisp}
WISP}
\end{subfigure}
%
%\hfill
%
\begin{subfigure}[t]{0.19\linewidth}
\centering
\includegraphics[width=\linewidth]{addr_change/figs/outdurvsaddrchange_hist_SAT}
\caption{
\label{fig:outdur_vs_addrchange_sat}
Satellite}
\end{subfigure}
%
\caption{
\label{fig:outdur_vs_addrchange}	
\figdone
Outage duration vs.~probability of address change for addresses from
various link types.}
%
\end{figure*}

