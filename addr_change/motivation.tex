% \section{Background and Motivation}

\section{Dynamic addressing background}

%@@all: I have fleshed out the Background and Motivation section, with
%a brief writeup about DHCP and PPP

% An IP address can be used to uniquely identify the end-host it is assigned to
% until the end-host's address changes for some reason. The duration of
% time that a dynamic IP address continues to be assigned to the same
% CPE device depends upon various causes that can induce the assigned IP
% address to change. In this section, we present techniques used for
% assigning dynamic addresses and the events and
% agents involved in dynamic address changes.

An IP address can be used to uniquely identify the end-host it is assigned to
until the end-host's address changes for some reason. The duration of
time that a dynamic IP address continues to be assigned to the same
CPE (Customer Premises Equipment) device depends upon various causes that can induce the assigned IP
address to change. Here, I present techniques used for
assigning dynamic addresses and the events and
agents involved in dynamic address changes.


\subsection{Dynamic Host Configuration Protocol}

ISPs often use the Dynamic Host Configuration Protocol
(DHCP)~\cite{rfc2131} for IP address assignment. DHCP issues an IP address to a host for a lease
duration configured by the ISP. The host will try to renew the lease
before it expires, typically half-way into the lease. However,
whether the same IP address is renewed, or a different one is
assigned, depends upon ISP policy.  We speculate that the
typical behavior of ISPs using DHCP is to renew the lease of the
currently assigned IP address, since one of the stated design goals
in the DHCP specification is that a DHCP client should be assigned the same address
in response to each request, whenever possible. Thus, we typically
only expect an ISP using DHCP, to change the address of a CPE, if
something happens to prevent the CPE from renewing its lease (like an outage).

% Further, on
% reboot, the previously assigned address may be reassigned, or
% alternately a new address may be issued, again, as dicated by ISP
% policy.

\subsection{Point-to-Point Protocol}

In some networks, end-hosts connect to an ISP using
point-to-point links. For these networks, the Point-to-Point Protocol
(PPP) first configures and establishes the point-to-point link~\cite{rfc1661}. Next,
a Network Control Protocol (NCP) like the Internet Protocol Control
Protocol (IPCP) configures IP addresses~\cite{rfc1332}. The PPP specification
notes that the link will remain configured for communication until the
link is actively closed down through network administrator
intervention or when an inactivity timer expires.

\subsection{Potential dynamic address change causes} 

Next, we identify the reasons dynamic addresses assigned using
the above techniques could change. We classify the following categories of address change:

\begin{itemize}
\item{\textbf{Changes after outages}} If the client is
  disconnected or loses power long enough to fail to renew a DHCP lease,
  its address may be assigned to another; when it returns,
  it may then get a new address. We call such changes
  \emph{outage-caused address changes}.

\item{\textbf{Changes after reboot/reconnect}} While we
  expect addresses assigned through traditional DHCP to change only when the
  outage duration is long enough to prevent lease renewal, addresses
  assigned through PPP can change upon outages of any duration. Any reboot or
  network reconnect event could cause the client
  to forget its prior address and request a new one, or the
  state associated with a connection may be lost.
  We call such
  address changes \emph{reboot-caused address changes}. 
  
  % We make no
  % distinction between very short power outages and reboots.

\item{\textbf{Administrative address changes}} A purpose of
  dynamic address assignment is to allow reconfiguration of
  the network; it is possible that a reconfiguration of the
  DHCP server will force a change to the subnet on which the
  client lies.  We expect such reassignment to be rare.
  
\item{\textbf{Periodic address changes}} We observe that
  some ISPs limit the session length associated with an
  address, causing a reassignment after a fixed duration,
  typically one day to one week depending on the ISP.
\end{itemize}

% We believe these four categories should cover all IP address
% changes we observed.  
Intuitively, the address change is
either caused by the ISP (administrative or periodic), or
caused by the client (or an interruption in network service
to the client) in a reboot or outage.

