\begin{figure*}
\begin{center}
\includegraphics[width=.4\linewidth]{timeouts/figs/20150206_high_1pctile_gt_300ms_one_vs_99_scatter-nosat-decimated}
\hspace{0.3in}\includegraphics[width=.4\linewidth]{timeouts/figs/20150206_high_1pctile_gt_300ms_one_vs_99_scatter-sat}
\end{center}
\caption{\label{fig:20150206_high_1pctile_gt_300ms_one_vs_99_scatter}Scatterplot of 1st and 99th percentile latencies for addresses with high values of both in survey IT63c; Left omits satellite-only ISPs; Right includes only satellite-only ISPs.}
\end{figure*}


\subsection{Are satellites involved?}

A reasonable hypothesis is that satellite links, widely
known for their necessarily high minimum latency, would also
be responsible for very high maximum latencies.
Transmissions via
geosynchronous satellite must transit 35,786km to a
satellite and back, leading to about 125 ms of one way
delay~\cite{minei-jsac99,barakat-geostationary}.  Another 125 ms for
the return trip yields a theoretical minimum
of 250ms. 
%  208 baycity
%    27 horizon
%  2696 hughes
%   365 iinet
%   116 ON_LINE
%   233 skylogic
%    93 skymesh
%    34 telesat
%   755 viasat

We expect satellite ISPs to have
high 1st percentile latencies, but we consider whether they
have high 99th percentile latencies as well. We use data from ISI survey IT63c
(20150206) for this analysis because it provides hundreds of ping samples per IP address, and we wish to study relatively few addresses in some detail. Figure~\ref{fig:20150206_high_1pctile_gt_300ms_one_vs_99_scatter}
shows the plot of 1st percentile latencies vs. the 99th percentile
latencies for addresses in this survey. We
separate addresses that the Maxmind database maps to
known satellite providers, including Hughes and ViaSat.  At left, we
show the overall distribution without addresses from known satellite
ISPs; at right, we show only satellite ISPs.  (Recall that the
precision of values just above the ISI timeout of three seconds is
limited to integers; this creates the horizontal bands.)  The
satellite-only ISPs plot shows that the 1st percentile RTT for these
addresses exceeds 500ms in all cases, showing that the RTTs are almost
double the theoretical minimum. There are
some points in the left plot that remain within the satellite-like
cluster; at least some of these are from rural broadband providers
that provide both satellite and other connectivity, such as xplorenet
in Canada, which had at least one IP address report with a below 0.5s first percentile.

Each satellite provider has a distinct cluster in this scatter
plot, and two smaller providers, Horizon and iiNet, have
clusters of reports that produce near-horizontal lines in
the graph, with varying 1st percentile but fairly consistent
99th percentile, as if queuing for these addresses is
capped but the base distance to the satellite varies by geography.

Although some satellite hosts do have remarkably high RTT
values---up to 517s---their 99th percentile values
are predominantly below 3s.  They do not have such
high 99th percentile values as the rest of the hosts with
over 0.3s first percentiles (those shown on the left
graph). Thus, satellite ASes accounted for very few of the high latency
addresses. 


