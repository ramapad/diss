
\begin{figure}[tb]
\begin{center}
% \centering
\includegraphics[width=3in]{timeouts/figs/proto_98th_pctiles_cdf}
\end{center}
\caption{\label{fig:icmp_vs_udp_vs_tcp}%
98th percentile RTTs associated with high-latency IP addresses using different probe protocols.  The first probe of a triplet (seq 0) often has a higher latency than the rest; TCP probes appear to have a similar distribution except for firewall-sourced responses.}
\end{figure}

\subsection{Is it ICMP?}

One might expect that high latencies could be a result of
preferential treatment against ICMP.
% pointed out by http://conferences.sigcomm.org/imc/2001/imw2001-papers/85.ps.gz
% http://conferences.sigcomm.org/imc/2001/imw2001-papers/85.pdf
RFC 1812 allows routers responding to ICMP to rate-limit
replies~\cite{rfc1812,ipmp}, however, this limitation of ICMP
should not substantially affect the results since each
address is meant to receive a ping from ISI once every
eleven minutes. Nevertheless, one can imagine firewalls or
similar devices that would interfere specifically with ICMP.

To evaluate this possibility, we selected high-latency
addresses from the IT63c (20150206) survey.  To these
addresses we sent a probe stream consisting of three ICMP
echo requests separated by one second, then 20 minutes
later, three UDP messages separated by one second, then
again 20 minutes later, three TCP ACK probes separated by one
second.
We avoided TCP SYNs
because they may appear to be associated with security 
vulnerability scanning.
We then consider the characteristics of these hosts
in terms of the difference between ICMP delay and TCP or UDP
delay.

% \paragraph{``High-latency'' addresses to sample}
\subsubsection*{``High-latency'' addresses to sample}

We choose the top 5\% of addresses when sorting by each of
the median, 80th, 90th and 95th percentile latencies.  Many
of these sets of addresses overlap: those who have among the
highest medians are also likely to be among the highest 80th
percentiles.  However, we considered these different sets to
be important so that the comparison would include both hosts
with high persistent latency and those with high occasional
latency.  After sampling 15,000 addresses from each of these four
sets, then removing duplicates, we obtain 53,875 addresses to probe. 

% We obtain addresses with high latencies from ISI's IT63c (20150206) dataset by selecting a random sample from those addresses whose percentile latencies were greater than 95\% of the probed addresses.
% For example, 95\% of all addresses have their 50\%th percentile latency below 1.381 seconds according to Table~\ref{tbl:grand}. 
% We chose 5\% of the 3,636,156 responsive addresses from IT63c (20150206) which had latencies above this value for our first set. Since it's possible that IP addresses with relatively high median latencies may have relatively low higher percentile latencies, we also obtain addresses which had their 80th, 90th and 95th percentile latencies greater than 95\% of all addresses. We find that 74\% of addresses are present in all the sets reflecting that IP addresses with high 50\% latencies are also likely to have high 80th, 90th and 9th percentile latencies. Subsequently, we randomly sample 15,000 addresses from each set and obtain 53,875 unique candidate addresses with high relative latency.
% 

From these addresses, we found that only 5,219 responded to
all probes from all protocols on April 29, 2015.  This is
somewhat expected: Only 27,579 responded to any probe from any
protocol.  

To complete the probing, we use
Scamper~\cite{luckie2010scamper} to send the probe stream
to each of the 
candidate addresses.
Note that
scamper uses a 2s timeout by default although the
timeout can be configured. Instead of setting an alternate
timeout in Scamper, we run tcpdump to collect all received
packets, effectively creating an ``indefinite'' timeout. This
allows us to observe packets that arrive arbitrarily late
since we continue to run tcpdump days after the Scamper 
code finished.

% \vspace{-0.1in}
% \paragraph{All protocols are treated the same (mostly)}
\subsubsection*{All protocols are treated the same (mostly)}

For each protocol, we select the 98th percentile RTT per address and
plot the distribution in Figure~\ref{fig:icmp_vs_udp_vs_tcp}. 
% Figure~\ref{fig:icmp_vs_udp_vs_tcp} shows the distribution
% of the 98th percentile latency for each address for ICMP, UDP and TCP.
%
We noticed two obvious features of
the data: that the first packet of the triplet often had a
noticeably different distribution of round trip times, and
that the TCP responses often had a mode around 200ms.
%
We will investigate
the ``first ping'' problem in Section~\ref{subsec:first_ping}.

The TCP responses appear to be generated by firewalls that
recognize that the acknowledgment is not part of a
connection and sent a RST without notifying the actual
destination: this cluster of responses all had the same TTL
and applied to all probes to entire /24 blocks.  That is,
for each address that had such a response, all other
addresses in that /24 had the same.

Ignoring the quick TCP responses apparently from a firewall,
it does not appear that any protocol has significant
preferential treatment among the high-latency hosts.  Of
course, this observation does not show that prioritization
does not occur along any of these paths; our assertion is
only that such prioritization, if it exists, is not a source
of the substantial latencies we observe.

% \begin{figure}
% \begin{center}
% \includegraphics[width=3in]{figs/icmp_vs_udp_vs_tcp}
% \end{center}
% \caption{\label{fig:is_it_proto}
% CDF of 98th percentile latencies per IP address for ICMP, UDP and TCP. We distinguish between sequence numbers. Mostly, all protocols have similar 98th percentile latencies. A few addresses have lower latencies for TCP, possibly because of a firewall.
% \end{figure}


% 
% 
% There were 539 addresses which had their 98th percentile values lower
% than 300ms. 464 of these addresses belong to a single AS (AS 26615)
% and all of them have identical TTL values.


