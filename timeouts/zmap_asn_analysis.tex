\newcommand{\elide}[1]{}

\newcommand{\hdrbar}[1]{\multicolumn{1}{c|}{\textbf{#1}}}
\begin{table*}[t]%
  \begin{center}%
  \begin{small}%
  \begin{tabular}{ll|rrr|rrr|rrr}
  % \begin{tabular}{r|rrr|rrr|rrr|rrr}
  % \begin{tabular}{rl|r|rr}
    % & & \multicolumn{3}{c|}{\textbf{April 2015}} &
    & & 
    \multicolumn{3}{c|}{\textbf{May 2015}} &
    \multicolumn{3}{c|}{\textbf{June 2015}} &
    \multicolumn{3}{c}{\textbf{July 2015}} \\ 
    % \hdr{ASN} & \hdr{$>$1s} & \%  & \hdr{Rank} &
    % \hline 
    \hdr{ASN} & \hdrbar{Owner} & 
    \hdr{$>$1s} & \%  & \hdrbar{Rank} &
    \hdr{$>$1s} & \%  & \hdrbar{Rank} &
    \hdr{$>$1s} & \% & \hdr{Rank} \\
    % \hdr{$>$1s} & \% & \hdr{Rank} \\
    \hline 
    26599 & TELEFONICA BRASIL & 
    % 26599 &
    % 2,941,446 & 79.275 & 1 & 
    3.56M & 80.4 & 1 & 
    3.87M & 77.5 & 1 &
    4.20M & 77.0 & 1\Tstrut \\

    26615 & Tim Celular S.A. &
    1.35M & 74.5 & 3 &
    1.42M & 71.5 & 2 &
    1.72M & 71.6 & 2 \\

    45609 & Bharti Airtel Ltd. &
    1.46M & 76.6 & 2 &
    1.21M & 81.0 & 3 &
    1.03M & 79.2 & 3 \\

    22394 & Cellco Partnership &
    0.55M & 73.4 & 8 &
    0.58M & 73.5 & 4 &
    0.63M & 72.7 & 4 \\

    1257 & TELE2 &
    0.67M & 69.5 & 5 &
    0.42M & 65.5 & 9 &
    0.58M & 67.4 & 5 \\

    27831 & Colombia Movil &
    0.53M & 68.8 & 9 &
    0.54M & 64.3 & 5 & 
    0.53M & 62.8 & 6 \\

    6306 & VENEZOLAN &
    0.69M & 77.3 & 4 &
    0.41M & 76.4 & 10 &
    0.40M & 75.7 & 10 \\

    9829 & National Internet Backbone &
    0.57M & 27.6 & 7 &
    0.43M & 30.9 & 7 &
    0.43M & 29.5 & 9 \\

    4134 & Chinanet &
    0.60M & 1.5 & 6 &
    0.38M & 0.9 & 11 &
    0.34M & 0.9 & 11 \\

    35819 & Etihad Etisalat (Mobily) & 
    0.42M & 54.0 & 10 &
    0.43M & 54.5 & 6 &    
    0.45M & 55.8 & 8 \\
  \end{tabular}
  \end{small}
  \end{center}
  \caption{\label{tbl:zmap_asns} Autonomous Systems sorted by the
    addresses summed across three Zmap scans for addresses that observed
    RTTs greater than 1s. The table shows for each AS: the number and
    percentage of addresses with RTT greater than 1s and the rank in that scan.}
\end{table*}

\begin{table*}[t]%
  \begin{center}%
  \begin{small}%
  \begin{tabular}{c|rr|rr|rr}
    &
    \multicolumn{2}{c|}{\textbf{May 2015}} &
    \multicolumn{2}{c|}{\textbf{June 2015}} &
    \multicolumn{2}{c}{\textbf{July 2015}} \\ 
    \hdrbar{Continent} & 
    \hdr{$>$1s} & \%  & 
    \hdr{$>$1s} & \%  & 
    \hdr{$>$1s} & \% \\\hline
    South America &
    7.27M & 26.7 &
    7.41M & 25.8 &
    8.05M & 26.9\Tstrut \\

    Asia &
    5.56M & 3.8 &
    4.73M & 3.4 &
    4.56M & 3.2 \\

    Europe &
    2.56M & 2.7 &
    2.09M & 2.2 &
    2.32M & 2.4 \\

    Africa &
    1.12M & 29.4 &
    1.20M & 30.3 &
    1.30M & 31.7 \\

    North America &
    0.93M & 1.0 &
    1.04M & 1.1 &
    1.14M & 1.2 \\

    Oceania &
    0.08M & 3.9 &
    0.08M & 3.7 &
    0.08M & 3.7 \\

\end{tabular}
  \end{small}
  \end{center}
  \caption{\label{tbl:zmap_conts} Continents sorted by the addresses summed across three Zmap scans for addresses that observed
    RTTs greater than 1s. The table shows for each AS: the number and
    percentage of addresses with RTT greater than 1s in that scan.}
\end{table*}


\begin{table*}[t]%
  \begin{center}%
  \begin{small}%
  \begin{tabular}{ll|rrr|rrr|rrr}
    & & 
    \multicolumn{3}{c|}{\textbf{May 2015}} &
    \multicolumn{3}{c|}{\textbf{June 2015}} &
    \multicolumn{3}{c}{\textbf{July 2015}} \\ 
    \hdr{ASN} & \hdrbar{Owner} & 
    \hdr{$>$100s} & \%  & \hdrbar{Rank} &
    \hdr{$>$100s} & \%  & \hdrbar{Rank} &
    \hdr{$>$100s} & \% & \hdr{Rank} \\
    \hline 
    26599 & TELEFONICA BRASIL & 
    51.9K & 1.2 & 1 &
    63.5K & 1.3 & 1 &
    77.6K & 1.4 & 1\Tstrut \\
    
    12430 & VODAFONE ESPANA S.A.U. &
    12.8K & 4.4 & 2 &
    11.6K  & 4.1 & 2 &
    14.6K & 5.2 & 3 \\

    26615 & Tim Celular S.A. &
    6.2K & 0.3 & 7 &
    9.4K & 0.5 & 3 &
    14.7K & 0.6 & 2 \\

    3352 & TELEFONICA DE ESPANA &
    8.5K & 0.2 & 3 & 
    7.3K & 0.1 & 5 &
    7.5K & 0.2 & 4 \\
    
    6306 & VENEZOLAN &
    7.5K & 0.8 & 5 &
    8.4K & 1.5 & 4 &
    6.6K & 1.2 & 6 \\

    22394 & Cellco Partnership &
    6.9K & 0.9 & 6 &
    6.6K & 0.8 & 6 &
    7.5K & 0.9 & 5 \\
    
    27831 & Colombia Movil &
    3.2K & 0.4& 10 &
    5.0K & 0.6 & 7 &
    5.2K & 0.6 & 7 \\

    45609 & Bharti Airtel Ltd. &
    7.8K & 0.4 & 4 &
    2.6K & 0.2 & 9 &
    2.9K & 0.2 & 9 \\

    35819 & Etihad Etisalat (Mobily) & 
    3.8K & 0.5 & 9 &
    3.9K & 0.5 & 8 &
    4.0K & 0.5 & 8 \\

    1257 & TELE2 &
    6.2K & 0.4 & 8 &
    1.7K & 0.3 & 14 &
    2.4K & 0.3 & 12 \\
    
  \end{tabular}
  \end{small}
  \end{center}
  \caption{\label{tbl:zmap_asns_100s} Autonomous Systems sorted by the
    addresses summed across three Zmap scans for addresses that observed
    RTTs greater than 100s. The table shows for each AS: the number and
    percentage of addresses with RTT greater than 100s and the rank in that scan.}
\end{table*}


\subsection{Autonomous Systems with the most high latency addresses}
\label{subsec:motivation}

Next, we investigate the ASes and geographic locations with the most high latency
addresses to identify relationships. For this analysis, we use Zmap scans from 2015 to
identify high latency addresses. Zmap pings every IPv4 address,
thereby covering addresses from all ASes. We chose
the May 22, Jun 21 and Jul 9 Zmap scans to study. These scans
were conducted at different times of the day, on different days of the week
and in different months, as shown in Table~\ref{tbl:scans}.
For each of these Zmap scans, we use Maxmind to find the ASN and geographic
location for every address that responded.
% , analyzing these
% responses allows us to recognize high-latency trends in
% Autonomous Systems and geographic locations.
% \vspace{-0.1in}
\subsubsection*{ASes most prone to RTTs greater than 1 second}

Figure~\ref{fig:grand_zmap} showed that the percentage of
addresses that sent high latency Echo Responses remained stable over time.
%
In particular, around 5\% of addresses observed RTTs greater than
a second in each scan. We refer to these addresses as \emph{turtles} and
investigate their distribution 
across Autonomous Systems to identify relationships.

For each Zmap scan, we found the turtles and identified
their AS, and ranked ASes by the number of contributed
turtles. Finally, we summed the turtles from each AS across the three
scans and sort ASes accordingly and show the top ten in Table~\ref{tbl:zmap_asns}. For
example, AS26615 had the second-largest sum of turtles across the three Zmap
scans, but was ranked third within the May 2015 scan.

Inspecting the owners of each of these Autonomous Systems reveals that a majority of them are cellular. 
%
AS26599 (TELEFONICA BRASIL), a cellular AS in Brazil, has the most
turtles, more than double that of the next largest AS in
each of the scans. 
%
The next two ASes, AS45609 (Bharti Airtel Ltd.), and AS26615 (Tim Celular), are also cellular, 
and so are 5 of
the remaining 7 ASes in the top 10. 

Also notable is the percentage of responding addresses
that are turtles for these ASes. Most of the cellular ASes have around
70\% of all probed addresses being turtles. 
AS9829, one of the two ASes with turtles accounting
for lower than 50\% of probed addresses, is known to offer other
services in addition to cellular. 
%
AS4134, with only 1\% of its probed addresses
being turtles, is also known to offer other services.
We believe that the cellular addresses observe high RTTs while
others do not, explaining the low ratio of probed addresses with
RTTs greater than 1 second.
%

%
Finally, nine ASes were observed in the top
ten in every scan. AS4134 was the only exception, but it ranked 11th in the June and
July scans. Thus, the Autonomous Sytems with the most
turtles also remain consistent over time.
%


Table~\ref{tbl:zmap_conts}  shows the continents with the most turtles. South America and Asia alone
account for around 75\% of all turtles.
%
Further, around a quarter of all
addresses in South America and a third of the addresses in Africa experienced
RTTs greater than 1s in each scan.
%
On the other hand, only 1\% of North America's addresses are turtles
(of which more than
half come from a single ASN: AS22394).

% This suggests that turtles are more prevalent in developing
% areas of the world.

\subsubsection*{ASes most prone to RTTs greater than 100 seconds}

% In each Zmap scan, we found around XX addresses with RTTs greater than
% 100s: we refer to these addresses as RTTgt100s.
%
Next, we investigate the Autonomous Systems of addresses with RTTs
greater than 100 seconds in the three Zmap scans: we refer to these addresses
as sleepy-turtles. 
We consider whether these addresses are different from turtles to
identify whether there is a different underlying cause.
%
Following the same process in identifying ASes and sorting them as in
Table~\ref{tbl:zmap_asns}, Table~\ref{tbl:zmap_asns_100s} shows Autonomous Systems that are
most prone to RTTs greater than 100 seconds.  
%

We find that sleepy-turtles exhibit similarities to turtles. Every Autonomous System in
Table~\ref{tbl:zmap_asns_100s} is cellular. Further, the ranks of the
Autonomous Systems remain stable over time across the scans. However,
there is more variation across the scans for the percentage of
sleepy-turtles among all probed addresses for
an AS. This suggests that the fraction of addresses experiencing RTTs
greater than 100 seconds is less stable
over time.
