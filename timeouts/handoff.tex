
\subsection{Patterns associated with RTTs greater than 100 seconds}

\newcommand{\noop}[1]{#1}
\begin{table}[t]
  \begin{center}%
    \begin{small}%
      \begin{tabular}{c|rrr}
        \hdrbar{Pattern} & \hdr{Pings} & \hdr{Events} & \hdr{Addrs}\\\hline
        \noop{Low latency, then decay} & 615 & 13 & 10\Tstrut\\
        \noop{Loss, then decay} & 1528 & 81 & 33\\
        \noop{Sustained high latency and loss} & 2994 & 21 & 14 \\
        \noop{High latency between loss} & 12 & 12 & 12
        % 
      \end{tabular}
    \end{small}
    \end{center}
  \caption[Patterns of latency and loss near high latency responses]{We observed distinct patterns of latency and loss near high latency responses,
    classifying all 5149 pings above 100 seconds from the sample.}
\label{tbl:handoff}
\end{table}
      
 Finally, we look at addresses with extraordinarily high
latencies ( greater than 100 seconds); in particular, we want to
understand whether these high latencies are an instance of a
first-ping-like behavior, where wireless negotiation or
buffering during intermittent connectivity creates the high
value, or, on the other hand, are instances of extreme
congestion.  To separate the two types of events, we consider
a sequence of probes, looking for whether or not the latency
diminishes after a ping beyond 100 seconds.

We sample 3,000 of 38,794 addresses whose 99th percentile
latency was greater than 100 seconds in the IT63c (20150206)
dataset.  Of this sample, 1,400 responded.
%
We sent each address 2000 ICMP Echo Request packets using
Scamper, spaced by 1 second.
%
To collect responses with very high delays without altering
the Scamper timeout, we simultaneously run tcpdump to
capture packets.
%

Ping samples that saw a round trip time above 100 seconds
exist in the context of a few very distinct patterns.
%
Often, a series of successive ping responses would be delivered
together almost simultaeously, leading to a steady decay in their
round trip times.
%
For example, after 136 seconds of no response
from IP address 191.225.110.96, we received all 136 responses over
a one second interval: every subsequent response's
round-trip latency was 1 second lower than the previous.
%
This pattern is sometimes preceded by a relatively low latency ping
($<$ 10
seconds) and at other times, follows a few lost pings: we distinguish
between these two cases and call the former \emph{Low latency, then decay}
and the latter \emph{Loss, then decay}.  It is possible that these
are both observing the same underlying action on the network, but
we leave them separate since there are substantially many of each.
%%% NEED TO EXPLAIN THE RULE.
%

Another characteristic pattern is that a high round trip time is
followed by
 several responses of even greater latency, possibly with intermittent losses.
%
This behavior is usually sustained for several minutes with latencies
remaining higher than normal ($>$10 seconds) throughout the duration:
we call this behavior \emph{Sustained high latency and loss}.
%
Finally, there are some cases where a single ping has a latency $>$
100 seconds and is preceded and followed by loss. We call these cases
\emph{High latency between loss}.

We count the number of occurrences of each pattern in Table~\ref{tbl:handoff}. 
%
For
each pattern, we show the number of pings greater than 100 seconds that were part of
that pattern, the number of instances of that
pattern occurring, and the number of unique addresses for which it
occurred.
%
We observe that the majority of events and addresses are \emph{Loss, then decay}, yet almost
twice as many pings are part of \emph{Sustained high latency and
  loss}.
%

