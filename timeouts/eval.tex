% \vfill\eject
\section{Recommended Timeout Values}
\label{sec:eval}
% \newcommand{\hdr}[1]{\multicolumn{1}{c}{\textbf{#1}}}

In this section, we analyze the ping latencies of all pings obtained
from ISI's Internet survey datasets from 2015 to find reasonable timeout values.
We demonstrate the effectiveness
of our matching scheme for recovering delayed responses from the
dataset. We then group the survey-detected responses and delayed
responses together to determine what timeout values would be necessary to 
recover various percentiles of responses. 
Some IP
addresses observe very high latencies in the ISI dataset; we verify that
these are real in Section~\ref{sec:verification} and examine causes in Section~\ref{sec:causes}.

% we validate
% our results by probing these IP addresses with the scamper tool~\cite{luckie2010scamper} and
% analyzing their latencies.

\subsection{Incorporating unmatched responses}

\begin{table}[tb]
  \begin{center}
    \begin{small}
  \begin{tabular}{c|rr}
    & \hdr{Packets} & \hdr{Addresses} \\
    \hline
    \textbf{Survey-detected } & 9,644,670,150 & 4,008,703\Tstrut\\
    % \hline
    \textbf{Naive matching} & 9,768,703,324 & 4,008,830 \\
    % \hline
    \textbf{Broadcast responses} & 33,775,148 & 9,942 \\
    % \hline
    \textbf{Duplicate responses} & 67,183,853 & 20,736 \\
    % \hline
    % \textbf{Filtered survey-detected} & 125,578,194,082 & 16,320,357 \\
    % \hline
    \textbf{Survey + Delayed} & 9,667,744,323 & 3,978,152 \\
    \end{tabular}
    \end{small}
  \end{center}
    \caption{Adding unmatched responses to survey-detected responses}

\label{tbl:filter_results}
\end{table}

\begin{figure*}[tb]
\begin{subfigure}[t]{0.47\linewidth}
\centering
\includegraphics[width=\linewidth]{timeouts/figs/cdf_basic_ping_ttl}
\caption{
\label{fig:basic_lat}
Before filtering
}
\end{subfigure}
%
\hfill
%
\begin{subfigure}[t]{0.47\linewidth}
\centering
\includegraphics[width=\linewidth]{timeouts/figs/cdf_filter_ping_ttl}
\caption{
\label{fig:filter_lat}
After filtering
}
\end{subfigure}
\caption{
\label{fig:lat} 
CDF of Percentile latency per IP address before and after filtering
  unexpected responses. Each point represents an IP address and each
  color represents the percentile from that IP address's response
  latencies. Before filtering unexpected responses, there are
  bumps caused by broadcast responses at 330s, 165s and 495s,
  fractions of the 11 minute (660s) probing interval.
}
\end{figure*}


ISI detected 9.64 Billion echo responses from 4 Million IP addresses in 2015
in the IT63w (20150117) and IT63c (20150206) datasets, as shown in the
first row of Table~\ref{tbl:filter_results}.
The next row shows the number of responses we would have obtained
if we had used a naive matching scheme where we simply matched each
unmatched response for an IP address with the last echo request for
that IP
address, without filtering unexpected responses.
The number of responses
increases by 1.3\% to 9.77 Billion; however, this includes responses
from addresses that received broadcast responses and duplicate
responses. 
%
After filtering unexpected responses, the number of IP addresses
reduces to 99.23\% of
the original addresses. 
%
Of 30,678 discarded IP addresses, 9,942 (32.4\%)
addresses were discarded because they also received broadcast
responses. 
%
The majority of discarded IP addresses, 20,736 (67.6\%) were addresses that
sent more than 4 echo responses in response to a single echo request. 

Though the number of discarded IP addresses is relatively small, removing
them eliminates responses that cluster around 330, 165, and 495 seconds.
Figure~\ref{fig:lat} shows the distribution of percentile latency per
IP address before and after filtering unexpected responses.  Comparing
these two graphs shows that the ``bumps'' in the CDF are removed by the filtering.

After discarding addresses, our matching technique yields 23,074,173
additional responses for the remaining addresses, giving us a total of
9.67 Billion Echo Responses from 3.98 Million IP addresses. We perform
our latency analysis on this combined dataset.

% They are caused
% because of the non-random sequence of probing, which meant that 0 and
% 1 were probed 330seconds apart.

% \newcommand{\bb}{~~~~~}
%%% \begin{table}[tp]%
%%%   \begin{center}%
%%%   \begin{small}%
%%%   \begin{tabular}{c@{\hspace{0.5em}}r|rrrrrrr}
%%%     &\multicolumn{8}{c}{\textbf{\% of pings}\vspace{0.3em}} \\
%%%     && \hdr{1\%} & \multicolumn{1}{c}{\textbf{50\%}} & \hdr{80\%} & \hdr{90\%} & \hdr{95\%} &
%%%     \hdr{98\%} & \hdr{99\%} \\\cline{2-9}
%%%     \parbox[t]{2mm}{\multirow{7}{*}{\rotatebox[origin=c]{90}{\textbf{\% of addresses}}}} & 
%%%     \textbf{1\%} & 0.01 & 0.05 & 0.11 & 0.13 & 0.16 & 0.29 & 0.34 \\
%%% %    \cline{2-9}
%%%     &\textbf{50\%} & 0.12 & 0.24 & 0.35 & 0.50 & 0.67 & 0.92 & 1.21 \\
%%% %    \cline{2-9}
%%%     &\textbf{80\%} & 0.19 & 0.33 & 0.57 & 1.34 & 2.20 & 5\bb & 7\bb \\
%%% %    \cline{2-9}
%%%     &\textbf{90\%} & 0.24 & 1.00 & 2.39 & 4\bb & 6\bb & 8\bb & 11\bb \\
%%% %    \cline{2-9}
%%%     &\textbf{95\%} & 0.31 & 2.43 & 5\bb & 6\bb & 7\bb & 12\bb & 18\bb \\
%%% %    \cline{2-9}
%%%     &\textbf{98\%} & 0.37 & 4\bb & 6\bb & 7\bb & 11\bb & 20\bb & 52\bb \\
%%% %    \cline{2-9}
%%%     &\textbf{99\%} & 0.43 & 4\bb & 6\bb & 8\bb & 14\bb & 55\bb & 159\bb \\
%%%     \end{tabular}
%%%   \end{small}
%%%   \end{center}
%%% 
%%% 
%%%     \caption{Minimum timeout in seconds that would have captured c\% of pings from r\% of IP
%%%       addresses in pings sent from Nov 2011 to October 2013 (where r is the row number and c is
%%%       the column number).}
%%% \label{tbl:grand}
%%% \end{table}

%\newcommand{\bb}{~~~~~}


\subsection{Recommended Timeout Values}

We now find retransmission thresholds which recover various
percentiles of responses for the IP addresses from the
combined dataset. For each IP address, we find the 1st, 50th, 80th,
90th, 95th, 98th and 99th percentile latencies. We then find the 1st, 50th,
80th, 90th, 95th, 98th and 99th percentiles of all the 1st percentile latencies. We repeat this for each
percentile and show the results in Table~\ref{tbl:grand_2015}.

\begin{table}[tb]
  % \begin{center}%
    \begin{small}%
      \hspace{-0.06in}%
  \begin{tabular}{l@{\hspace{0.5em}}r|rrrrrrr}
    &\multicolumn{8}{c}{\textbf{\% of pings}} \\
    && \hdr{1\%} & \multicolumn{1}{c}{\textbf{50\%}} & \hdr{80\%} & \hdr{90\%} & \hdr{95\%} &
    \hdr{98\%} & \hdr{99\%} \\\cline{2-9}
    \multirow{7}{*}{\rotatebox[origin=lb]{90}{\textbf{\% of addresses}}} & 
    \textbf{1\%} & 0.01 & 0.03 & 0.04 & 0.07 & 0.10 & 0.13 & 0.18\Tstrut \\
%    \cline{2-9}
    &\textbf{50\%} & 0.16 & 0.19 & 0.21 & 0.26 & 0.42 & 0.53 & 0.64 \\
%    \cline{2-9}
    &\textbf{80\%} & 0.19 & 0.26 & 0.33 & 0.43 & 0.54 & 0.74 & 1.21 \\
%    \cline{2-9}
    &\textbf{90\%} & 0.22 & 0.31 & 0.42 & 0.57 & 0.84 & 1.61 & 3\bb \\
%    \cline{2-9}
    &\textbf{95\%} & 0.25 & 1.42 & 2.38 & 3\bb & 5\bb & 9\bb & 15\bb \\
%    \cline{2-9}
    &\textbf{98\%} & 0.30 & 1.94 & 4\bb & 6\bb & 12\bb & 41\bb & 78\bb \\
%    \cline{2-9}
    &\textbf{99\%} & 0.33 & 2.31 & 4\bb & 8\bb & 22\bb & 76\bb & 145\bb \\
    \end{tabular}
    \end{small}
    % \end{center}

\vspace{\baselineskip}

    \caption{Minimum timeout in seconds that would have captured c\% of pings from r\% of IP
      addresses in the IT63w (20150117) and IT63c (20150206) datasets (where r is the row number and c is
      the column number).}
\label{tbl:grand_2015}
\end{table}

The 1st percentile of an address's latency will be close to the ideal latency that its link
can provide. We find that the 1st percentile latency is below 330ms for 99\%
of IP addresses: most addresses are capable of
responding with low latency. Further, 50\% of pings from 50\% of the
addresses have latencies below 190ms, showing that latencies tend to
be low in general.

However, we see that a substantial fraction of IP addresses also have
surprisingly high latencies. For instance, to capture 95\% of pings from 95\%
addresses requires waiting 5 seconds.  Restated, at least 5\% of
pings from 5\% of addresses have latencies higher than 5 seconds. Thus, even
setting a timeout as high as 5 seconds will infer a false loss rate of 5\%
for these addresses.

Note that retrying lost pings cannot be used as a substitute for
setting a longer timeout since a retried ping is not an independent
sample of latency. 
Whatever caused the first one to be delayed is likely to cause the
followup pings to be delayed as well, as we show in Section~\ref{sec:causes}.

At the extreme, we see 1\% of pings from 1\% of addresses
having latency above 145 seconds! These latencies are so high that
we investigate these addresses further.  \emph{We now
  consider 60 seconds to be a reasonable timeout to balance
  progress with response rate, at least when studying
  outages and latencies, although an ideal timeout may vary
  for different settings.}  A timeout of 60 seconds easily
covers 98\% of pings to 98\% of addresses, yet does not seem
long enough to slow measurements unnecessarily.

