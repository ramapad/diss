% \section{Importance of probe timeouts}  
\section{Challenges in selecting a timeout for probing techniques}
\label{sec:motivation}

% Paper stuff
% In this section, we describe why it is important to choose an appropriate
% timeout for active probes, especially when used
% for outage detection. We also describe measurement
% studies with particular attention to what timeouts were used and how those
% timeouts were chosen.

Conventional wisdom suggests that active probes on the Internet should timeout
after a few seconds. The belief is that after a few seconds there is a very
small chance that a probe and response will still exist in the
network. Once a probe times out, the prober can free the state
associated with the probe, thereby reclaiming memory.

Conventional wisdom also suggests that a single timed out probe is
insufficient to reason about end-host failures, due to potential
random loss on the Internet. For most probing systems, any timed out
active probes are followed up with retransmissions to increase the
confidence that a lack of response is due to an outage event and not
due to random loss on the Internet. These followup probes will also
have a timeout that is generally the same as the first attempt.

Setting correct timeouts is critical for probing-based remote outage
detection techniques. These techniques infer outages based upon lost
probes and probe response loss is dependent upon the prober's
timeout. Additonally, since probe timeouts trigger followup probes,
setting appropriate timeouts is vital to these techniques.

However, choosing an appropriate timeout is
challenging. Selecting a timeout value that is too low will ignore delayed
responses and might add to congestion by performing retransmissions to an
already congested host. Timeout values that are too high will delay
retransmissions that can confirm an outage. In addition, too-high timeouts
increase the amount of state that needs to be maintained at a prober, since
every probe will need to be stored until either the probe times out,
or the response arrives.

% Even for studies that don't focus upon outages, selecting a good 
% timeout is important. For instance, in the ISI surveys we study, most probes solicit
% no responses. To the best of our knowledge, this is the first paper
% that investigates this
% broad lack of responses to see if researchers are simply using timeouts that
% are too short.

\subsection{Timeouts used in outage and connectivity studies}

Outage detection systems such as Trinocular~\cite{trinocular}
and Thunderping~\cite{pingin} tend to use a 3 second timeout for active
probes because it is the default TCP SYN/ACK
timeout~\cite{rfc1122}. Both techniques will not infer outages if a
single response is delayed beyond the timeout, since they send
follow-up probes to confirm suspected outages. However, if a series of
responses are delayed beyond the timeout, both techniques can
potentially infer false probe-loss and therefore, false
outages. % Ideally, we would like to detect these events as \emph{sleep}
% events, since the probe responses are delayed, not lost.

Internet performance monitoring systems use a wide range of probe
timeouts. On the shorter side, iPlane~\cite{iplane} and
Hubble~\cite{hubble} send ICMP echo requests with a 2 second
timeout. iPlane declares a host unresponsive after one failed
retransmission. Hubble waits two minutes after a failed probe then
retransmits probes six times and finally declares reachability with
traceroutes. On the longer side, Feamster et
al.~\cite{measuring-effects} used a one hour timeout after each
probe. However, they chose a long timeout to avoid errors due to clock
drift between their probing and probed hosts; they did not do so to
account for links that have excessive
delays. PlanetSeer~\cite{planetseer} assumed that four consecutive TCP
timeouts (3.2-16 seconds) indicates a path anomaly.

% internal
It is especially important for connectivity measurements from probing
hardware placed inside networks to have timeouts because of the limited memory
in the probing hardware. The RIPE Atlas~\cite{atlas} probing hardware sends
continuous pings to various hosts on the Internet to observe connectivity. The
timeout for their ICMP echo requests is 1
second~\cite{atlas-mailing-list-post}. 
The SamKnows probing hardware uses a 3 second timeout for ICMP echo requests
sent during loaded intervals~\cite{samknows}.

We started this study with the expectation that these
timeout values might need minor adjustment to account for
large buffers in times of congestion; what we found was
quite different.

% dunno what to do with this text
%Other studies have analyzed ping latencies. Cai et.
%al~\cite{cai2011understanding} use the variance of ping latencies to
%distinguish between low-bitrate edge links and broadband connections. Pelsser
%et. al~\cite{pelsser2013paris} study if pings are suitable to measure delay and
%jitter for applications and they find that latency variance is high for probes
%belonging to different flows, and that variance is much less for probes
%belonging to the same flow. Our focus in this work is specifically upon
%utilizing ping latency data to find appropriate retransmission thresholds.

% it should
% be able to distinguish between a lost response and a delayed
% response. to beProbe retransmission is crucial for outage detection using active
% probes. Both Thunderping and Trinocular utilize adaptive retransmission to
% distinguish between probes lost due to end-host outages and probes
% lost due to general loss on the Internet\rama{Do we have a citation for how much general loss on the
%   Internet is?}. However, outage detection requires that a delayed
% probe not be interpreted as a lost probe. Sending more packets to
% an already congested host can be harmful, thus the timeout for
% retransmission should capture most packets. At the same time, a very
% high timeout can result in retransmissions occurring after long
% periods of time, resulting in delayed outage detection. Further, high
% timeouts require high memory usage since all packets in transit need
% to be kept track of. 

% Currently, both Thunderping and Trinocular use fixed timeout durations
% before retransmission.
 % Thunderping uses a 5s timeout while Trinocular
% uses a 3 second timeout.

% Our goal in this paper is to find reasonable timeout thresholds by
% performing an analysis of ping latency. 
% We study the distribution of
% ping latency in a representative dataset and find timeout thresholds
% that can recover X percentile of probe responses from any IP address.

% \textbf{Outage detection with active probes}: The Thunderping
% project~\cite{schulman2011} detects outages in residential hosts
% caused by severe weather. Thunderping uses a timeout of 5s before retransmission.
% Trinocular detects outages in all /24 blocks on the
% Internet. Trinocular retransmits packets after a 3s timeout.

% \textbf{Latency of pings}: Heidemann et al. study how the variance in
% ping latency can be used to distinguish between different types of
% links in~\cite{cai2011understanding}. Pelsser et al. examine how the
% variance of ping latency on a path can be reduced by employing common
% flow IDs. 

