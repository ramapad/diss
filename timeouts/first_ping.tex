
\newcommand{\countInitialList}{236,937}
\newcommand{\countNotprobed}{1,994}
\newcommand{\countResponsives}{83,174}
\newcommand{\countMaxPrelimAboveMaxRest}{51,009}
\newcommand{\countDropsToMedianAndToMax}{51,646}
\newcommand{\countDropsToMedianNotToMax}{11,874}
\newcommand{\countFirstProbeNoResponse}{8,329}
\newcommand{\countNegativeDrop}{10,910}
\newcommand{\countDropsToMedianNotInitial}{3,079}
\newcommand{\countSecondProbeNoResponse}{1,311}
\newcommand{\countTooFewResponses}{415}
\newcommand{\countInitialSlashTwentyFours}{1,887}
\newcommand{\countResponsiveSlashTwentyFours}{1,230}
\newcommand{\countDroppingSlashTwentyFours}{1,083}
\newcommand{\countUnresponsive}{151,769}

\subsection{Is it the first ping?}
\label{subsec:first_ping}

% The latencies measured in Section~\ref{sec:eval} are
RTTs that are consistently greater than a second are sufficiently
high that interactive application traffic would seem impractical with these
delays.
We suspected that the latencies measured by ISI and Zmap
might not be typical of application traffic.

We considered two broad explanations---extraordinary
persistent latency due to oversized queues associated with
low-bandwidth links, or extraordinary temporary, initial
latency due to MAC-layer time slot negotiation or device
wake-up.

In this section, we find that the latter appears to be a
more likely explanation, qualitatively consistent with prior
investigations of GPRS performance
characteristics~\cite{gprsweb}, but showing quantitatively more significant delay.


We extracted \countInitialList{} IP addresses from the 20150206 ISI 
dataset (February 2015), including all addresses with 
a median RTT of at least one second.   To select only responsive
addresses that still had high latency, for each of these 
IP addresses, we sent two pings, separated by five seconds, 
with a timeout of 60 seconds.   We omit \countUnresponsive{} 
addresses that did not respond to either probe and 
\countNotprobed{} addresses that responded, on average, 
within 200ms.

%% todo -- want to say that 80s is long enough for
%% whatever process to reset. 
Of the \countResponsives{} addresses that remain, we 
wait approximately 80 seconds before sending ten pings, 
once per second with the same 60-second timeout.  We 
next classify how the round trip time of the first ping, $RTT_1$, differs from
those of the rest of the responded pings, $RTT_2 \ldots RTT_{n}$, where $n$ may be smaller than 10 if responses are missing.   For most of these addresses,
\countDropsToMedianAndToMax{}, the first
response took longer than the \emph{maximum} of the
rest.  This suggests that roughly 2/3 of high latency
observations are a result of negotiation or wake-up rather
than random latency variation or persistent congestion.
For \countDropsToMedianNotToMax, $median(RTT_2 \ldots RTT_{n}) < RTT_1 < max(RTT_2 \ldots RTT_{n})$, i.e., the first response
took longer than the \emph{median}, but not the maximum, of
the rest.  The first response was smaller than the
median of the rest for a comparable \countNegativeDrop.
That the first is above or below the median in roughly 
equal measure suggests that for these addresses there is 
little observed penalty to the first ping.
Finally, we omit analysis of \countFirstProbeNoResponse{} addresses
because we did not receive a response to, at least, the first probe, even though
they did respond to the initial pair of probes, 
and we omit an additional \countTooFewResponses{} addresses that did
respond to the first probe, but not to at least four probes overall
(i.e., we require $n \geq 4$ before computing the median or maximum
for comparison).

\begin{figure}[tb]
\begin{center}
\includegraphics[width=3in]{timeouts/figs/initial_dual_cdf}
\end{center}
\caption{\label{fig:initial_drops_cdf}Bottom: Difference between initial latency and second probe latency; values around 1 indicate that both responses arrive at about the same time, values near zero indicate that the RTTs were about the same.  The second line includes only those where $RTT_1 > max(RTT_2 \ldots RTT_n)$.  Top: The probability that, given $RTT_1 - RTT_2$ on the $x$-axis, that $RTT_1 > max(RTT_2 \ldots RTT_n)$.}
\end{figure}

% \vspace{-0.1in}

\subsubsection*{Can the overestimate be detected?}  We show in 
Figure~\ref{fig:initial_drops_cdf} the differences between the
first and second round trip times for all those that had a 
first and second response. (\countSecondProbeNoResponse{} 
addresses responded to the first but not the second).  Rarely,
latency increases from first to the second (yielding a negative 
difference) or decreases sufficient to indicate reordering (yielding 
a difference greater than one second).  Typical among these 
addresses is for the second ping to be one second less than the first, that
is, for both responses to arrive at about the same time.

We infer that a measurement approach that sent a second
probe after one second could detect this behavior.  The top
graph of Figure~\ref{fig:initial_drops_cdf} shows the
probability that the maximum will be less than the first based on the difference
between the first two latencies.  (When the RTT difference
exceeds 1 at the right edge of the upper graph, there are very
few samples in an environment of substantial reordering.)
Any significant drop from $RTT_1$ to $RTT_2$ is
indicative of an overestimate with high probability.

\begin{figure}[tb]
\begin{center}
\includegraphics[width=3in]{timeouts/figs/drops_to_min_cdf}
\end{center}
\caption{\label{fig:drops_to_min}Difference between initial latency and observed minimum.  The typical setup time is below four seconds.}
\end{figure}

\subsubsection*{How long does the negotiation or wake-up process take, and how 
large is the overestimate?}  We observe that this can be
estimated by comparing the first round trip time to the lowest 
seen among the ten probes.  Of course, if the negotiation takes
15 seconds, the first probe rtt will take at most 9 seconds 
longer than the last, so this data set will treat all instances of
a setup time between 10 and 60 seconds as taking 9. We show in Figure~\ref{fig:drops_to_min}
the differences between $RTT_1$ and $min(RTT_2 \ldots RTT_n)$
for those \countDropsToMedianAndToMax{} addresses that had a higher 
first rtt than the maximum of the rest.  The median is 1.37 seconds, and 90\% of the
differences are below 4 seconds.  Only 2\% of the samples are above 8.5 seconds,
suggesting that we do not underestimate this time substantially, and thus conclude
that the wake-up or negotiation process generally takes from one-half to four seconds.

\begin{figure}[tb]
\begin{center}
\includegraphics[width=3in]{timeouts/figs/prefix_percentage_cdf}
\end{center}
\caption{\label{fig:prefix_percentage_cdf}Percentage of addresses in a /24 prefix showing 
  a drop from the initial to the maximum.}
\end{figure}
% \neil{maybe drop the following par} 
\subsubsection*{Are the addresses that show a high initial
ping scattered across the IP address space or clustered into
/24s?}  The \countInitialList{} IP addresses that we decided
to probe initially are from only
\countInitialSlashTwentyFours{} ``/24'' prefixes.  This is
somewhat fewer prefixes than would be expected, given that
there are 3.6M addresses in 34K prefixes in the overall
20150206 dataset.  That is, as one might expect, greater than
one second latencies do seem to be a property of the
networks associated with selected prefixes.  The
\countResponsives{} addresses that responded are from only
\countResponsiveSlashTwentyFours{} prefixes.  We show the percentage
of responsive addresses within each prefix that dropped from the 
initial ping to the maximum of the rest in Figure~\ref{fig:prefix_percentage_cdf}.  Several
prefixes did not have an initial latency greater than the maximum; these typically had very few responsive addresses.  In other
prefixes, most addresses showed a reduction.
Finally, the
\countDropsToMedianAndToMax{} that showed a reduction from the
initial ping are from only \countDroppingSlashTwentyFours{}
prefixes.  Of the 161 prefixes that had only one address with 
above one-second median latency, only 39 showed a reduced from the
initial RTT to the maximum of the rest.  Taken together, we believe
this distribution of addresses across relatively few prefixes indicates
that the wake-up behavior is associated with some providers but not
restricted to them.

% /24's:   34395
% addrs: 3636156
