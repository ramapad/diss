%% 
%% * could it be routing? (if affecting all of a /24 at the same time;  I don't believe this happens)
%% 
%% * could it be handoff?  (should be occasional / rare (though not necessarily so), within some networks but not all, and apply to networks with already high latencies)
%% 
%% * could it be solar interference on satellites?  (seek diurnal patterns s.t. for one (15 minute?) period out of the 24 hours, there's a problem)
%% 
%% * does it affect thought-to-be-wired links?  (seek large cable-provider networks and fiber providers, determine whether there are high latency guys in there.)
%% 
%% * can the devices be fingerprinted? (e.g., nmap)

\section{Why do pings take so long?}
\label{sec:causes}

In this section, we aim to determine what causes high RTTs. We
investigate the RTTs of satellite
links and find that they account for a small fraction of high RTT
addresses. We follow up with an analysis of Autonomous Systems and
geographic locations that are most prone to two potentially different
types of high RTTs: RTTs greater than 1s and RTTs greater than
100s. We then investigate addresses that exhibit each type of RTT and
find potential explanations.
% check if the first ping to an address in a continuous stream of pings is subject to
% higher latencies than the rest. Finally, we investigate particularly high
% latencies and identify patterns associated with their occurrence.

\ninput{timeouts/satellites}

\ninput{timeouts/zmap_asn_analysis}

\ninput{timeouts/first_ping}

\ninput{timeouts/handoff}

\subsection{Summary}
High latencies appear to be a property mainly of cellular
  Autonomous Systems, though a few also appear on satellite
  links. Latencies in the ISI data that are regularly above one second seem to be caused
by the first-ping behavior associated with several addresses, where
the first ping in a stream of pings has higher latency than the
rest. Egregiously high latencies, i.e., latencies greater than a
hundred seconds, occur in two broad patterns. In the first, latencies
steadily decay with each probe, as if clearing a backlog. In the second, latencies are continuously
high and are accompanied by loss, as if the network link is oversubscribed.
