\begin{figure*}
  \begin{center}
  \includegraphics[width=0.99\textwidth]{timeouts/figs/pctile_var_over_time}
  \end{center}
  \caption{\label{fig:pctile_var_over_time}Top: Minimum timeout
    required to capture the $c^{th}$ percentile latency
    sample from the $c^{th}$ percentile address in each
    survey, organized by time.  Each point represents the
    timeout required to capture, e.g., 95\% of the responses
    from 95\% of the addresses.  The 1\% line is indicative
    of the minimum.  Bottom: Response rate for each survey;
    symbols represent which vantage point was used.  Surveys
    from Japan with very few successes are not plotted on
    the top graph.}
\end{figure*}


\subsection{Is it a particular survey or vantage point?}

ISI survey data are collected from four vantage points at
different times.  Vantage points are identified by initial
letter, and are in Marina del Rey,
California, ``w''; Ft. Collins, Colorado, ``c''; Fujisawa-shi, Kanagawa,
Japan, ``j''; and Athens, Greece, ``g''.  

In this section, we look at summary metrics of each of the
surveys.  In Figure~\ref{fig:pctile_var_over_time}, our
intent was to ensure that the results were consistent from
one survey to the next, but we found a surprising result as
well.  The consistency of values is apparent: the median
ping from the median address remains near 200ms for the
duration.  However, there are exceptions in the following
data sets: IT59j (20140515), IT60j (20140723), IT61j
(20141002), IT62g (20141210). 
These
higher sampled latencies are coincident with a substantial
reduction in the fraction of responses that are matched: in
typical ISI surveys, 20\% of pings receive a response; in
these, between 0.02\% and 0.2\% see a response.  It appears
that these data sets should not be considered further.
Additionally, it54c (20130524) it54j (20130618) and it54w (20130430)
were flagged by ISI as having high latency variation due to 
a software error~\cite{isi-notes-54}.

Ignoring the outliers, trends are apparent.  The timeout
necessary to capture 95\% of responses from 95\% of
addresses increased from near two seconds in 2007 to near
five seconds in 2011.  (We note that the apparent stability
of this line may be misleading; since the $y$-axis is a log
scale and our latency estimates are only precise to integer
seconds when greater than 3, small variations will be lost.)
The 98th percentile latency from the 98th percentile address
has increased steadily since 2011, and the 99th increased
from a modest 20 seconds in 2011 to a surprising 140 in
2013.  These latency observations are not isolated to
individual traces.

In sum, high latency is increasing, and although some surveys
show atypical statistics, early 2015 datasets that we focus on
appear typical of expected performance.






