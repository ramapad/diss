
\chapter{Mitigating false inferences due to dynamic addressing}

\label{cpt:addr_change}

% TODO: Index section numbers to the roadmap

In this chapter, I begin by describing how dynamic addressing can lead
probing-based outage detection techniques to make false inferences
about outages. 

Next, I describe work with colleagues to empirically
measure the frequency of dynamic addressing and the durations for
which addresses are assigned to residential home router devices in
several networks around the world and the effect of outages upon
dynamic reassignment~\cite{addrchange-reasons}. The measurements we
used are sourced from the RIPE NCC's Atlas project, which deploys small devices, called probes, that
conduct measurements from globally distributed
networks~\cite{atlas}. The RIPE Atlas dataset offers measurements that allow us to
determine when an IP address change occurred and what the addresses
were before and after the change. In addition, the dataset includes many
measurements that provide context about what was happening around the
time of the address change. I was able to use these measurements to
detect when RIPE Atlas probes rebooted and were not sending pings
(indicating a power outage) and when their pings were not getting
responses (indicating a network outage). In a study with colleagues of active RIPE
Atlas probes in 2015, we found 3,038 RIPE Atlas probes with address
changes hosted across 929 ISPs and 156
countries~\cite{addrchange-reasons}. Using the measurements from RIPE Atlas, I identify networks
where addresses are typically stable. 

Finally, I discuss a technique to identify outages even in networks
where dynamic reassignment is common. Using a complementary dataset
that allows checking if an address for which a probing-based technique
detected an outage has remained the same before and after the detected
outage, we are able to confirm outages even in networks where dynamic
reassignment is common.

\section{IP addresses can be proxies for end users}

Academia and industry often rely on a simplifying assumption that IP
addresses uniquely identify
end-hosts~\cite{p2pfilesharing,p2pavail,sen2004analyzing,sekar2006multi,anomalousdns,kuhrer2015going,xie2005worm,jung2004empirical,fabian2007botnet,stone2009your,andriesse2015reliable,fail2ban,spamhaus,cbl,sorbs}. This
assumption allows researchers to track end host behavior over
time~\cite{anomalousdns, kuhrer2015going, pingin}, or to count
participating users in peer-to-peer
systems~\cite{p2pfilesharing,p2pavail,sen2004analyzing}. Many
organizations create blacklists of suspicious IP addresses based on
previously observed malicious traffic associated with those
addresses~\cite{fail2ban,spamhaus,cbl,sorbs}. 

Probing-based techniques like Thunderping~\cite{pingin} make a similar
assumption: a probed address is representative of a residential
customer's Internet connection. Many residences have at least one
device with a public IP address~\cite{cgn-imc16}, typically the home
router. When a home router's address stops responding
to pings, it could be evidence of a residential Internet outage.

All of these applications would benefit from understanding how often
and when dynamic addresses assigned to user devices change.

\section{Probing-based techniques can make false outage inferences due
to dynamic addressing}

\label{sec:addrchange-false-inferences}

When probing-based remote outage detection techniques send probes to an address, they expect that the
address continues to be assigned to the same end-host for the entirety
of the probing duration. Depending upon how a dynamic address gets
reassigned, these techniques can make false inferences about outages in two ways:

\begin{itemize}

\item{\emph{Detecting false outages} Probing-based remote outage detection techniques detect outages
    when a previously responsive address stops responding to
    probes. However, If a dynamic address being probed is
withdrawn from its host and is not assigned to any other host, active probes to the address will no longer
elicit responses. These techniques will infer false
probe-loss, leading them to infer false outages.}

\item{\emph{Detecting false outage duration} These techniques detect outage
    duration by continuing to probe an unresponsive address. When the
    address starts responding to probes again, the outage is inferred
    to end. If a home router with a
    public dynamic address has an outage and at some point during the outage,
    the dynamic address is reassigned to some other home router
    which responds to probes, probing-based remote outage detection techniques would infer that the outage ended incorrectly.
% For ISPs that use DHCP
%     for address assignment, we would expect dynamically
%     assigned addresses to stick around on the end-host until an
%     outage occurs. However, upon the occurrence of the outage, if the
%     outage is long enough, they can get reassigned to another host,
%     especially, if the outage is longer than the DHCP lease
%     duration. Here, RODWAP techniques can detect the outage itself but
%     can perhaps not detect outages that are long
}

% TODO: CITE http://www.umiacs.umd.edu/~tdumitra/courses/ENEE757/Fall15/papers/Stone-Gross09.pdf
\end{itemize}

My approach to mitigating these false inferences is to analyze how
frequently and for what reasons dynamic addresses are reassigned, in
various networks. Using the results of these analysis, I identify
networks where addresses are typically stable. 

% TODO: Think about whether I want the following text in the diss.

% I will use the results of these analyses to build a model
% of how likely an inference about an outage using a probing-based
% remote outage detection technique is
% a false inference caused by dynamic addressing. For example,
% preliminary work with colleagues has revealed that some European ISPs change addresses upon
% very small outages and are particularly likely to change addresses at certain
% times of the day~\cite{addrchange-reasons}. These results will inform
% my model to not attempt detection of 
% outage duration for these ISPs, and to discard outages
% detected at times that are particularly likely to have dynamic address
% changes. The model will ultimately yield stable addresses who either
% do not undergo dynamic reassignment for months at a time, or who get
% reassigned but in a predictable manner. Thunderping limits itself to
% probing addresses in the U.S. where dynamic reassignment is
% uncommon~\cite{addrchange-reasons}; stable addresses from the
% model can help us detect outages in new areas.




% The results from the dynamic DNS dataset are preliminary in
% scale and based on a short measurement to show potential.
% Further, our study of RIPE Atlas data showed us that the cause of
% address changes is important. I intend to couple our
% outage-detection tool to probe addresses corresponding to
% the dynamic DNS domains while fetching their A records.  We
% can thus identify outages that occur near the reassignment,
% allowing us to infer if an address-change was caused by an outage and
% feed results into the model. Further, if the dynamic DNS result
% indicates that a probed address had recently been reassigned, then the
% detected false positive outage can be filtered.
 
% \subsubsection{Open DNS resolvers}

% Since 2010, various studies have reported on the existence of more
% than 15 million 'open' DNS resolvers on the
% Internet~\cite{openresolver, schomp2014clientsidedns, kuhrer2014exit,
%   kuhrer2015going}. These DNS resolvers are 'open' because they will resolve a DNS query sent from arbitrary IP
% addresses on the Internet. Previous studies have found that more than
% three-quarters of open DNS resolvers are likely to be
% residential~\cite{schomp2014dnsvul, schomp2014clientsidedns}. I
% propose two potential approaches to fingerprint these open DNS
% resolvers and track address changes.

% \paragraph{DNS caches}
% Open DNS resolvers often cache previous
% lookups~\cite{schomp2014dnsvul}. My insight is that these caches can
% be used to fingerprint open DNS resolvers, allowing us to track when
% their IP addresses change. I plan to do this in two phases.

% First, I will find open DNS resolvers on the Internet. I propose to register a domain and deploy an Authoritative DNS server
% for it. Then I intend to perform a one-time scan of the entire IPv4
% address space by sending a DNS request for a subdomain within the
% domain we control to all the IPv4 addresses on the Internet. Each DNS
% request I send to a target IP address will embed the target address
% into the request, similar to the approach used by Dagon et
% al.~\cite{dagon2008corrupted}. The Open DNS resolvers will route the
% request to our Authoritative DNS server.  At the authoritative DNS
% server, I will note the target IP address to which this request was
% sent and generate a unique fingerprint for the device at this address,
% and embed this fingerprint in my response. When these responses
% reach the open DNS resolvers, each will now contain its unique
% fingerprint in its cache.

% Next, I will periodically inspect the caches of known open DNS resolvers.
% I will issue periodic DNS requests for the subdomain we
% control (with the target IP address embedded in the request) to all
% the addresses that contacted our Authoritative DNS server. If we
% obtain the fingerprint that we had previously issued to that address,
% we know that the device continues to be assigned that address. If we
% find that an address is no longer returning the expected cache
% fingerprint, we know that the address has changed. I then propose to
% issue DNS requests to related addresses (as described in
% Section~\ref{sec:last_mile}) with the \emph{old} target IP address
% embedded in the request. If the device is present on any of those
% addresses, then we will obtain the expected fingerprint. Upon finding
% the device at a new address, we will update
% our local mapping and note that the fingerprint is now available at
% this new address.

% \paragraph{Anomalous Open DNS Resolvers}

% Of the 30 million Open DNS Resolvers on the Internet, around 17
% million are \emph{anomalous}~\cite{anomalousdns}, i.e.,
% instead of sending DNS responses with a source port of 53, they
% respond with a non-standard source port. Kaizer et al. ~\cite{anomalousdns} found that
% these devices are primarily residential ADSL modems. Not only do these
% devices use a non-standard source port, DNS requests can be made to
% these devices in such a way that the source ports are assigned
% \emph{sequentially}. My insight is that we can use this sequential
% assignment of source ports to fingerprint anomalous open DNS resolvers.

% The first part of our approach here is similar to my approach with
% the DNS caches: I will find open DNS resolvers that are anomalous. After registering a domain and deploying an Authoritative DNS server
% for it, I will perform a one-time scan of the entire IPv4
% address space by sending a DNS request for a subdomain within the
% domain we control to all the IPv4 addresses on the Internet. Each DNS
% request I send to a target IP address will embed the target address
% into the request as before. However, instead of embedding responses with
% unique fingerprints from the authoritative DNS server, we simply
% monitor the source ports that issue DNS responses from each
% address. If it's a non-standard port, we flag the device as an
% anomalous open DNS resolver.

% Next, I will periodically inspect the source ports used by anomalous
% open DNS resolver responses. Since we know which
%   addresses the anomalous open DNS resolvers are located at, I
%   periodically issue DNS queries to these addresses. As long as the
%   source port for successive requests to an address continues to be
%   sequential, I can state with high confidence that the address has
%   not changed. The source ports for these devices typically vary
%   between 10,000 to 30,000; thus there is only a small likelihood that
%   another device coincidentally happens to have the next value in
%   sequence. If we find that a response doesn't arrive, or that one
%   arrives but the source port is not sequential, then we know that the
%   device's address has been reassigned. As in the DNS cache approach,
%   I will then look for the expected source port in DNS responses from
%   requests sent to related addresses to find the device again.


% \subsection{Confirming that detected outages are accurate}

% After mitigating false positive outages, I propose the use of datasets
% from RIPE Atlas probes~\cite{atlas} as ground truth to confirm that the remaining
% outages are indeed true positives. In previous work, I had inferred outages occurring on
% RIPE Atlas probes by looking for gaps when probes did not perform 
% measurements that they were scheduled to~\cite{addrchange-reasons}. By
% probing IP addresses at which RIPE Atlas probes are also deployed, I
% will compare outages we detect against outages inferred from RIPE
% Atlas datasets and validate whether our detected outages are accurate.