
\section{Conclusions}

Using a seven year dataset collected by probing residential IP addresses in the U.S., I showed that a variety of weather conditions can inflate the likelihood of Internet dropouts. I quantified this inflation and show that it varies depending upon the type of weather, link type, and geographic location. 
% We also showed that the time to recover from a dropout increases during weather events.

Even ignoring times when hurricanes were active, all link types see more failures during thunderstorms---fiber addresses, the most resilient to thunderstorms still observed an additional dropout every 11 days, while satellite addresses, the most susceptible, observed an additional dropout every day. High wind speeds result in a super-linear increase in dropout probability for wired links while higher precipitation results in particularly pronounced increases in dropout probability for wireless links.

The extent to which weather conditions can inflate the probability of dropouts varies considerably with geography. States in the Midwest are susceptible to dropouts during rain while states in the south experience dropouts much more often in the snow: addresses in Mississippi, for example, experience an additional dropout every 4 days. 

The reliability analyses in this chapter were performed using dropouts of individual IP addresses. Although dynamic addressing and user behavior also constitute dropouts, the key observation that allowed the use of dropouts for reliability comparisons is that the inflation in dropout rates during the occurrence of severe weather conditions is due to the additional outages that occur. Confounding factors such as dynamic addressing and user behavior do not positively correlate with peak diurnal failure periods; therefore, the increase in dropout rate during a weather condition is equivalent to the increase in outage rate during that condition. 