
%TODO: Find out what's different about ns-proposal used in reliability proposal
% \documentclass[letterpaper,10pt]{article}
\documentclass[10pt,onecolumn]{article}
\usepackage[letterpaper, margin=1.25in]{geometry}

\PassOptionsToPackage{hyphens}{url}
\usepackage[hyphens]{url}
\usepackage{hyperref}
\hypersetup{breaklinks=true}
\urlstyle{same}
\usepackage{balance}

% TODO: I seem to have used most of these from the reliability
% proposal. Find out what the role of each command is.
\usepackage{ifpdf,floatrow} 

% nspring 10.35pt palatino
\renewcommand*{\rmdefault}{ppl} % Roman default
\usepackage{fix-cm}
% \renewcommand{\normalsize}{\fontsize{10.35pt}{12.5pt}\selectfont}
% \renewcommand{\normalsize}{\fontsize{10pt}{12.5pt}\selectfont}
% endspring

%% ns - replacement for times that lacks stupidity.
% \usepackage{mathptmx}
% \usepackage[scaled=.90]{helvet}
% \usepackage{courier}
% 
\usepackage[override]{cmtt} % make tt font tighter / less ugly

% \usepackage{makeidx}  % allows for indexgeneration

\usepackage[square,comma,numbers,sort&compress]{natbib}

\usepackage{color}
\usepackage[table]{xcolor}
\definecolor{orange}{RGB}{255,127,0}

\newcommand{\rama}[1]{{\color{red}[\todo{rama: #1}]}}
\newcommand{\ns}[1]{{\color{green}[ns: #1]}}
% \newcommand{\ignore}[1]{}

% \setlength{\textwidth}{6.5in}
% \setlength{\textheight}{9in}
% \setlength{\topmargin}{0in}
% \setlength{\headheight}{0in}
% \setlength\columnsep{.30in}
% \setlength{\headsep}{0in}
% \setlength{\oddsidemargin}{0pt}
% \setlength{\evensidemargin}{0pt}
\setlength{\textfloatsep}{.1in plus 0.05in}
\setlength{\itemsep}{-10pt}


\usepackage{enumitem}
\setlist[itemize]{leftmargin=*}
% \usepackage{slashbox}
\usepackage{graphicx}
\usepackage{amsmath,amssymb}
\usepackage{verbatimbox}
\usepackage{boxedminipage}
\usepackage{multirow}
% \usepackage{subfigure}
\usepackage{xspace}
\usepackage[]{pdfpages}

\usepackage[utf8x]{inputenc}
\usepackage{ucs}

%boxed,vlined,,linesnumbered,commentsnumbered
\usepackage[vlined]{algorithm2e}
\providecommand{\SetAlgoLined}{\SetLine}
\providecommand{\DontPrintSemicolon}{\dontprintsemicolon}

% \renewcommand\thesubfigure{(\alph{subfigure})}

%\usepackage{algorithm}
%\usepackage[noend]{algpseudocode}

\hyphenation{map-orphans}

% \input{commands} % TODO: Find out what this does!

% Find out how \ninput is different from \input

% \title{Thesis Proposal: Detecting outages per IP address remotely}
\title{Probing-based measurement of last-mile Internet reliability}
% \title{Measuring Internet reliability remotely for
%   individual users with probing-based techniques}
% \title{Thesis Proposal: Remote probing-based outage detection for
%   individual IP addresses}

\author{Ramakrishna Padmanabhan}

\begin{document}

% Define table specific commands
\newcommand{\bb}{~~~~~}
\newcommand{\hdr}[1]{\multicolumn{1}{c}{\textbf{#1}}}
% define ``struts'', as suggested by Claudio Beccari in
%    a piece in TeX and TUG News, Vol. 2, 1993.
\newcommand\Tstrut{\rule{0pt}{2.2ex}}         % = `top' strut
\newcommand\Bstrut{\rule[-0.9ex]{0pt}{0pt}}   % = `bottom' strut

\maketitle


\begin{abstract}

% We use the Internet without know that the Internet is actually super reliable. It's based more upon: hey, the Internet seems reliable. 

% The Internet is used today for communication

% Detection of Internet outages, which are potentially rare events, demands broad and longitudinal measurements of users' Internet connections.
% Internet reliability is increasingly important as the applications we use increasingly depend upon the Internet.
% Internet reliability is increasingly important as a variety of services that we use migrate to the Internet.
% Internet reliability is increasingly important as the applications that we depend upon migrate to the Internet.
Internet reliability for home users is increasingly important as a variety of services that we use migrate to the Internet. Yet, we lack authoritative measures of residential Internet reliability. Measuring reliability requires the detection of Internet outage events experienced by home users. But residential Internet outages are rare events. Further, they can vary in the number of affected users: a large power failure can result in thousands of users' Internet outages whereas a fault in the cable connecting a user's home router to the ISP may affect only a single user's connection. Thus, detecting residential Internet outages requires broad and longitudinal measurements of individual users' Internet connections. However, such measurements of Internet reliability are challenging to obtain accurately and at scale. % The second step is to use detected outages to reason about Internet reliability across different dimensions such as ISPs, media-types, geographical areas, and challenging environments.

Probing-based remote outage detection techniques can scale but their accuracy is questionable. These techniques detect Internet outages across time as well as across the IPv4 address space by sending active probes, such as pings and traceroutes, to users' IP addresses and use probe responses to infer Internet connectivity. However, they can infer false outages since their foundational assumption can sometimes be invalid: that the lack of response to an active probe is indicative of failure. In this dissertation, I defend the following thesis: \emph{It is possible to remotely and accurately detect substantial outages experienced by any device with a stable public IP address that typically responds to active probes and use these outages to compare reliability across ISPs, media-types, geographical areas, and weather conditions}.

In the first part of the dissertation, I address the inaccuracy of probing-based techniques' detected outages and show how to use probe responses to correctly detect outages. I illustrate two potential scenarios where the foundational assumption of probing-based techniques is invalid. In the first scenario, responses are delayed beyond the prober's timeout, leading these techniques to infer packet-loss instead of delay. In the second scenario, these techniques can falsely infer packet-loss when the address they are probing gets dynamically reassigned. I examine how often delayed responses and dynamic reassignment occur across ISPs to quantify the inaccuracy of these techniques. I show how outages can be inferred correctly even in networks with dynamic reassignment using complementary datasets that can reveal whether an address was dynamically reassigned before, during, and after a detected outage for that address.

% Prior work has focused upon detecting edge Internet outages for larger aggregates of addresses, such as those belonging to the same country, BGP prefix, or /24 address block. 

% In the second part of the dissertation, I demonstrate the need for probing-based techniques to measure individual addresses' outages; otherwise, they risk missing outages. Although individual outages are not direct measures of reliability---they can occur independently because users disable equipment---individual outages are still rare events. My approach is to use simultaneous outages of individual addresses that are related to each other to identify a \emph{dependent} failure event. I show that these failure events do not knock out the address block containing the addresses in the majority of times; current techniques that detect outages at the address-block level would miss these events.

% In the final part of the dissertation, I show how to study the effect of challenging environments upon the reliability of a group of addresses by analyzing the inflation in outage rate for that group during its presence. I use the insight that the statistical change in outage rate in different challenging environments (e.g. thunderstorm) can quantitatively expose actual outage “inflation”. 

In the second part of the dissertation, I motivate why the detection of individual addresses' outages is required for analyzing residential reliability. An individual address typically represents one residential customer; therefore, detecting outages for individual addresses can allow capturing even small outages. Prior probing-based techniques focus upon the detection of outages affecting entire BGP prefixes or /24 address blocks. Here, I quantitavely demonstrate the extent to which prior techniques can miss residential outages. I show that even individual address outages occur rarely in most networks. When multiple simultaneous outages of related individual addresses occur, there is likely a common underlying cause. With this insight, I develop and evaluate an approach to find outage events that are unlikely to have occurred independently. I show that the majority of such dependent events do not affect entire /24 address blocks or BGP prefixes, and are therefore not likely to be detected by existing techniques which look for outages at these granularities. 

In the final part of the dissertation, I show how to use individual addresses' outages detected by probing-based techniques to assess Internet reliability across media-types, geographical areas, and weather conditions. Individual outages are not direct measures of reliability: they can occur independently because users disable equipment or can be observed falsely due to dynamic address renumbering. I use the insight that the statistical change in outage rate in different challenging environments (e.g. thunderstorm) can quantitatively expose actual outage “inflation”. I show how to study the effect of challenging environments upon the reliability of a group of addresses by analyzing the inflation in outage rate for that group during its presence.

This dissertation's contributions will help achieve comprehensive measurements of Internet reliability that can be used to identify vulnerable networks and their challenges, inform which enhancements can help networks improve reliability, and evaluate the efficacy of deployed enhancements over time.

% comprehensive datasets of Internet reliability that can aid in improving reliability by targeting problem regions or adapting strategies that have proven to be successful
\end{abstract}


% \section{Preliminaries}

\section{Introduction}


% With online accessibility of
% devices ranging from temperature control systems to baby monitors in
% this Internet of Things era, our dependence upon the Internet for
% increasingly many facets of our daily lives only continues to
% rise.

% With the ability to access a
% variety of devices online in this Internet of Things era, ranging from
% temperature control systems to baby monitors, our dependence upon the
% Internet for increasingly many facets of our daily lives only
% continues to rise. 

% As the range of Internet services that we rely upon increases, so does our reliance upon
% the Internet. 

% Talk more about how important the Internet is?

Residential Internet reliability is increasingly important as a variety of
services that we use migrate to the Internet. Internet users today can
communicate with each other, perform financial transactions, plan
their travel, and even obtain critical services such as health
monitoring~\cite{ideal-life, remote-health-elderly} and emergency
services~\cite{emergency-voip-voipfone, emergency-voip-fcc} from their
homes. Our dependence upon the Internet will rise further as more of
our home devices become connected in this Internet of Things
era. Consequently, continuous availability of the Internet and
resilience is vital, and the reliability of the Internet is of
interest to stakeholders across the board, from government regulators and
Internet Service Providers, to users.

% TODO: How do I connect the next sentence neatly to the previous
% paragraphs? Especially when the previous paragraphs will go on to
% talk about all the interested parties at length.

% TODO: Look into billionts for citations about regulator interest.

Broad and longitudinal
measurements of users' Internet reliability in different circumstances
can identify vulnerable networks and their challenges, can inform
which enhancements can help these networks improve reliability, and
can evaluate the efficacy of deployed enhancements.

% Measuring
% Internet reliability at the individual user level will therefore prove invaluable in assessing
% the current state of Internet connectivity and in developing future
% enhancements.

% By detecting outage events and their duration, we
% can reason about a particular user's Internet reliability and compare
% it to other users' reliability across geography, ISPs, and
% media-types.

% Talk about how Internet reliability is difficult and who is
% interested. FCC. Even ISPs (Cite the "Nevermind" paper by Nick
% Duffield which claims that ISPs are typically reactive and wait for
% customers to call). Common users.

Yet, we lack authoritative measures of last-mile Internet reliability. The first step towards measuring users' Internet
reliability is to detect \emph{Internet outages}---events that prevent
users from communicating over the Internet. Since we expect outage
events to be rare, detecting them requires broad and longitudinal
measurements of individual users. However, such measurements of
Internet reliability at the individual user level are challenging to
obtain accurately and at scale. The next step towards measuring users'
reliability is to segregate detected outages into categories
that suggest their cause. Once outages are categorized in this
manner, we can estimate the reliability of an ISP by considering the
subset of outages that affected solely that ISP.

Existing techniques that measure individual-user-level outages can be
grouped into on-premises outage detection techniques and remote probing-based
outage detection techinques. On-premises techniques, such as
RIPE Atlas~\cite{atlas}, SamKnows~\cite{samknows}, and
BISmark~\cite{bismark-main-bib}, measure diverse aspects of
users' Internet connections, but
measure relatively few users. These techniques 
deploy dedicated hardware at user premises that continuously conduct ping,
traceroute, DNS measurements etc.; some of these
measurements can be used to infer Internet connectivity problems. Whereas hardware
based techniques have fundamental scaling difficulties owing to
manufacturing and deployment costs, hundreds of millions of
IP addresses respond to active probes~\cite{timeouts}. Since many
residences have at least one device with a public IP address ~\cite{cgn-imc16}
(typically the home router), these IP addresses can be probed
remotely, from 
vantage points that we control, to measure their connectivity. Thunderping~\cite{pingin} and Trinocular~\cite{trinocular} adopt this
approach to outage detection, taking a
complementary approach to on-premises techniques: they focus upon
measuring only connectivity but do so for many users. Since these
techniques can send probes remotely from servers under their control, without requiring any user
involvement, they are able to detect outages across time
as well as across the IPv4 address space.
% However, existing techniques do
% not study latency and therefore do not identify performance
% degradation resulting from high delay.
However, probing-based remote outage detection techniques can
make false inferences about outages when some scenarios
occur~\cite{timeouts, addrchange-reasons}. Further, existing
techniques have not 
attempted to categorize detected outages by their likely cause.
% However, measuring residential
% outages is challenging because of the scale: there are millions of
% residential links to measure. 
% Second, users can voluntarily power
% down their home Internet equipment and it is challenging to
% distinguish between voluntary user shutdowns and an outage at the last-mile link.
 
% ; even multihomed last mile links for business connectivity
% often share the same upstream hardware, representing a single point of 
% failure~\cite{twcable-business-web}. 
% Last-mile links lack the redundancy of the Internet's core;
% thus an outage of the last-mile link is likely to cut off users'
% Internet connectivity altogether.

% The second challenge is that it can be hard to distinguish
% between an outage at the last-mile link and voluntary user shutdowns
% of their Internet connections.
% We do not have a good understanding of the reliabity of Internet
% connectivity for end-users. Understanding the reliability of Internet
% connectivity for end-users requires understanding the reliability of
% the \emph{last mile link} connecting an end-user to the Internet. This
% is because the core of the Internet is designed to be redundant but
% last mile links typically are not.
%TODO: Perhaps borrow sentence from enduser.tex about business last-mile links.


% Thesis statement comments

% Older versions of thesis statement:
% \emph{For any end-host with a publicly assigned IP address that has the ability to respond to active probes, it is possible to remotely isolate accurately determine connectivity problems experienced by that end-host's last mile link.}
% \emph{For any end-host with an IP address that has the ability to
% respond to active probes, it is possible to remotely detect outages
% experienced by that end-host's last mile link.}

% Bobby's comments: Send traceroutes, but also to related addresses. What happens when all addresses change en-masse? Can we somehow identify such instances?
% Neil suggested that I should replace end-host with 'Internet accessible device'. Can I assert that a device with a public IP address is by definition Internet accessible.
% \emph{It is possible to remotely and accurately detect outages experienced by any device with a public IP address that typically responds to active probes.}

% Come up with definition of accuracy of outages.

% and use it to compare reliability across ISPs, media-types and
% geographical areas

I argue in this thesis that 
\emph{It is possible to remotely and accurately detect substantial outages
  experienced by any device with a stable public IP address that typically
  responds to active probes and use these outages to compare
  reliability across ISPs, media-types and geographical areas.} To demonstrate the thesis, I make the following
initial contributions:

% TODO: 
% It is possible to remotely and accurately estimate the reliability of
% an ISP's customer's device so long as the device has
% a stable public IP address that typically
%   responds to active probes


In Section~\ref{sec:related}, I define terms in the thesis statement, place the problem of outage detection
at the individual user level in the context of related work, and describe the challenges
that probing-based remote outage detection techniques will need to address. These
techniques study outages by sending active probes (such as ping's echo
requests) and use probe responses to infer outages. They assume that a
response to an active probe indicates a working path to the probed
user device and that lack of response is indicative of failure. I
illustrate two scenarios where this assumption can be
invalid, leading to potentially false outage inferences.
 % I
% illustrate two potential scenarios where this assumption is
% invalid---when responses are delayed beyond the prober's timeout and
% when the probed address is dynamically reassigned. I also highlight why
% Internet outages voluntarily caused by the
% residential user need to be segregated.
% I also describe the challenge that remote probing-based techniques face
% in detecting outages specifically in the last-mile.

In Section~\ref{sec:timeouts}, I investigate the prevalence of delayed
probe responses due to early timeout. The
lack of response to an active probe isn't always indicative of loss;
for example, when
responses are delayed beyond the prober's timeout, the response
eventually arrives but the prober would never see the response because
it timed out too early. I report how commonly responses are delayed
beyond timeouts in
networks around the world and propose techniques to mitigate this
problem. 
% Decouple probe retransmission
% and loss. Possibly identify that only cellular guys have long delays?
% Also, this will ensure that we trigger retransmission upon high
% latency/loss and actually identify loss as loss and high latency as
% high latency.

In Section~\ref{sec:addr_change}, I investigate how dynamic addressing can
lead remote probing-based outage detection techniques to make false inferences about outages and techniques to
mitigate these false inferences. My approach to mitigating
these false inferences is rooted in building a model that characterizes
the probability that a dynamic address is reassigned at any point of time. I describe preliminary
work which shows the feasibility of building such a model and detail
proposed work to gather new datasets that can feed into the
model. I will ultimately use this model to find candidate stable Internet
addresses for probing.

In Section~\ref{sec:last_mile}, I discuss an approach to determine
which of the detected outages are consistent with the failure of an
ISP's operated equipment, and which outages can be attributed to other
causes such as power outages or users voluntarily shutting down their home
Internet equipment. My approach is to probe addresses that are related
to the address that is already being probed. I propose the use of multiple criteria
to find related addresses such as geography, ISP, and network
topology. I describe how we can then use simultaneous outages of these
related addresses (or the lack thereof) to categorize outages and
estimate Intrenet reliability along various dimensions.

% We show in this proposal that confounding factors can cause RODWAP
% techniques to sometimes
% infer false outages. 

% However, we argue that RODWAP techniques can
% be used for accurate outage detection by identifying and mitigating
% confounding factors. 
 % This has
% been the basis of existing active probe based techniques that detect
% loss, and thereby outage events, such as Thunderping~\cite{pingin} and
% Trinocular~\cite{quan2013trinocular}.

% In spite of , we argue that it is possible to remotely detect
% outages on the last mile link using active probes for any end-host
% with an IP address that responds to active probes.

% We investigate potential causes that
% would lead existing active probe based outage detection approaches to
% falsely infer loss and describe proposed work to mitigate detection of
% false loss in Section~\ref{sec:timeouts} and
% Section~\ref{sec:addr_change}. We also propose 

% We show in this proposal that probe-loss need not always be due to
% lossy links.


% \begin{itemize}
%   \item{\bf{False probe-loss inference due to early timeout:}} 
%     Traditionally, active probe based approaches time out probes after a few seconds. Responses that arrive after the timeout will be reported as lost. When this happens, existing techniques would confuse high delay with probe-loss.
%   \item{\bf{False probe-loss inference due to IP address change:}}
%     Consider an IP address that was previously responsive. If the host to whom that IP address was assigned changed its IP address as a result of dynamic addressing or mobility, and if the probed IP address is not reassigned to any host, then echo responses will cease to arrive and existing techniques would infer false probe-loss.
% \end{itemize}

% After analyzing causes of false probe-loss, we will investigate techniques to remove false probe-loss. Once we remove false probe-loss, we will be left with all instances of true probe-loss and probe-delays. This will give us the ability to identify that \emph{some} link on the path to the destination is experiencing connectivity problems. However, since we are specifically interested in identifying connectivity problems on the last-mile link we require another step. In this step, we will conduct TTL-limited probing from multiple vantage points to identify which link exhibits connectivity problems.


\section{Related Work}

Previous work
studied the performance of DHCP in small campus
networks~\cite{dhcpwatch, dhcp-gatech} and
settings where smartphone usage is
widespread~\cite{dhcp-smartphones} and developed techniques to
reduce network address utilization and DHCP broadcast traffic. The goal of those studies was
to improve the performance of DHCP by tuning configuration.

Conceptually, so long as there is some uniquely identifying feature
that remains constant across a host's address change,
it is possible to track IP address changes over time for that host.
Several studies have used this broad
method~\cite{udmap,census-survey, zmap-dhcp,
maier2009dominant, dhcp-gatech, peering-shroud, dhcp-dimes}. 
UDmap~\cite{udmap} studied
dynamic address properties using Hotmail user login
traces where the user's login serves as the identifying
feature. Casado et al.~\cite{peering-shroud} tracked clients using HTTP cookies when
clients access a CDN. 
Other studies~\cite{maier2009dominant,census-survey} used continuous
responsiveness of an address itself as the identifying feature, assuming
that an address that responds continuously belongs to the same user and that
when an address stops responding to pings, it has been reassigned.

While we share the same goal as these studies, our approach diverges
in that we are interested in the events associated with an address
change.
Previous studies lacked
access to end-host information that could reveal the cause of an
address change. One exception, Maier et al.~\cite{maier2009dominant},
used access to the Radius server of a European DSL provider from one
urban area to
identify why DSL sessions terminated, and noted that the DSL
provider often limited Radius session length to 24 hours in that area.
We extend this result to several ISPs in countries from Europe, Asia,
and South America, and identify other typical session length limits.
%
Argon et al.~\cite{dhcp-dimes} used periodic measurements
from end-hosts in the DIMES infrastructure~\cite{netdimes}.
DIMES software installed on an end-user computer is
different from RIPE Atlas hardware probes primarily in that it reports back
only every 30-60 minutes (as opposed to RIPE Atlas's 3 minutes),
the agent can be installed on
laptops that move (as opposed to RIPE Atlas probes that could
move, but do not), the hosts running DIMES are often
powered down (resulting in limited uptime), and DIMES hosts
appear to have static IP addresses more often (they reported 60\% had
only one address).  Nevertheless,
Argon et al. observed that some small ISPs exhibited 
address alternation with a 24 hour periodicity. In IPv6, the RFC for
privacy extensions for stateless address autoconfiguration recommends
that IPv6 addresses be changed every 24 hours~\cite{rfc4941} and
empirical results by Plonka and Berger found that more than 90\% of
client IPv6 addresses were ephemeral~\cite{akamai-v6addr-usage}. We
showed that 24 hour defaults are not uncommon in IPv4 as well.

These studies relied on relatively uncontrolled observations of
the address assigned to a device or user, both in terms of whether
the devices are active, whether the users connect using multiple devices,
and how frequently samples are provided.  As a consequence, 
the dynamic IP address churn rates reported by these studies
vary.  While UDmap reported that over 30\% of IP addresses have
inter-user durations of 1--3 days~\cite{udmap},  Heidemann et
al. reported that 90\% of IP addresses were occupied for less than a
day~\cite{census-survey}.  Maier et al.~\cite{maier2009dominant} reported that a
major European ISP had per-user median durations of just 20 minutes during
their study in 2009 (we did not observe this duration in 2015).
We believe that the perspective of a device using the dynamically assigned
network is necessary for understanding the reasons behind the address change
and for getting precise information about the duration that any
address is held. Further, since RIPE Atlas probes provide continuous,
longitudinal measurements enabling the inference of successive
addresses assigned to a CPE device, we perform the first
analysis of dynamic prefixes from which devices are assigned
successive addresses.


\input{timeouts}

\chapter{Mitigating false inferences due to dynamic addressing}

\label{sec:addr_change}

In this section, I describe how dynamic addressing can lead
probing-based outage detection
techniques to make false inferences about outages and describe
techniques to obtain a list of stable addresses for probing.

Academia and industry often rely on a simplifying assumption that IP addresses 
uniquely identify end-hosts~\cite{p2pfilesharing,p2pavail,sen2004analyzing,sekar2006multi,anomalousdns,kuhrer2015going,xie2005worm,jung2004empirical,fabian2007botnet,stone2009your,andriesse2015reliable,fail2ban,spamhaus,cbl,sorbs}.
This assumption allows researchers to track end host
behavior over time~\cite{anomalousdns, kuhrer2015going, pingin}, or to count participating users in peer-to-peer
systems~\cite{p2pfilesharing,p2pavail,sen2004analyzing}. Many organizations create blacklists of suspicious IP
addresses based on previously observed malicious traffic associated
with those addresses~\cite{fail2ban,spamhaus,cbl,sorbs}. 

When probing-based remote outage detection techniques send probes to an address, they expect that the
address continues to be assigned to the same end-host for the entirety
of the probing duration. Depending upon how a dynamic address gets
reassigned, these techniques can make false inferences about outages in two ways:

\begin{itemize}

\item{\emph{Detecting false outages} Probing-based remote outage detection techniques detect outages
    when a previously responsive address stops responding to
    probes. However, If a dynamic address being probed is
withdrawn from its host and is not assigned to any other host, active probes to the address will no longer
elicit responses. These techniques will infer false
probe-loss, leading them to infer false outages.}

\item{\emph{Detecting false outage duration} These techniques detect outage
    duration by continuing to probe an unresponsive address. When the
    address starts responding to probes again, the outage is inferred
    to end. If a user device  with a
    public dynamic address has an outage and at some point during the outage,
    the dynamic address is reassigned to some other user device
    which responds to probes, probing-based remote outage detection techniques would infer that the outage ended incorrectly.
% For ISPs that use DHCP
%     for address assignment, we would expect dynamically
%     assigned addresses to stick around on the end-host until an
%     outage occurs. However, upon the occurrence of the outage, if the
%     outage is long enough, they can get reassigned to another host,
%     especially, if the outage is longer than the DHCP lease
%     duration. Here, RODWAP techniques can detect the outage itself but
%     can perhaps not detect outages that are long
}

% TODO: CITE http://www.umiacs.umd.edu/~tdumitra/courses/ENEE757/Fall15/papers/Stone-Gross09.pdf
\end{itemize}

My approach to mitigating these false inferences is to analyze how
frequently and for what reasons dynamic addresses are
reassigned. I will use the results of these analyses to build a model
of how likely an inference about an outage using a probing-based
remote outage detection technique is
a false inference caused by dynamic addressing. For example,
preliminary work with colleagues has revealed that some European ISPs change addresses upon
very small outages and are particularly likely to change addresses at certain
times of the day~\cite{addrchange-reasons}. These results will inform
my model to not attempt detection of 
outage duration for these ISPs, and to discard outages
detected at times that are particularly likely to have dynamic address
changes. The model will ultimately yield stable addresses who either
do not undergo dynamic reassignment for months at a time, or who get
reassigned but in a predictable manner. Thunderping limits itself to
probing addresses in the U.S. where dynamic reassignment is
uncommon~\cite{addrchange-reasons}; stable addresses from the
model can help us detect outages in new areas.

% In the rest of this section, we provide background about 

\section{Dynamic addressing background}

An IP address can be used to uniquely identify the end-host it is assigned to
until the end-host's address changes for some reason. The duration of
time that a dynamic IP address continues to be assigned to the same
CPE (Customer Premises Equipment) device depends upon various causes that can induce the assigned IP
address to change. Here, I present techniques used for
assigning dynamic addresses and the events and
agents involved in dynamic address changes.

ISPs often use the Dynamic Host Configuration Protocol
(DHCP)~\cite{rfc2131} for IP address assignment. DHCP issues an IP address to a host for a lease
duration configured by the ISP. The host will try to renew the lease
before it expires, typically half-way into the lease. However,
whether the same IP address is renewed, or a different one is
assigned, depends upon ISP policy.  We speculate that the
typical behavior of ISPs using DHCP is to renew the lease of the
currently assigned IP address, since one of the stated design goals
in the DHCP specification is that a DHCP client should be assigned the same address
in response to each request, whenever possible. Thus, we typically
only expect an ISP using DHCP, to change the address of a CPE, if
something happens to prevent the CPE from renewing its lease (like an outage).
% Further, on
% reboot, the previously assigned address may be reassigned, or
% alternately a new address may be issued, again, as dicated by ISP
% policy.

In some networks, end-hosts connect to an ISP using
point-to-point links. For these networks, the Point-to-Point Protocol
(PPP) first configures and establishes the point-to-point link~\cite{rfc1661}. Next,
a Network Control Protocol (NCP) like the Internet Protocol Control
Protocol (IPCP) configures IP addresses~\cite{rfc1332}. The PPP specification
notes that the link will remain configured for communication until the
link is actively closed down through network administrator
intervention or when an inactivity timer expires.

\subsubsection{Potential dynamic address change causes} 

Next, we identify the reasons dynamic addresses assigned using
the above techniques could change. We classify the following categories of address change:

\begin{itemize}
\item{\textbf{Changes after outages}} If the client is
  disconnected or loses power long enough to fail to renew a DHCP lease,
  its address may be assigned to another; when it returns,
  it may then get a new address. We call such changes
  \emph{outage-caused address changes}.

\item{\textbf{Changes after reboot/reconnect}} While we
  expect addresses assigned through traditional DHCP to change only when the
  outage duration is long enough to prevent lease renewal, addresses
  assigned through PPP can change upon outages of any duration. Any reboot or
  network reconnect event could cause the client
  to forget its prior address and request a new one, or the
  state associated with a connection may be lost.
  We call such
  address changes \emph{reboot-caused address changes}. 
  
  % We make no
  % distinction between very short power outages and reboots.

\item{\textbf{Administrative address changes}} A purpose of
  dynamic address assignment is to allow reconfiguration of
  the network; it is possible that a reconfiguration of the
  DHCP server will force a change to the subnet on which the
  client lies.  We expect such reassignment to be rare.
  
\item{\textbf{Periodic address changes}} We observe that
  some ISPs limit the session length associated with an
  address, causing a reassignment after a fixed duration,
  typically one day to one week depending on the ISP.

\end{itemize}

\subsection{Building a global model of dynamic address change}

Conceptually, so long as there is some uniquely identifying feature
that remains constant across a device's address change,
it can be possible to track IP address changes over time. Several studies have used this broad
method~\cite{udmap, census-survey, zmap-dhcp,
maier2009dominant, dhcp-gatech, peering-shroud, dhcp-dimes}. 
UDmap~\cite{udmap} studied
dynamic address properties using Hotmail user login
traces where the user's login serves as the identifying
feature. Casado et al.~\cite{peering-shroud} tracked clients using HTTP cookies when
clients access a CDN. Other studies~\cite{zmap-dhcp, census-survey} used continuous
responsiveness of an address itself as the identifying feature, assuming
that an address that responds continuously belongs to the same user and that
when an address stops responding to pings, it has been
reassigned. 

However, these studies report conflicting results about the frequency
of address changes. While UDmap reported that over 30\% of IP addresses have
inter-user durations of 1--3 days~\cite{udmap},  Heidemann et
al. reported that 90\% of IP addresses were occupied for less than a
day~\cite{census-survey}.  Maier et al.~\cite{maier2009dominant} reported that a
major European ISP had per-user median durations of just 20 minutes during
their study in 2009 whereas our work in 2015 did not observe this
duration~\cite{addrchange-reasons}. These differences are likely due
to the different biases associated with each study: Maier et
al.~\cite{maier2009dominant} studied one European ISP in an area,
UDmap studied only Hotmail users, while studies that use continuous
responsiveness of addresses over time~\cite{zmap-dhcp, census-survey} could
potentially confuse the occurrence of an outage with an address change.

Considering these conflicting results,
I first analyze the feasibility of building a dynamic addressing
model by describing preliminary results from a novel dataset~\cite{addrchange-reasons}. The
results show that a global model is indeed feasible, but that it will
require multiple, independent, diverse datasets that track address changes across
the world. % Then I describe proposed
% work to gather other complementary datasets.

% However, most of these studies do not account for the cause of the address change,
% studying dynamic address durations in isolation. We show in this
% proposal that studying the causes of dynamic address change is vital
% especially for RODWAP techniques. The occurrence of outages can cause
% dynamic address reassignment.

\subsection{Preliminary results towards a global model}

The RIPE NCC's Atlas project deploys small devices, called probes, that
conduct measurements from globally distributed
networks~\cite{atlas}. The RIPE Atlas dataset offers measurements that allow us to
determine when an IP address change occurred and what the addresses
were before and after the change. In addition, the dataset includes many
measurements that provide context about what was happening around the
time of the address change. I was able to use these measurements to
detect when RIPE Atlas probes rebooted and were not sending pings
(indicating a power outage) and when their pings were not getting
responses (indicating a network outage). In a study with colleagues of active RIPE
Atlas probes in 2015, we found 3,038 RIPE Atlas probes with address
changes hosted across 929 ISPs and 156 countries~\cite{addrchange-reasons}.

\subsubsection{Some ISPs change addresses periodically}


ISPs can assign dynamic addresses for as long as they wish.
In DHCP, long leases simplify administration, while short
leases can be more efficient in reclaiming unused addresses.
DHCP leases, however, are meant to be renewable by devices
that are still active.  In this section, we look at periodic
address reassignment: instances where a device changes
address periodically, despite actively using the address.

If ISPs intentionally renumber after specific durations, we would
expect those address durations to be prominent in a distribution
of all address durations belonging to that ISP. We initially
considered studying distributions
of raw address durations, similar to the analyses by
Maier et al.~\cite{maier2009dominant} and
Moura et al.~\cite{zmap-dhcp}, but found that short address-durations
were overrepresented. When trying to reason about the expected duration that an address will
continue to be assigned to the CPE, we would like to know the fraction
of total time that each duration accounted for. This latter notion is more useful to find whether an
ISP is using periodic durations consistently, since the modes at
intervals on the scale of days will be more visible. 

To capture this notion we
define a metric, the \emph{total time fraction}. For a given probe and an address duration $d$,
we define the total time fraction for $d$ as the fraction of time spent by the probe in durations of length $d$.
We compute the total time fraction for a given probe and a duration
$d$ by obtaining the total address
time for the probe, and computing the fraction of the total
address time that was accounted for by address durations of 
length $d$. For a probe $p$, if $n(d)$ is the number of times the probe had an address duration
$d$ and $D$ is an array containing all address durations that were assigned to
the probe, the total time fraction for the address duration $d$ is
given by:

$f^p_d =  d \times n(d) / \Sigma(D)$

We use a similar procedure for computing the total time fraction
considering all probes in an ISP, country, or continent. We believe that  
%the distribution of 
the total time fraction offers a better representation of 
the
probability that an address was assigned for a certain
duration than a simple inspection of the address durations. 

\subsubsection*{North American addresses are assigned longer than other addresses}

\begin{figure}[tb]
  % \centering
  \begin{center}
    \includegraphics[width=3in]{figs/conts_a_all_ip_durs_connlogs_wtd_cdf}
  \end{center}
  \caption{\label{fig:conts_all_durs} 
    %% Dynamic address-durations
    %%     weighted by address-durations by continent.
    Cumulative distribution of total time fraction by continent. 
    Modes (vertical segments
    in the CDF) indicate periodic renumbering.  Addresses in North America
    are relatively long lived and free of periodic renumbering.}
\end{figure}

We begin by inspecting how address durations vary across
continents.  We expected that address scarcity might affect
address durations, leading to longer durations in North
America and shorter durations in Asia.
We use RIPE
Atlas's probe database 
to find the country to which each probe belongs. Next, we aggregate the
address durations of probes by their respective countries and
subsequently, to their continents.
Figure~\ref{fig:conts_all_durs} shows the cumulative distribution of
the total time fraction for each
continent, i.e., the y-axis shows the fraction of total address duration accounted for by durations less than the x-axis value. 
%% Figure~\ref{fig:conts_all_durs} shows a CDF of address durations, weighted as described above, for each
%% continent.
The number in
parentheses in the legend for each continent shows the total
 address duration for that continent in years ($\Sigma(D)$).

In Europe, Asia, Africa, and South America, address durations exhibit well-defined modes,
mostly at intervals that are multiples of 24 hours. The most common mode is
exactly at 24 hours: the total time fraction for European addresses at
24 hours is 0.16, African addresses is also 0.16, and Asian addresses is 0.07.
One week address durations are also common in Europe, with the total
time fraction at 1 week equaling 0.08.
South American addresses exhibit multiple modes: their total time
fraction is 0.11 at
12 hours, 0.07 at 28 hours, 0.09 at 48 hours, and 0.03 at 192 hours (8
days). 

The curves for North America and Oceania do not have well-defined modes,
suggesting that ISPs in these continents do not periodically change
addresses. Further, North American probes typically retain their dynamic
addresses for much longer durations than other continents; North
American addresses spent more
than half of the total time in address durations longer than 50
days. This suggests that IP addresses can be used as end-host
identifiers in North America for several weeks.

% \subsubsection*{Periodic renumbering is common in some parts of the world}

% \begin{figure}[tb]
%   % \centering
%   \begin{center}
%     \includegraphics[width=3in]{figs/DE_asns_a_all_ip_durs_connlogs_wtd_cdf}
%   \end{center}
%   \caption{\label{fig:DE_asns_all_durs}
% %
% 	%% Dynamic address-durations weighted by address-durations for ASes in Germany.
%     Cumulative distribution of total time fractions for ASes in Germany.
%         Many
%       German ISPs appear to change addresses every 24 hours. However,
%       some ISPs have more stable addresses.
%   %%@@rama: quantify
%   }
% \end{figure}


% Next, we investigate how the periodic renumbering behavior of ISPs
% correlates with the country in which they operate. Germany has more
% than a hundred RIPE Atlas probes deployed across several ISPs, thus we
% study their address durations in Figure~\ref{fig:DE_asns_all_durs} for
% ISPs with probes that contributed at least 3 years of total time. Many
% ISPs in Germany change addresses every 24 hours: 77\% of the duration
% in DTAG (AS 3320), 76\% in Telefonica1 (AS 6805), 74\% in Telefonica2
% (AS 13184), and 29\% in Vodafone (AS 3209), is 24 hours. We observe
% that the 'other' ISPs also have a mode at 24 hours, suggesting that
% German ISPs are particularly likely to renumber every 24
% hours. However, this behavior is not universal: Kabel Deutschland (AS
% 31334) and Kabel BW (AS29562) do not exhibit a mode at 24 hours;
% instead, more than 90\% of their total address duration was spent in
% durations longer than two weeks. These results suggest that periodic renumbering behavior can exhibit
% some geographic correlation, but is likely
% largely caused by ISP policy. 

% We found 20 ISPs in the RIPE Atlas dataset that we label as \emph{periodic} because
% these ISPs renumber many of their customers after they have held their
% address for a specific duration. This time limit varies across
% ISPs. Of 2272 dynamically assigned probes in the dataset, 193 (8.5\%)
% change addresses periodically with a period of 24 hours, and 123
% (5.4\%) do so with a period of one week. Periodic renumbering occurred
% most commonly in central European countries like Germany, Austria,
% Poland, and Croatia. Some ISPs in Russia, Kazakhstan, Mauritius, and
% South America, also periodically renumber.

% Private communication with a large European ISP confirmed that the ISP renumbers every 24
% hours, since the ISP considers this scheme to be more 'privacy secure' although
% there is no government regulation that forces this feature. The ISP
% also reported that it uses PPPoE instead of DHCP for its DSL
% lines (which accounted for the vast majority of its customers). Since
% periodic behavior would be atypical of DHCP but consistent with PPP
% techniques for address assignment, we speculate that periodic
% renumbering is a property of ISPs that use PPP.

\subsubsection*{Periodic address changes are some times synchronized}

% \begin{figure}[th]
%  \centering
%     \includegraphics[width=3in]{figs/weekly_3215_a_periodicrenums_per_h24_connlogs_bar}
%   \caption{\label{fig:3215_renums_per_h24}Periodic address changes
%     in Orange appear more evenly distributed among the hours of the
%     day.}
% \end{figure}

\begin{figure}[th]
  \centering
    \includegraphics[width=3in]{figs/daily_3320_a_periodicrenums_per_h24_connlogs_bar}
  \caption{\label{fig:3320_renums_per_h24}Periodic address changes
    are more likely in some hours for Deutsche Telekom.}
\end{figure}

We imagine two broad strategies for daily renumbering:
either leaving each customer on an independent, free-running
clock that resets after 24 hours, or synchronizing all
address changes to an off-peak time when few would be
interrupted.  Both seem reasonable strategies: independent
clocks seem simple to implement, synchronized address
changes seem more likely to shuffle addresses since many
addresses are made available during the synchronized
interval.  Probing-based outage detection techniques would benefit from knowing which strategy
is being used: if an ISP is known to change addresses at specific
times of the day, we can account for this behavior in the model.

 We expect
that plotting the time of day at which addresses change for
each ISP will expose whether the renumbering is
synchronized. For the German ISP, Deutsche Telekom AG (DTAG), Figure~\ref{fig:3320_renums_per_h24} shows the hour of
the day when an address change occurred after the address had been
assigned for 24 hours; DTAG assigns periodic durations more often during some
hours of the day. In private correspondence with a large European ISP,
we learned that many CPE devices come with an option to choose the
time at which they should disconnect and reconnect to receive a new
address, as a privacy feature. Figure~\ref{fig:3320_renums_per_h24}
supports this deployment scenario, observing almost three quarters of
all periodic address changes between hours 24 to 6 (in GMT). However,
some CPEs do not have this feature because a quarter of the periodic
address changes happen at other hours of the day.

\subsubsection{Some ISPs are more likely to change addresses upon
  outages}

Here, we investigated how outages occurring at the CPE (customer
premises equipment), due to loss of power or network connectivity
affect the likelihood of address changes. % We quantify how frequently
% and for which probes an outage event at the CPE device appears to
% cause the reassignment of its IP address.
% We quantify how the relationship between outage 


% \subsubsection*{Renumbering behavior upon outages varies across ISPs}

% For each individual probe, we considered the conditional
% probability of an address change given a detected
% outage. $P(ac|nw)$
% represents the conditional probability that an address change occurred
% given a network outage and $P(ac|pw)$ represents the same for a power outage. We estimated this probability using the
% fraction of outages occurring contemporaneously with an address change (out of the
% total number of outages).  We show the distribution of
% these probabilities by probe to estimate whether the group
% of probes (by geography or ISP) is dominated by those that
% always or seldom change addresses on an outage.

% \begin{figure}[tb]
%   % \centering
%   \begin{center}
%     \includegraphics[width=3in]{figs/top_asns_frac_norenums_over_totalnos_cdf}
%   \end{center}
%   \caption{\label{fig:top_asns_frac_norenums_over_totalnos}
% Distribution of $P(ac|nw)$ per probe for the ASes with the most probes
% that had at least one address change. Probes in DTAG, Orange, and BT, are far more likely to change addresses upon a
%     network outage than probes in Verizon and LGI.}
% \end{figure}


% We find that the likelihood of address change upon an outage event
% differs across ASes. Figure~\ref{fig:top_asns_frac_norenums_over_totalnos} shows
% the CDF of $P(ac|nw)$ for the five ASes
% that host the most probes with at least one address change and at least three
% network outage events. We find that probes in ASes that periodically renumber---Orange, DTAG, and BT---have high $P(ac|nw)$ compared to probes from ASes that
% do not periodically renumber, LGI and Verizon. Around half of the probes in both Orange and
% DTAG had  $P(ac|nw)$  equal to 1: every network outage was accompanied
% by an address change! $P(ac|pw)$ was also similar
% for these ISPs. We found 10 more ASes whose probes were particularly
% likely to renumber upon outages: all of them are in Europe and 7 of them
% also periodically renumber. Private communication with a large European
% ISP whose probes consistently had an address change upon outage confirmed that they use PPPoE and Radius to assign addresses for
% their DSL lines. We expect that this property can be used as evidence in inferring a device's link
% type.

% \subsubsection*{For some ISPs, most outages result in address changes}

\begin{figure}[t]
  \includegraphics[width=1.5in]{figs/6830_combined_merged_bar}~~~
  \includegraphics[width=1.5in]{figs/3215_combined_merged_bar}

  \caption{\label{fig:outagedurs} The likelihood of an address change (renumbering)
    given network or power outages of different durations in LGI (left)
    and Orange (right).  The top graph is a histogram; the
    complete bar represents the number of outages observed
    across all probes in that AS.  The lightly-shaded bar
    extends for those outages that also saw an address
    change.  The lower graph shows the same data as a
    percentage.  Although relatively few outages
    lasted longer than a day, the majority of these were
    coincident with an address change in both ISPs. However,
    Orange (right) changed addresses even on the shortest
    outages.}
\end{figure}

Dynamic addresses assigned using DHCP should typically retain
their addresses as long as they continue to renew their lease half-way
into the lease duration as the standard
recommends~\cite{rfc2131}. However, an outage could
prevent them from renewing their lease. Depending upon the address
churn at the time, the address they had previously been assigned may
be reassigned to another device.  In this way, an outage
longer than half a lease duration could potentially cause an address
change. To investigate the effect of outage duration on the likelihood
of address change, we analyzed the conditional probability of an
address change given the occurrence of network or power outages of different durations
for probes from LGI (AS 6830) and
Orange (AS 3215) in Figure~\ref{fig:outagedurs}. 

The behavior upon outages for the two ISPs is strikingly
different. LGI's behavior appears consistent with what we would 
expect for dynamic addresses assigned using DHCP: fewer than
3\% of outages of up to an hour resulted in an address
change.  More than 25\% of outage
durations that lasted at least twelve hours resulted in an address
change. This behavior is consistent with a DHCP lease duration on the
order of a few hours.  Not every outage longer than twelve
hours resulted in an address change, consistent with DHCP 
behavior when a client returns after an expired lease and the
previously assigned address
is still available.

For Orange, we found that even very short outages resulted in
address changes. 91\% of outages that lasted less than five
minutes resulted in an address change, and for every outage duration
longer than five minutes and shorter than three hours, more than 75\% occurred with an
address change. For outages between three hours to three days
long, the percentage of address changes was closer to 50\%, suggesting
the presence of some CPE devices that do not renumber upon every outage. However, as the
outage duration increases beyond 3 days, almost every outage results
in an address change.

% Private communication with a large European ISP
% confirmed that this behavior is expected for PPPoE based DSL lines in
% that ISP: any reboot/reconnect event will result in the assignment of a new
% address from the ISP's dynamic address pool. Since outages of such short durations can result in an
% address change, a simple reboot of the CPE (resulting in a power
% outage), or unplugging and replugging the network cable (resulting in a network outage), can change the dynamic address assigned to the end-user.%  That
% end-users can change their dynamically assigned address 
% has implications for researchers and operators who use IP
% addresses to identify end-hosts, particularly when IP addresses are
% being used to blacklist malicious actors.

\subsubsection*{Preliminary results suggest that building a global
  model is feasible}

Preliminary results offer promise that modeling the likelihood of
address change can help prevent false inferences about outages and
their durations. For ISPs that change periodically and/or synchronously, the model can
predict when probe-loss is more likely due to
address changes than outages. For ISPs that change addresses upon most
outages, the model can inform in which ISPs outage duration detection
is particularly error-prone. For other ISPs which change addresses
mostly upon longer outages, the model can be used to estimate the
likelihood that an inferred outage ended falsely.

Though the results from RIPE Atlas are promising, they are potentially
biased toward technical users like most measurement infrastructure,
and biased toward European deployment. Building the global model of dynamic address change will
require multiple, independent, diverse datasets that track address changes across
the world.

\subsection{Proposed work to gather complementary datasets}

IP address changes can
be tracked over time if there exists some uniquely identifying feature
that remains constant across the device's address change. I investigate the use of datasets which have this
property to study dynamic addresses:

% \subsubsection{Download manager logs}

% A large CDN's download manager installed on users’ desktop
% and laptops records log lines when events such as a file download
% occur. Each such log line contains a unique installation ID, the
% user’s current public IPv4 address, and the timestamp. 

\subsubsection{Dynamic DNS services}

Websites such as dyn.com~\cite{dyn} provide dynamic DNS. Dynamic
DNS is a service that allows users with a dynamic IP address to host
web services, by providing DNS services that can be easily updated to
reflect changes in users' IP addresses. Users of Dynamic DNS Services
run a daemon provided by the dynamic DNS provider, which is responsible for
determining the publicly visible IP address, and updating the A
record(s) for the user's domain(s). 

I propose to track IP address changes using domain names
registered with dynamic DNS services. Since the domain name of a user
maps to her current IP address, we can use the domain name as a
fingerprint, and detect changes in IP addresses for each domain name
over time, by periodically obtaining the 'A' record associated with
each domain name. 

\paragraph{Geographic correlation of dynamic behavior}

\begin{figure}[tb]
% \centering
\begin{center}
\includegraphics[height=1.5in]{figs/did_dname_get_renum}
\end{center}
\caption{\label{fig:addr_change_per_ctry}
IP address renumbering in dynamic DNS domains over a week: Black
represents dynamic DNS domains which experienced at least one address
change, while grey represents domains whose addresses remained the
same. Renumbering behavior appears to be correlated with geographic
location.}
\end{figure}

As a proof of concept, I report on a preliminary result from this
approach: corroborating the geographic relationships in Figure~\ref{fig:conts_all_durs} 
while extending to countries not well represented by RIPE.
I
obtained 3000 dynamic DNS domains from three different dynamic DNS
services: 2000 from afraid.org~\cite{afraid}, 600 from dyn~\cite{dyn} and 400 from
noip.com~\cite{noip} and fetched the 'A' records from their respective
nameservers once every hour. I collected this data for a week, and
then inspected how many of these domains experienced at least one
address change during this time. Figure~\ref{fig:addr_change_per_ctry}
shows the number of domains that had at least one 
address change and the domains that had none. The y-axis is in log-scale. 
Address changes in Asian and Latin American countries appear
more prevalent, with more than a third of all domains in these
countries seeing at least one address-change. On the other hand,
Northern European countries observe fewer than 6\% of their domain names
experiencing an address change. Address changes are uncommon in
North America: only 3\% of domain names in the US and 6\% of domain
names in Canada observed an address change.

The results from the dynamic DNS dataset are preliminary in
scale and based on a short measurement to show potential.
Further, our study of RIPE Atlas data showed us that the cause of
address changes is important. I intend to couple our
outage-detection tool to probe addresses corresponding to
the dynamic DNS domains while fetching their A records.  We
can thus identify outages that occur near the reassignment,
allowing us to infer if an address-change was caused by an outage and
feed results into the model. Further, if the dynamic DNS result
indicates that a probed address had recently been reassigned, then the
detected false positive outage can be filtered.
 
\subsubsection{Open DNS resolvers}

Since 2010, various studies have reported on the existence of more
than 15 million 'open' DNS resolvers on the
Internet~\cite{openresolver, schomp2014clientsidedns, kuhrer2014exit,
  kuhrer2015going}. These DNS resolvers are 'open' because they will resolve a DNS query sent from arbitrary IP
addresses on the Internet. Previous studies have found that more than
three-quarters of open DNS resolvers are likely to be
residential~\cite{schomp2014dnsvul, schomp2014clientsidedns}. I
propose two potential approaches to fingerprint these open DNS
resolvers and track address changes.

\paragraph{DNS caches}
Open DNS resolvers often cache previous
lookups~\cite{schomp2014dnsvul}. My insight is that these caches can
be used to fingerprint open DNS resolvers, allowing us to track when
their IP addresses change. I plan to do this in two phases.

First, I will find open DNS resolvers on the Internet. I propose to register a domain and deploy an Authoritative DNS server
for it. Then I intend to perform a one-time scan of the entire IPv4
address space by sending a DNS request for a subdomain within the
domain we control to all the IPv4 addresses on the Internet. Each DNS
request I send to a target IP address will embed the target address
into the request, similar to the approach used by Dagon et
al.~\cite{dagon2008corrupted}. The Open DNS resolvers will route the
request to our Authoritative DNS server.  At the authoritative DNS
server, I will note the target IP address to which this request was
sent and generate a unique fingerprint for the device at this address,
and embed this fingerprint in my response. When these responses
reach the open DNS resolvers, each will now contain its unique
fingerprint in its cache.

Next, I will periodically inspect the caches of known open DNS resolvers.
I will issue periodic DNS requests for the subdomain we
control (with the target IP address embedded in the request) to all
the addresses that contacted our Authoritative DNS server. If we
obtain the fingerprint that we had previously issued to that address,
we know that the device continues to be assigned that address. If we
find that an address is no longer returning the expected cache
fingerprint, we know that the address has changed. I then propose to
issue DNS requests to related addresses (as described in
Section~\ref{sec:last_mile}) with the \emph{old} target IP address
embedded in the request. If the device is present on any of those
addresses, then we will obtain the expected fingerprint. Upon finding
the device at a new address, we will update
our local mapping and note that the fingerprint is now available at
this new address.

\paragraph{Anomalous Open DNS Resolvers}

Of the 30 million Open DNS Resolvers on the Internet, around 17
million are \emph{anomalous}~\cite{anomalousdns}, i.e.,
instead of sending DNS responses with a source port of 53, they
respond with a non-standard source port. Kaizer et al. ~\cite{anomalousdns} found that
these devices are primarily residential ADSL modems. Not only do these
devices use a non-standard source port, DNS requests can be made to
these devices in such a way that the source ports are assigned
\emph{sequentially}. My insight is that we can use this sequential
assignment of source ports to fingerprint anomalous open DNS resolvers.

The first part of our approach here is similar to my approach with
the DNS caches: I will find open DNS resolvers that are anomalous. After registering a domain and deploying an Authoritative DNS server
for it, I will perform a one-time scan of the entire IPv4
address space by sending a DNS request for a subdomain within the
domain we control to all the IPv4 addresses on the Internet. Each DNS
request I send to a target IP address will embed the target address
into the request as before. However, instead of embedding responses with
unique fingerprints from the authoritative DNS server, we simply
monitor the source ports that issue DNS responses from each
address. If it's a non-standard port, we flag the device as an
anomalous open DNS resolver.

Next, I will periodically inspect the source ports used by anomalous
open DNS resolver responses. Since we know which
  addresses the anomalous open DNS resolvers are located at, I
  periodically issue DNS queries to these addresses. As long as the
  source port for successive requests to an address continues to be
  sequential, I can state with high confidence that the address has
  not changed. The source ports for these devices typically vary
  between 10,000 to 30,000; thus there is only a small likelihood that
  another device coincidentally happens to have the next value in
  sequence. If we find that a response doesn't arrive, or that one
  arrives but the source port is not sequential, then we know that the
  device's address has been reassigned. As in the DNS cache approach,
  I will then look for the expected source port in DNS responses from
  requests sent to related addresses to find the device again.


\subsection{Confirming that detected outages are accurate}

After mitigating false positive outages, I propose the use of datasets
from RIPE Atlas probes~\cite{atlas} as ground truth to confirm that the remaining
outages are indeed true positives. In previous work, I had inferred outages occurring on
RIPE Atlas probes by looking for gaps when probes did not perform 
measurements that they were scheduled to~\cite{addrchange-reasons}. By
probing IP addresses at which RIPE Atlas probes are also deployed, I
will compare outages we detect against outages inferred from RIPE
Atlas datasets and validate whether our detected outages are accurate.

\section{Analyzing the effect of external factors upon Internet reliability}
\label{sec:weather}

One aspect of measuring Internet reliability is to determine if the occurrence of certain events adversely affects Internet connectivity. Consider the occurrence of adverse weather conditions for instance: prior work has shown that Internet outages occur more frequently during times of precipitation~\cite{pingin}. However, this work was preliminary in nature and was performed over a short duration (three months). Further, it treated every instance of a previously responsive address failing to respond to pings as an outage, ignoring the effects of dynamic addressing.

In this section, I discuss an approach to quantify the effect of external factors, such as the occurrence of various weather conditions, upon Internet connectivity of residential addresses using measurements from the Thunderping probing system~\cite{pingin}. 

\subsection{Thunderping measurement system}

Thunderping probes addresses during times of severe weather. 

\subsection{Find the inflation in dropout rate}

The key insight here is that determining the inflation in \emph{dropout} rate when event(s) occur captures the increased likelihood of the \emph{outage} rate. 

% \begin{figure*}[t]
% %
% \begin{subfigure}[t]{0.47\linewidth}
% \centering
% \includegraphics[width=\linewidth]{figs/frate_by_timeofweek_jan11todec17_scatter}
% \caption{
% \label{fig:frate_lts_timeofweek}
% Dropout probability has significant diurnal variation.
% }
% \end{subfigure}
% %
% \hfill
% %
% \begin{subfigure}[t]{0.47\linewidth}
% \centering
% \includegraphics[width=\linewidth]{figs/addresshoursbywtyp_by_timeofweek_scatter}
% \caption{
% \label{fig:weather_timeofweek}
% Different weather conditions are prominent at different times.
% }
% \end{subfigure}
% %
% \caption{
%  Weather does not occur most often during hours of the week when there are an inflated number of dropouts. 
% }
% \end{figure*}



 

\section{Categorizing outages using simultaneous failures of related addresses}
\label{sec:last_mile}

% Only true positive outages remain after filtering false positive
% outages using the proposed techniques in the previous sections, and
% some applications may be able to use these outages without any additional
% processing. 
% For example, consider a peer-to-peer file storage network
% that keeps track of live peers; this network can use our outage feed
% as is to update its list of live peers.

% On the other hand, when 

When analyzing the reliability of a particular ISP, we need
to find the subset of outages that affected only
that ISP. Doing so ensures that ISPs offering services in challenged
areas do not have their reliability lowered by events such as power
outages or users voluntarily shutting down their home Internet
equipment.

I describe three categories under which detected outages can be 
placed. Each category provides hints about the likely cause of outages
placed in that category. For example, power outages or undersea cable cuts can affect addresses from multiple
ISPs; I term events which result in outages for
many providers' addresses \emph{multiple-ISP outages}. Users in some
geographic areas are particularly likely to shut down their Internet
equipment~\cite{grover2013peeking} but users elsewhere may also power
off their equipment when faced with certain circumstances, such as
approaching thunderstorms. I call such an event a \emph{user-caused
outage}. On the other hand, consider an ISP experiencing a failure in
its networking infrastructure resulting in an outage affecting only
addresses belonging to this ISP: these are outages that should bring down the ISP's reliability
estimate. I term these events \emph{shared-ISP outages}.
Probing-based remote outage detection techniques will detect outages
when all of these scenarios occur since previously responsive
addresses will cease responding in all these scenarios.

I develop and evaluate an approach for segregating outages into different categories
based upon the insight that outages occurring simultaneously in time
for addresses that are related by virtue of sharing geography, ISP, or
network topology,
could share a common cause. For example, if we detect addresses from
many ISPs within geographically proximate regions failing simultaneously in time, we are likely observing a
multiple-ISP outage. If we detect addresses from only a single ISP
failing simultaneously in time, we are potentially observing a
shared-ISP outage. If the detected outage does not appear to
have happened simultaneously in time with other outages of related
addresses, I term it an \emph{isolated} outage. User-caused outages would
likely manifest as isoltated outages. Thus, evidence
of simultaneous failure of multiple
``related'' addresses can be used to distinguish between the different
categories of outages. 

However, it is possible that this approach can incorrectly classify detected
outages into the wrong outage category. For example, if
unrelated addresses happen to fail close together in time,
these outages could be detected as
``simultaneous'' and could be classified into either the multiple-ISP
or shared-ISP categories, depending upon the ISPs to which
the unlrelated addresses belonged. It is also possible that a
multiple-ISP outage is categorized as shared-ISP because
other affected ISPs' addresses had
not been probed at that time. Similarly, if a set of related addresses failed
simultaneously but only one of these addresses was being
probed, we would not observe these outages to be simultaneous. 

While the above false classifications can potentially be mitigated by
intelligent probing schemes, there exists a limitation that cannot:
once an outage has beeen detected to be \emph{isolated}, the proposed approach cannot distinguish between
a user-caused outage and an outage that only affected a single user
but was caused by lack of power or by a network failure. It is possible
that some users regularly power off their equipment at night; if
so, we could
check if a particular user's address consistently experiences outages
between certain hours and categorize these outages as user-caused. But
this solution would not be successful in categorizing all instances of user-caused outages.
% This scenario is not an example of a false
% classification; instead, it is a limitation of the simultaneous outage
% detection approach.

Though it is important to reduce false classifications,
it is also important to balance the probing volume. Sending probes
too quickly to an already challenged network could exacerbate the
problem. Also, keeping the probing volume towards individual
destination addresses low will increase the overall number of
destination addresses we can probe. 

The proposed approach has two tasks:
(a) find addresses that are ``related'' to a given address and are therefore
likely to fail together and
(b) design a probing scheme that balances false outage classification
rates with the number of sent probes.

\subsection{Finding related addresses}

Given any address to probe, a key task for the proposed approach is to
find addresses that are ``related'' to this address since such addresses are likely to share
fate with the given address and experience outages simultaneously. For instance, addresses can be
related by geography when they are physically proximate to each
other; a power outage can result in simultaneous outages
of many geographically proximate addresses. Alternately, addresses can be related by ISP, when they
share the same provider. Addresses can also be related by network
topology: for instance, they can share the same last-hop
router~\cite{hobbit-use-hobbit-imc-instead} or they can be assigned
from the same dynamic addressing
pool~\cite{addrchange-reasons}. 

To successfully segregate outages into their categories,
addresses related by a range of criteria need to be chosen because
outages with different causes can cause a different set of related
addresses to fail together. For example, if a
tree branch damages a network cable that serves an apartment building,
it is possible that only addresses that are related by network
topology will fail together. On the other hand, a failure in the
network infrastructure for an ISP could cause many addresses belonging to that ISP to
fail together across several geographic areas. A power outage can cause
addresses belonging to many different ISPs in a geographic region to
fail together.

Existing datasets can be used to find a set of \emph{candidate related
addresses}, according to
each criterion. Given an address, let $cra$ be the set of
candidate related addresses. To find addresses that are related to a
given address according to geography or ISP, we can use the
MaxMind database~\cite{MaxMind}, a popular commercial service used in
several large Internet measurement projects~\cite{censys-about,
usc-sandy, heidemann-diurnal}. Although we would expect intuitively
that addresses related by network topology can be found by simply
enumerating numerically adjacent addresses, preliminary work has shown
that subsequent dynamically addresses assigned to the same device are
often numerically distant~\cite{addrchange-reasons}. Thus, it is possible that even
numerically adjacent addresses are not related by network topology and
conversely, that numerically distant addresses are. Addresses from the same
dynamic addressing pool are available from preliminary
work~\cite{addrchange-reasons} and from other proposed work in
Section~\ref{sec:addr_change}. Addresses with the same last-hop router
as a given address are available from Lee and
Spring~\cite{hobbit-use-hobbit-imc-instead}. 

% After finding the set of candidate related addresses, $cra$, for each given
% address, the next step is to select a subset of these addresses for probing which
% will maximize the likelihood of finding simultaneous outages when they
% occur. I call this subset of addresses the probed related addresses ($pra$). Addresses that are topologically related (same last-hop router
% or same dynamic addressing pools) are likely to share fate upon
% outages; thus, I will first select topologically related addresses
% to a given address. Selecting addresses related by other criteria can
% also prove to be helpful for shedding light upon potential causes and
% the extent of outages. For example, power outages and network outages
% in an area can result in simultaneous outages for multiple IP
% addresses in that geographic area. By selecting related
% addresses from that geographic area, we can potentially disambiguate a power
% outage from a network outage: if addresses in that area from multiple
% ISPs fail simultaneously, the outage is likely a power outage and if
% addresses from only a single ISP failed, the cause is likely a network
% outage. If we also select addresses from multiple
% geographic areas, we can estimate the severity and potential cause of
% the outage. For example, if addresses fail simultaneously
% across a large geographic area, a severe outage has likely occurred,
% perhaps at some core infrastructure. 

% Although increasing the size of the set of probed related addresses,
% $pra$, can increase the likelihood of finding simultaneous outages, it will also result in higher
% probe volume. Thus, an important component of this task is to balance
% the addresses in $pra$ such that the probe volume remains reasonable
% while also maximizing our ability to observe simultaneous
% outages. My plan is to start with a small $pra$ set consisting of 11
% addresses: two topologically related, four geographically related
% and five for severity estimation. In the next section, I illustrate
% how probe traffic can remain low even when we have 11
% additional addresses to probe in the $pra$.


\subsection{Probing related addresses}

The other key task in the proposed approach is to probe the given
address and a subset of addresses from its $cra$ in such a manner as to
minimize false outage classifications while
keeping the probing volume low. Probing a larger subset of addresses
from the $cra$ will increase the likelihood of observing
simultaneous outages since there is a higher
likelihood that we are probing addresses that fail together. Probing more frequently
increases the accuracy of the measured simultaneity; for example,
probing all addresses every second will allow us to measure
simultaneous outages that occurred in the same second. Less frequent probes will
reduce accuracy by increasung the likelihood that two outages
which were not actually simultaneous are reported to be simultaneous. On the other hand, probing many addresses and
probing more frequently increase the probing volume. 

My first approach toward resolving these tradeoffs is to vary the number of
probed addresses from the $cra$ and the probing volume towards them and empirically
determine the resulting false classification rates. 
Ideally, a
complete list of ``ground truth'' outages per category would
exist against which the detected outages can be
compared. Although a complete list does not exist, partial lists of
network outages are available from the outages mailing
list~\cite{outages-mailing-list} and of power outages from the
U.S. Department of Energy~\cite{power-outages-us-official-list}. For
multiple-ISP
and shared-ISP outages, I intend to compare the detected
outages against the known outages in these lists for various combinations of probed addresses and
probing volume. While these partial lists can confirm that some
multiple-ISP and shared-ISP outages were indeed
categorized correctly, they
cannot inform if isolated outages were
categorized incorrectly into the multiple-ISP or shared-ISP
categories, since these lists are not exhaustive.

For determining how often isolated outages are falsely categorized into
the multiple-ISP or shared-ISP categories at various probing volumes, 
I intend to investigate outages detected by existing studies such as
Thunderping~\cite{pingin} and the ISI survey~\cite{census-survey}.
Using these existing measurements, I will first estimate the probability that
unrelated addresses that were probed fail ``simultaneously'' over
different windows of time. Thunderping pings thousands of
addresses every day in areas with severe weather alerts and each
address receives approximately a ping every minute when Thunderping
uses 11 vantage points. The ISI survey uses a different probing scheme: it sends probes
to every address in a /24 prefix once every 11 minutes over a two week
period for 24,000 /24 prefixes. With such a large volume of addresses being
probed, both Thunderping and the ISI survey detect many ``simultaneous'' outages; some of
these are likely true positive simultaneous outages and others
false. Simultaneous outages observed in these
datasets for addresses in unrelated ISPs and disparate geographic areas are
likely false positives. I intend to use such false positives to
estimate the probability of detecting false simultaneous outages---and
therefore, falsely categorizing outages into multiple-ISP or
shared-ISP---using
these techniques' default probing scheme. To obtain the false positive
rate for less frequent probing volumes, I intend to simulate the
false positives that would have resulted if probes had been send at
lower frequencies to a smaller subset of probed addresses. Similarly,
I will assess the false negative rate by looking for related addresses
that failed together in these datasets. Thunderping probes addresses
in similar geographic areas and the ISI survey probes all addresses in
a /24; thus, both schemes already probe some related addresses. I will
simulate lower probing-volume schemes and determine how many true simultaneous
outages affecting various related addresses will be missed and estimate
the false negative rate.

\subsection{Using categorized outages to estimate reliability}

After validating that the outage categorizations are accurate, I will
use the categorized outages to estimate ISP-level, media-type-level,
and geographical-area-level reliability using both reliability metrics
I defined in Section~\ref{sec:related}. For ISP-level and
media-type-level reliability analyses, I will only consider shared-ISP
outages. For comparing reliability across different geographies, I will consider multiple-ISP outages as
well. Also, I will perform an analysis of the proportion of
Internet-connected devices over time; this analysis will include every
discovered outage, including isolated ones.

% When analyzing if a particular geographical area has reliable
% Interent connectivity,
% multiple-ISP and shared-ISP outages should count as outages,
% but not user-caused ones. And when comparing ISP-level or
% media-type-level reliability, only shared-ISP outages should
% be treated as outages.

% The other key task in the proposed approach is to probe the given
% address and the addresses in its $pra$ in such as a manner as to
% find simultaneous outages with precision while keeping the probing
% volume low. If we probe all the $pra$ addresses every second, we
% can observe simultaneous outages with the precision of a second. Thus,
% observing just two simultaneous outages in a given second may be
% enough to convince us that these outages are involuntary, since two
% users are very unlikely to power off their equipment in the exact same
% second. On the other hand, if we only probe $pra$ addresses once every
% hour, we likely need more addresses to fail together simultaneously,
% because the likelihood that two users decided to voluntarily power off
% their devices in the same hour is higher.

% My idea to balance probing volume and simultaneous outage detection
% precision is to probe $pra$ addresses only when necessary for
% simultaneous outage detection. Whenever the given address appears to
% be experiencing an outage, addresses in $pra$ will need to be probed
% to check for simultaneity. I will probe the
% given address every minute; one ping per minute is the approximate
% probe-rate that an address receives from Thunderping when Thunderping
% has 11 vantage points. Upon a lost probe to the given address, I will
% immediately initiate probes to all the $pra$ addresses as well as to
% the given address using the probing scheme adopted by
% Thnderping~\cite{pingin}. During this phase, all addresses in the
% $pra$ as well as the given address will receive at least a ping every
% minute from each vantage point (assuming 11 vantage points). I will
% retry lost probes to any of these addresses with exponential backoff. This scheme
% will ensure that an outage that affects any subset of the given
% address and addresses in the $pra$ will be detected in the same minute.

% Since this probing scheme relies upon addresses in the $pra$ for
% simultaneous outage detection, an important task is to ensure that
% the addresses in $pra$ remain responsive to probes. Thus, even when the given address
% is responsive to probes, I will probe every $pra$ address
% once every 11 minutes from a single vantage point. When an address in
% the $pra$ is no longer responsive, I will replace it with another
% address from the $cra$.

% By ensuring that probes are only sent when necessary, I believe that
% the probe volume will be low enough to not trigger abuse
% reports. Consider the case when we select 11 $pra$ addresses
% for a given address from the same ISP and we use 11 vantage points. The probe rate towards that ISP
% in the absence of lost probes will only be increased by 1 probe per
% minute (since each $pra$ address will only receive a single
% probe every 11 minutes). This increase of 1 probe per minute is
% similar to the increase in Thunderping's probe rate, if Thunderping
% had selected a single extra IP address to probe in that ISP. When
% probes to the given address are lost, then the probe rate towards that
% ISP increases by a factor of 11. However, this increase in probe rate
% is justified because we can now detect simultaneous outages in the
% minute that they occur.


% Why should we send them traceroutes?

\section{Conclusion}

In this proposal, I described the problem of measuring last-mile Internet
reliability and illustrated how remote probing-based outage
detection techniques have the potential to measure Internet
reliability for individual users broadly, longitudinally, and
accurately. In spite of their potential, these techniques can make
false inferences about outages in two scenarios: when probe
responses are delayed beyond timeouts and when addresses get dynamically
reassigned. I described preliminary work which studied how commonly
probes are delayed beyond responses and described measurements of
dynamic addressing across the world that can help build a model of
dynamic addressing. For each outlined scenario, I proposed approaches
that can mitigate false outage inferences when that scenario
occurs. Finally, I discussed an approach to segregate outages into
categories that suggest cause, and how we can use these categorized
outages to study Internet reliability along different dimensions.


\clearpage % TODO: Find out what clearpage does?
\bibliographystyle{plain}
\bibliography{longnames,outages}

\end{document}
