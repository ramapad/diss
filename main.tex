
%TODO: Find out what's different about ns-proposal used in reliability proposal
% \documentclass[letterpaper,10pt]{article}
%% \documentclass[10pt,onecolumn]{article}
\documentclass[10pt,onecolumn]{ns-article}
% \newcommand{\mydriver}{pdflatex}
% \documentclass[12pt,\mydriver]{thesis-2}
%% \usepackage[letterpaper, margin=1.25in]{geometry} % For giving to neil


\PassOptionsToPackage{hyphens}{url}
\usepackage[hyphens]{url}
\usepackage{hyperref}
\hypersetup{breaklinks=true}
\urlstyle{same}
\usepackage{balance}

% TODO: I seem to have used most of these from the reliability
% proposal. Find out what the role of each command is.
\usepackage{ifpdf,floatrow} 

% nspring 10.35pt palatino
\renewcommand*{\rmdefault}{ppl} % Roman default
\usepackage{fix-cm}
% \renewcommand{\normalsize}{\fontsize{10.35pt}{12.5pt}\selectfont}
% \renewcommand{\normalsize}{\fontsize{10pt}{12.5pt}\selectfont}
% endspring

%% ns - replacement for times that lacks stupidity.
% \usepackage{mathptmx}
% \usepackage[scaled=.90]{helvet}
% \usepackage{courier}
% 
% \usepackage[override]{cmtt} % make tt font tighter / less ugly

% \usepackage{makeidx}  % allows for indexgeneration

\usepackage[square,comma,numbers,sort&compress]{natbib}

\usepackage{titlesec}
   \titleformat{\chapter}
      {\normalfont\large}{Chapter \thechapter:}{1em}{}

\usepackage{color}
\usepackage[table]{xcolor}
\definecolor{orange}{RGB}{255,127,0}
\newcommand{\rama}[1]{{\color{red}[\todo{rama: #1}]}}
\newcommand{\ns}[1]{{\color{green}[ns: #1]}}
\newcommand{\etal}{et~al.\xspace}
\providecommand{\ie}{\emph{i.e.,} }
\providecommand{\eg}{\emph{e.g.,} }
\providecommand{\cf}{\emph{cf.,} }
\providecommand{\vs}{\emph{vs.} }
\providecommand{\etc}{\emph{etc.}}   
\providecommand{\ione}{\emph{(i)} }
\providecommand{\itwo}{\emph{(ii)} }
\providecommand{\ithree}{\emph{(iii)} }
\providecommand{\ifour}{\emph{(iv)} }
\providecommand{\ifive}{\emph{(v)} }

% \newcommand{\ignore}[1]{}

% Rk: The following commands came from the NSF reliability proposal
% \setlength{\textwidth}{6.5in}
% \setlength{\textheight}{9in}
% \setlength{\topmargin}{0in}
% \setlength{\headheight}{0in}
% \setlength\columnsep{.30in}
% \setlength{\headsep}{0in}
% \setlength{\oddsidemargin}{0pt}
% \setlength{\evensidemargin}{0pt}

% Rk: The following commands are from mainthesis.tex from UMD's
% official style guide. Revisit and fix to get tables.
% \newcommand{\tbsp}{\rule{0pt}{18pt}} %used to get a vertical distance after \hline
% \renewcommand{\baselinestretch}{2}
% \setlength{\textwidth}{5.9in}
% \setlength{\textheight}{9in}
% \setlength{\topmargin}{-.50in}
% %\setlength{\topmargin}{0in}    %use this setting if the printer makes the the top margin 1/2 inch instead of 1 inch
% \setlength{\oddsidemargin}{.55in}
% \setlength{\parindent}{.4in}
% \pagestyle{empty}

\setlength{\textfloatsep}{.1in plus 0.05in}
\setlength{\itemsep}{-10pt}


\usepackage{enumitem}
\setlist[itemize]{leftmargin=*}
% \usepackage{slashbox}
\usepackage{graphicx}
\usepackage{amsmath,amssymb}
\usepackage{verbatimbox}
\usepackage{boxedminipage}
\usepackage{multirow}
% \usepackage{subfigure}
\usepackage[labelformat=simple]{subcaption}
% \usepackage{subfig}
\usepackage{xspace}
\usepackage[]{pdfpages}

\usepackage[utf8x]{inputenc}
\usepackage{ucs}

% \newcounter{FileStack}
% \let\OrigInput\input
% \newcommand{\ninput}[1]{%
%   \stepcounter{FileStack}
%   \expandafter\let
%   \csname NameStack\theFileStack\endcsname
%   \ThisFile
%   \def\ThisFile{#1}%
%   \OrigInput{#1}%
%   \expandafter\let\expandafter
%   \ThisFile
%   \csname NameStack\theFileStack\endcsname
%   \addtocounter{FileStack}{-1}%
% }

%boxed,vlined,,linesnumbered,commentsnumbered
\usepackage[vlined]{algorithm2e}
\providecommand{\SetAlgoLined}{\SetLine}
\providecommand{\DontPrintSemicolon}{\dontprintsemicolon}





\newcommand{\planetlab}{PlanetLab\xspace}

% \renewcommand\thesubfigure{(\alph{subfigure})}

%\usepackage{algorithm}s}
\title{Analyzing Internet reliability remotely for
  individual users with probing-based techniques}
% \title{Thesis Proposal: Remote probing-based outage detection for
%   individual IP addresses}

\author{Ramakrishna Padmanabhan}

\begin{document}

% Define table specific commands
\newcommand{\bb}{~~~~~}
\newcommand{\hdr}[1]{\multicolumn{1}{c}{\textbf{#1}}}
% define ``struts'', as suggested by Claudio Beccari in
%    a piece in TeX and TUG News, Vol. 2, 1993.
\newcommand\Tstrut{\rule{0pt}{2.2ex}}         % = `top' strut
\newcommand\Bstrut{\rule[-0.9ex]{0pt}{0pt}}   % = `bottom' strut

\maketitle


\begin{abstract}

% We use the Internet without know that the Internet is actually super reliable. It's based more upon: hey, the Internet seems reliable. 

% The Internet is used today for communication

% Detection of Internet outages, which are potentially rare events, demands broad and longitudinal measurements of users' Internet connections.
% Internet reliability is increasingly important as the applications we use increasingly depend upon the Internet.
% Internet reliability is increasingly important as a variety of services that we use migrate to the Internet.
% Internet reliability is increasingly important as the applications that we depend upon migrate to the Internet.
Internet reliability is increasingly important as a variety of services that we use migrate to the Internet. Yet, we lack authoritative measures of last-mile Internet reliability. The first step towards measuring last-mile reliability is to detect Internet outage events experienced by users. Since Internet outages are rare events, detecting them requires broad and longitudinal measurements; however, such measurements of Internet reliability at the individual user level are challenging to obtain accurately and at scale. The second step is to use detected outages to reason about Internet reliability across different dimensions such as ISPs, media-types, and geographical areas.

Probing-based remote outage detection techniques can scale but their accuracy is questionable. These techniques detect Internet outages across time as well as across the IPv4 address space by sending active probes, such as pings and traceroutes, to users' IP addresses and use probe responses to infer Internet connectivity. However, they can infer false outages since their foundational assumption can sometimes be invalid: that the lack of response to an active probe is indicative of failure. I illustrate two potential scenarios where this assumption is invalid. In the first scenario, responses are delayed beyond the prober's timeout, leading these techniques to infer packet-loss instead of delay. In the second scenario, these techniques can falsely infer packet-loss when the address they are probing gets dynamically reassigned. I examine how commonly delayed responses and dynamic reassignment occur across ISPs to quantify the inaccuracy of these techniques. 

Next, I demonstrate how detected outages can be used to perform meaningful assessments of Internet reliability. One aspect of Internet reliability is the study of how an external factor (like the occurence of thunderstorms) affects Internet connectivity; I show how to study the effect of such a factor upon the reliability of a group of addresses by studying the \emph{inflation} in outage rate for that group during its presence. Measuring the inflation in outage rate mitigates the effects of false outages. I also develop and evaluate an approach to segregate outages into categories that suggest their cause. Outages could result from a variety of causes, such as power outages, voluntary shutdown of users' home Internet equipment, network outages due to an ISP's infrastructure failure etc. When assessing the reliability of a particular ISP, we would ideally consider only the subset of outages that affect solely that ISP. I propose a technique to segregate outages by detecting simultaneous outages of ``related'' groups of addresses (addresses may be related by geography, ISP, or network topology). Simultaneous outages can serve as evidence that a detected outage affected multiple users in a particular ISP.

Implementing these proposed techniques will help achieve comprehensive measurements of Internet reliability that can be used to identify vulnerable networks and their challenges, inform which enhancements can help networks improve reliability, and evaluate the efficacy of deployed enhancements over time.

% comprehensive datasets of Internet reliability that can aid in improving reliability by targeting problem regions or adapting strategies that have proven to be successful
\end{abstract}

\newpage


% \section{Preliminaries}

\section{Introduction}


% With online accessibility of
% devices ranging from temperature control systems to baby monitors in
% this Internet of Things era, our dependence upon the Internet for
% increasingly many facets of our daily lives only continues to
% rise.

% With the ability to access a
% variety of devices online in this Internet of Things era, ranging from
% temperature control systems to baby monitors, our dependence upon the
% Internet for increasingly many facets of our daily lives only
% continues to rise. 

% As the range of Internet services that we rely upon increases, so does our reliance upon
% the Internet. 

% Talk more about how important the Internet is?

Residential Internet reliability is increasingly important as a variety of
services that we use migrate to the Internet. Internet users today can
communicate with each other, perform financial transactions, plan
their travel, and even obtain critical services such as health
monitoring~\cite{ideal-life, remote-health-elderly} and emergency
services~\cite{emergency-voip-voipfone, emergency-voip-fcc} from their
homes. Our dependence upon the Internet will rise further as more of
our home devices become connected in this Internet of Things
era. Consequently, continuous availability of the Internet and
resilience is vital, and the reliability of the Internet is of
interest to stakeholders across the board, from government regulators and
Internet Service Providers, to users.

% TODO: How do I connect the next sentence neatly to the previous
% paragraphs? Especially when the previous paragraphs will go on to
% talk about all the interested parties at length.

% TODO: Look into billionts for citations about regulator interest.

Broad and longitudinal
measurements of users' Internet reliability in different circumstances
can identify vulnerable networks and their challenges, can inform
which enhancements can help these networks improve reliability, and
can evaluate the efficacy of deployed enhancements.

% Measuring
% Internet reliability at the individual user level will therefore prove invaluable in assessing
% the current state of Internet connectivity and in developing future
% enhancements.

% By detecting outage events and their duration, we
% can reason about a particular user's Internet reliability and compare
% it to other users' reliability across geography, ISPs, and
% media-types.

% Talk about how Internet reliability is difficult and who is
% interested. FCC. Even ISPs (Cite the "Nevermind" paper by Nick
% Duffield which claims that ISPs are typically reactive and wait for
% customers to call). Common users.

Yet, we lack authoritative measures of last-mile Internet reliability. The first step towards measuring users' Internet
reliability is to detect \emph{Internet outages}---events that prevent
users from communicating over the Internet. Since we expect outage
events to be rare, detecting them requires broad and longitudinal
measurements of individual users. However, such measurements of
Internet reliability at the individual user level are challenging to
obtain accurately and at scale. The next step towards measuring users'
reliability is to segregate detected outages into categories
that suggest their cause. Once outages are categorized in this
manner, we can estimate the reliability of an ISP by considering the
subset of outages that affected solely that ISP.

Existing techniques that measure individual-user-level outages can be
grouped into on-premises outage detection techniques and remote probing-based
outage detection techinques. On-premises techniques, such as
RIPE Atlas~\cite{atlas}, SamKnows~\cite{samknows}, and
BISmark~\cite{bismark-main-bib}, measure diverse aspects of
users' Internet connections, but
measure relatively few users. These techniques 
deploy dedicated hardware at user premises that continuously conduct ping,
traceroute, DNS measurements etc.; some of these
measurements can be used to infer Internet connectivity problems. Whereas hardware
based techniques have fundamental scaling difficulties owing to
manufacturing and deployment costs, hundreds of millions of
IP addresses respond to active probes~\cite{timeouts}. Since many
residences have at least one device with a public IP address ~\cite{cgn-imc16}
(typically the home router), these IP addresses can be probed
remotely, from 
vantage points that we control, to measure their connectivity. Thunderping~\cite{pingin} and Trinocular~\cite{trinocular} adopt this
approach to outage detection, taking a
complementary approach to on-premises techniques: they focus upon
measuring only connectivity but do so for many users. Since these
techniques can send probes remotely from servers under their control, without requiring any user
involvement, they are able to detect outages across time
as well as across the IPv4 address space.
% However, existing techniques do
% not study latency and therefore do not identify performance
% degradation resulting from high delay.
However, probing-based remote outage detection techniques can
make false inferences about outages when some scenarios
occur~\cite{timeouts, addrchange-reasons}. Further, existing
techniques have not 
attempted to categorize detected outages by their likely cause.
% However, measuring residential
% outages is challenging because of the scale: there are millions of
% residential links to measure. 
% Second, users can voluntarily power
% down their home Internet equipment and it is challenging to
% distinguish between voluntary user shutdowns and an outage at the last-mile link.
 
% ; even multihomed last mile links for business connectivity
% often share the same upstream hardware, representing a single point of 
% failure~\cite{twcable-business-web}. 
% Last-mile links lack the redundancy of the Internet's core;
% thus an outage of the last-mile link is likely to cut off users'
% Internet connectivity altogether.

% The second challenge is that it can be hard to distinguish
% between an outage at the last-mile link and voluntary user shutdowns
% of their Internet connections.
% We do not have a good understanding of the reliabity of Internet
% connectivity for end-users. Understanding the reliability of Internet
% connectivity for end-users requires understanding the reliability of
% the \emph{last mile link} connecting an end-user to the Internet. This
% is because the core of the Internet is designed to be redundant but
% last mile links typically are not.
%TODO: Perhaps borrow sentence from enduser.tex about business last-mile links.


% Thesis statement comments

% Older versions of thesis statement:
% \emph{For any end-host with a publicly assigned IP address that has the ability to respond to active probes, it is possible to remotely isolate accurately determine connectivity problems experienced by that end-host's last mile link.}
% \emph{For any end-host with an IP address that has the ability to
% respond to active probes, it is possible to remotely detect outages
% experienced by that end-host's last mile link.}

% Bobby's comments: Send traceroutes, but also to related addresses. What happens when all addresses change en-masse? Can we somehow identify such instances?
% Neil suggested that I should replace end-host with 'Internet accessible device'. Can I assert that a device with a public IP address is by definition Internet accessible.
% \emph{It is possible to remotely and accurately detect outages experienced by any device with a public IP address that typically responds to active probes.}

% Come up with definition of accuracy of outages.

% and use it to compare reliability across ISPs, media-types and
% geographical areas

I argue in this thesis that 
\emph{It is possible to remotely and accurately detect substantial outages
  experienced by any device with a stable public IP address that typically
  responds to active probes and use these outages to compare
  reliability across ISPs, media-types and geographical areas.} To demonstrate the thesis, I make the following
initial contributions:

% TODO: 
% It is possible to remotely and accurately estimate the reliability of
% an ISP's customer's device so long as the device has
% a stable public IP address that typically
%   responds to active probes


In Section~\ref{sec:related}, I define terms in the thesis statement, place the problem of outage detection
at the individual user level in the context of related work, and describe the challenges
that probing-based remote outage detection techniques will need to address. These
techniques study outages by sending active probes (such as ping's echo
requests) and use probe responses to infer outages. They assume that a
response to an active probe indicates a working path to the probed
user device and that lack of response is indicative of failure. I
illustrate two scenarios where this assumption can be
invalid, leading to potentially false outage inferences.
 % I
% illustrate two potential scenarios where this assumption is
% invalid---when responses are delayed beyond the prober's timeout and
% when the probed address is dynamically reassigned. I also highlight why
% Internet outages voluntarily caused by the
% residential user need to be segregated.
% I also describe the challenge that remote probing-based techniques face
% in detecting outages specifically in the last-mile.

In Section~\ref{sec:timeouts}, I investigate the prevalence of delayed
probe responses due to early timeout. The
lack of response to an active probe isn't always indicative of loss;
for example, when
responses are delayed beyond the prober's timeout, the response
eventually arrives but the prober would never see the response because
it timed out too early. I report how commonly responses are delayed
beyond timeouts in
networks around the world and propose techniques to mitigate this
problem. 
% Decouple probe retransmission
% and loss. Possibly identify that only cellular guys have long delays?
% Also, this will ensure that we trigger retransmission upon high
% latency/loss and actually identify loss as loss and high latency as
% high latency.

In Section~\ref{sec:addr_change}, I investigate how dynamic addressing can
lead remote probing-based outage detection techniques to make false inferences about outages and techniques to
mitigate these false inferences. My approach to mitigating
these false inferences is rooted in building a model that characterizes
the probability that a dynamic address is reassigned at any point of time. I describe preliminary
work which shows the feasibility of building such a model and detail
proposed work to gather new datasets that can feed into the
model. I will ultimately use this model to find candidate stable Internet
addresses for probing.

In Section~\ref{sec:last_mile}, I discuss an approach to determine
which of the detected outages are consistent with the failure of an
ISP's operated equipment, and which outages can be attributed to other
causes such as power outages or users voluntarily shutting down their home
Internet equipment. My approach is to probe addresses that are related
to the address that is already being probed. I propose the use of multiple criteria
to find related addresses such as geography, ISP, and network
topology. I describe how we can then use simultaneous outages of these
related addresses (or the lack thereof) to categorize outages and
estimate Intrenet reliability along various dimensions.

% We show in this proposal that confounding factors can cause RODWAP
% techniques to sometimes
% infer false outages. 

% However, we argue that RODWAP techniques can
% be used for accurate outage detection by identifying and mitigating
% confounding factors. 
 % This has
% been the basis of existing active probe based techniques that detect
% loss, and thereby outage events, such as Thunderping~\cite{pingin} and
% Trinocular~\cite{quan2013trinocular}.

% In spite of , we argue that it is possible to remotely detect
% outages on the last mile link using active probes for any end-host
% with an IP address that responds to active probes.

% We investigate potential causes that
% would lead existing active probe based outage detection approaches to
% falsely infer loss and describe proposed work to mitigate detection of
% false loss in Section~\ref{sec:timeouts} and
% Section~\ref{sec:addr_change}. We also propose 

% We show in this proposal that probe-loss need not always be due to
% lossy links.


% \begin{itemize}
%   \item{\bf{False probe-loss inference due to early timeout:}} 
%     Traditionally, active probe based approaches time out probes after a few seconds. Responses that arrive after the timeout will be reported as lost. When this happens, existing techniques would confuse high delay with probe-loss.
%   \item{\bf{False probe-loss inference due to IP address change:}}
%     Consider an IP address that was previously responsive. If the host to whom that IP address was assigned changed its IP address as a result of dynamic addressing or mobility, and if the probed IP address is not reassigned to any host, then echo responses will cease to arrive and existing techniques would infer false probe-loss.
% \end{itemize}

% After analyzing causes of false probe-loss, we will investigate techniques to remove false probe-loss. Once we remove false probe-loss, we will be left with all instances of true probe-loss and probe-delays. This will give us the ability to identify that \emph{some} link on the path to the destination is experiencing connectivity problems. However, since we are specifically interested in identifying connectivity problems on the last-mile link we require another step. In this step, we will conduct TTL-limited probing from multiple vantage points to identify which link exhibits connectivity problems.




\chapter{Measuring Residential Internet
  Reliability: a Primer}

\label{cpt:bg}

In this chapter, I provide background and definitions related to the
thesis statement. Then I present related work in measuring Interne
outages in general, and residential Internet outages at the individual
user level in particular. I discuss probing-based techniques to detect
outages remotely in detail and illustrate scenarios where they could
make false inferences about outages. I also illustrate that measuring
Internet reliability using detected outages has nuances and show how
some classes of detected outages need to be treated differently,
depending upon the Internet reliability measure under consideration.

% In this dissertation, I focus
% upon how common outages are and therefore use the \emph{rate} at which
% outages occur. Another metric is the proportion of total measured time detected as outage events.


\section{Background and definitions}

Intuitively, a reliable Internet connection is one that works
continuously. In other words, it experiences no outages. 

Measuring Internet reliability, therefore, necessitates measuring
Internet outages and then using measured outages in a metric that
represents some property of outages. Depending upon the application,
the appropriate outage metric may vary.

The goal of this dissertation is to provide broad, longitudinal, and accurate measurements of
Internet reliability across ISPs, media-types, and geographic
locations in a variety of circumstances. Such measurements can help
users choose from their available Internet options and can inform ISPs
about potential problems in their networks. To achieve this goal, I
propose the following thesis and define terms in the thesis as follows:



\emph{It is possible to remotely and accurately detect substantial outages
  experienced by any device with a stable public IP address that typically
  responds to active probes and use these outages to compare
  reliability across ISPs, media-types and geographical areas.}


\begin{itemize}

\item {\emph{Device with a stable public IP address}: This is a device
    connected to the Internet, like a
home-router, to which an ISP has assigned a public IP address such
that the
assignment is either static, or dynamic in a manner that allows the
duration of dynamic assignment to be estimated.}

\item {\emph{Substantial outage}: I define a substantial outage to be an event where a device
    with an Internet connection is unable to send or receive any
    packets for at least 10 minutes.}

\item {\emph{Accuracy of outage detection}: An outage detection technique is accurate when it
correctly identifies every substantial outage event experienced by an Internet-connected-device, along with its
duration. There are no time-periods when the address
experiences a substantial outage but it goes undetected (false
negatives). Similarly, there are no time-periods classified as
outages when the destination address is able to receive packets from the
Internet (false positives).}

\item {\emph{Reliability}: I define two measures of reliability: one is the raw count of outage
events over measured time and the other is the proportion of total measured time
detected as outage events.} % When estimating a particular ISP's reliability, I
% only use the subset of outages that solely affected that ISP's
% addresses.}
% When estimating the reliability of a geographical area, I
% consider the subset of outages that affected only that geographical
% area.

\end{itemize}

\section{Related work}

The architects of the Internet predicted that network outages could
occur, and designed the Internet to have the ability to route around
outages~\cite{clark-darpa}. As predicted, a variety of factors cause
outages in the Internet, including optical fiber
cuts~\cite{fiber-cuts}, routing and infrastructure
failures~\cite{backbone-failures-1999, ratulbgp}, and
hurricanes~\cite{pingin}.

%TODO: Cite ratulbgp somewhere, network-black-holes, feamster:sigmetrics:failures

Large Internet outages that can affect packets from thousands of
Internet hosts have received attention from the research
community~\cite{censorship-outages, trinocular, hubble, paxson-e2e,
hubble, netdiagnoser, lifeguard, poiroot,
phillipa-outages-mailing-list, california-fault-lines,
delayed-routing-convergence, consensus-routing, routing-e2e-path-perf,
voip-bgp-convergence}. Outages occurring in the Internet's core can
cause Internet path failures; researchers have investigated transient
Internet path failures caused by route
changes~\cite{delayed-routing-convergence, consensus-routing,
routing-e2e-path-perf, voip-bgp-convergence} and longer path failures
caused by infrastructure device outages~\cite{paxson-e2e, hubble,
netdiagnoser, lifeguard, poiroot, phillipa-outages-mailing-list,
california-fault-lines}. Dainotti et~al.~\cite{dainotti-imc11} observe
Internet Background Radiation traffic sent to IPv4 darknets to detect
outages affecting entire countries.

% Other studies detect outages at the
% country-level~\cite{censorship-outages} and at the network prefix
% level~\cite{trinocular, hubble}.

Another class of techniques detects outages at the Internet's edge,
for network prefixes or address blocks, but
does not target outages of individual users' Internet
connections. Hubble studies reachability problems affecting BGP
prefixes~\cite{hubble}. Trinocular detects outages affecting /24
address blocks. Richter
et~al.~\cite{advancing-outage-art} use the observation point of a
large CDN to detect periods of reduced activity from /24 address
blocks consistent with outages. CAIDA's IODA
system~\cite{ioda-project-page} detects outages affecting countries, ASNs, and geographic provinces using three complementary
datasets: BGP updates from Routviews~\cite{routeviews} and RIPE RIS~\cite{ripe-ris}, active probing data
obtained with CAIDA's implementation of the Trinocular methodology,
and IBR data using the technique introduced by Dainotti et~al.~\cite{dainotti-imc11}. 


 % Industry provides some options to
% study failures but they either focus solely on websites that are
% down~\cite{downdetector, outageanalyzer, isitdownrightnow, downforeveryoneorjustme}, or offer services to monitor
% large customer networks~\cite{thousandeyes}. 
However, outages that affect individual users have received comparatively less
attention~\cite{pingin, grover2013peeking, disco, alwayson}. In the rest of this
section, I classify these efforts to detect outages into on-premises
outage detection techniques and remote probing-based outage detection
techniques, and
discuss their approaches and challenges in detail.

% Why do I care only about complete outages and not partial ones?
% Because they are easy to define! :P
% Because they are more likely to be last-mile link. Aha! Yes, so a
% complete outage means we will have the ability to isolate the fault

% We define a link to experience an outage when it experiences peformance
% degradation resulting in unusually high loss and/or delay. When a link
% experiences delay but no loss, we define that link to be \emph{sleepy}
% and we refer to the event as a \emph{sleep}. When a link experiences
% loss but no delay, we define that link to be \emph{lossy} and we refer
% to the event as a loss. When a link experiences complete loss, i.e.,
% all packets on that link are dropped, we define that link to be
% \emph{out} and we refer to the event as an \emph{outage}. Note that by
% definition, every \emph{outage} event is also a \emph{loss}
% event. When a link experiences delay and loss, we define that link to
% be \emph{sleepy-lossy} and we refer to the event as a
% \emph{sleep-loss}. When we speak of a single probe being lost/delayed,
% we will refer to it a \emph{probe-loss} /\emph{probe-sleep}.



% Several studies have tried to detect outages. 

% Find which ones study outages using passive techniques?

% Find which ones study outages at not the last-mile

% Find which ones study outages at the last-mile but using dedicated
% hardware (Ark, RIPE Atlas)

\subsection{On-premises outage detection techniques}
% The redundancy present in the core of the Internet is
% mostly absent in residential networks, owing to the high cost of
% deploying redundant last mile links. Even multihomed last mile links for business connectivity
% often share the same upstream hardware, representing a single point of 
% failure. Residential link failures directly impact end-users and as a result,
% are of interest to service providers, policy makers, and the end-users themselves.

% Most residential end-users today lack the means to understand the
% reliability of their Internet connectivity over time, and of comparing
% reliability across competing ISPs. They have to rely upon speedtest
% tools which can offer estimates of connectivity over a few seconds but
% not over longer timescales. 
Recognizing the need for long term measurements of residential
Internet performance, policymakers such as the FCC from the U.S., and
Ofcom from the U.K. have deployed the SamKnows hardware
platform~\cite{samknows} inside residences to measure residential
Internet connections continuously by performing active and passive
measurements and reporting their results to users, ISPs, and policy
makers. RIPE NCC, the European RIR, has pioneered the RIPE
Atlas~\cite{atlas} project and Sundaresan et al. the BISmark
project~\cite{bismark-main-bib}, which also study user connectivity
using dedicated hardware measurement devices on user
premises. On-premises techniques can also use measurements from
software deployed on user machines: the DIMES project~\cite{netdimes}
and DASU are two notable examples~\cite{Dasu:NSDI2013}.

% TODO: I don't like the "as done in the DIMES project" above.

% I mention this later, so no need to talk about it now. 
% ; however, this approach
% is not well suited to detecting outages since the DIMES software is
% often installed on laptops~\cite{dhcp-dimes}.

% To
% offset some of the scalability costs, on-premises outage detection systems can also be software-based 

% TODO: Consider if any of these techniques needs to be described in
% additional detail
% Disco~\cite{disco} uses Kleinberg's burst detection to detect
% events where many RIPE Atlas probes disconnect from
% their infrastructure in a correlated manner.


% In essence,
% they are also probe-based outage detection techniques, in that the
% absence of any probes indicates an outage; however, they are
% on-premises and not remote.
Hardware-based approaches can offer accurate reports about
Internet connectivity since the hardware devices are designed to make
measurements continuously as long as they are powered. These techniques have the
ability to perform a range of other measurements such as DNS anycast
tests that can identify which instance of a root-server is closest,
and even passive measurement of the websites that users
access. However, these approaches are fundamentally limited in scale
since their hardware is expensive, distributing the hardware to users
is time consuming, and convincing users to keep their hardware running
is challenging. For example, the RIPE Atlas project, which began in
2010 and has been continuously expanding across the world, has fewer than 10,000 probes that are currently making measurements, out
of more than 15,000 distributed probes.

While some of these costs can be offset by utilizing measurements from
deployed software on user systems~\cite{netdimes, dhcp-dimes, Dasu:NSDI2013} or using a combination of hardware
and software measurements~\cite{IMC2014-Broadband-bischof}, deploying software widely remains
challenging. Separating user behavior, such as turning off their laptops, from
Internet outage events presents additional challenges for these techniques~\cite{dhcp-dimes}.
 
\subsection{Probing-based remote outage detection techniques}

% TODO: Talk about Zmap. It cannot do adaptive probing, being
% stateless, and hence cannot be used for individual outage detection.

Probing-based remote outage detection techniques can detect
connectivity problems remotely through active probing from servers
under reseacher control. Though this approach will prevent 
certain types of measurements, such as DNS anycast tests, it can measure
Internet connectivity for individual users at scale. However,
existing techniques can infer false outages in some scenarios as I
illustrate next.

Probing-based remote outage detection techniques study connectivity problems by
sending active probes (such as ping's echo requests) and use probe
responses to infer connectivity problems. For example, an
echo-response from the end-host indicates that its network connection
is working. If a previously responsive destination ceases to respond
to probes, current techniques infer that the destination could be
experiencing connectivity problems. Thunderping~\cite{pingin},
Trinocular~\cite{trinocular}, and Zmap~\cite{durumeric2013zmap}, have
used this technique to detect outages, albeit at different scales. I
discuss each approach in detail next.

\subsubsection{Trinocular detects failures of /24 address blocks}

Trinocular pings addresses in ~4M /24 address blocks and
uses the responses to detect Internet outages affecting entire blocks. Using historical
data from the ISI census~\cite{census-survey}, it models the responsiveness of
blocks and finds subsets of addresses within each block that are
likely to respond to pings. The system pings a few of these addresses
from each block at random and probes them in 11-minute
rounds. Trinocular then employs Bayesian inference to reason about
responses from blocks. When a block's responsiveness is lower than
expected, Trinocular probes the block at a faster rate and eventually
detects an outage when the follow-up probes also indicate the block's
lack of Internet connectivity.

\subsubsection{Thunderping detects failures of individual addresses
  during severe weather}

Thunderping pings
sampled addresses from multiple ISPs in geographic areas in the United
States. Originally designed to evaluate how weather affects Internet
outages, the system uses Planetlab vantage points to ping 100 IPv4
addresses from multiple ISPs in U.S. counties with active
weather alerts. Each address is pinged from multiple Planetlab vantage
points (at least 3) every 11 minutes, and addresses in a county are
pinged six hours before, during, and after a weather alert for that
county. 


\subsubsection{Zmap was used to study Internet outages during
  Hurricane Sandy}

Zmap is an active probing technique designed to send packets of a
specified type (such as ICMP echo) to all IPv4 addresses at
very fast speeds (under an hour in 2013~\cite{durumeric2013zmap},
under 5 minutes today~\cite{zippier-zmap}. A key to these speeds is that the
Zmap tool chooses to not store state at the prober; instead, response
packets are matched with sent ones by encoding destination-specific data
in the sent packets. By using cyclic generators, Zmap probes
destination addresses in a random order, reducing probing burden on
individual ISPs. However, Zmap's decision to not store state comes
with a trade off: probe retransmissions upon the detection of a lost
probe is difficult. The Zmap
tool was used to detect Internet outages during Hurricane
Sandy~\cite{durumeric2013zmap}. % However, finding smaller Internet failure
% events with the Zmap tool is challenging.

% Since outages are infrequent and can affect small parts of
% the address space, finding them would require running repeated Zmap
% scans of the entire address space on the order of minutes. Even if this is technically feasible,
% it remains an open problem whether such an aggressive probing scheme
% is warranted.

% The following was from related work in corrfails, don't thin it's
% necessary here.
% The key difference of this work from Trinocular is that we do not assume
% that correlated failures span entire /24-address blocks; instead, we
% look for correlated failures of addresses related by geography and
% ISP. While we share Trinocular's intuiton that dependent events will
% affect related addresses, Trinocular's notion of relatedness is solely
% that of belonging to the same /24. With the IPv4 address space
% breaking up, we hypothesized that addresses in disparate /24s may be
% affected by a correlated failure event. Further, a power outage may
% affect a few addresses but from several different ISPs.

% TODO: I probably don't need this part about scaling at all. 
% \subsubsection{Can scale, but not indefinitely}

% While probing-based outage detection techniques can scale to probing hundreds
% of thousands of addresses, they cannot scale indefinitely. Very high
% probe volume can cause traffic to be viewed as malicious and can
% result in probes getting blacklisted and in abuse
% reports~\cite{durumeric2013zmap}. Further, high probe volume increases
% the state that needs to be stored by the prober. While probing
% schemes like Zmap have circumvented this problem by not storing
% state~\cite{durumeric2013zmap}, adaptive retransmission to confirm a
% suspected outage requires the storage of state.

% Trinocular is an outage detection system that employs active
% probes to detect outages for entire /24 prefixes. It uses historical
% measurements to 

% Thunderping~\cite{pingin}, detects last-mile
% link outages for individual residential links during times of predicted severe weather
% conditions using pings, and correlates outages with weather
% conditions. The US National Weather service issues severe weather alerts for areas that are
% likely to experience conditions of severe weather; Thunderping uses
% these alerts to select geographic areas to study. Using
% Maxmind, a popular geolocation service, it then finds IP addresses in
% these geographic areas and pings these addresses from multiplee
% PlanetLab vantage points before, during, and after the weather
% event. We use the results of these pings to infer outages
% and correlate them with observed weather conditions to measure
% the effect of weather upon residential Internet connectivity.


% \begin{figure}[tb]
% % \centering
% \begin{center}
% \includegraphics[width=3in]{figs/pingin_real_deal_v13}
% \end{center}
% \caption{\label{fig:thunderping} Thunderping detects outages in
%   last-mile links during times of predicted severe weather. It uses
%   weather alerts from NOAA to find areas that are likely to be
%   affected by severe weather. It then pings IP addresses in those
%   areas before, during and after the weather alerts from multiple
%   PlanetLab vantage points and uses the
%   results to infer outages.}
% \end{figure}

% \subsubsection{Improving the accuracy of remote probing based outage
%   detection techniques}

% Can measure reliability inaccurately
\section{Probing-based remote outage detection techniques can
be inaccurate}

% I define a probed destination address to undergo an ``outage'' event
% when the address is unable to send or receive any Internet packets. The ideal outage detection technique should be capable of
% identifying every outage event, along with its
% duration. There should be no time-periods when the destination address
% experiences an outage but the outage is undetected (false
% negatives). Also, there should be no time-periods classified as
% outages when the destination address is able to receive packets from the
% Internet (false positives).

Probing-based remote outage detection techniques can infer false
negative and false positive outages as a consequence of their foundational
assumption: that a response to an active probe indicates a working path to the probed
IP address and that lack of response is indicative of
failure. False negatives can occur when the probe rate
to a destination address is low, so that very short outages
experienced by the address go undetected. For example, with
Thunderping's probing scheme of sending a ping every 11 minutes from
each of its vantage points to a destination address~\cite{pingin}, it is possible that an outage lasting
shorter than 11 minutes is not observed by each vantage
point. Increasing the probe rate can limit the maximum duration
of these false negative outages; remote probing based outage detection
techniques must tradeoff the rate
with which they probe a given destination and the duration of the
longest outage that they can fail to detect. 

While false negative outages can be controlled by probing faster,
false positive outages pose a potentially larger accuracy problem. Current
techniques can make false positive inferences about
outages in the following scenarios:

% \begin{itemize}

% \item{\bf{Confusing delay with loss:}}

\subsection{Confusing delay with loss}

Traditionally, active probe based approaches time out probes after a
few seconds. Thunderping~\cite{pingin} and
Trinocular~\cite{trinocular} time out probes after a few
seconds. Responses that arrive after the timeout will be reported as
lost. When this happens, existing techniques would infer loss though
the responses are in fact merely delayed. Chapter~\ref{cpt:timeouts}
presents a measurement study on probe response latencies in networks
around the world and discusses approaches to disambiguate delayed
probes from lost probes.

% TODO: Raise whether extreme delay is an outage.

\subsection{Making false inferences about outages due to dynamic
      addressing}
% \item{\bf{Making false inferences about outages due to dynamic
%       addressing:}}

 Consider an IP address that was previously responsive. If the host
to which that IP address was assigned changed its IP address as a
result of dynamic addressing, and if the probed IP address is not
reassigned to any host, then echo responses will cease to
arrive. Existing techniques would thus infer false probe-loss and
consequently, false outages. Consider an alternate scenario where the
probed IP address has an outage. Suppose that at some point during the
outage, the IP address is reassigned to some other end-host which
responds to probes. Existing techniques would infer that the arrival
of responses signals the end of the outage and would infer that the
outage ended prematurely.  I address how to mitigate false inferences
due to dynamic address reassignment in Chapter~\ref{cpt:addr_change}.

% \item{\bf{Some outages can falsely lower inferred reliability}}
% When analyzing ISP-level or media-type-level reliability, our
% reliability inferences for an ISP should only be based upon outages that
% affected solely that ISP's customers. However, power outages and
% undersea cable cuts can result in outages to many ISPs'
% customers. Also, users sometimes choose to voluntarily shut down their home Internet
% equipment~\cite{grover2013peeking}. If a probing-based remote outage detection technique happens to
% measure an address during these times, probes sent to that address
% will cease to arrive, leading to the inference of an outage. When
% measuring ISP-level or media-type-level reliability, these outages
% must be filtered.

% \end{itemize}

% the rest of the proposal.
 % false probe-loss inference due to early timeout in
% Section~\ref{sec:timeouts} and false probe-loss inference due to IP
% address change in Section~\ref{sec:addr_change}.

\section{Analyzing Internet reliability using detected outages}
% depending upon the question under investigation

Internet reliability can be measured along several dimensions,
depending upon the application. For example, users who require the
Internet for work, or who use health monitoring equipment that needs a
continuously working Internet connection~\cite{ideal-life,
remote-health-elderly}, may be interested in finding which ISP and
media-type in their area provides the most reliable Internet
connection. A per-ISP or per-media-type reliability metric would be
appropriate for this application since these metrics can facilitate
comparisons.

Another application of Internet reliability could be the
identification of geographic regions and challenging conditions with
particularly poor connectivity. A per-region reliability metric could
allow policymakers and ISPs to identify problematic areas and drive
Internet infrastructure deployment in such areas.

Reliability can also be measured along various combinations of ISP,
media-type, geographic region, and challenging conditions. Such
measures can help find the most reliable ISP and media-type for a
geographic region that is particularly susceptible to a challenging
weather condition (such as blizzards, for example). Important
infrastucture in those areas can then use the most reliable ISP and
media-type combination.

\subsection{Not all outages are relevant to Internet reliability
measures}

Even after mitigating errors in outage inferences due to high latency
and dynamic address reassignment, some false outages may
remain. 

Additionally, the set of detected outages that should be considered in
reliability metrics can vary depending upon the application. Here are
three potential applications with different requirements:

\begin{itemize}

\item{Suppose the goal is to measure the effectiveness of broadcasting
    critical information (such as severe weather alerts or Amber
    alerts) over the Internet. An Internet reliability metric, such as
    the rate of Internet outages over time, offers a sense of how many
    users in each ISP or geographic region can be reached through such
    a broadcast. For this application, all Internet
    outages---including outages due to users turning off their home
    Internet equipment---should be represented in the metric. }

\item {When comparing Internet reliability across geographic regions,
    perhaps to identify areas with particularly poor connectivity, outages due to
    user behavior should not be considered in the reliability
    metric. Grover et al. report that users sometimes voluntarily
    power off their home Internet
    equipment~\cite{grover2013peeking}. Probing-based techniques would
  detect such instances as Internet outages since a previously
  responsive address ceases to respond to probes. Without accounting
  for such outages, we may overestimate the outage rate in a region.}
 
\item {When comparing reliability across ISPs, the reliability metric
should ideally only consider outages that each ISP was responsible
for. Doing so ensures that ISPs offering services in challenged areas
do not have their reliability lowered by events such as power outages
or user behavior.}

\end{itemize}

\subsection{Estimating how a challenging condition affects Internet
  reliability}

Suppose we wish to assess the effect of a challenging condition or environment---like
the presence of a thunderstorm---upon the Internet reliability of a
group of addresses. This group of addresses could be a set of
addresses that share some relationship to each other: they could
belong to the same ISP, media-type, geographic region etc. Such an
assessment would help identify areas and networks that are
particularly prone to Internet connectivity problems in certain types
of weather. Chapter~\ref{cpt:weather} describes how to perform these assessments.

\subsection{Categorizing outages by their potential cause}

Since the set of detected outages that should go into a reliability
metric varies by application, depending upon the outage's cause, there
is a need to classify detected outages. Chapter~\ref{cpt:corrfails}
discusses a technique that uses simultaneous failures of related
addresses to achieve this classification.

\chapter{When should probes time out?}

% \section{Understanding false probe-loss inference due to early
%   timeout}
\section{Mitigating false inferences due to early timeout}
\label{sec:timeouts}

In this section, I describe how probe responses delayed beyond
timeouts used by current probing-based techniques can lead to false
probe-loss inferences, and thereby to false outage inferences. I 
describe work with colleagues which
investigated the prevalence of delayed responses in the
Internet. Using these results, I propose an approach that will minimize responses
delayed beyond timeouts.

\subsection{Challenges in selecting a timeout for probing techniques}

Conventional wisdom suggests that active probes on the Internet should timeout
after a few seconds. The belief is that after a few seconds there is a very
small chance that a probe and response will still exist in the
network. Once a probe times out, the prober can free the state
associated with the probe, thereby reclaiming memory.

Conventional wisdom also suggests that a single timed out probe is
insufficient to reason about end-host failures, due to potential random loss on
the Internet. % When a probe experiences a timeout, it could be because
% the end-host's last-mile link is \emph{out}, but it could also be because the
% probe was lost elsewhere along the path.
For most probing systems, any timed out active probes are followed up with
retransmissions to increase the confidence that a lack of response is due to
an outage event and not due to random loss on the Internet. These followup probes will also have a timeout that
is generally the same as the first attempt. 

Setting correct timeouts is critical for
probing-based remote outage detection techniques. These techniques infer outages
based upon lost probes and probe response loss is
dependent upon the prober's timeout. Additonally, since probe timeouts trigger followup probes, setting appropriate
timeouts is vital to these techniques. However, choosing an appropriate timeout is
challenging. Selecting a timeout value that is too low will ignore delayed
responses and might add to congestion by performing retransmissions to an
already congested host. Timeout values that are too high will delay
retransmissions that can confirm an outage. In addition, too-high timeouts
increase the amount of state that needs to be maintained at a prober, since
every probe will need to be stored until either the probe times out,
or the response arrives.

% Internet performance monitoring systems use a wide range of probe
% timeouts. On the
% shorter side, iPlane~\cite{iplane} and Hubble~\cite{hubble} send ICMP echo requests with a 2 second
% timeout. iPlane declares a host unresponsive after one failed retransmission. Hubble waits two minutes after a failed probe then retransmits probes six times and finally declares reachability with traceroutes. On the longer side, Feamster
% et al.~\cite{measuring-effects} used a one hour timeout after each probe. However,
% they chose a long timeout to avoid errors due to clock drift between their
% probing and probed hosts; they did not do so to account for links that have
% excessive delays. PlanetSeer~\cite{planetseer} assumed that four consecutive
% TCP timeouts (3.2-16 seconds) indicates a path anomaly. 

Outage detection systems such as Trinocular~\cite{trinocular}
and Thunderping~\cite{pingin} tend to use a 3 second timeout for active
probes because it is the default TCP SYN/ACK
timeout~\cite{rfc1122}. Both techniques will not infer outages if a
single response is delayed beyond the timeout, since they send
follow-up probes to confirm suspected outages. However, if a series of
responses are delayed beyond the timeout, both techniques can
potentially infer false probe-loss and therefore, false
outages. % Ideally, we would like to detect these events as \emph{sleep}
% events, since the probe responses are delayed, not lost.

\subsection{Investigating the prevalence of delayed responses}

Here, I describe work with colleagues that measured how frequently responses
to active probes are delayed beyond timeouts set by existing
approaches. We began by studying ping latencies from Internet-wide surveys~\cite{census-survey} conducted by ISI,
including 9.64 billion ICMP Echo Responses from 4 million different IP
addresses in 2015, and identified addresses that are particularly likely
to be subject to high delay.  We then \emph{verified} the high latencies
by repeating measurements using other probing techniques, comparing the
statistics of various surveys, and investigating high-latency
behavior of ICMP compared to UDP and TCP.  Finally, we
explained these distributions by isolating satellite links,
considering sequences of latencies at a higher sampling rate,
and classifying a complete sample of the Internet address
space through a modified Zmap client. In this proposal, I will highlight
the most relevant results; detailed analyses are available in our IMC
2015 paper~\cite{timeouts}.

\subsubsection{ISI survey data reveals minute-long latencies}
% \subsubsection{ISI survey data reveals minute-long latencies in recent
% surveys}

ISI has conducted Internet wide
surveys~\cite{census-survey} since 2006. Each survey includes pings sent to approximately 24,000 /24
address blocks, meant to represent 1\% of all allocated IPv4
address space.  Once an address block is included, ICMP echo
request probes are sent to all 256 addresses in the selected
/24 address blocks once every 11 minutes, typically for two
weeks. Though the probing scheme for these surveys set a timeout
threshold of 3 seconds~\cite{census-survey}, ISI records every
received packet, including potential responses to probes that were delayed
beyond the prober's timeout. We determined which received packets were
valid responses and included them in our latency analysis.

We then analyzed the ping latencies of all pings obtained
from ISI's Internet survey datasets from January and February 2015 to find reasonable timeout values. For each IP address, we found the 1st, 50th, 80th,
90th, 95th, 98th and 99th percentile latencies. We then found the 1st, 50th,
80th, 90th, 95th, 98th and 99th percentiles of all the 1st percentile latencies. We repeated this for each
percentile and show the results in Table~\ref{tbl:grand_2015}.

\begin{table}[tb]
  % \begin{center}%
    \begin{small}%
      \hspace{-0.06in}%
  \begin{tabular}{l@{\hspace{0.5em}}r|rrrrrrr}
    &\multicolumn{8}{c}{\textbf{\% of pings}} \\
    && \hdr{1\%} & \multicolumn{1}{c}{\textbf{50\%}} & \hdr{80\%} & \hdr{90\%} & \hdr{95\%} &
    \hdr{98\%} & \hdr{99\%} \\\cline{2-9}
    \multirow{7}{*}{\rotatebox[origin=lb]{90}{\textbf{\% of addresses}}} & 
    \textbf{1\%} & 0.01 & 0.03 & 0.04 & 0.07 & 0.10 & 0.13 & 0.18\Tstrut \\
%    \cline{2-9}
    &\textbf{50\%} & 0.16 & 0.19 & 0.21 & 0.26 & 0.42 & 0.53 & 0.64 \\
%    \cline{2-9}
    &\textbf{80\%} & 0.19 & 0.26 & 0.33 & 0.43 & 0.54 & 0.74 & 1.21 \\
%    \cline{2-9}
    &\textbf{90\%} & 0.22 & 0.31 & 0.42 & 0.57 & 0.84 & 1.61 & 3\bb \\
%    \cline{2-9}
    &\textbf{95\%} & 0.25 & 1.42 & 2.38 & 3\bb & 5\bb & 9\bb & 15\bb \\
%    \cline{2-9}
    &\textbf{98\%} & 0.30 & 1.94 & 4\bb & 6\bb & 12\bb & 41\bb & 78\bb \\
%    \cline{2-9}
    &\textbf{99\%} & 0.33 & 2.31 & 4\bb & 8\bb & 22\bb & 76\bb & 145\bb \\
    \end{tabular}
    \end{small}
    % \end{center}

\vspace{\baselineskip}

    \caption{Minimum timeout in seconds that would have captured c\% of pings from r\% of IP
      addresses in two ISI survey datasets from early 2015 (where r is the row number and c is
      the column number).}
\label{tbl:grand_2015}
\end{table}

The 1st percentile of an address's latency will be close to the ideal latency that its link
can provide. We found that the 1st percentile latency is below 330ms for 99\%
of IP addresses: most addresses are capable of
responding with low latency. Further, 50\% of pings from 50\% of the
addresses have latencies below 190ms, showing that latencies tend to
be low in general. 

However, we see that a substantial fraction of IP addresses also have
surprisingly high latencies. For instance, to capture 95\% of pings from 95\%
addresses requires waiting 5 seconds.  Restated, at least 5\% of
pings from 5\% of addresses have latencies higher than 5 seconds. Thus, even
setting a timeout as high as 5 seconds will infer a false loss rate of 5\%
for these addresses. At the extreme, we see 1\% of pings from 1\% of addresses
having latency above 145 seconds!


\begin{figure*}
  \begin{center}
  \includegraphics[width=\textwidth]{figs/pctile_var_over_time_for_proposal}
  \end{center}
  \caption{\label{fig:pctile_var_over_time}Top: Minimum timeout
    required to capture the $c^{th}$ percentile latency sample from
    the $c^{th}$ percentile address in each survey, organized by time.
    Each point represents the timeout required to capture, e.g., 95\%
    of the responses from 95\% of the addresses.}
\end{figure*}

\subsubsection{ISI survey data shows that high latencies are a recent phenomenon}

These unusually high latencies led us to perform a longitudinal
analysis of ISI's surveys and investigate if these high latencies have
occurred consistently over time. We selected the minimum timeouts that would
have captured 95\% of pings from 95\% of addresses, 98\% of pings from
98\% of addresses, and 99\% of pings from 99\% of addresses and show
these values in each survey from 2006 to 2015 in
Figure~\ref{fig:pctile_var_over_time}. We observe that the minimum
timeout that would have captured 95\% of pings from 95\% of addresses
increased from 2s in 2011 to 5s in 2015, and the value for 99\% of
pings from 99\% increased from 20s to 140s during the same
period. These results suggest that high latencies are a relatively
recent phenomenon. 



\subsubsection{Zmap data shows that high latencies are more prevalent
in some ASes than others}

Some of the latencies in Table~\ref{tbl:grand_2015} are so high that
we considered if they could be artifacts of ISI's probing scheme. The
ISI survey results are derived from repeated pings to 1\% of the
routed Internet. The Zmap project~\cite{durumeric2013zmap} offers a different
perspective, sending a single probe to the entire Internet.

Though Zmap is stateless and does not measure latencies by default, we
modified Zmap to measure latencies. We did so by extending the ICMP
probing module in the Zmap scanner to embed the probe send time into
the echo request. When an echo response is received, Zmap provides
both the time of the response as well as the embedded probe send time,
allowing us to estimate the latency. Zmap has performed these scans
since April 2015.

\begin{figure}[tb]
% \centering
\begin{center}
\includegraphics[width=3in]{figs/grand_zmap}
\end{center}
\caption{\label{fig:grand_zmap}%
Distribution of RTTs for all Zmap scans performed between April to
September 2015. Around 5\%
of addresses have latencies greater than 1s in each scan, and 0.1\% of addresses observed latencies in excess of 75s.
}
\end{figure}

Our first goal was to confirm that Zmap also observed high
latencies. Figure~\ref{fig:grand_zmap} shows the distribution of latencies in 17 Zmap scans conducted between April to
September 2015. Most responses arrive with low latency, having a median latency lower than
250ms for each scan. However, ~5\% of addresses responded with RTTs
greater than 1 second in each scan. Further, 0.1\% of addresses
responded with latencies exceeding 75 seconds in each scan. These
results corroborate the high latencies observed in the ISI data and
demonstrate that typical timeouts would miss a significant fraction of responses.

However, are these high latencies spread randomly across all addresses
in the Internet? Or instead, are some addresses particularly likely to
experience high latencies?
% The former would be the case if core
% routers in the Internet experience congestion, which could potentially
% delay packets for a wide swath of the Internet's addresses. The latter
% would happen if the cause of the high latencies is something to do
% with the last-mile link, so that a few addresses experience higher delay
% owing to some aspect of their last-mile link.
To find how high latencies are distributed across the Internet, we
investigated which Autonomous Systems' addresses are particularly
likely to have high latencies. For this analysis, we used three Zmap
scans conducted in 2015 to identify high latency addresses, conducted
on May 22, Jun 21 and Jul 9. These scans were conducted at different
times of the day, on different days of the week and in different
months. For each of these Zmap scans, we used Maxmind to find the ASN
and geographic location for every address that responded.

\newcommand{\hdrbar}[1]{\multicolumn{1}{c|}{\textbf{#1}}}
\begin{table*}[t]%
  \begin{center}%
  \begin{small}%
  \begin{tabular}{ll|rrr|rrr|rrr}
  % \begin{tabular}{r|rrr|rrr|rrr|rrr}
  % \begin{tabular}{rl|r|rr}
    % & & \multicolumn{3}{c|}{\textbf{April 2015}} &
    & & 
    \multicolumn{3}{c|}{\textbf{May 2015}} &
    \multicolumn{3}{c|}{\textbf{June 2015}} &
    \multicolumn{3}{c}{\textbf{July 2015}} \\ 
    % \hdr{ASN} & \hdr{$>$1s} & \%  & \hdr{Rank} &
    % \hline 
    \hdr{ASN} & \hdrbar{Owner} & 
    \hdr{$>$1s} & \%  & \hdrbar{Rank} &
    \hdr{$>$1s} & \%  & \hdrbar{Rank} &
    \hdr{$>$1s} & \% & \hdr{Rank} \\
    % \hdr{$>$1s} & \% & \hdr{Rank} \\
    \hline 
    26599 & TELEFONICA BRASIL & 
    % 26599 &
    % 2,941,446 & 79.275 & 1 & 
    3.56M & 80.4 & 1 & 
    3.87M & 77.5 & 1 &
    4.20M & 77.0 & 1\Tstrut \\

    26615 & Tim Celular S.A. &
    1.35M & 74.5 & 3 &
    1.42M & 71.5 & 2 &
    1.72M & 71.6 & 2 \\

    45609 & Bharti Airtel Ltd. &
    1.46M & 76.6 & 2 &
    1.21M & 81.0 & 3 &
    1.03M & 79.2 & 3 \\

    22394 & Cellco Partnership &
    0.55M & 73.4 & 8 &
    0.58M & 73.5 & 4 &
    0.63M & 72.7 & 4 \\

    1257 & TELE2 &
    0.67M & 69.5 & 5 &
    0.42M & 65.5 & 9 &
    0.58M & 67.4 & 5 \\

    27831 & Colombia Movil &
    0.53M & 68.8 & 9 &
    0.54M & 64.3 & 5 & 
    0.53M & 62.8 & 6 \\

    6306 & VENEZOLAN &
    0.69M & 77.3 & 4 &
    0.41M & 76.4 & 10 &
    0.40M & 75.7 & 10 \\

    9829 & National Internet Backbone &
    0.57M & 27.6 & 7 &
    0.43M & 30.9 & 7 &
    0.43M & 29.5 & 9 \\

    4134 & Chinanet &
    0.60M & 1.5 & 6 &
    0.38M & 0.9 & 11 &
    0.34M & 0.9 & 11 \\

    35819 & Etihad Etisalat (Mobily) & 
    0.42M & 54.0 & 10 &
    0.43M & 54.5 & 6 &    
    0.45M & 55.8 & 8 \\
  \end{tabular}
  \end{small}
  \end{center}
  \caption{\label{tbl:zmap_asns} Autonomous Systems sorted by the
    addresses summed across three Zmap scans for addresses that observed
    RTTs greater than 1s. The table shows for each AS: the number and
    percentage of addresses with RTT greater than 1s and the rank in that scan.}
\end{table*}


Inspecting the Autonomous Systems and countries of addresses with high latencies
reveals that a majority of them belong to cellular ASes in South
America and Asia, as shown in Table~\ref{tbl:zmap_asns}. AS26599
(TELEFONICA BRASIL), a cellular AS in Brazil, has the most addresses
with latencies exceeding 1s---more than double that of the next
largest AS in each of the scans. The next two ASes, AS45609 (Bharti
Airtel Ltd.), and AS26615 (Tim Celular), are also cellular, and so are
5 of the remaining 7 ASes in the top 10 ASes with the most addresses
with latencies exceeding a second. Also notable is that more than 70\%
of all responding addresses in these ASes had latencies exceeding a
second. 

% Include the next para if you really want to talk about the
% experiments that appear to confirm that we're reaching cellular devices.
% We conducted additional experiments upon some addresses that
% were particularly likely to have high latencies from the ISI dataset,
% and confirmed that 

While the results from the ISI and Zmap datasets reveal that high
latencies exceeding typical timeouts occur in the Internet, they also
show that these latencies are not uniformly distributed across all
addresses. This observation lies at the root of my proposed work to
set timeouts for probe-based remote outage detection systems.

\subsubsection{Proposed work: Set timeouts based upon
  destination addresses}

The wide variation in observed latencies for IP addresses around the
world indicate that probers should set timeout values
based upon the addresses that they are probing. Even a 3s
timeout may suffice for 90\% of addresses in the ISI survey since 90\% of addresses respond
within 3s for 99\% of the pings sent to them. My proposed work is to find expected latency values
associated with the IP addresses that need to be probed, and to set
their timeouts accordingly.
 
I propose to find expected latencies for any IP address on the
Internet by analyzing historical and current ping data, available from
the Zmap project~\cite{censys-icmp}. Zmap has continued to perform
their scans of the IPv4 Internet, averaging one scan per week since
April 2015. For each IP address that has consistently responded to
pings, I expect to have roughly 100 samples. I will calculate expected
latencies for all addresses using their own latencies weighted by the
number of observed samples and will also include latency samples of
other ``related'' addresses. Related addresses can be addresses
belonging to the same /24 network, addresses belonging to the same
ISP, addresses sharing the same last-hop router, addresses from the
same dynamically addressed pools etc; I describe related addresses in
more detail in Section~\ref{sec:last_mile}.

Once I have determined the expected latencies for all IP addresses
that respond to pings in the IPv4 Internet, the next task is to
determine appropriate per-address timeouts based upon the destinations
that need to be probed. Given any address to probe, I will modify the probing scheme to
set timeouts that are just high enough as to capture almost all
responses (say 99.9\%) from that address. Setting adaptive timeouts this way will achieve the
twin goals of capturing most responses while also keeping the state
required at the prober low.

\chapter{Mitigating false inferences due to dynamic addressing}

\label{sec:addr_change}

In this section, I describe how dynamic addressing can lead
probing-based outage detection
techniques to make false inferences about outages and describe
techniques to obtain a list of stable addresses for probing.

Academia and industry often rely on a simplifying assumption that IP addresses 
uniquely identify end-hosts~\cite{p2pfilesharing,p2pavail,sen2004analyzing,sekar2006multi,anomalousdns,kuhrer2015going,xie2005worm,jung2004empirical,fabian2007botnet,stone2009your,andriesse2015reliable,fail2ban,spamhaus,cbl,sorbs}.
This assumption allows researchers to track end host
behavior over time~\cite{anomalousdns, kuhrer2015going, pingin}, or to count participating users in peer-to-peer
systems~\cite{p2pfilesharing,p2pavail,sen2004analyzing}. Many organizations create blacklists of suspicious IP
addresses based on previously observed malicious traffic associated
with those addresses~\cite{fail2ban,spamhaus,cbl,sorbs}. 

When probing-based remote outage detection techniques send probes to an address, they expect that the
address continues to be assigned to the same end-host for the entirety
of the probing duration. Depending upon how a dynamic address gets
reassigned, these techniques can make false inferences about outages in two ways:

\begin{itemize}

\item{\emph{Detecting false outages} Probing-based remote outage detection techniques detect outages
    when a previously responsive address stops responding to
    probes. However, If a dynamic address being probed is
withdrawn from its host and is not assigned to any other host, active probes to the address will no longer
elicit responses. These techniques will infer false
probe-loss, leading them to infer false outages.}

\item{\emph{Detecting false outage duration} These techniques detect outage
    duration by continuing to probe an unresponsive address. When the
    address starts responding to probes again, the outage is inferred
    to end. If a user device  with a
    public dynamic address has an outage and at some point during the outage,
    the dynamic address is reassigned to some other user device
    which responds to probes, probing-based remote outage detection techniques would infer that the outage ended incorrectly.
% For ISPs that use DHCP
%     for address assignment, we would expect dynamically
%     assigned addresses to stick around on the end-host until an
%     outage occurs. However, upon the occurrence of the outage, if the
%     outage is long enough, they can get reassigned to another host,
%     especially, if the outage is longer than the DHCP lease
%     duration. Here, RODWAP techniques can detect the outage itself but
%     can perhaps not detect outages that are long
}

% TODO: CITE http://www.umiacs.umd.edu/~tdumitra/courses/ENEE757/Fall15/papers/Stone-Gross09.pdf
\end{itemize}

My approach to mitigating these false inferences is to analyze how
frequently and for what reasons dynamic addresses are
reassigned. I will use the results of these analyses to build a model
of how likely an inference about an outage using a probing-based
remote outage detection technique is
a false inference caused by dynamic addressing. For example,
preliminary work with colleagues has revealed that some European ISPs change addresses upon
very small outages and are particularly likely to change addresses at certain
times of the day~\cite{addrchange-reasons}. These results will inform
my model to not attempt detection of 
outage duration for these ISPs, and to discard outages
detected at times that are particularly likely to have dynamic address
changes. The model will ultimately yield stable addresses who either
do not undergo dynamic reassignment for months at a time, or who get
reassigned but in a predictable manner. Thunderping limits itself to
probing addresses in the U.S. where dynamic reassignment is
uncommon~\cite{addrchange-reasons}; stable addresses from the
model can help us detect outages in new areas.

% In the rest of this section, we provide background about 

\section{Dynamic addressing background}

An IP address can be used to uniquely identify the end-host it is assigned to
until the end-host's address changes for some reason. The duration of
time that a dynamic IP address continues to be assigned to the same
CPE (Customer Premises Equipment) device depends upon various causes that can induce the assigned IP
address to change. Here, I present techniques used for
assigning dynamic addresses and the events and
agents involved in dynamic address changes.

ISPs often use the Dynamic Host Configuration Protocol
(DHCP)~\cite{rfc2131} for IP address assignment. DHCP issues an IP address to a host for a lease
duration configured by the ISP. The host will try to renew the lease
before it expires, typically half-way into the lease. However,
whether the same IP address is renewed, or a different one is
assigned, depends upon ISP policy.  We speculate that the
typical behavior of ISPs using DHCP is to renew the lease of the
currently assigned IP address, since one of the stated design goals
in the DHCP specification is that a DHCP client should be assigned the same address
in response to each request, whenever possible. Thus, we typically
only expect an ISP using DHCP, to change the address of a CPE, if
something happens to prevent the CPE from renewing its lease (like an outage).
% Further, on
% reboot, the previously assigned address may be reassigned, or
% alternately a new address may be issued, again, as dicated by ISP
% policy.

In some networks, end-hosts connect to an ISP using
point-to-point links. For these networks, the Point-to-Point Protocol
(PPP) first configures and establishes the point-to-point link~\cite{rfc1661}. Next,
a Network Control Protocol (NCP) like the Internet Protocol Control
Protocol (IPCP) configures IP addresses~\cite{rfc1332}. The PPP specification
notes that the link will remain configured for communication until the
link is actively closed down through network administrator
intervention or when an inactivity timer expires.

\subsubsection{Potential dynamic address change causes} 

Next, we identify the reasons dynamic addresses assigned using
the above techniques could change. We classify the following categories of address change:

\begin{itemize}
\item{\textbf{Changes after outages}} If the client is
  disconnected or loses power long enough to fail to renew a DHCP lease,
  its address may be assigned to another; when it returns,
  it may then get a new address. We call such changes
  \emph{outage-caused address changes}.

\item{\textbf{Changes after reboot/reconnect}} While we
  expect addresses assigned through traditional DHCP to change only when the
  outage duration is long enough to prevent lease renewal, addresses
  assigned through PPP can change upon outages of any duration. Any reboot or
  network reconnect event could cause the client
  to forget its prior address and request a new one, or the
  state associated with a connection may be lost.
  We call such
  address changes \emph{reboot-caused address changes}. 
  
  % We make no
  % distinction between very short power outages and reboots.

\item{\textbf{Administrative address changes}} A purpose of
  dynamic address assignment is to allow reconfiguration of
  the network; it is possible that a reconfiguration of the
  DHCP server will force a change to the subnet on which the
  client lies.  We expect such reassignment to be rare.
  
\item{\textbf{Periodic address changes}} We observe that
  some ISPs limit the session length associated with an
  address, causing a reassignment after a fixed duration,
  typically one day to one week depending on the ISP.

\end{itemize}

\subsection{Building a global model of dynamic address change}

Conceptually, so long as there is some uniquely identifying feature
that remains constant across a device's address change,
it can be possible to track IP address changes over time. Several studies have used this broad
method~\cite{udmap, census-survey, zmap-dhcp,
maier2009dominant, dhcp-gatech, peering-shroud, dhcp-dimes}. 
UDmap~\cite{udmap} studied
dynamic address properties using Hotmail user login
traces where the user's login serves as the identifying
feature. Casado et al.~\cite{peering-shroud} tracked clients using HTTP cookies when
clients access a CDN. Other studies~\cite{zmap-dhcp, census-survey} used continuous
responsiveness of an address itself as the identifying feature, assuming
that an address that responds continuously belongs to the same user and that
when an address stops responding to pings, it has been
reassigned. 

However, these studies report conflicting results about the frequency
of address changes. While UDmap reported that over 30\% of IP addresses have
inter-user durations of 1--3 days~\cite{udmap},  Heidemann et
al. reported that 90\% of IP addresses were occupied for less than a
day~\cite{census-survey}.  Maier et al.~\cite{maier2009dominant} reported that a
major European ISP had per-user median durations of just 20 minutes during
their study in 2009 whereas our work in 2015 did not observe this
duration~\cite{addrchange-reasons}. These differences are likely due
to the different biases associated with each study: Maier et
al.~\cite{maier2009dominant} studied one European ISP in an area,
UDmap studied only Hotmail users, while studies that use continuous
responsiveness of addresses over time~\cite{zmap-dhcp, census-survey} could
potentially confuse the occurrence of an outage with an address change.

Considering these conflicting results,
I first analyze the feasibility of building a dynamic addressing
model by describing preliminary results from a novel dataset~\cite{addrchange-reasons}. The
results show that a global model is indeed feasible, but that it will
require multiple, independent, diverse datasets that track address changes across
the world. % Then I describe proposed
% work to gather other complementary datasets.

% However, most of these studies do not account for the cause of the address change,
% studying dynamic address durations in isolation. We show in this
% proposal that studying the causes of dynamic address change is vital
% especially for RODWAP techniques. The occurrence of outages can cause
% dynamic address reassignment.

\subsection{Preliminary results towards a global model}

The RIPE NCC's Atlas project deploys small devices, called probes, that
conduct measurements from globally distributed
networks~\cite{atlas}. The RIPE Atlas dataset offers measurements that allow us to
determine when an IP address change occurred and what the addresses
were before and after the change. In addition, the dataset includes many
measurements that provide context about what was happening around the
time of the address change. I was able to use these measurements to
detect when RIPE Atlas probes rebooted and were not sending pings
(indicating a power outage) and when their pings were not getting
responses (indicating a network outage). In a study with colleagues of active RIPE
Atlas probes in 2015, we found 3,038 RIPE Atlas probes with address
changes hosted across 929 ISPs and 156 countries~\cite{addrchange-reasons}.

\subsubsection{Some ISPs change addresses periodically}


ISPs can assign dynamic addresses for as long as they wish.
In DHCP, long leases simplify administration, while short
leases can be more efficient in reclaiming unused addresses.
DHCP leases, however, are meant to be renewable by devices
that are still active.  In this section, we look at periodic
address reassignment: instances where a device changes
address periodically, despite actively using the address.

If ISPs intentionally renumber after specific durations, we would
expect those address durations to be prominent in a distribution
of all address durations belonging to that ISP. We initially
considered studying distributions
of raw address durations, similar to the analyses by
Maier et al.~\cite{maier2009dominant} and
Moura et al.~\cite{zmap-dhcp}, but found that short address-durations
were overrepresented. When trying to reason about the expected duration that an address will
continue to be assigned to the CPE, we would like to know the fraction
of total time that each duration accounted for. This latter notion is more useful to find whether an
ISP is using periodic durations consistently, since the modes at
intervals on the scale of days will be more visible. 

To capture this notion we
define a metric, the \emph{total time fraction}. For a given probe and an address duration $d$,
we define the total time fraction for $d$ as the fraction of time spent by the probe in durations of length $d$.
We compute the total time fraction for a given probe and a duration
$d$ by obtaining the total address
time for the probe, and computing the fraction of the total
address time that was accounted for by address durations of 
length $d$. For a probe $p$, if $n(d)$ is the number of times the probe had an address duration
$d$ and $D$ is an array containing all address durations that were assigned to
the probe, the total time fraction for the address duration $d$ is
given by:

$f^p_d =  d \times n(d) / \Sigma(D)$

We use a similar procedure for computing the total time fraction
considering all probes in an ISP, country, or continent. We believe that  
%the distribution of 
the total time fraction offers a better representation of 
the
probability that an address was assigned for a certain
duration than a simple inspection of the address durations. 

\subsubsection*{North American addresses are assigned longer than other addresses}

\begin{figure}[tb]
  % \centering
  \begin{center}
    \includegraphics[width=3in]{figs/conts_a_all_ip_durs_connlogs_wtd_cdf}
  \end{center}
  \caption{\label{fig:conts_all_durs} 
    %% Dynamic address-durations
    %%     weighted by address-durations by continent.
    Cumulative distribution of total time fraction by continent. 
    Modes (vertical segments
    in the CDF) indicate periodic renumbering.  Addresses in North America
    are relatively long lived and free of periodic renumbering.}
\end{figure}

We begin by inspecting how address durations vary across
continents.  We expected that address scarcity might affect
address durations, leading to longer durations in North
America and shorter durations in Asia.
We use RIPE
Atlas's probe database 
to find the country to which each probe belongs. Next, we aggregate the
address durations of probes by their respective countries and
subsequently, to their continents.
Figure~\ref{fig:conts_all_durs} shows the cumulative distribution of
the total time fraction for each
continent, i.e., the y-axis shows the fraction of total address duration accounted for by durations less than the x-axis value. 
%% Figure~\ref{fig:conts_all_durs} shows a CDF of address durations, weighted as described above, for each
%% continent.
The number in
parentheses in the legend for each continent shows the total
 address duration for that continent in years ($\Sigma(D)$).

In Europe, Asia, Africa, and South America, address durations exhibit well-defined modes,
mostly at intervals that are multiples of 24 hours. The most common mode is
exactly at 24 hours: the total time fraction for European addresses at
24 hours is 0.16, African addresses is also 0.16, and Asian addresses is 0.07.
One week address durations are also common in Europe, with the total
time fraction at 1 week equaling 0.08.
South American addresses exhibit multiple modes: their total time
fraction is 0.11 at
12 hours, 0.07 at 28 hours, 0.09 at 48 hours, and 0.03 at 192 hours (8
days). 

The curves for North America and Oceania do not have well-defined modes,
suggesting that ISPs in these continents do not periodically change
addresses. Further, North American probes typically retain their dynamic
addresses for much longer durations than other continents; North
American addresses spent more
than half of the total time in address durations longer than 50
days. This suggests that IP addresses can be used as end-host
identifiers in North America for several weeks.

% \subsubsection*{Periodic renumbering is common in some parts of the world}

% \begin{figure}[tb]
%   % \centering
%   \begin{center}
%     \includegraphics[width=3in]{figs/DE_asns_a_all_ip_durs_connlogs_wtd_cdf}
%   \end{center}
%   \caption{\label{fig:DE_asns_all_durs}
% %
% 	%% Dynamic address-durations weighted by address-durations for ASes in Germany.
%     Cumulative distribution of total time fractions for ASes in Germany.
%         Many
%       German ISPs appear to change addresses every 24 hours. However,
%       some ISPs have more stable addresses.
%   %%@@rama: quantify
%   }
% \end{figure}


% Next, we investigate how the periodic renumbering behavior of ISPs
% correlates with the country in which they operate. Germany has more
% than a hundred RIPE Atlas probes deployed across several ISPs, thus we
% study their address durations in Figure~\ref{fig:DE_asns_all_durs} for
% ISPs with probes that contributed at least 3 years of total time. Many
% ISPs in Germany change addresses every 24 hours: 77\% of the duration
% in DTAG (AS 3320), 76\% in Telefonica1 (AS 6805), 74\% in Telefonica2
% (AS 13184), and 29\% in Vodafone (AS 3209), is 24 hours. We observe
% that the 'other' ISPs also have a mode at 24 hours, suggesting that
% German ISPs are particularly likely to renumber every 24
% hours. However, this behavior is not universal: Kabel Deutschland (AS
% 31334) and Kabel BW (AS29562) do not exhibit a mode at 24 hours;
% instead, more than 90\% of their total address duration was spent in
% durations longer than two weeks. These results suggest that periodic renumbering behavior can exhibit
% some geographic correlation, but is likely
% largely caused by ISP policy. 

% We found 20 ISPs in the RIPE Atlas dataset that we label as \emph{periodic} because
% these ISPs renumber many of their customers after they have held their
% address for a specific duration. This time limit varies across
% ISPs. Of 2272 dynamically assigned probes in the dataset, 193 (8.5\%)
% change addresses periodically with a period of 24 hours, and 123
% (5.4\%) do so with a period of one week. Periodic renumbering occurred
% most commonly in central European countries like Germany, Austria,
% Poland, and Croatia. Some ISPs in Russia, Kazakhstan, Mauritius, and
% South America, also periodically renumber.

% Private communication with a large European ISP confirmed that the ISP renumbers every 24
% hours, since the ISP considers this scheme to be more 'privacy secure' although
% there is no government regulation that forces this feature. The ISP
% also reported that it uses PPPoE instead of DHCP for its DSL
% lines (which accounted for the vast majority of its customers). Since
% periodic behavior would be atypical of DHCP but consistent with PPP
% techniques for address assignment, we speculate that periodic
% renumbering is a property of ISPs that use PPP.

\subsubsection*{Periodic address changes are some times synchronized}

% \begin{figure}[th]
%  \centering
%     \includegraphics[width=3in]{figs/weekly_3215_a_periodicrenums_per_h24_connlogs_bar}
%   \caption{\label{fig:3215_renums_per_h24}Periodic address changes
%     in Orange appear more evenly distributed among the hours of the
%     day.}
% \end{figure}

\begin{figure}[th]
  \centering
    \includegraphics[width=3in]{figs/daily_3320_a_periodicrenums_per_h24_connlogs_bar}
  \caption{\label{fig:3320_renums_per_h24}Periodic address changes
    are more likely in some hours for Deutsche Telekom.}
\end{figure}

We imagine two broad strategies for daily renumbering:
either leaving each customer on an independent, free-running
clock that resets after 24 hours, or synchronizing all
address changes to an off-peak time when few would be
interrupted.  Both seem reasonable strategies: independent
clocks seem simple to implement, synchronized address
changes seem more likely to shuffle addresses since many
addresses are made available during the synchronized
interval.  Probing-based outage detection techniques would benefit from knowing which strategy
is being used: if an ISP is known to change addresses at specific
times of the day, we can account for this behavior in the model.

 We expect
that plotting the time of day at which addresses change for
each ISP will expose whether the renumbering is
synchronized. For the German ISP, Deutsche Telekom AG (DTAG), Figure~\ref{fig:3320_renums_per_h24} shows the hour of
the day when an address change occurred after the address had been
assigned for 24 hours; DTAG assigns periodic durations more often during some
hours of the day. In private correspondence with a large European ISP,
we learned that many CPE devices come with an option to choose the
time at which they should disconnect and reconnect to receive a new
address, as a privacy feature. Figure~\ref{fig:3320_renums_per_h24}
supports this deployment scenario, observing almost three quarters of
all periodic address changes between hours 24 to 6 (in GMT). However,
some CPEs do not have this feature because a quarter of the periodic
address changes happen at other hours of the day.

\subsubsection{Some ISPs are more likely to change addresses upon
  outages}

Here, we investigated how outages occurring at the CPE (customer
premises equipment), due to loss of power or network connectivity
affect the likelihood of address changes. % We quantify how frequently
% and for which probes an outage event at the CPE device appears to
% cause the reassignment of its IP address.
% We quantify how the relationship between outage 


% \subsubsection*{Renumbering behavior upon outages varies across ISPs}

% For each individual probe, we considered the conditional
% probability of an address change given a detected
% outage. $P(ac|nw)$
% represents the conditional probability that an address change occurred
% given a network outage and $P(ac|pw)$ represents the same for a power outage. We estimated this probability using the
% fraction of outages occurring contemporaneously with an address change (out of the
% total number of outages).  We show the distribution of
% these probabilities by probe to estimate whether the group
% of probes (by geography or ISP) is dominated by those that
% always or seldom change addresses on an outage.

% \begin{figure}[tb]
%   % \centering
%   \begin{center}
%     \includegraphics[width=3in]{figs/top_asns_frac_norenums_over_totalnos_cdf}
%   \end{center}
%   \caption{\label{fig:top_asns_frac_norenums_over_totalnos}
% Distribution of $P(ac|nw)$ per probe for the ASes with the most probes
% that had at least one address change. Probes in DTAG, Orange, and BT, are far more likely to change addresses upon a
%     network outage than probes in Verizon and LGI.}
% \end{figure}


% We find that the likelihood of address change upon an outage event
% differs across ASes. Figure~\ref{fig:top_asns_frac_norenums_over_totalnos} shows
% the CDF of $P(ac|nw)$ for the five ASes
% that host the most probes with at least one address change and at least three
% network outage events. We find that probes in ASes that periodically renumber---Orange, DTAG, and BT---have high $P(ac|nw)$ compared to probes from ASes that
% do not periodically renumber, LGI and Verizon. Around half of the probes in both Orange and
% DTAG had  $P(ac|nw)$  equal to 1: every network outage was accompanied
% by an address change! $P(ac|pw)$ was also similar
% for these ISPs. We found 10 more ASes whose probes were particularly
% likely to renumber upon outages: all of them are in Europe and 7 of them
% also periodically renumber. Private communication with a large European
% ISP whose probes consistently had an address change upon outage confirmed that they use PPPoE and Radius to assign addresses for
% their DSL lines. We expect that this property can be used as evidence in inferring a device's link
% type.

% \subsubsection*{For some ISPs, most outages result in address changes}

\begin{figure}[t]
  \includegraphics[width=1.5in]{figs/6830_combined_merged_bar}~~~
  \includegraphics[width=1.5in]{figs/3215_combined_merged_bar}

  \caption{\label{fig:outagedurs} The likelihood of an address change (renumbering)
    given network or power outages of different durations in LGI (left)
    and Orange (right).  The top graph is a histogram; the
    complete bar represents the number of outages observed
    across all probes in that AS.  The lightly-shaded bar
    extends for those outages that also saw an address
    change.  The lower graph shows the same data as a
    percentage.  Although relatively few outages
    lasted longer than a day, the majority of these were
    coincident with an address change in both ISPs. However,
    Orange (right) changed addresses even on the shortest
    outages.}
\end{figure}

Dynamic addresses assigned using DHCP should typically retain
their addresses as long as they continue to renew their lease half-way
into the lease duration as the standard
recommends~\cite{rfc2131}. However, an outage could
prevent them from renewing their lease. Depending upon the address
churn at the time, the address they had previously been assigned may
be reassigned to another device.  In this way, an outage
longer than half a lease duration could potentially cause an address
change. To investigate the effect of outage duration on the likelihood
of address change, we analyzed the conditional probability of an
address change given the occurrence of network or power outages of different durations
for probes from LGI (AS 6830) and
Orange (AS 3215) in Figure~\ref{fig:outagedurs}. 

The behavior upon outages for the two ISPs is strikingly
different. LGI's behavior appears consistent with what we would 
expect for dynamic addresses assigned using DHCP: fewer than
3\% of outages of up to an hour resulted in an address
change.  More than 25\% of outage
durations that lasted at least twelve hours resulted in an address
change. This behavior is consistent with a DHCP lease duration on the
order of a few hours.  Not every outage longer than twelve
hours resulted in an address change, consistent with DHCP 
behavior when a client returns after an expired lease and the
previously assigned address
is still available.

For Orange, we found that even very short outages resulted in
address changes. 91\% of outages that lasted less than five
minutes resulted in an address change, and for every outage duration
longer than five minutes and shorter than three hours, more than 75\% occurred with an
address change. For outages between three hours to three days
long, the percentage of address changes was closer to 50\%, suggesting
the presence of some CPE devices that do not renumber upon every outage. However, as the
outage duration increases beyond 3 days, almost every outage results
in an address change.

% Private communication with a large European ISP
% confirmed that this behavior is expected for PPPoE based DSL lines in
% that ISP: any reboot/reconnect event will result in the assignment of a new
% address from the ISP's dynamic address pool. Since outages of such short durations can result in an
% address change, a simple reboot of the CPE (resulting in a power
% outage), or unplugging and replugging the network cable (resulting in a network outage), can change the dynamic address assigned to the end-user.%  That
% end-users can change their dynamically assigned address 
% has implications for researchers and operators who use IP
% addresses to identify end-hosts, particularly when IP addresses are
% being used to blacklist malicious actors.

\subsubsection*{Preliminary results suggest that building a global
  model is feasible}

Preliminary results offer promise that modeling the likelihood of
address change can help prevent false inferences about outages and
their durations. For ISPs that change periodically and/or synchronously, the model can
predict when probe-loss is more likely due to
address changes than outages. For ISPs that change addresses upon most
outages, the model can inform in which ISPs outage duration detection
is particularly error-prone. For other ISPs which change addresses
mostly upon longer outages, the model can be used to estimate the
likelihood that an inferred outage ended falsely.

Though the results from RIPE Atlas are promising, they are potentially
biased toward technical users like most measurement infrastructure,
and biased toward European deployment. Building the global model of dynamic address change will
require multiple, independent, diverse datasets that track address changes across
the world.

\subsection{Proposed work to gather complementary datasets}

IP address changes can
be tracked over time if there exists some uniquely identifying feature
that remains constant across the device's address change. I investigate the use of datasets which have this
property to study dynamic addresses:

% \subsubsection{Download manager logs}

% A large CDN's download manager installed on users’ desktop
% and laptops records log lines when events such as a file download
% occur. Each such log line contains a unique installation ID, the
% user’s current public IPv4 address, and the timestamp. 

\subsubsection{Dynamic DNS services}

Websites such as dyn.com~\cite{dyn} provide dynamic DNS. Dynamic
DNS is a service that allows users with a dynamic IP address to host
web services, by providing DNS services that can be easily updated to
reflect changes in users' IP addresses. Users of Dynamic DNS Services
run a daemon provided by the dynamic DNS provider, which is responsible for
determining the publicly visible IP address, and updating the A
record(s) for the user's domain(s). 

I propose to track IP address changes using domain names
registered with dynamic DNS services. Since the domain name of a user
maps to her current IP address, we can use the domain name as a
fingerprint, and detect changes in IP addresses for each domain name
over time, by periodically obtaining the 'A' record associated with
each domain name. 

\paragraph{Geographic correlation of dynamic behavior}

\begin{figure}[tb]
% \centering
\begin{center}
\includegraphics[height=1.5in]{figs/did_dname_get_renum}
\end{center}
\caption{\label{fig:addr_change_per_ctry}
IP address renumbering in dynamic DNS domains over a week: Black
represents dynamic DNS domains which experienced at least one address
change, while grey represents domains whose addresses remained the
same. Renumbering behavior appears to be correlated with geographic
location.}
\end{figure}

As a proof of concept, I report on a preliminary result from this
approach: corroborating the geographic relationships in Figure~\ref{fig:conts_all_durs} 
while extending to countries not well represented by RIPE.
I
obtained 3000 dynamic DNS domains from three different dynamic DNS
services: 2000 from afraid.org~\cite{afraid}, 600 from dyn~\cite{dyn} and 400 from
noip.com~\cite{noip} and fetched the 'A' records from their respective
nameservers once every hour. I collected this data for a week, and
then inspected how many of these domains experienced at least one
address change during this time. Figure~\ref{fig:addr_change_per_ctry}
shows the number of domains that had at least one 
address change and the domains that had none. The y-axis is in log-scale. 
Address changes in Asian and Latin American countries appear
more prevalent, with more than a third of all domains in these
countries seeing at least one address-change. On the other hand,
Northern European countries observe fewer than 6\% of their domain names
experiencing an address change. Address changes are uncommon in
North America: only 3\% of domain names in the US and 6\% of domain
names in Canada observed an address change.

The results from the dynamic DNS dataset are preliminary in
scale and based on a short measurement to show potential.
Further, our study of RIPE Atlas data showed us that the cause of
address changes is important. I intend to couple our
outage-detection tool to probe addresses corresponding to
the dynamic DNS domains while fetching their A records.  We
can thus identify outages that occur near the reassignment,
allowing us to infer if an address-change was caused by an outage and
feed results into the model. Further, if the dynamic DNS result
indicates that a probed address had recently been reassigned, then the
detected false positive outage can be filtered.
 
\subsubsection{Open DNS resolvers}

Since 2010, various studies have reported on the existence of more
than 15 million 'open' DNS resolvers on the
Internet~\cite{openresolver, schomp2014clientsidedns, kuhrer2014exit,
  kuhrer2015going}. These DNS resolvers are 'open' because they will resolve a DNS query sent from arbitrary IP
addresses on the Internet. Previous studies have found that more than
three-quarters of open DNS resolvers are likely to be
residential~\cite{schomp2014dnsvul, schomp2014clientsidedns}. I
propose two potential approaches to fingerprint these open DNS
resolvers and track address changes.

\paragraph{DNS caches}
Open DNS resolvers often cache previous
lookups~\cite{schomp2014dnsvul}. My insight is that these caches can
be used to fingerprint open DNS resolvers, allowing us to track when
their IP addresses change. I plan to do this in two phases.

First, I will find open DNS resolvers on the Internet. I propose to register a domain and deploy an Authoritative DNS server
for it. Then I intend to perform a one-time scan of the entire IPv4
address space by sending a DNS request for a subdomain within the
domain we control to all the IPv4 addresses on the Internet. Each DNS
request I send to a target IP address will embed the target address
into the request, similar to the approach used by Dagon et
al.~\cite{dagon2008corrupted}. The Open DNS resolvers will route the
request to our Authoritative DNS server.  At the authoritative DNS
server, I will note the target IP address to which this request was
sent and generate a unique fingerprint for the device at this address,
and embed this fingerprint in my response. When these responses
reach the open DNS resolvers, each will now contain its unique
fingerprint in its cache.

Next, I will periodically inspect the caches of known open DNS resolvers.
I will issue periodic DNS requests for the subdomain we
control (with the target IP address embedded in the request) to all
the addresses that contacted our Authoritative DNS server. If we
obtain the fingerprint that we had previously issued to that address,
we know that the device continues to be assigned that address. If we
find that an address is no longer returning the expected cache
fingerprint, we know that the address has changed. I then propose to
issue DNS requests to related addresses (as described in
Section~\ref{sec:last_mile}) with the \emph{old} target IP address
embedded in the request. If the device is present on any of those
addresses, then we will obtain the expected fingerprint. Upon finding
the device at a new address, we will update
our local mapping and note that the fingerprint is now available at
this new address.

\paragraph{Anomalous Open DNS Resolvers}

Of the 30 million Open DNS Resolvers on the Internet, around 17
million are \emph{anomalous}~\cite{anomalousdns}, i.e.,
instead of sending DNS responses with a source port of 53, they
respond with a non-standard source port. Kaizer et al. ~\cite{anomalousdns} found that
these devices are primarily residential ADSL modems. Not only do these
devices use a non-standard source port, DNS requests can be made to
these devices in such a way that the source ports are assigned
\emph{sequentially}. My insight is that we can use this sequential
assignment of source ports to fingerprint anomalous open DNS resolvers.

The first part of our approach here is similar to my approach with
the DNS caches: I will find open DNS resolvers that are anomalous. After registering a domain and deploying an Authoritative DNS server
for it, I will perform a one-time scan of the entire IPv4
address space by sending a DNS request for a subdomain within the
domain we control to all the IPv4 addresses on the Internet. Each DNS
request I send to a target IP address will embed the target address
into the request as before. However, instead of embedding responses with
unique fingerprints from the authoritative DNS server, we simply
monitor the source ports that issue DNS responses from each
address. If it's a non-standard port, we flag the device as an
anomalous open DNS resolver.

Next, I will periodically inspect the source ports used by anomalous
open DNS resolver responses. Since we know which
  addresses the anomalous open DNS resolvers are located at, I
  periodically issue DNS queries to these addresses. As long as the
  source port for successive requests to an address continues to be
  sequential, I can state with high confidence that the address has
  not changed. The source ports for these devices typically vary
  between 10,000 to 30,000; thus there is only a small likelihood that
  another device coincidentally happens to have the next value in
  sequence. If we find that a response doesn't arrive, or that one
  arrives but the source port is not sequential, then we know that the
  device's address has been reassigned. As in the DNS cache approach,
  I will then look for the expected source port in DNS responses from
  requests sent to related addresses to find the device again.


\subsection{Confirming that detected outages are accurate}

After mitigating false positive outages, I propose the use of datasets
from RIPE Atlas probes~\cite{atlas} as ground truth to confirm that the remaining
outages are indeed true positives. In previous work, I had inferred outages occurring on
RIPE Atlas probes by looking for gaps when probes did not perform 
measurements that they were scheduled to~\cite{addrchange-reasons}. By
probing IP addresses at which RIPE Atlas probes are also deployed, I
will compare outages we detect against outages inferred from RIPE
Atlas datasets and validate whether our detected outages are accurate.
\chapter{Analyzing weather's effect on Internet Reliability}
\label{cpt:weather}

\section{Analyzing the effect of external factors upon Internet reliability}
\label{sec:weather}

One aspect of measuring Internet reliability is to determine if the occurrence of certain events adversely affects Internet connectivity. Consider the occurrence of adverse weather conditions for instance: prior work has shown that Internet outages occur more frequently during times of precipitation~\cite{pingin}. However, this work was preliminary in nature and was performed over a short duration (three months). Further, it treated every instance of a previously responsive address failing to respond to pings as an outage, ignoring the effects of dynamic addressing.

In this section, I discuss an approach to quantify the effect of external factors, such as the occurrence of various weather conditions, upon Internet connectivity of residential addresses using measurements from the Thunderping probing system~\cite{pingin}. 

\subsection{Thunderping measurement system}

Thunderping probes addresses during times of severe weather. 

\subsection{Find the inflation in dropout rate}

The key insight here is that determining the inflation in \emph{dropout} rate when event(s) occur captures the increased likelihood of the \emph{outage} rate. 

% \begin{figure*}[t]
% %
% \begin{subfigure}[t]{0.47\linewidth}
% \centering
% \includegraphics[width=\linewidth]{figs/frate_by_timeofweek_jan11todec17_scatter}
% \caption{
% \label{fig:frate_lts_timeofweek}
% Dropout probability has significant diurnal variation.
% }
% \end{subfigure}
% %
% \hfill
% %
% \begin{subfigure}[t]{0.47\linewidth}
% \centering
% \includegraphics[width=\linewidth]{figs/addresshoursbywtyp_by_timeofweek_scatter}
% \caption{
% \label{fig:weather_timeofweek}
% Different weather conditions are prominent at different times.
% }
% \end{subfigure}
% %
% \caption{
%  Weather does not occur most often during hours of the week when there are an inflated number of dropouts. 
% }
% \end{figure*}



 
\section{Introduction}

Wather-related damage to vital infrastructure
(in this case the power grid) can lead to significant economic harm.
%
Yet, %  in the Internet era,
little is known about the economic impact of
weather-induced outages on the most pervasive infrastructure that people use to
access the Internet: residential last-mile links.
%
For massive last-mile outages, telcos are required by U.S.~policy~\cite{cfr49}
to report the outage to the FCC.
%
However, the minimum reporting threshold is astronomical: the outage must be at
least 30 minutes in duration, and it must have affected tens of thousands of
customers~\cite{cfr49}.
%
Researchers have also studied widespread link failures in the Internet,
like undersea cable cuts~\cite{chan-pam11, bilski-ccis09}, natural
disasters~\cite{heidemann-sandy}, and backbone router
failures~\cite{iannaccone-imw02}. 

In practice, most weather events are much more localized and not severe enough
to generate such a large outage.
%
For decades, this everyday weather has been known to lead to to smaller scale
outages of telecom infrastructure.
%
For example, early telephone and cable television engineering documents
describe how to avoid moisture in wires because it impedes signal
propagation~\cite{aiee09-jewett, toct66-smith}.  Also, rain attenuates
satellite signals above 10~GHz~\cite{ieee75-hogg}.  Finally, point-to-point
wireless links can experience multipath fading due to objects moving in the
wind~\cite{ieeecm01-bolcskei}. In short, residential links are vulnerable to
everyday weather because residential equipment and wiring are often installed
outdoors: wind can blow trees onto overhead wires, heat can cause equipment to
fail, and rain can seep into underground equipment cabinets.
 
Surprisingly, for these everyday weather conditions, there are no public
statistics on the frequency or magnitude of the outages they induce (directly
and indirectly).
%
This could be a problem for Internet-based companies because they do not know
how many customers they are losing to nature, and for regulators because they
do not know how significant the problem is, and which conditions and geographic
areas deserve their attention.
%
In this work we resolve this issue: we provide the first comprehensive study
that identifies the correlation between everyday weather and residential
Internet last-mile outages.
%
Specifically, we quantify the absolute increase in the number of outages
observed during weather, when compared to non-weather periods.

%To make this information useful to the intended audience of this work (i.e.,
%policymakers, web companies, and operators), 

Quantifying the relationship between occurrences of weather and an increase in
outages cannot be answered with a short term %, broadly collected
study.
%
The data set needs to be longitudinal because weather is
\emph{seasonal}---certain weather conditions only happen at certain times of
year---and because some weather events are \emph{rare} enough that providers in a 
specific location may not be adequately prepared.
%
Targeted probing is needed because weather is \emph{localized}: at any time
only specific geographic locations are exposed to weather conditions.
%
Broad observation of outages of several links will capture correlated outages
of several hosts, such as the work by Heidemann et al.~\cite{imc08-heidemann,
trinocular}, but it will not reveal failures of individual links as may be
the case for weather.
%
Although some systems can obtain detailed measurements at residential
gateways~\cite{ripe-atlas,sundaresan-sigcomm11}, the limited deployment of
these measurement systems make them inadequate for studying the scale needed to
observe many different weather conditions, multiple times, in different
geographic areas.
%
Therefore, we performed a seven year longitudinal study with targeted
measurements of residential links surviving weather events.

In 2011, we introduced a measurement system for this task called
ThunderPing~\cite{schulman-imc11}.
%
For the past seven years ThunderPing has been following forecasts of weather in
the U.S. and pinging a sample of 100 hosts from each last-mile provider in the
area for six hours before, during, and six hours after the forecasted weather
event.
%
The focus of our initial paper on ThunderPing was its probing methodology, but
it also included a preliminary study that looked at 66 days of data.
%
Given how limited the data set was, we were unable to draw statistically 
significant conclusions and we saw only one season, summer, of one
year. % ---therefore we could not, with any confidence, quantify the increase in
% outages during weather events.
%
We also did not have enough data to explore variations in effect of
weather by geography,
%
nor could we explore if the likelihood of failure varies
with continuous weather conditions (e.g., wind speed),
 
In this paper, the time totaled across all responsive links
exposed to different weather events is in the \textbf{centuries}.
%
For example,
we have observed a total of \emph{100 centuries} of DSL links exposed to cold
weather.
%
This large data set enabled us to address all of these significant limitations
of our prior preliminary study.
 
There is a challenge with quantifying how weather correlates with
outages: outages are relatively uncommon events, and thus every outage
is a significant event.
%
%This means every single outage is a significant event.
%
This is compounded by the fact that we wish to analyze subsets of our
data to focus on, say, particular link types or locations.
%% Yet, we need to slice and dice the data to isolate different properties
%% of residential links that may be related to their failure likelihood:
%% namely their link type and location.
%
With so few outages observed compared to the time that links are responsive,
it is difficult to determine if different weather causes a statistically significant
increase in outages.
%
To address this issue, we borrow statistical tools from epidemiology
that enable us to reason about the \emph{inflation} in dropout
probability, and to establish statistical significance to our results,
even though failures happen at relatively low rates.
%
We detail this approach in Section~\ref{sec:hazard}, as we believe it
to be of general use to the community.


%% known as
%% \emph{hazard rate}.
%% %
%% This tool enables us to establish statistical significance of our results, even
%% though there are relatively few failure events.
%% %
%% This metric enables us to measure the \emph{increase} in probability of a failure
%% that correlates with a specific type, or intensity, of weather.
%% 
%% We expect that this metric may be useful for web and cloud service operators
%% because they can use it to quantify how many of the outages that they observe
%% are likely due to weather alone, and thus estimate how much revenue they are
%% losing to weather outages.
%% %
%% It may be useful for regulators because they can determine if the added number of
%% failures from weather is significant enough to warrant new policy focused on
%% addressing weather outages, such as advocating for undergrounding
%% infrastructure as it has a demonstrated improvement on robustness to
%% weather~\cite{undergrounding}.

Another challenge is this metric could be artificially inflated by
weather conditions coinciding with daily network state changes such as
maintenance or renumbering~\cite{addrchange-imc}.
%
We verify that weather does not appear to be positively correlated with
peak diurnal failure periods, and to do recovery time analysis we
select networks and outage durations unlikely to suffer from address
renumbering, allowing us to ensure that we are probing the same
residential link before and after the outage.

\paragraph{Observations and Contributions}
%
We present a dataset spanning seven years, all weather conditions, and
76~billion responsive pings to 8.7~million hosts throughout the U.S. 
%
We apply techniques from epidemiology to attribute statistically
significant rates of dropout to individual weather conditions.
%
Our key findings span four broad areas of analysis:

\begin{itemize}

\item \textbf{Link type variations} (\S\ref{sec:inflations}): Different link
types experience weather in highly varying ways.  For instance, compared to
wired link types (cable, DSL, fiber), wireless link types (WISPs and satellite)
experience greater increases in dropout rates during rainy conditions and high
temperatures, but often \emph{decreases} in dropout rates in snow and cold
temperatures.

\item \textbf{Geographic variations} (\S\ref{sec:geography}): Different
geographic regions can be affected to varying degrees.  For instance,
Midwestern U.S.~states are more prone to failures in thunderstorms and rain
than coastal states.  Southern states are more prone to failures in snow than
other states.

\item \textbf{Continuous variable analysis} (\S\ref{sec:continuous}): Most link
types have highly nonlinear dropout rates with respect to changes in
temperature, wind speed, and precipitation.  For temperature, dropout rates are
typically non-monotonic; satellite links drop out more in moderate temperatures
than low or high temperatures.

% \item \textbf{Recovery times} (\S\ref{sec:recovery}): The amount of time to
% recover from a dropout varies significantly across different weather
% conditions.  Thunderstorms take the longest for cable links to recover from.

\end{itemize}

Our findings have ramifications on how network outage detection and analysis
should be performed; limiting measurements to any particular geographic region,
link type, or time of year can introduce statistically significant bias.
%
We believe our results also have implications for network administrators and
policy-makers; an increased use of satellite links in the Midwestern U.S.~has
resulted in those states' increased dropout rates in rainy weather.
%
We will be making all of our data and code publicly available.

% New contributions =============================================================

% Can quantify weather because we have more data

% Tons more data allowed us to compute confidence intervals

% Double checking if we have more failures due to maintenance periods and what not

% Failure duration 

% Double checking if failure duration with NetSession data

% Each of the PLOTS, DUH!

% Extra text {{{

% Among our findings, we show that temperature has a
% surprising relationship with link reliability: extreme
% temperature correlates with failures of typical link types,
% while satellite links and dialup links see high failure at
% modest temperatures.  Wind degrades reliability proportional
% to the square of wind speed, and the amount of hourly
% precipitation, in contrast, has a less pronounced effect on
% reliability.  However, the type of precipitation has a
% substantial effect---freezing rain, hail, and thunderstorm
% often double the probability of seeing a failure, depending 
% on link type.

% 
% A location can be exposed to the same weather patterns year over year, then one
% year a severe storm such as Hurricane Sandy in New York City in 2012.
% %
% This is important for our study because it will reveal how resilient last-mile
% Internet links are to failures when an anomalous weather event occurs that
% providers are not prepared for.
% 
% Therefore, the seven years worth of data that we collected for this study
% provides an opportunity to quantify the effect of weather on a particular
% location.
%

% Unlike the measurements needed to understand
% these failures, the collecting measurements needed to understand residential link failures
% requires a particularly broad set of hosts. Weather 
% varies by location and time, and how residential hosts fail in weather 
% depends on location, link type, and Internet service provider.

% Evaluation of weather-related failures in this work presented in
% Section~\ref{sec:failurerate} are backed with more pings which enable
% statistical comparisons between failures in different weather conditions, and
% observations of continuous weather conditions such as wind speed and
% temperature.

% For the continuous weather measurements, we observe clear
% relationships between the probability of failure depending on the residential
% link type~\todo{nums}. For the discrete weather measurements, we observe~\todo{nums}.

% With both breadth and depth, the observations of residential link
% failures can also be correlated with other possible causes: We study weather as
% a likely cause of outages to gain confidence that outages can be detected
% remotely; this confidence enables future comparisons of reliability measures
% across providers and countries.


% Weather can cause fixed \lastmile transmission media to
% fail~\cite{schulman-imc11}. 
% 
% Also there are so many different ISPs that it is difficult to get a global
% picture of the weather-related failures.
% 
% We know a lot about how loss of vital infrastructure affects business from the supplier side.
% %
% We do not however know what the costs are of having customers disconnected due to weather.
% In this work we deliver on that promise
% 
% 
% Unlike other vital infrastructure, residential Internet depends on the operation
% of other vital infrastructure to function.
% %
% Also, it has a lower barrier for a user to shut it down on their own, in
% particular to avoid damage of their equipment during storms.
% %
% If it does turn out that users are doing this, this is still valuable knowledge
% for operators as those customers will not be able to access their networks.
% 
% 
% (1) Categorical
%  - overall
%  - geographic (rama finds the worst link type/geography combination)
%}}}


\section{Data Set}

This section describes the data we collected and its initial
processing.
%
We start with a definition of the partially interpreted data
we seek: ``dropouts,'' where an address fails to respond in
the context of otherwise ``responsive hours'' of an address.
%
Dropouts and \emph{disruptions} from Chapter~\ref{cpt:corrfails} are synonymous.
%
We next briefly review the ThunderPing data probing system and
present brief statistics about the raw active probing data.
% 
Then, we review the weather data, particularly how and
where it is collected and how we handle hurricanes.
%
We conclude by describing the benefits (and limitations) of this
data for our study of weather-related effects.

\subsection{Dropouts, Defined} %{{{

A dropout happens when the address attached to a residential link transitions from
being responsive to pings from multiple vantage points, to being unresponsive
from all of the vantage points.
% 
Specifically, we define a residential link ``dropout'' as an hour when at least three
vantage points pinging a host and receiving replies suddenly experience 11 minutes
(an entire probing interval) where they do not receive a 
reply before a five second timeout.
%
This dropout occurs within a ``responsive address hour,'' a
continuous observation of an IP address in known weather
conditions.  A responsive hour may or may not include a
dropout, and the ratio of dropouts to responsive hours is a
measure of outage likelihood.  Responsive hours add: two
addresses both observed in the same hour or one address
observed for two hours in the same conditions are equivalent.

Our selection of three vantage points is based on prior work's selection
of three vantage points to observe outages~\cite{trinocular}.
%
Our selection of a five second timeout for ping responses is based on our
prior work that observed that most ping replies to residential
hosts are received within five seconds~\cite{timeouts}.
%
Our selection of one hour as the time period for a dropout is based on the fact
that the weather data we collected consists of hourly reports.
%
Considering at most one dropout per probed address per hour will diminish the
number of observed dropouts from individual links, if they should alternate
between responsive and unresponsive states: there can be at most one per hour,
not five (due to the 11 minute probing interval).

% To contrast with a dropout, a ``responsive hour'' is an hour
% in which weather is experienced, but the address does not
% experience a dropout.
% We define responsive hour later in Sec 2.4. And we do it
%   correctly there!
% nspring - I think it needs to be here, corrected if necessary.

Observing a dropout is a sign that a residential link may (but may not) have
experienced an outage:
%
\emph{Dropouts are a superset of outages.}
%
Dropouts can also occur if the device re-attaches to the
network with a new address after only momentary
disconnection, typically through re-association of a PPP
session for a DSL modem, but potentially through
administrative renumbering of prefixes. For our purposes, we
expect these events to occur independent of weather, such
that the two events can be studied separately.
%
% Restated, when a residential link experiences a prolonged outage, a prolonged
% dropout will also occur.
%
%Other network changes can also lead to dropouts.
%
%IP renumbering is the most common source of dropouts that are not outages. 
%IP renumbering is when residential providers change the IP addresses assigned
%to their customers either via forced renumbering~\cite{forced-renumbering},
%DHCP lease expiration~\cite{dhcp-lease}, or PPPoE sessions
%%expiring~\cite{pppoe}.  For our purposes, we expect these events (address
%reassignment or premise equipment reboot) to occur independent
%of weather, such that the two events can be studied separately.
%
We confirm that dropouts during typical maintenance intervals are
unlikely to correlate with weather in
Section~\ref{subsec:independence_of_maintenance}.
% , and we
% revisit DHCP in Section~\ref{sec:recovery}.

%
% \todo{insert that our approach for dealing with this is statistical?}

In short, by observing dropouts, we will be able to observe how residential
links behave during weather, at the scale necessary to make quantitative
conclusions about weather's effect on residential links in the U.S.
%
% We must also be careful because we observe the effects of
% other network changes that do not correspond with outages.
%}}}

% \subsection{How did we observe if there were dropouts on residential links during weather?} %{{{

\subsection{Dropouts, collected}

We briefly summarize the methodology of ``ThunderPing'': our probing
system that has been running for seven years.
%
More details about ThunderPing can be found in our preliminary work in
%ThunderPing was described in our preliminary work in
IMC~2011~\cite{pingin}.
%, details can be found there.
%
% nspring - I know I wrote this, but I'm not feeling it any more.
% surgical strike preferred.
%Although our initial probing methodology was sufficient for data collection, our
%analysis methods were inadequate in that they did not consider the background
%rate of failures and they were too ambitious in assuming outage duration could
%be understood for all link types.  We describe our revised method in 
%Section~\ref{sec:method}.

The ThunderPing probing methodology is as follows:
%
For every forecast of severe weather provided by the US National Weather
Service, ThunderPing pings a sample of 100 residential hosts from each
provider in the affected region.  The affected region is specified by FIPS code, which roughly
corresponds to counties in the U.S.
%
The probing starts up to six hours before the forecast event, continues during the
event, and terminates six hours after the event, regardless of whether the weather
materializes.
 
The residential hosts ThunderPing pings during each weather event are selected
from a master list of residential hosts classified by provider (reverse DNS
name) and geographic location (FIPS code).  We classify link type by
provider, when the provider implies a well-defined link type; (typically rural) providers that
use a variety of media types to provide connectivity are included under ``All'' link
types with the rest, but are not classified further.  We determine location using a 
MaxMind database from the same year for choosing which addresses to probe, but from the same
month for analysis. Although there are errors in both classifications, a location error
would be expected to cause an underestimate of the effect of weather by placing a host
not in the forecast region falsely into the area of weather effect.
 
ThunderPing sends pings to each of these hosts from up to 10 geographically
dispersed \planetlab vantage points every 11 minutes.  This interval is due to~\cite{imc08-heidemann}.
When a \planetlab node fails, we replace it, but if the number of working vantage points
drops below three, we discard observations at that time as untrustworthy.
%
% Having more than the three vantage points needed to detect dropouts allows us
% to continue probing uninterrupted even in the inevitable case that a
% \planetlab node fails.
% %
When there are at least three, we require
that all active vantage points do not have a response in order to label the event as a dropout.
  
ThunderPing retransmits failed ICMP requests: when a vantage point sees a
lack of ping response it retries that ping with an exponential backoff up to 10
times within the 11~minute probing interval.  Therefore, a dropout will typically require
at least 30 failed ICMP requests.

ThunderPing has been running for seven years, and has
collected 76 billion responsive pings to 8.7 million
residential addresses.
 
\subsection{Weather, classified}

To quantify the effect of weather on dropouts, we needed to determine
what weather residential links were exposed to when a dropout did or did not
occur.

The US National Weather Service (NWS) operates a 
network of 900 automated ``ASOS'' weather stations.
%
These weather stations are typically located at airports.
%
The NWS weather stations record hourly observations of 24 weather
variables in METAR format and make those available~\cite{metar-ftp}.
% nspring: I feel like "we downloaded it" is implied.
%
%For all of the geographic locations that we probed, across all of the time
%periods that we probed, we downloaded METAR data
%from~\cite{metar-ftp}.
%
 
There are two types of weather information: categories that
account for the common precipitation types (e.g.,
thunderstorm, hail, snow) and continuous variables (e.g.,
wind speed, precipitation quantity).

We annotate each responsive address hour for an address with the
corresponding weather information associated with the geographically
closest weather station to that address. Doing so allows us to find
the number of responsive hours and dropout address hours in specific
weather conditions. 

\paragraph{Hurricanes are special}
%
Severe events are among the most important failure events for us to study how the
Internet is affected, as the Internet is increasingly relied on as the primary
mode of communication in an emergency~\cite{emergency-voip-fcc}.
However, severe events have the potential to overwhelm the typical and
obscure interesting observations.

% hurricane_times = [
%    ['October 7 2017 00:00:00', 'October 10 2017 00:00:00'], # Hurricane Nate                  
%    ['September 9 2017 00:00:00', 'September 13 2017 00:00:00'], # Hurricane Irma              
%    ['August 24 2017 00:00:00', 'September 1 2017 00:00:00'], # Hurricane Harvey                
%    ['October 6 2016 00:00:00', 'October 9 2016 00:00:00'], # Hurricane Matthew                
%    ['September 1 2016 00:00:00', 'September 3 2016 00:00:00'], # Hurricane Hermine            
%    ['July 3 2014 00:00:00', 'July 5 2014 00:00:00'], # Hurricane Arthur, don't seem to have an\
% y data                                                                                          
%    ['October 28 2012 00:00:00', 'November 2 2012 00:00:00'], # Hurricane Sandy                
%    ['August 25 2012 00:00:00', 'August 31 2012 00:00:00'], # Hurricane Isaac                  
%    ['August 26 2011 00:00:00', 'August 30 2011 00:00:00'], # Hurricane Irene, don't seem to ha\
% ve any data                                                                                    
% ]

The following hurricanes made US landfall during our
measurement:
Irene (Aug 26--30, 2011),
Isaac (Aug 25--31, 2012),
Sandy (Oct 28--Nov 1, 2012),
Arthur (Jul 3--5, 2014),
Hermine (Sept 1--3, 2016),
Matthew (Oct 6--9, 2016),
Harvey (Oct 6--9, 2017),
Irma (Sept 9--13, 2017), and
Nate (Oct 7--10, 2017)~\cite{noaa-sotc}.
Hurricanes manifest as a combination of weather features and
are so pronounced that their contribution to thunderstorm or
rain outages would be disproportionate.\footnote{It is
  disappointing to realize the irony that the most
  significant weather events are also the least
  surprising.}  We thus omit them from categorical weather
classification (e.g., Figure~\ref{fig:inflatedfrate_by_wtyp_ci_whiskers}).
However, we consider data from Hurricane events
when studying continuous variables (inches of rain and wind speed,
for example, where these extremes are clearly distinguishable).
%
Collectively, these hurricane times account for less than 3\% of
%
%All these hurricane times put together account for less than 3\% of
responsive address hours and 4\% of dropout hours.

\begin{table}
  \hspace{-0.1in}\tiny
\ninput{weather/fragments/summary_table_vert}
  \caption{\label{tbl:summary}Summary of data set for large ISPs
	classified by link type.  ``All'' comprises data from ISPs not
	included in this sample.
	(For this table, we count D.C.~as a state.)
	}
\end{table}



\subsection{Data, Summarized} % Comparing with our IMC 2011 Study}

% \todo{ alter this to cover table 1 and mention the old data in comparison. } 

This data set comprises observations from January 2011 to December 2017,
though only 1467 days included sufficiently many operating vantage points to
classify a responsive address hour.

We show per-ISP highlights in Table~\ref{tbl:summary}.  We
observe major providers such as Comcast, Qwest, and ViaSat
in all fifty states (and DC).  Of the 1.77 Billion
responsive address hours from Table~\ref{tbl:summary}, 139M
(8\%) were hours where responsive addresses experienced
rain, 66M (4\%) snow, and 19M (1\%) thunderstorm.

Contrasted to our preliminary study~\cite{pingin}, this
covers nearly 22 times the duration (compared to 66 days),
and includes roughly 60 times as many dropout events (likely
because those days were in spring and early summer).%
% sentiment in next section :
% is collected over enough time that we can draw
% conclusions with confidence about the increase in dropouts correlated with
% many different types of weather in different geographic locations.
%
\footnote{In the public reviews of the IMC 2011 paper, all of the reviewers stated that
they wished the dataset was more comprehensive so conclusions could be made
about the effects of weather on residential links.}
%
% The dataset in this work covers many seasons several times and contains a 
% severe weather events.
%
%
% At 1467 days, the data in this work spans a period of time 22
% times longer than our preliminary study in 2011.
%

%}}}

\subsection{Why this data?} %{{{

Others have studied outages and collected broad IP
responsiveness data.  Here we describe the benefits
of our data, addressing its limitations in Section~\ref{sec:datasux}.

Our data provides a view on outages of individual addresses,
including isolated outages of ``customer premises''
equipment or singly-connected links that are most exposed.
We rely on statistics to identify a significant change in likelihood of
failure, rather than rely upon large outages of infrastructure
common to a larger aggregate prefix to signify significance.
Every residential link is wired with its
own infrastructure: every residence can have different equipment installed
in different ways and has its own resident network administrator.
As a concrete example, we expect to observe
the effect of water infiltration in the network interface
device (the demarcation point connecting premise phone
wiring to the provider).  (We discuss the flip side of this
coin below.)

Our data is of a scale large enough to compare link
types, providers, geography, and across time.  Seven years
of data make it feasible to observe multiple instances of
both severe and common weather events.  Rare events include
a fair number of tornadoes and virtually unique events such
as snow in Louisiana.  Many observations of similar weather
increase the confidence in our dropout probability estimates,
making it possible to split the data and identify the differences
between, for example, heavy and light rain on wireless ISPs in
Kentucky.  The sampling approach---providing data for each 
provider in an area---ensures that even less-used network links
and providers are well-represented, permitting a comparison with
satellites and wireless ISPs that might be poorly represented
in end host measurement probes~\cite{samknows, sundaresan-sigcomm11,
  ripe-atlas} or when using provider-specific
data~\cite{conext10-jin,cisco-transponder,alpha-transponder}.

Our data includes data from times not subject to interesting
weather: the method probes before and after forecast weather
alerts.  ``Typical'' weather occurs particularly when the
forecast does not materialize or the forecast is for a
long-term event (e.g., summer fire warnings).  With these
measurements, we can establish a baseline for the rate of
dropouts in common weather conditions.  Probing after the
weather also permits measuring recovery time as we wait for
previously responsive addresses to return.

Our data is not sensitive to link failures elsewhere in the
Internet or to PlanetLab vantage point failures.
Restated, with multiple vantage points, catastrophic Internet link outages,
such as the fiber cuts during the ``Baltimore tunnel fire'' in
2001~\cite{bmore-tunnel-fire} will only be considered as an outage if all
vantage points are unable to communicate with the host over the residential
link.  As described above, without three active vantage points,
we make no decision about address responsiveness.

\subsection{Dataset Limitations} % What are the limitations of our data set?} %{{{
\label{sec:datasux}

% \paragraph{Probing is biased to links and time periods where there are weather
% forecasts}
%
The essence of ThunderPing is to selectively probe only when
there is a weather alert forecast for an area, which biases the data
toward time periods where there is some atypical weather present.
Obviously, regions that experience temperate weather are 
unlikely to be represented, and we thus do not attempt to quantify
what fraction of all residential network outages are caused by weather.
More subtly, during the interval around
forecast severe weather, the weather conditions may not be
ideal: our estimate of the background dropout rate is likely inflated 
by proximity to potentially severe weather, thus causing us to underestimate
the quantitative effect of that weather.

% \todo{Aaron- did I capture what you wanted (prior text commented below)?}

% Since ThunderPing only probes when there is a weather forecast for an area, we
% may not probe some residential links that received few or no forecasted weather
% over the course of our seven year study.
% %
% Also, by focusing only on links that are expected to experience weather, we may
% have bias toward hosts that will dropout if our weather assignment is
% incorrect. 
% %
% The result of this is that our baseline number of dropouts may be inflated.

Our approach relies upon active probing to gain breadth
across hundreds of providers, but there are limits to this
breadth: providers may administratively filter ICMP requests
and home routers may decline to respond.  We assume that
providers and end hosts that filter are no more or less
vulnerable to weather and that these features do not affect
our conclusions.

Our data set does not identify the cause of an individual
dropout. Our analysis seeks to correlate observations of
dropouts with weather events under the expectation that a
change in probability of outage is related to the weather.
Should a user turn off equipment nightly, this is
independent of weather and will not not be a factor; should a user
unplug equipment when lightning is nearby, such would
contribute to the probability of dropouts in thunderstorm.
Residential Internet infrastructure is also explicitly reliant
on residential electrical power, and we do not isolate power
failures.  We expect network service outages to be more common
than power outages, for power outages to occur only in the most
severe of weather conditions, and for power outages not
to correlate with link type.
% In part, this is why we omit hurricanes: prior
% studies of network outages in hurricanes were likely to be
% measurements of power loss.~\cite{zmap}

Finally, AT\&T, one of the largest DSL and fiber providers in the US does
not assign reverse DNS names to their residential customers. As such, they
were not included in our master list of residential links that we probe
with ThunderPing.  
%}}}


\section{Categorizing outages using simultaneous failures of related addresses}
\label{sec:last_mile}

% Only true positive outages remain after filtering false positive
% outages using the proposed techniques in the previous sections, and
% some applications may be able to use these outages without any additional
% processing. 
% For example, consider a peer-to-peer file storage network
% that keeps track of live peers; this network can use our outage feed
% as is to update its list of live peers.

% On the other hand, when 

When analyzing the reliability of a particular ISP, we need
to find the subset of outages that affected only
that ISP. Doing so ensures that ISPs offering services in challenged
areas do not have their reliability lowered by events such as power
outages or users voluntarily shutting down their home Internet
equipment.

I describe three categories under which detected outages can be 
placed. Each category provides hints about the likely cause of outages
placed in that category. For example, power outages or undersea cable cuts can affect addresses from multiple
ISPs; I term events which result in outages for
many providers' addresses \emph{multiple-ISP outages}. Users in some
geographic areas are particularly likely to shut down their Internet
equipment~\cite{grover2013peeking} but users elsewhere may also power
off their equipment when faced with certain circumstances, such as
approaching thunderstorms. I call such an event a \emph{user-caused
outage}. On the other hand, consider an ISP experiencing a failure in
its networking infrastructure resulting in an outage affecting only
addresses belonging to this ISP: these are outages that should bring down the ISP's reliability
estimate. I term these events \emph{shared-ISP outages}.
Probing-based remote outage detection techniques will detect outages
when all of these scenarios occur since previously responsive
addresses will cease responding in all these scenarios.

I develop and evaluate an approach for segregating outages into different categories
based upon the insight that outages occurring simultaneously in time
for addresses that are related by virtue of sharing geography, ISP, or
network topology,
could share a common cause. For example, if we detect addresses from
many ISPs within geographically proximate regions failing simultaneously in time, we are likely observing a
multiple-ISP outage. If we detect addresses from only a single ISP
failing simultaneously in time, we are potentially observing a
shared-ISP outage. If the detected outage does not appear to
have happened simultaneously in time with other outages of related
addresses, I term it an \emph{isolated} outage. User-caused outages would
likely manifest as isoltated outages. Thus, evidence
of simultaneous failure of multiple
``related'' addresses can be used to distinguish between the different
categories of outages. 

However, it is possible that this approach can incorrectly classify detected
outages into the wrong outage category. For example, if
unrelated addresses happen to fail close together in time,
these outages could be detected as
``simultaneous'' and could be classified into either the multiple-ISP
or shared-ISP categories, depending upon the ISPs to which
the unlrelated addresses belonged. It is also possible that a
multiple-ISP outage is categorized as shared-ISP because
other affected ISPs' addresses had
not been probed at that time. Similarly, if a set of related addresses failed
simultaneously but only one of these addresses was being
probed, we would not observe these outages to be simultaneous. 

While the above false classifications can potentially be mitigated by
intelligent probing schemes, there exists a limitation that cannot:
once an outage has beeen detected to be \emph{isolated}, the proposed approach cannot distinguish between
a user-caused outage and an outage that only affected a single user
but was caused by lack of power or by a network failure. It is possible
that some users regularly power off their equipment at night; if
so, we could
check if a particular user's address consistently experiences outages
between certain hours and categorize these outages as user-caused. But
this solution would not be successful in categorizing all instances of user-caused outages.
% This scenario is not an example of a false
% classification; instead, it is a limitation of the simultaneous outage
% detection approach.

Though it is important to reduce false classifications,
it is also important to balance the probing volume. Sending probes
too quickly to an already challenged network could exacerbate the
problem. Also, keeping the probing volume towards individual
destination addresses low will increase the overall number of
destination addresses we can probe. 

The proposed approach has two tasks:
(a) find addresses that are ``related'' to a given address and are therefore
likely to fail together and
(b) design a probing scheme that balances false outage classification
rates with the number of sent probes.

\subsection{Finding related addresses}

Given any address to probe, a key task for the proposed approach is to
find addresses that are ``related'' to this address since such addresses are likely to share
fate with the given address and experience outages simultaneously. For instance, addresses can be
related by geography when they are physically proximate to each
other; a power outage can result in simultaneous outages
of many geographically proximate addresses. Alternately, addresses can be related by ISP, when they
share the same provider. Addresses can also be related by network
topology: for instance, they can share the same last-hop
router~\cite{hobbit-use-hobbit-imc-instead} or they can be assigned
from the same dynamic addressing
pool~\cite{addrchange-reasons}. 

To successfully segregate outages into their categories,
addresses related by a range of criteria need to be chosen because
outages with different causes can cause a different set of related
addresses to fail together. For example, if a
tree branch damages a network cable that serves an apartment building,
it is possible that only addresses that are related by network
topology will fail together. On the other hand, a failure in the
network infrastructure for an ISP could cause many addresses belonging to that ISP to
fail together across several geographic areas. A power outage can cause
addresses belonging to many different ISPs in a geographic region to
fail together.

Existing datasets can be used to find a set of \emph{candidate related
addresses}, according to
each criterion. Given an address, let $cra$ be the set of
candidate related addresses. To find addresses that are related to a
given address according to geography or ISP, we can use the
MaxMind database~\cite{MaxMind}, a popular commercial service used in
several large Internet measurement projects~\cite{censys-about,
usc-sandy, heidemann-diurnal}. Although we would expect intuitively
that addresses related by network topology can be found by simply
enumerating numerically adjacent addresses, preliminary work has shown
that subsequent dynamically addresses assigned to the same device are
often numerically distant~\cite{addrchange-reasons}. Thus, it is possible that even
numerically adjacent addresses are not related by network topology and
conversely, that numerically distant addresses are. Addresses from the same
dynamic addressing pool are available from preliminary
work~\cite{addrchange-reasons} and from other proposed work in
Section~\ref{sec:addr_change}. Addresses with the same last-hop router
as a given address are available from Lee and
Spring~\cite{hobbit-use-hobbit-imc-instead}. 

% After finding the set of candidate related addresses, $cra$, for each given
% address, the next step is to select a subset of these addresses for probing which
% will maximize the likelihood of finding simultaneous outages when they
% occur. I call this subset of addresses the probed related addresses ($pra$). Addresses that are topologically related (same last-hop router
% or same dynamic addressing pools) are likely to share fate upon
% outages; thus, I will first select topologically related addresses
% to a given address. Selecting addresses related by other criteria can
% also prove to be helpful for shedding light upon potential causes and
% the extent of outages. For example, power outages and network outages
% in an area can result in simultaneous outages for multiple IP
% addresses in that geographic area. By selecting related
% addresses from that geographic area, we can potentially disambiguate a power
% outage from a network outage: if addresses in that area from multiple
% ISPs fail simultaneously, the outage is likely a power outage and if
% addresses from only a single ISP failed, the cause is likely a network
% outage. If we also select addresses from multiple
% geographic areas, we can estimate the severity and potential cause of
% the outage. For example, if addresses fail simultaneously
% across a large geographic area, a severe outage has likely occurred,
% perhaps at some core infrastructure. 

% Although increasing the size of the set of probed related addresses,
% $pra$, can increase the likelihood of finding simultaneous outages, it will also result in higher
% probe volume. Thus, an important component of this task is to balance
% the addresses in $pra$ such that the probe volume remains reasonable
% while also maximizing our ability to observe simultaneous
% outages. My plan is to start with a small $pra$ set consisting of 11
% addresses: two topologically related, four geographically related
% and five for severity estimation. In the next section, I illustrate
% how probe traffic can remain low even when we have 11
% additional addresses to probe in the $pra$.


\subsection{Probing related addresses}

The other key task in the proposed approach is to probe the given
address and a subset of addresses from its $cra$ in such a manner as to
minimize false outage classifications while
keeping the probing volume low. Probing a larger subset of addresses
from the $cra$ will increase the likelihood of observing
simultaneous outages since there is a higher
likelihood that we are probing addresses that fail together. Probing more frequently
increases the accuracy of the measured simultaneity; for example,
probing all addresses every second will allow us to measure
simultaneous outages that occurred in the same second. Less frequent probes will
reduce accuracy by increasung the likelihood that two outages
which were not actually simultaneous are reported to be simultaneous. On the other hand, probing many addresses and
probing more frequently increase the probing volume. 

My first approach toward resolving these tradeoffs is to vary the number of
probed addresses from the $cra$ and the probing volume towards them and empirically
determine the resulting false classification rates. 
Ideally, a
complete list of ``ground truth'' outages per category would
exist against which the detected outages can be
compared. Although a complete list does not exist, partial lists of
network outages are available from the outages mailing
list~\cite{outages-mailing-list} and of power outages from the
U.S. Department of Energy~\cite{power-outages-us-official-list}. For
multiple-ISP
and shared-ISP outages, I intend to compare the detected
outages against the known outages in these lists for various combinations of probed addresses and
probing volume. While these partial lists can confirm that some
multiple-ISP and shared-ISP outages were indeed
categorized correctly, they
cannot inform if isolated outages were
categorized incorrectly into the multiple-ISP or shared-ISP
categories, since these lists are not exhaustive.

For determining how often isolated outages are falsely categorized into
the multiple-ISP or shared-ISP categories at various probing volumes, 
I intend to investigate outages detected by existing studies such as
Thunderping~\cite{pingin} and the ISI survey~\cite{census-survey}.
Using these existing measurements, I will first estimate the probability that
unrelated addresses that were probed fail ``simultaneously'' over
different windows of time. Thunderping pings thousands of
addresses every day in areas with severe weather alerts and each
address receives approximately a ping every minute when Thunderping
uses 11 vantage points. The ISI survey uses a different probing scheme: it sends probes
to every address in a /24 prefix once every 11 minutes over a two week
period for 24,000 /24 prefixes. With such a large volume of addresses being
probed, both Thunderping and the ISI survey detect many ``simultaneous'' outages; some of
these are likely true positive simultaneous outages and others
false. Simultaneous outages observed in these
datasets for addresses in unrelated ISPs and disparate geographic areas are
likely false positives. I intend to use such false positives to
estimate the probability of detecting false simultaneous outages---and
therefore, falsely categorizing outages into multiple-ISP or
shared-ISP---using
these techniques' default probing scheme. To obtain the false positive
rate for less frequent probing volumes, I intend to simulate the
false positives that would have resulted if probes had been send at
lower frequencies to a smaller subset of probed addresses. Similarly,
I will assess the false negative rate by looking for related addresses
that failed together in these datasets. Thunderping probes addresses
in similar geographic areas and the ISI survey probes all addresses in
a /24; thus, both schemes already probe some related addresses. I will
simulate lower probing-volume schemes and determine how many true simultaneous
outages affecting various related addresses will be missed and estimate
the false negative rate.

\subsection{Using categorized outages to estimate reliability}

After validating that the outage categorizations are accurate, I will
use the categorized outages to estimate ISP-level, media-type-level,
and geographical-area-level reliability using both reliability metrics
I defined in Section~\ref{sec:related}. For ISP-level and
media-type-level reliability analyses, I will only consider shared-ISP
outages. For comparing reliability across different geographies, I will consider multiple-ISP outages as
well. Also, I will perform an analysis of the proportion of
Internet-connected devices over time; this analysis will include every
discovered outage, including isolated ones.

% When analyzing if a particular geographical area has reliable
% Interent connectivity,
% multiple-ISP and shared-ISP outages should count as outages,
% but not user-caused ones. And when comparing ISP-level or
% media-type-level reliability, only shared-ISP outages should
% be treated as outages.

% The other key task in the proposed approach is to probe the given
% address and the addresses in its $pra$ in such as a manner as to
% find simultaneous outages with precision while keeping the probing
% volume low. If we probe all the $pra$ addresses every second, we
% can observe simultaneous outages with the precision of a second. Thus,
% observing just two simultaneous outages in a given second may be
% enough to convince us that these outages are involuntary, since two
% users are very unlikely to power off their equipment in the exact same
% second. On the other hand, if we only probe $pra$ addresses once every
% hour, we likely need more addresses to fail together simultaneously,
% because the likelihood that two users decided to voluntarily power off
% their devices in the same hour is higher.

% My idea to balance probing volume and simultaneous outage detection
% precision is to probe $pra$ addresses only when necessary for
% simultaneous outage detection. Whenever the given address appears to
% be experiencing an outage, addresses in $pra$ will need to be probed
% to check for simultaneity. I will probe the
% given address every minute; one ping per minute is the approximate
% probe-rate that an address receives from Thunderping when Thunderping
% has 11 vantage points. Upon a lost probe to the given address, I will
% immediately initiate probes to all the $pra$ addresses as well as to
% the given address using the probing scheme adopted by
% Thnderping~\cite{pingin}. During this phase, all addresses in the
% $pra$ as well as the given address will receive at least a ping every
% minute from each vantage point (assuming 11 vantage points). I will
% retry lost probes to any of these addresses with exponential backoff. This scheme
% will ensure that an outage that affects any subset of the given
% address and addresses in the $pra$ will be detected in the same minute.

% Since this probing scheme relies upon addresses in the $pra$ for
% simultaneous outage detection, an important task is to ensure that
% the addresses in $pra$ remain responsive to probes. Thus, even when the given address
% is responsive to probes, I will probe every $pra$ address
% once every 11 minutes from a single vantage point. When an address in
% the $pra$ is no longer responsive, I will replace it with another
% address from the $cra$.

% By ensuring that probes are only sent when necessary, I believe that
% the probe volume will be low enough to not trigger abuse
% reports. Consider the case when we select 11 $pra$ addresses
% for a given address from the same ISP and we use 11 vantage points. The probe rate towards that ISP
% in the absence of lost probes will only be increased by 1 probe per
% minute (since each $pra$ address will only receive a single
% probe every 11 minutes). This increase of 1 probe per minute is
% similar to the increase in Thunderping's probe rate, if Thunderping
% had selected a single extra IP address to probe in that ISP. When
% probes to the given address are lost, then the probe rate towards that
% ISP increases by a factor of 11. However, this increase in probe rate
% is justified because we can now detect simultaneous outages in the
% minute that they occur.


% Why should we send them traceroutes?

\section{Conclusion and future work}

In this dissertation, I described how to measure residential Internet
reliability remotely using probing-based techniques. These techni

First, I showed
how to detect Internet outages accurately using these techniques by
analyzing and mitigating potential scenarios that can cause these
techniques to make false inferences about detected outages. 

 can measure Internet
reliability for individual users broadly, longitudinally, and
accurately. In spite of their potential, these techniques can make
false inferences about outages in two scenarios: when probe
responses are delayed beyond timeouts and when addresses get dynamically
reassigned. I described preliminary work which studied how commonly
probes are delayed beyond responses and described measurements of
dynamic addressing across the world that can help build a model of
dynamic addressing. For each outlined scenario, I proposed approaches
that can mitigate false outage inferences when that scenario
occurs. Finally, I discussed an approach to segregate outages into
categories that suggest cause, and how we can use these categorized
outages to study Internet reliability along different dimensions.


\clearpage % TODO: Find out what clearpage does?
\bibliographystyle{plain}
\bibliography{longnames,outages}

\end{document}
