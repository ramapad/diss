
\section{Conclusion}

In this proposal, I described the problem of measuring last-mile Internet
reliability and illustrated how remote probing-based outage
detection techniques have the potential to measure Internet
reliability for individual users broadly, longitudinally, and
accurately. In spite of their potential, these techniques can make
false inferences about outages in two scenarios: when probe
responses are delayed beyond timeouts and when addresses get dynamically
reassigned. I described preliminary work which studied how commonly
probes are delayed beyond responses and described measurements of
dynamic addressing across the world that can help build a model of
dynamic addressing. For each outlined scenario, I proposed approaches
that can mitigate false outage inferences when that scenario
occurs. Finally, I discussed an approach to segregate outages into
categories that suggest cause, and how we can use these categorized
outages to study Internet reliability along different dimensions.
