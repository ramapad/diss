
\chapter{Conclusion}

In this dissertation, I described how to measure residential Internet
reliability remotely using probing-based techniques. While having the
ability to measure broadly, these techniques' outage inferences can be
inaccurate. My contributions have improved their accuracy, and have
allowed their detected outages to be used in metrics for comparing
residential Internet reliability of various ISPs, media types, and
geographic regions in different weather conditions. I showed how to detect Internet outages accurately using
probing-based techniques by analyzing and mitigating potential
scenarios that can cause these techniques to make false inferences
about detected outages. In Chapter~\ref{cpt:timeouts}, I investigated
how frequently probe responses can be delayed beyond commonly used
timeouts by analyzing ping response latencies from IP
addresses across the world in a variety of networks. In
Chapter~\ref{cpt:addr_change}, I analyzed dynamic addressing patterns
in ISPs to find networks where addresses are stable. I also showed how
to detect outages in networks where dynamic reassignment is common,
using complementary datasets that can provide information on whether a
device's IP address has changed. Chapter~\ref{cpt:corrfails}
demonstrated the need for detecting individual address outages when
measuring residential reliability. In Chapter~\ref{cpt:weather}, I
compared the reliability of ISPs, media-types, and geographic regions
across several weather conditions.


