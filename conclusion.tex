
\chapter{Conclusion and future work}

In this dissertation, I described how to measure residential Internet
reliability remotely using probing-based techniques. While having the
ability to measure broadly, these techniques' outage inferences can be
inaccurate. My contributions have improved their accuracy, and have
allowed their detected outages to be used in metrics for comparing
residential Internet reliability of various ISPs, media types, and
geographic regions in different weather conditions.

In Chapter~\ref{} began by showing how to detect Internet outages accurately using
probing-based techniques by analyzing and mitigating potential
scenarios that can cause these techniques to make false inferences
about detected outages. By analyzing ping response latencies from IP
addresses across the world in a variety of networks, I investigated
how frequently probe responses can be delayed beyond commonly used
timeouts. I next analyzed dynamic addressing patterns in ISPs to find
networks where addresses are stable. I also showed how to detect
outages in networks where dynamic reassignment is common, using
complementary datasets that can provide information on whether a
device's IP address has changed.


\section{Future Work}

% \subsection{Building a global model of dynamic address change}
\subsection{Tracking devices across IP addresses using IDs on a global
scale}

%Though the results from RIPE Atlas are promising, they are potentially
% biased toward technical users like most measurement infrastructure,
% and biased toward European deployment.

In Chapter~\ref{}, I showed how to use IDs from complementary datasets
to (i) analyze dynamic addressing patterns and (ii) to confirm
outages. However, the RIPE Atlas dataset was biased towards European
and North American networks. While the CDN dataset was broader, it
could still only offer confirmation for 1\% of Thunderping's detected outages.

With more sources of IDs such as those from RIPE Atlas, it may be
possible to model the likelihood of
address change can help prevent false inferences about outages and
their durations. For ISPs that change periodically and/or synchronously, the model can
predict when probe-loss is more likely due to
address changes than outages. For ISPs that change addresses upon most
outages, the model can inform in which ISPs outage duration detection
is particularly error-prone. For other ISPs which change addresses
mostly upon longer outages, the model can be used to estimate the
likelihood that an inferred outage ended falsely.

Orthogonally, If every outage detected by a probing-based technique could be
confirmed through a complementary dataset with IDs, then detected
outages would be a hundred percent accurate. Additionally, the
analysis of outage recovery durations for all of these outages will be
possible. 



% A less lofty goal is to simply analyze more ISPs.

The key to tracking IP address(es) assigned to a home router over time
is to associate some uniquely identifying feature (an ID) that remains
constant across the home router's address
changes. Chapter~\ref{cpt:addr_change} showed two examples of IDs: in the
case of RIPE Atlas probes', the probe ID remained unchanged and in the
case of the CDN software, the installation ID remained unchanged. I
presented two applications of such IDs: (i) it is possible to analyze dynamic
address assignment practices (ii) it is possible to confirm a detected
outage if the ID associated with the address was the same before and
after the outage.

A challenge, however, is to obtain sources of IDs that can scale
across the Internet. IP address changes can
be tracked over time if there exists some uniquely identifying feature
that remains constant across the device's address change. There are several potential datasets which have this
property:

\subsubsection{Dynamic DNS services}

Websites such as dyn.com~\cite{dyn} provide dynamic DNS. Dynamic
DNS is a service that allows users with a dynamic IP address to host
web services, by providing DNS services that can be easily updated to
reflect changes in users' IP addresses. Users of Dynamic DNS Services
run a daemon provided by the dynamic DNS provider, which is responsible for
determining the publicly visible IP address, and updating the A
record(s) for the user's domain(s). 

IP address changes can be tracked using the domain names
registered with dynamic DNS services. Since the domain name of a user
maps to her current IP address, we can use the domain name as a
fingerprint, and detect changes in IP addresses for each domain name
over time, by periodically obtaining the 'A' record associated with
each domain name. 

\paragraph{Geographic correlation of dynamic behavior}

\begin{figure}[tb]
% \centering
\begin{center}
\includegraphics[height=1.5in]{figs/did_dname_get_renum}
\end{center}
\caption{\label{fig:addr_change_per_ctry}
IP address renumbering in dynamic DNS domains over a week: Black
represents dynamic DNS domains which experienced at least one address
change, while grey represents domains whose addresses remained the
same. Renumbering behavior appears to be correlated with geographic
location.}
\end{figure}

As a proof of concept, I report on a preliminary result from this
approach: corroborating the geographic relationships in Figure~\ref{fig:conts_all_durs} 
while extending to countries not well represented by RIPE.
I
obtained 3000 dynamic DNS domains from three different dynamic DNS
services: 2000 from afraid.org~\cite{afraid}, 600 from dyn~\cite{dyn} and 400 from
noip.com~\cite{noip} and fetched the 'A' records from their respective
nameservers once every hour. I collected this data for a week, and
then inspected how many of these domains experienced at least one
address change during this time. Figure~\ref{fig:addr_change_per_ctry}
shows the number of domains that had at least one 
address change and the domains that had none. The y-axis is in log-scale. 
Address changes in Asian and Latin American countries appear
more prevalent, with more than a third of all domains in these
countries seeing at least one address-change. On the other hand,
Northern European countries observe fewer than 6\% of their domain names
experiencing an address change. Address changes are uncommon in
North America: only 3\% of domain names in the US and 6\% of domain
names in Canada observed an address change.



\subsection{Classifying IP addresses}

Probing-based techniques that seek to detect residential Internet
outages need a list of addresses classified as residential. More
broadly, a classification of the IP address space into residential,
enterprise, campus etc., can benefit any system that uses IP addresses
as a proxy for measurement, including IP address based host-reputation
systems~\cite{fail2ban,spamhaus,cbl,sorbs}. Recent work has also shown
that ISPs are increasingly likely to deploy Carrier Grade NATs (CGNs),
where tens of residential Internet connections are multiplexed over a single
public IPv4 address~\cite{cgn-imc16}.
%  and approaches to count
% participating users in peer-to-peer
% systems~\cite{p2pfilesharing,p2pavail,sen2004analyzing}.

In this dissertation, I relied upon classifications of addresses as
residential using reverse DNS based schemes from prior
work~\cite{pingin}. Many ISPs include hints about an address's
intended use in
the reverse DNS entry of that address. Recent research has further improved address
classification with reverse DNS names~\cite{youndo-rdns}. However,
it is not mandatory for ISPs to provide meaningful reverse DNS
names. Some large ISPs, such as AT\&T do not provide reverse DNS names
for most of their addresses, resulting in their addresses' under-representation in
Thunderping data as seen in Chapter~\ref{cpt:weather}.

An orthogonal approach to address classification is to use datasets with
some uniquely identifying feature (an ID) that can be used to track IP
addresses over time. By analyzing how many IDs are associated with an IP
address simultaneously and over time, I show in preliminary work that it is possible to infer
how the ISP is using the address~\cite{shared-addrs-apnic-blog, shared-addrs-aims}. An address that 



\subsection{Identifying outage causes}



\subsection{Exploring the tradeoff between outage detection and ICMP blocking}



\subsection{Measuring reliability in IPv6 networks}

