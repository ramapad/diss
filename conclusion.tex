
\section{Conclusion and future work}

In this dissertation, I described how to measure residential Internet
reliability remotely using probing-based techniques. These techni

First, I showed
how to detect Internet outages accurately using these techniques by
analyzing and mitigating potential scenarios that can cause these
techniques to make false inferences about detected outages. 

 can measure Internet
reliability for individual users broadly, longitudinally, and
accurately. In spite of their potential, these techniques can make
false inferences about outages in two scenarios: when probe
responses are delayed beyond timeouts and when addresses get dynamically
reassigned. I described preliminary work which studied how commonly
probes are delayed beyond responses and described measurements of
dynamic addressing across the world that can help build a model of
dynamic addressing. For each outlined scenario, I proposed approaches
that can mitigate false outage inferences when that scenario
occurs. Finally, I discussed an approach to segregate outages into
categories that suggest cause, and how we can use these categorized
outages to study Internet reliability along different dimensions.
