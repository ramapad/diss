
\chapter{Towards categorizing outages by
  their cause}
\label{cpt:corrfails}

Since Internet reliability measures are application-dependent,
different measures may need to consider different subsets of detected
outages.

In this chapter, I introduce a taxonomy of outages based upon their
likely cause.

% When analyzing the reliability of a particular ISP, we need
% to find the subset of outages that affected only
% that ISP. Doing so ensures that ISPs offering services in challenged
% areas do not have their reliability lowered by events such as power
% outages or users voluntarily shutting down their home Internet
% equipment.

I describe three categories under which detected outages can be 
placed. Each category provides hints about the likely cause of outages
placed in that category. For example, power outages or undersea cable cuts can affect addresses from multiple
ISPs; I term events which result in outages for
many providers' addresses \emph{multiple-ISP outages}. Users in some
geographic areas are particularly likely to shut down their Internet
equipment~\cite{grover2013peeking} but users elsewhere may also power
off their equipment when faced with certain circumstances, such as
approaching thunderstorms. I call such an event a \emph{user-caused
outage}. On the other hand, consider an ISP experiencing a failure in
its networking infrastructure resulting in an outage affecting only
addresses belonging to this ISP: these are outages that should bring down the ISP's reliability
estimate. I term these events \emph{shared-ISP outages}.
Probing-based remote outage detection techniques will detect outages
when all of these scenarios occur since previously responsive
addresses will cease responding in all these scenarios.

I develop and evaluate an approach for segregating outages into different categories
based upon the insight that outages occurring simultaneously in time
for addresses that are related by virtue of sharing geography or ISP
could share a common cause. For example, if we detect addresses from
many ISPs within geographically proximate regions failing simultaneously in time, we are likely observing a
multiple-ISP outage. If we detect addresses from only a single ISP
failing simultaneously in time, we are potentially observing a
shared-ISP outage. If the detected outage does not appear to
have happened simultaneously in time with other outages of related
addresses, I term it an \emph{isolated} outage. User-caused outages would
likely manifest as isoltated outages. Thus, evidence
of simultaneous failure of multiple
``related'' addresses can be used to distinguish between the different
categories of outages. 

