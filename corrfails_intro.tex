
\chapter{The need for measuring individual address outages}
\label{cpt:corrfails}

In this chapter, I develop and evaluate an approach to detect
dependent Internet failure events that affect multiple residential
users simultaneously using measurements of individual address outages
gathered with the Thunderping technique. 

I define a dependent Internet failure event to be one which affects
multiple customers due to a shared underlying cause (such as a network
or power outage). My insight is that such events can affect multiple
addresses that are related to each other simultaneously.

Using a novel approach based
upon Binomial hypothesis testing, I detect instances of per-address
outage events that are unlikely to have happened by random chance and
flag these as dependent Internet outages. Next, I characterize these
outages and present results that challenge conventional wisdom on how
such outages affect Internet address blocks. I show that many of these
outages would be missed by existing techniques that do not perform individual
address outage detection, and instead detect outages affecting entire
network prefixes.

% begin by providing background on why residential
% reliability measurements could require measurements of individual
% addresses' outages. 

\section{Background: dependent residential outages can be small}

Residential Internet connections are vulnerable. The last-mile link
connecting home routers to their ISP is typically not multi-homed and
is therefore a single point of failure. Further, last-mile links can
be damaged by exposure to the elements or by broken tree limbs blown
by the wind. Thus, residential outages could be small in the number of
affected users: they may affect only the users in an apartment block
or can even affect only a couple of neighboring homes.

% However, an individual address's outage need not necessarily be due to
% an outage; users may also choose to voluntarily turn off their home
% Internet equipment~\cite{grover2013peeking}.

However, prior probing-based techniques typically focus upon detecting failures
that affect many users and may miss observing smaller residential
failures.  They detect outages that affect entire BGP
prefixes~\cite{hubble} or /24 address
blocks~\cite{trinocular}. Residential outages may not be large enough
to be detected at these granularities.

