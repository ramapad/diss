
\chapter{The need for measuring individual address outages}
\label{cpt:corrfails}

In this chapter, I develop and evaluate an approach to detect
dependent \emph{Internet disruption} events that affect multiple residential
addresses simultaneously using measurements of individual address disruptions
gathered with the Thunderping technique. Borrowing terminology from
Richter et al.~\cite{advancing-outage-art}, I define an Internet
disruption event for an address to be the abrupt loss of response to active probing
from that address.

% I define a dependent Internet disruption event to be one which affects
% multiple customers due to a shared underlying cause (such as a network
% or power outage). My insight is that such events can affect multiple
% addresses that are related to each other simultaneously.
Techniques that detect outages at the Internet's edge often seek
disruption events affecting a substantial set of addresses. The set of addresses may comprise
those belonging to the same /24
address block~\cite{trinocular,advancing-outage-art}, BGP
prefix~\cite{hubble}, or country~\cite{dainotti-imc11}.  
Techniques seek such disruption events because individually, each large disruption has
impact and their size makes them easier to confirm, e.g., with operators. In contrast, disruptions
affecting only a few users are harder to detect with confidence.  For example, the
lack of response from a single address might best be explained by a
user switching off their home router---hardly an outage.
However, residential Internet
outages may be limited to a small neighborhood or apartment block; prior
techniques are likely to miss such events.

% In the rest of the chapter, I describe work with colleagues where we
% use a novel approach based upon Binomial hypothesis testing to
% detect instances of per-address disruption events that are unlikely to
% have happened independently and flag these as dependent Internet
% disruption events. Next, we characterize these dependent disruption
% events and present results that challenge conventional wisdom on how
% such disruptions affect Internet address blocks. We show that many of
% these events would be missed by existing techniques that do not
% perform individual address outage detection.

In the rest of this chapter, I describe work with colleagues where we
demonstrate a technique that detects disruption events with
quantifiable confidence, by investigating the potential dependence
between disruptions of multiple IP addresses in a principled way. We
apply a simple statistical method to a large dataset of active probing
measurements towards residential Internet users in the US. We find
times when multiple addresses experience a disruption simultaneously
such that they are unlikely to have occurred independently; we call
the occurrence of such events \emph{dependent disruptions}. We
characterize these dependent disruption events and present results
that challenge conventional wisdom on how such disruptions affect
Internet address blocks. We show that many of these events would be
missed by existing techniques that do not perform individual address
outage detection.

% begin by providing background on why residential
% reliability measurements could require measurements of individual
% addresses' outages. 

\section{Background: dependent residential outages can be small}

Residential Internet connections are vulnerable. The last-mile link
connecting home routers to their ISP is typically not multi-homed and
is therefore a single point of failure. Further, last-mile links can
be damaged by exposure to the elements or by broken tree limbs blown
by the wind. Thus, residential outages may be limited to a small
neighborhood or apartment block.

