% \section{Data Set}

% We use data from Thunderping.

\section{Results}

% \begin{figure}[t]
% \centering
% \includegraphics[width=.5\linewidth]{figs/dmin_vs_nout_660_jan17todec17_glorious_scatter}
% \caption{
% \label{fig:dmin_vs_nouts}
% Correlated dropouts detected in 2017 vs. $D_{min}$.}
% \end{figure}


\subsection{Correlated dropouts across ISPs}

Correlated dropouts occur across most of the large ISPs pinged by Thunderping.

\begin{figure}[t]
\centering
\includegraphics[width=.5\linewidth]{corrfails/figs/corrfails_by_isp_660_jan17todec17_filter0_bar}
\caption{
\label{fig:corr_dropouts_across_isps}
Correlated dropouts by ISP.}
\end{figure}

We detected 20,831 correlated dropouts in 2017 using the Binomial
test. We first grouped these dropouts by ISP to see if any ISPs have an
unusual number of correlated dropouts. Figure~\ref{fig:corr_dropouts_across_isps} shows the top 10
ISPs with correlated dropouts. These top ISPs together account for 65\% of all detected correlated dropouts.

% % How about saying we will consider a couple of different approaches? In
% % one, we will consider all correlated dropouts. In another, we will
% % consider the correlated dropouts that affect at least 95\% of the
% % addresses.

% % Give an ISP-level breakdown of both.

% % Perhaps begin with: is it dynamic addressing?

% % If it's dynamic addressing, does it depend upon the size?

\subsection{Correlated dropouts occur more frequently at night}

Prior work has shown that outages tended to happen more frequently
during maintenance intervals. Here, we investigate if the correlated
dropouts we detected also follow similar patterns.

\begin{figure}[t]
\centering
\includegraphics[width=.5\linewidth]{corrfails/figs/corrfails_by_timeofweek_660_jan17todec17_filter0_bar}
\caption{
\label{fig:corr_dropouts_timeofweek}
Correlated dropouts that began in each hour of the week. 'Mon' on the
bottom xaxis refers to midnight on Monday in UTC time. On the top xaxis, 'Mon' refers to midnight at UTC-6 (CST). }
\end{figure}

% \begin{figure}[t]
% \centering
% \includegraphics[width=.5\linewidth]{figs/corrfails_by_timeofweek_660_jan17todec17_filter0_asncomcast_bar}
% \caption{
% \label{fig:corr_dropouts_timeofweek_comcast}
% Comcast: Correlated dropouts that began in each hour of the week. 'Mon' on the
% bottom xaxis refers to midnight on Monday in UTC time. On the top xaxis, 'Mon' refers to midnight at UTC-6 (CST). }
% \end{figure}

% \begin{figure}[t]
% \centering
% \includegraphics[width=.5\linewidth]{figs/corrfails_by_timeofweek_660_jan17todec17_filter0_asn209_bar}
% \caption{
% \label{fig:corr_dropouts_timeofweek_qwest}
% Qwest: Correlated dropouts that began in each hour of the week. 'Mon' on the
% bottom xaxis refers to midnight on Monday in UTC time. On the top xaxis, 'Mon' refers to midnight at UTC-6 (CST). }
% \end{figure}

% \begin{figure}[t]
% \centering
% \includegraphics[width=.5\linewidth]{figs/corrfails_by_timeofweek_660_jan17todec17_filter0_asn7155_bar}
% \caption{
% \label{fig:corr_dropouts_timeofweek_viasat}
% Viasat: Correlated dropouts that began in each hour of the week. 'Mon' on the
% bottom xaxis refers to midnight on Monday in UTC time. On the top xaxis, 'Mon' refers to midnight at UTC-6 (CST). }
% \end{figure}

For each hour of the week, we found how many correlated dropouts
started, and show these in
Figure~\ref{fig:corr_dropouts_timeofweek}.
As expected, there are more
correlated dropouts close to midnight, CST, on weekday nights,
corroborating previous work.

% \subsection{Correlated dropouts can recover together}

% Since Thunderping continues to probe an IP address even after it
% becomes unresponsive until the end of the weather alert, Thunderping
% can observe when the address becomes responsive again. This
% responsiveness may signal that the outage has ended.

% However, just because an address starts responding to pings again does not necessarily
% mean that it has recovered from a failure: it is possible that the
% address has been reassigned to some other device which did not have a
% failure. 

% Our insight is that multiple addresses that dropout together and also
% recover together offer evidence that: (a) the event was an outage. (b)
% the outage event has indeed ended. 

% Table~\ref{tbl:nouts_vs_nrecs} shows how many addresses recovered in a
% correlated manner when depending upon the number of addresses that
% were affected for a correlated dropout. For many correlated dropouts,
% irrespective of their size, we observe the majority of addresses recover.

% % \begin{table}[th]
% % %  \scriptsize
% %   \centering
% %   \hspace{-0.04in}\tiny
% %   \begin{tabular}{c|c|c|c|c|c|c|c|c|c|c|c|}
% % \textbf{Dropouts} & \multicolumn{11}{c|}{Correlated recoveries} \\
% %     & \textbf{0} & \textbf{2} &
% %     \textbf{3} & \textbf{5} & \textbf{10}  & \textbf{25}  &
% %     \textbf{50}  & \textbf{100} & \textbf{250}  & \textbf{500}  &
% %     \textbf{1000}\\
% %     \hline
% % 2 & 5390 (47\%) & 5994 (52\%) & 0 (0\%) & 0 (0\%) & 0 (0\%) & 0 (0\%) & 0 (0\%) & 0 (0\%) & 0 (0\%) & 0 (0\%) & 0 (0\%)\\
% % 3 & 3733 (42\%) & 2939 (33\%) & 2119 (24\%) & 0 (0\%) & 0 (0\%) & 0 (0\%) & 0 (0\%) & 0 (0\%) & 0 (0\%) & 0 (0\%) & 0 (0\%)\\
% % 5 & 3388 (32\%) & 3148 (30\%) & 1691 (16\%) & 2077 (20\%) & 0 (0\%) & 0 (0\%) & 0 (0\%) & 0 (0\%) & 0 (0\%) & 0 (0\%) & 0 (0\%)\\
% % 10 & 1659 (17\%) & 2224 (23\%) & 1414 (15\%) & 1926 (20\%) & 2167 (23\%) & 0 (0\%) & 0 (0\%) & 0 (0\%) & 0 (0\%) & 0 (0\%) & 0 (0\%)\\
% % 25 & 209 (4\%) & 465 (9\%) & 452 (9\%) & 787 (16\%) & 1401 (29\%) & 1437 (30\%) & 0 (0\%) & 0 (0\%) & 0 (0\%) & 0 (0\%) & 0 (0\%)\\
% % 50 & 16 (1\%) & 40 (3\%) & 39 (3\%) & 99 (8\%) & 206 (16\%) & 423 (34\%) & 396 (32\%) & 0 (0\%) & 0 (0\%) & 0 (0\%) & 0 (0\%)\\
% % 100 & 3 (0\%) & 6 (1\%) & 11 (2\%) & 22 (4\%) & 48 (8\%) & 84 (15\%) & 128 (23\%) & 235 (43\%) & 0 (0\%) & 0 (0\%) & 0 (0\%)\\
% % 250 & 1 (0\%) & 0 (0\%) & 0 (0\%) & 0 (0\%) & 11 (7\%) & 11 (7\%) & 19 (13\%) & 47 (32\%) & 55 (38\%) & 0 (0\%) & 0 (0\%)\\
% % 500 & 0 (0\%) & 0 (0\%) & 0 (0\%) & 0 (0\%) & 0 (0\%) & 0 (0\%) & 1 (7\%) & 0 (0\%) & 3 (23\%) & 9 (69\%) & 0 (0\%)\\
% % 1000 & 0 (0\%) & 0 (0\%) & 0 (0\%) & 0 (0\%) & 1 (25\%) & 0 (0\%) & 0 (0\%) & 0 (0\%) & 0 (0\%) & 1 (25\%) & 2 (50\%)\\
% %     \end{tabular}
% %   \caption{\label{tbl:nouts_vs_nrecs_without_nonrecs} Number of addresses that
% %     recovered for correlated dropouts affecting different numbers of addresses.
% %   }
% % \end{table}


% \begin{table}[th]
% %  \scriptsize
%   \centering
%   \hspace{-0.04in}\tiny
%   \begin{tabular}{c|c|c|c|c|c|c|c|c|c|c|c|c|c|}
% \textbf{Dropouts} & \multicolumn{13}{c|}{Correlated recoveries} \\
%     & \textbf{-2} & \textbf{-1} & \textbf{0} & \textbf{2} &
%     \textbf{3} & \textbf{5} & \textbf{10}  & \textbf{25}  &
%     \textbf{50}  & \textbf{100} & \textbf{250}  & \textbf{500}  &
%     \textbf{1000}\\
%     \hline
% 2 & 623 (24\%) & 286 (11\%) & 425 (16\%) & 1259 (49\%) & 0 (0\%) & 0 (0\%) & 0 (0\%) & 0 (0\%) & 0 (0\%) & 0 (0\%) & 0 (0\%) & 0 (0\%) & 0 (0\%)\\
% 3 & 476 (16\%) & 283 (10\%) & 463 (16\%) & 741 (25\%) & 1006 (34\%) & 0 (0\%) & 0 (0\%) & 0 (0\%) & 0 (0\%) & 0 (0\%) & 0 (0\%) & 0 (0\%) & 0 (0\%)\\
% 5 & 520 (11\%) & 304 (6\%) & 761 (15\%) & 1050 (21\%) & 915 (18\%) & 1401 (28\%) & 0 (0\%) & 0 (0\%) & 0 (0\%) & 0 (0\%) & 0 (0\%) & 0 (0\%) & 0 (0\%)\\
% 10 & 349 (7\%) & 184 (3\%) & 568 (11\%) & 888 (17\%) & 799 (15\%) & 1159 (22\%) & 1377 (26\%) & 0 (0\%) & 0 (0\%) & 0 (0\%) & 0 (0\%) & 0 (0\%) & 0 (0\%)\\
% 25 & 169 (5\%) & 65 (2\%) & 136 (4\%) & 259 (8\%) & 250 (8\%) & 470 (14\%) & 861 (26\%) & 1067 (33\%) & 0 (0\%) & 0 (0\%) & 0 (0\%) & 0 (0\%) & 0 (0\%)\\
% 50 & 47 (4\%) & 13 (1\%) & 18 (2\%) & 23 (2\%) & 31 (3\%) & 77 (7\%) & 146 (14\%) & 358 (34\%) & 335 (32\%) & 0 (0\%) & 0 (0\%) & 0 (0\%) & 0 (0\%)\\
% 100 & 15 (3\%) & 2 (0\%) & 4 (1\%) & 7 (1\%) & 5 (1\%) & 23 (5\%) & 31 (6\%) & 71 (14\%) & 147 (30\%) & 193 (39\%) & 0 (0\%) & 0 (0\%) & 0 (0\%)\\
% 250 & 2 (1\%) & 0 (0\%) & 1 (1\%) & 0 (0\%) & 0 (0\%) & 1 (1\%) & 8 (5\%) & 17 (11\%) & 16 (10\%) & 52 (34\%) & 56 (37\%) & 0 (0\%) & 0 (0\%)\\
% 500 & 0 (0\%) & 0 (0\%) & 0 (0\%) & 0 (0\%) & 0 (0\%) & 0 (0\%) & 0 (0\%) & 0 (0\%) & 2 (13\%) & 0 (0\%) & 2 (13\%) & 11 (73\%) & 0 (0\%)\\
% 1000 & 0 (0\%) & 0 (0\%) & 0 (0\%) & 0 (0\%) & 0 (0\%) & 0 (0\%) & 1 (33\%) & 0 (0\%) & 0 (0\%) & 0 (0\%) & 0 (0\%) & 1 (33\%) & 1 (33\%)\\
%     \end{tabular}
%   \caption{\label{tbl:nouts_vs_nrecs} The number of addresses that
%     recovered (columns) for correlated dropouts affecting different numbers of
%     addresses (rows). -2 indicates that no addresses that dropped out
%     recovered. -1 indicates that only one address recovered. The other
%     numbers show how many of the (at least two) addresses that
%     recovered did so in a correlated manner.
%   }
% \end{table}

% \begin{table}[th]
% %  \scriptsize
%   \centering
%   \hspace{-0.04in}\tiny
%   \begin{tabular}{c|c|c|c|c|c|c|c|c|c|c|c|c|c|}
% \textbf{Dropouts} & \multicolumn{13}{c|}{Correlated recoveries} \\
%     & \textbf{-2} & \textbf{-1} & \textbf{0} & \textbf{2} &
%     \textbf{3} & \textbf{5} & \textbf{10}  & \textbf{25}  &
%     \textbf{50}  & \textbf{100} & \textbf{250}  & \textbf{500}  &
%     \textbf{1000}\\
%     \hline
% 2 & 48 (26\%) & 16 (9\%) & 47 (25\%) & 76 (41\%) & 0 (0\%) & 0 (0\%) & 0 (0\%) & 0 (0\%) & 0 (0\%) & 0 (0\%) & 0 (0\%) & 0 (0\%) & 0 (0\%)\\
% 3 & 34 (13\%) & 8 (3\%) & 50 (19\%) & 86 (32\%) & 90 (34\%) & 0 (0\%) & 0 (0\%) & 0 (0\%) & 0 (0\%) & 0 (0\%) & 0 (0\%) & 0 (0\%) & 0 (0\%)\\
% 5 & 58 (8\%) & 24 (3\%) & 53 (7\%) & 152 (21\%) & 178 (25\%) & 247 (35\%) & 0 (0\%) & 0 (0\%) & 0 (0\%) & 0 (0\%) & 0 (0\%) & 0 (0\%) & 0 (0\%)\\
% 10 & 50 (6\%) & 19 (2\%) & 49 (5\%) & 84 (9\%) & 158 (18\%) & 251 (28\%) & 290 (32\%) & 0 (0\%) & 0 (0\%) & 0 (0\%) & 0 (0\%) & 0 (0\%) & 0 (0\%)\\
% 25 & 21 (3\%) & 12 (2\%) & 23 (4\%) & 48 (7\%) & 42 (6\%) & 77 (12\%) & 226 (35\%) & 201 (31\%) & 0 (0\%) & 0 (0\%) & 0 (0\%) & 0 (0\%) & 0 (0\%)\\
% 50 & 5 (2\%) & 3 (1\%) & 4 (2\%) & 4 (2\%) & 9 (4\%) & 18 (8\%) & 49 (21\%) & 92 (39\%) & 52 (22\%) & 0 (0\%) & 0 (0\%) & 0 (0\%) & 0 (0\%)\\
% 100 & 3 (3\%) & 0 (0\%) & 1 (1\%) & 0 (0\%) & 0 (0\%) & 5 (5\%) & 14 (13\%) & 19 (17\%) & 41 (38\%) & 26 (24\%) & 0 (0\%) & 0 (0\%) & 0 (0\%)\\
% 250 & 0 (0\%) & 0 (0\%) & 0 (0\%) & 0 (0\%) & 0 (0\%) & 0 (0\%) & 3 (7\%) & 4 (9\%) & 5 (12\%) & 18 (42\%) & 13 (30\%) & 0 (0\%) & 0 (0\%)\\
% 500 & 0 (0\%) & 0 (0\%) & 0 (0\%) & 0 (0\%) & 0 (0\%) & 0 (0\%) & 0 (0\%) & 0 (0\%) & 0 (0\%) & 0 (0\%) & 0 (0\%) & 2 (100\%) & 0 (0\%)\\
% 1000 & 0 (0\%) & 0 (0\%) & 0 (0\%) & 0 (0\%) & 0 (0\%) & 0 (0\%) & 0 (0\%) & 0 (0\%) & 0 (0\%) & 0 (0\%) & 0 (0\%) & 0 (0\%) & 1 (100\%)\\
%     \end{tabular}
%   \caption{\label{tbl:nouts_vs_nrecs_comcast} For Comcast, the number of addresses that
%     recovered (columns) for correlated dropouts affecting different numbers of
%     addresses (rows). -2 indicates that no addresses that dropped out
%     recovered. -1 indicates that only one address recovered. The other
%     numbers show how many of the (at least two) addresses that
%     recovered did so in a correlated manner.
%   }
% \end{table}


% \begin{table}[th]
% %  \scriptsize
%   \centering
%   \hspace{-0.04in}\tiny
%   \begin{tabular}{c|c|c|c|c|c|c|c|c|c|c|c|c|c|}
% \textbf{Dropouts} & \multicolumn{13}{c|}{Correlated recoveries} \\
%     & \textbf{-2} & \textbf{-1} & \textbf{0} & \textbf{2} &
%     \textbf{3} & \textbf{5} & \textbf{10}  & \textbf{25}  &
%     \textbf{50}  & \textbf{100} & \textbf{250}  & \textbf{500}  &
%     \textbf{1000}\\
%     \hline
% 2 & 27 (20\%) & 11 (8\%) & 16 (12\%) & 80 (60\%) & 0 (0\%) & 0 (0\%) & 0 (0\%) & 0 (0\%) & 0 (0\%) & 0 (0\%) & 0 (0\%) & 0 (0\%) & 0 (0\%)\\
% 3 & 43 (32\%) & 11 (8\%) & 13 (10\%) & 23 (17\%) & 45 (33\%) & 0 (0\%) & 0 (0\%) & 0 (0\%) & 0 (0\%) & 0 (0\%) & 0 (0\%) & 0 (0\%) & 0 (0\%)\\
% 5 & 66 (18\%) & 26 (7\%) & 69 (19\%) & 89 (24\%) & 55 (15\%) & 60 (16\%) & 0 (0\%) & 0 (0\%) & 0 (0\%) & 0 (0\%) & 0 (0\%) & 0 (0\%) & 0 (0\%)\\
% 10 & 65 (11\%) & 24 (4\%) & 109 (19\%) & 131 (22\%) & 94 (16\%) & 99 (17\%) & 61 (10\%) & 0 (0\%) & 0 (0\%) & 0 (0\%) & 0 (0\%) & 0 (0\%) & 0 (0\%)\\
% 25 & 20 (5\%) & 11 (3\%) & 36 (9\%) & 54 (14\%) & 34 (9\%) & 86 (22\%) & 102 (26\%) & 57 (14\%) & 0 (0\%) & 0 (0\%) & 0 (0\%) & 0 (0\%) & 0 (0\%)\\
% 50 & 1 (1\%) & 1 (1\%) & 6 (8\%) & 6 (8\%) & 4 (5\%) & 12 (16\%) & 10 (14\%) & 21 (29\%) & 12 (16\%) & 0 (0\%) & 0 (0\%) & 0 (0\%) & 0 (0\%)\\
% 100 & 1 (4\%) & 0 (0\%) & 1 (4\%) & 2 (7\%) & 1 (4\%) & 2 (7\%) & 0 (0\%) & 1 (4\%) & 8 (30\%) & 11 (41\%) & 0 (0\%) & 0 (0\%) & 0 (0\%)\\
% 250 & 0 (0\%) & 0 (0\%) & 0 (0\%) & 0 (0\%) & 0 (0\%) & 0 (0\%) & 0 (0\%) & 1 (8\%) & 2 (17\%) & 5 (42\%) & 4 (33\%) & 0 (0\%) & 0 (0\%)\\
% 500 & 0 (0\%) & 0 (0\%) & 0 (0\%) & 0 (0\%) & 0 (0\%) & 0 (0\%) & 0 (0\%) & 0 (0\%) & 1 (33\%) & 0 (0\%) & 1 (33\%) & 1 (33\%) & 0 (0\%)\\
% 1000 & 0 (0\%) & 0 (0\%) & 0 (0\%) & 0 (0\%) & 0 (0\%) & 0 (0\%) & 1 (50\%) & 0 (0\%) & 0 (0\%) & 0 (0\%) & 0 (0\%) & 1 (50\%) & 0 (0\%)\\
%     \end{tabular}
%   \caption{\label{tbl:nouts_vs_nrecs_qwest} For Qwest, the number of addresses that
%     recovered (columns) for correlated dropouts affecting different numbers of
%     addresses (rows). -2 indicates that no addresses that dropped out
%     recovered. -1 indicates that only one address recovered. The other
%     numbers show how many of the (at least two) addresses that
%     recovered did so in a correlated manner.
%   }
% \end{table}

% \begin{table}[th]
% %  \scriptsize
%   \centering
%   \hspace{-0.04in}\tiny
%   \begin{tabular}{c|c|c|c|c|c|c|c|c|c|c|c|c|c|}
% \textbf{Dropouts} & \multicolumn{13}{c|}{Correlated recoveries} \\
%     & \textbf{-2} & \textbf{-1} & \textbf{0} & \textbf{2} &
%     \textbf{3} & \textbf{5} & \textbf{10}  & \textbf{25}  &
%     \textbf{50}  & \textbf{100} & \textbf{250}  & \textbf{500}  &
%     \textbf{1000}\\
%     \hline
% 2 & 7 (9\%) & 19 (24\%) & 22 (28\%) & 31 (39\%) & 0 (0\%) & 0 (0\%) & 0 (0\%) & 0 (0\%) & 0 (0\%) & 0 (0\%) & 0 (0\%) & 0 (0\%) & 0 (0\%)\\
% 3 & 5 (3\%) & 11 (7\%) & 45 (30\%) & 57 (38\%) & 32 (21\%) & 0 (0\%) & 0 (0\%) & 0 (0\%) & 0 (0\%) & 0 (0\%) & 0 (0\%) & 0 (0\%) & 0 (0\%)\\
% 5 & 5 (1\%) & 10 (2\%) & 111 (25\%) & 144 (33\%) & 85 (19\%) & 88 (20\%) & 0 (0\%) & 0 (0\%) & 0 (0\%) & 0 (0\%) & 0 (0\%) & 0 (0\%) & 0 (0\%)\\
% 10 & 3 (0\%) & 7 (1\%) & 65 (10\%) & 159 (24\%) & 161 (25\%) & 156 (24\%) & 104 (16\%) & 0 (0\%) & 0 (0\%) & 0 (0\%) & 0 (0\%) & 0 (0\%) & 0 (0\%)\\
% 25 & 1 (0\%) & 0 (0\%) & 5 (1\%) & 18 (5\%) & 38 (11\%) & 81 (24\%) & 104 (31\%) & 90 (27\%) & 0 (0\%) & 0 (0\%) & 0 (0\%) & 0 (0\%) & 0 (0\%)\\
% 50 & 2 (2\%) & 0 (0\%) & 1 (1\%) & 0 (0\%) & 1 (1\%) & 7 (5\%) & 29 (22\%) & 63 (49\%) & 26 (20\%) & 0 (0\%) & 0 (0\%) & 0 (0\%) & 0 (0\%)\\
% 100 & 1 (2\%) & 0 (0\%) & 0 (0\%) & 0 (0\%) & 0 (0\%) & 0 (0\%) & 0 (0\%) & 2 (5\%) & 24 (57\%) & 15 (36\%) & 0 (0\%) & 0 (0\%) & 0 (0\%)\\
% 250 & 0 (0\%) & 0 (0\%) & 0 (0\%) & 0 (0\%) & 0 (0\%) & 0 (0\%) & 1 (6\%) & 0 (0\%) & 2 (12\%) & 5 (29\%) & 9 (53\%) & 0 (0\%) & 0 (0\%)\\
% 500 & 0 (0\%) & 0 (0\%) & 0 (0\%) & 0 (0\%) & 0 (0\%) & 0 (0\%) & 0 (0\%) & 0 (0\%) & 0 (0\%) & 0 (0\%) & 1 (33\%) & 2 (67\%) & 0 (0\%)\\
% 1000 & 0 (0\%) & 0 (0\%) & 0 (0\%) & 0 (0\%) & 0 (0\%) & 0 (0\%) & 0 (0\%) & 0 (0\%) & 0 (0\%) & 0 (0\%) & 0 (0\%) & 0 (0\%) & 0 (0\%)\\
%     \end{tabular}
%   \caption{\label{tbl:nouts_vs_nrecs_viasat} For Viasat, the number of addresses that
%     recovered (columns) for correlated dropouts affecting different numbers of
%     addresses (rows). -2 indicates that no addresses that dropped out
%     recovered. -1 indicates that only one address recovered. The other
%     numbers show how many of the (at least two) addresses that
%     recovered did so in a correlated manner.
%   }
% \end{table}


% \begin{figure}[t]
% \centering
% \includegraphics[width=.5\linewidth]{figs/corrdropouts_durs_660_jan17todec17_filter0_cdf}
% \caption{
% \label{fig:corrdropouts_durs}
% Distribution of durations of correlated dropouts. The curve includes the
% duration of a detected correlated dropout as many times as there were addresses
% that recovered in a correlated manner. For example, if 10 addresses
% dropped out and 5 recovered in a correlated manner, and if the
% duration of the recovery was 1h, then we would consider 1h five
% times. Around two-thirds of
% detected dropouts lasted less than an hour.}
% \end{figure}


% \subsection{Correlated dropouts may be prefix reassignment}

% Recent work has shown that the inactivity of addresses does not
% necessarily indicate an Internet failure event.

\subsection{Correlated dropouts do not always knock out entire  /24s}

We observe that many correlated dropouts did not affect all addresses
in a /24. Prior work that detect outages affecting /24 address blocks
can miss these events~\cite{trinocular, advancing-outage-art}.

Of 20,831 detected correlated dropouts, 12,825(61\%) have at least one
responsive address in \emph{all} the /24s that the correlated dropout
affected. Even if some of the /24s have addresses that failed by
random chance, the  assumption requires that at least one
/24 (the one with the correlated failure) should be 'knocked out',
i.e., have no responsive addresses. This assumption is incorrect 61\%
of the time.

\begin{figure}[t]
\centering
\includegraphics[width=.5\linewidth]{corrfails/figs/outpers24_vs_uppers24_all_660_jan17todec17_scatter}
\caption{
\label{fig:outpers24_vs_uppers24_all}
Addresses out in a /24 vs. addresses alive in a /24, for all /24s with
an address that dropped out during a detected correlated
dropout event for that state-ASN.}
\end{figure}

Next, we investigate how many addresses were alive in a /24 which had
an address that may have been involved in a correlated dropout, in Figure~\ref{fig:outpers24_vs_uppers24_all}
For each correlated dropout, for each /24 involved in the correlated
dropout, we first found how many addresses dropped out. Next, we found
how many addresses continued to respond in the /24. The 20,831
detected dropouts affected addresses from 92,777 /24s. Each of these /24s
is a point in Figure~\ref{fig:outpers24_vs_uppers24_all}.

It it is hard to reason about /24s with less than Dmin failures, since
the addresses in those /24s may have failed independently. However,
recall that Dmin gives the minimum dropouts necessary to infer a
correlated dropout across the entire U.S. state and ASN
aggregate. Thus, when more than Dmin failures happen \emph{in the same
  /24}, that /24's addresses are highly likely to have been affected
by the correlated dropout. 

\begin{figure}[t]
\centering
\includegraphics[width=.5\linewidth]{corrfails/figs/outpers24minusdmin_vs_uppers24_all_660_jan17todec17_scatter}
\caption{
\label{fig:outpers24minusdmin_vs_uppers24_all}
Minimum addresses out in a /24 vs. addresses alive in a /24, for all /24s with
at least Dmin address that dropped out during a detected correlated
dropout event for that state-ASN.}
\end{figure}

 Thus, we next find just the /24s with at least Dmin failures; there
were 14,702 such /24s of the 92,277 total. For these /24s, we
subtracted Dmin-1 from the total failed addresses: this gives us the
minimum number of addresses that failed in the /24 as part of the
correlated dropout (we subtract Dmin - 1 since there could have been
Dmin - 1 independent
failures). Figure~\ref{fig:outpers24minusdmin_vs_uppers24_all} shows
the 14,702 /24s' minimum number of addresses that failed in the /24
along the xaxis and the number of addresses that continued to respond
along the yaxis. 10,164 (69\%) /24s had at least one responsive
address. 5,850 (40\%) had at least 10 responsive addresses. 1,691 /24s
had at least 10 addresses that were part of the correlated dropout. Of
those, 550 (33\%) had at least 10 responsive addresses.

% These results highlight the value in detecting outages at the
% individual address level.

% Thus, prior work that detects 

% \subsection{Correlated dropouts can be multi-ISP}

% We observe correlated dropouts occurring simultaneously across
% multiple ISPs. We confirm that some of them correlate with large known
% power outages reported in the media.

