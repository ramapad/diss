
\subsubsection*{How many addresses are disrupted dependently?}

\begin{figure}[t]
    \centering
    \includegraphics[width=0.5\linewidth]{corrfails/figs/dmin_vs_nout_660_jan17todec17_extra_glorious_scatter}
   \caption[$D_{min}$ vs observed disruptions, for each
   detected dependent disruption event]{
   \label{fig:dmin_vs_nout}
For each detected correlated disruption event,
Figure~\ref{fig:dmin_vs_nout} shows the $D_{min}$ value on the x-axis and
the corresponding number of observed disruptions on the y-axis. 62\% of
the 20,831 detected events had more than $D_{min}$ observed
disruptions.  The scatterplot adds a random gaussian offset to both $x$ and $y$ with mean of 0.1, clamped at 0.45, to show density.
 }
\end{figure}
%
\hfill
%

The Binomial test does not say that \emph{all} of the
addresses that were observed to be disrupted during a dependent
event were disrupted in a dependent manner. Consider if $D_{min}$ is 4 and we
detect an event where 7 addresses were disrupted. The Binomial Test
shows us that the event took place with very low probability. However,
that does not necessarily mean all 7 addresses were disrupted in a dependent manner;
up to 3 of them could have been disrupted independently with up to 99.99\%
probability.

We call the set of addresses in a state-ASN aggregate that were
disrupted in the time-bin of a dependent event the observed
group of addresses that were disrupted, or the \emph{observed
  disrupted group} for short. Of the observed disrupted group, our assumption is that some were disrupted together in a
dependent manner: we call this subset the actual group of
addresses that were disrupted, or \emph{actual disrupted group}. We obtain a minimum bound on the actual
disrupted group by subtracting $D_{min} - 1$
from the observed disrupted group. For the 20,831 dependent disruption
events, the total addresses in all the observed disrupted groups is 229,413 and
the total addresses in all the minimum actual groups is 165,328.

We study the relationship between $D_{min}$ for a state-ASN
aggregate on the x-axis and the
corresponding number of addresses in the observed group of disrupted addresses (on the
y-axis) in Figure~\ref{fig:dmin_vs_nout}. Each point corresponds to one
of the 20,831 detected events. Sometimes, a state-ASN
aggregate had such low $P_d$ that even a single disruption in a
11-minute bin occurred with less than 0.01\% probably and therefore
had a $D_{min}$ value of 1. However, since we are looking for unlikely
disruptions of multiple addresses, all our detected events observed at
least two addresses that were disrupted in the same time-bin.

12,911 (62\%) detected events observed \emph{more} than $D_{min}$
disruptions, corroborating the result from Figure~\ref{fig:err_prob}
that most detected events would have been detected even with a
stricter threshold. 

We detected dependent disruption events with various sizes as shown in
Figure~\ref{fig:dmin_vs_nout}. There are
693 (3\%) events with more than 50 observed disrupted addresses. For the largest
detected event, we observed 913 addresses experience disruptions in
the same time-bin in AS33489 (Comcast) in
Florida at 2017-09-13T20:33 UTC time. This detected event correlates
to the minute with a known failure event for Comcast
that was discussed in the Outages mailing list~\cite{comcast-sep-13-2017-large-outage-mailinglist}. However, for most of the events, the size of the observed
group of disrupted addresses is small: there were 2,593 (12\%)
with two, 2,969 (14\%) with three, 2,776 (13\%) with four, and 2,175
(10\%) with five observed disrupted addresses. These results highlight the
ability of our technique to detect even small sized disruptions with
confidence.


\section{Properties of dependent disruptions}

In this section, we study various properties of dependent
disruptions. For some properties, we conduct additional analyses on
specific ISPs in the Thunderping dataset: Comcast (cable), Qwest (DSL)
and Viasat (Satellite). These are three ISPs whose addresses are
pinged frequently by Thunderping (as seen in
Figure~\ref{fig:iggy_ips_frac_per_as})
and where we were able to
detect in excess of a thousand dependent disruption events (3109
events for Comcast, 1855 for Viasat, 1734 for Qwest).

\section{Dependent disruption events across ISPs}

\label{sec:corrfails_by_isp}

% Dependent disruptions occur across most of the large ISPs pinged by Thunderping.

\begin{figure}[t]
\centering
\includegraphics[width=.5\linewidth]{corrfails/figs/corrfails_by_isp_660_jan17todec17_filter0_bar}
\caption[Dependent disruption events detected per ISP]{
\label{fig:corrfails_across_isps}
Figure~\ref{fig:corrfails_across_isps} shows the number of
dependent disruption events detected per ISP. Note that these numbers
are more a reflection of addresses sampled and pinged in the
Thunderping dataset than any major underlying problem in their
infrastructure.}
\end{figure}

We grouped dependent disruption events by ISP to check if any ISPs contribute an
unusual number of events. Figure~\ref{fig:corrfails_across_isps} shows the top 15
ISPs with dependent disruption events. Most of the ISPs from
Figure~\ref{fig:iggy_ips_frac_per_as} are represented here as well, suggesting
that no ISPs are unduly biasing our results. These top 15 ISPs together account
for 13,643 (65\%) of all detected events.

We emphasize that these results are not meant to reflect any
underlying problems with these ISPs; the Thunderping system samples
and pings large ISPs more frequently and consequently, finds more
disrupted addresses in them. The purpose of this analysis is to ensure
that no ISP contributes unduly many events.

\subsection{Dependent disruptions are more frequent at night for some
ISPs}

\begin{figure*}[t]
\begin{subfigure}[t]{0.32\linewidth}
\centering
\includegraphics[width=\linewidth]{corrfails/figs/corrfails_by_timeofweek_660_jan17todec17_filter0_asncomcast_bar}
\caption{
Comcast
\label{fig:corrfails_timeofweek_comcast}
}
\end{subfigure}
%
\hfill
%
\begin{subfigure}[t]{0.32\linewidth}
\centering
\includegraphics[width=\linewidth]{corrfails/figs/corrfails_by_timeofweek_660_jan17todec17_filter0_asn209_bar}
\caption{
Qwest
\label{fig:corrfails_timeofweek_qwest}
}
\end{subfigure}
%
\hfill
%
\begin{subfigure}[t]{0.32\linewidth}
\centering
\includegraphics[width=\linewidth]{corrfails/figs/corrfails_by_timeofweek_660_jan17todec17_filter0_asn7155_bar}
\caption{
Viasat
\label{fig:corrfails_timeofweek_viasat}
}
\end{subfigure}
\caption[Dependent disruption events that began in each hour of the week]{
\label{fig:corrfails_tow_asn}
%Investigating addresses with high loss rates
Dependent disruption events that began in each hour of the week. 'Mon' on the
bottom x-axis refers to midnight on Monday in UTC time. On the top x-axis, 'Mon' refers to midnight at UTC-6 (CST). 
}
\end{figure*}

% TODO: Add more citations for the midnight local time thing. I think
% Matthew's router reboots work has something? I know we cite other
% people in Richter et. al.
 
Recent work has shown that disruptions tend to happen more frequently
during maintenance intervals close to midnight local
time~\cite{advancing-outage-art}. To obtain this result, Richter et
al.~used proprietary data from a content delivery network, collected at
the granularity of every hour.  Here, we investigate if our technique
can identify similar patterns of dependent disruptions.

Figure~\ref{fig:corrfails_tow_asn} shows that individual ISPs can have
different behavior. Comcast and Viasat have more dependent disruption
events occurring close to midnight, CST, on weekday nights. Qwest, on
the other hand, does not appear to have a clearly discernible pattern.
%
Our results confirm those from prior work~\cite{advancing-outage-art},
lending credence to our technique.
%
Moreover, we are able to do so using public (Thunderping) data and a
granularity of every 11 minutes.


\subsection{Dependent disruptions can recover together}

\label{sec:rec}

Here, we investigate whether dependent disruption events are
accompanied by \emph{dependent recovery}. Since Thunderping continues
to probe an IP address even after it becomes unresponsive until the end
of the weather alert, it can observe when the address becomes
responsive again. This responsiveness may signal that the disruption
for the address has ended. Multiple addresses that are disrupted
together and also recover together offer evidence that: (a)~the event
was indeed dependent and (b)~the event has ended, allowing estimation
of the disruption's duration.

Most dependent disruptions also have correlated recoveries.
%
Of 20,831 dependent disruption events, 6,869 (33\%) had \emph{all}
disrupted addresses 
% in the observed disrupted group 
recover during the same 11-minute time-bin.
%
Further, 14,789 (71\%) disruption events had at least half of the
disrupted addresses recover together.
%
Across all of the 20,831 dependent disruption events, there were
229,413 disrupted addresses in total.
%
Of these, 121,648 (53\%) disrupted addresses---from 15,117 (73\%) 
disruption events---exhibited a dependent recovery with other addresses
from that same group.
%
This indicates that dependent recovery is quite common.
%

\begin{table*}[th]
%  \scriptsize
  \centering
  \hspace{-0.04in}\tiny
  \begin{tabular}{c|c|c|c|c|c|c|c|c|c|c|c|}
\textbf{Disruptions} & \multicolumn{11}{c|}{Correlated recoveries} \\
    & \textbf{-2} & \textbf{-1} & \textbf{0} & \textbf{2} &
    \textbf{3} & \textbf{$\leq$ 5} & \textbf{$\leq$ 10}  &
    \textbf{$\leq$ 50}  & \textbf{$\leq$ 100} & \textbf{$\leq$ 500}  &
    \textbf{$\leq$ 1000}\\
    \hline
2 & 623 (24\%) & 286 (11\%) & 425 (16\%) & 1259 (49\%) & 0 (0\%) & 0 (0\%) & 0 (0\%) & 0 (0\%) & 0 (0\%) & 0 (0\%) & 0 (0\%)\\
3 & 476 (16\%) & 283 (10\%) & 463 (16\%) & 741 (25\%) & 1006 (34\%) & 0 (0\%) & 0 (0\%) & 0 (0\%) & 0 (0\%) & 0 (0\%) & 0 (0\%)\\
$\leq$ 5 & 520 (11\%) & 304 (6\%) & 761 (15\%) & 1050 (21\%) & 915 (18\%) & 1401 (28\%) & 0 (0\%) & 0 (0\%) & 0 (0\%) & 0 (0\%) & 0 (0\%)\\
$\leq$ 10 & 349 (7\%) & 184 (3\%) & 568 (11\%) & 888 (17\%) & 799 (15\%) & 1159 (22\%) & 1377 (26\%) & 0 (0\%) & 0 (0\%) & 0 (0\%) & 0 (0\%)\\
$\leq$ 50 & 216 (5\%) & 78 (2\%) & 154 (4\%) & 282 (7\%) & 281 (6\%) & 547 (13\%) & 1007 (23\%) & 1760 (41\%) & 0 (0\%) & 0 (0\%) & 0 (0\%)\\
$\leq$ 100 & 15 (3\%) & 2 (0\%) & 4 (1\%) & 7 (1\%) & 5 (1\%) & 23 (5\%) & 31 (6\%) & 218 (44\%) & 193 (39\%) & 0 (0\%) & 0 (0\%)\\
$\leq$ 500 & 2 (1\%) & 0 (0\%) & 1 (1\%) & 0 (0\%) & 0 (0\%) & 1 (1\%) & 8 (5\%) & 35 (21\%) & 52 (31\%) & 69 (41\%) & 0 (0\%)\\
$\leq$ 1000 & 0 (0\%) & 0 (0\%) & 0 (0\%) & 0 (0\%) & 0 (0\%) & 0 (0\%) & 1 (33\%) & 0 (0\%) & 0 (0\%) & 1 (33\%) & 1 (33\%)\\
    \end{tabular}
  \caption[Dependent recovery]{\label{tbl:nouts_vs_nrecs} The number of addresses that
    recovered (columns) for dependent disruptions affecting different numbers of
    addresses (rows). -2 indicates that no addresses that dropped out
    were observed to have recovered. -1 indicates that only one address recovered. The other
    numbers show how many of the (at least two) addresses that
    recovered did so in a correlated manner.
  }
\end{table*}

We also tested whether the likelihood of dependent recovery is a
function of the number of addresses in the observed disrupted group.
%
It is possible
that disruptions with fewer addresses in the observed disrupted group
tended to experience correlated recovery more frequently. 
%
As the number of
addresses in the observed disrupted group increases, do the number of
addresses that recover in a correlated manner also increase?

Table~\ref{tbl:nouts_vs_nrecs} answers this question. The $-2$ and
$-1$ columns show events where there is insufficient data from the
Thunderping dataset to determine recovery; $-2$ shows events where
none of the addresses in the observed disrupted group responded to
Thunderping's pings after the disruption and $-1$ shows events where only
one of the addresses responded to Thunderping's pings after the
disruption. The rest of the columns show how many events recovered in a
correlated manner. We observe that for the majority of events,
irrespective of the number of addresses in the observed disrupted
group, more than 50\% recover together.


\subsubsection*{Recovery times are often shorter than an hour}


\begin{figure*}[t]
\begin{subfigure}[t]{0.47\linewidth}
\centering
\includegraphics[width=\linewidth]{corrfails/figs/raw_corrdropouts_durs_660_jan17todec17_filter0_cdf}
\caption{
\label{fig:corrdropouts_durs}
}
\end{subfigure}
%
\hfill
%
\begin{subfigure}[t]{0.47\linewidth}
  \centering
  \includegraphics[width=\linewidth]{corrfails/figs/nrecs_vs_durs_660_jan17todec17_glorious_scatter}
  \caption{
    \label{fig:nrecs_vs_durs}
}
\end{subfigure}
%
% \hfill
%
\caption[Durations of dependent disruption events]{
\label{fig:rec_durs}
% Caption would go here
(a)~The distribution of durations of dependent dropouts for all addresses that
recovered in a correlated manner. 60\% of addresses recovered in less than an hour.
(b)~For dependent dropout events where at least two addresses recovered,
this shows the number of addresses that recovered on the x-axis and the
corresponding recovery duration for the event on the y-axis. Dependent
dropout events vary in their duration irrespective of the number of
affected addresses.
}
\end{figure*}

Next, we turn our attention to the time it takes dependent disruptions
to recover.
%
Figure~\ref{fig:corrdropouts_durs} shows that 60\% of recovered
addresses recovered in less than an hour.
%
Our technique is able to identify this, because we operate at the
precision of the 11-minute time-bins from standard outage detection
datasets.
%
Conversely, recent work that finds disruptions spanning an entire
calendar hour~\cite{advancing-outage-art} would miss these disruptions.


Next, we examine whether short recovery durations can be attributable
to small disruption events: that is, do the recoveries appear quick
because only a couple hosts were disrupted?
%
Figure~\ref{fig:nrecs_vs_durs} shows that the answer is no: Even
dependent disruptions with hundreds of addresses that recovered
together often last less than an hour.

\subsection{Dependent disruptions can be multi-ISP}

\begin{figure}[t]
\centering
\includegraphics[width=3.3in]{corrfails/figs/corrfails_over_time_660_jan17todec17_scatter_no_nils}
\caption[Multi-ISP dependent disruption events over time]{
\label{fig:multiisp_over_time}
Multi-ISP dependent disruption events over time: several ISPs in the same
state have simultaneous disruption events on 333 occasions. Here, we
show how many events occurred on each day of the year in 2017. Days
with many multi-ISP events often correlate with days with large known power outages.}
\end{figure}

\begin{figure}[t]
\centering
\includegraphics[width=.5\linewidth]{corrfails/figs/corrfails_over_irma}
\caption[Multi-ISP dependent disruption events during Hurricane Irma]{
\label{fig:corrfails_over_irma}
Multi-ISP dependent disruption events during Hurricane Irma in Florida (FL), Georgia (GA),
and South Carolina (SC). Of 111 events during this time, 15 affected 3
ISPs simultaneously and 96 affected 2.}
\end{figure}

Dependent disruption events can also span multiple ISPs within a
single state: these events indicate a fault of infrastructure shared
by the ISP or their customers. 
%
Here, we broaden our analysis to examine whether the dependent
disruption events we detected are
correlated across multiple ISPs within the same state.

We observe 333 instances where multiple ISPs in the same state had simultaneous dependent
disruption events, and we are able to
confirm that many occurred on days when the media reported large power
outages in those areas. Figure~\ref{fig:multiisp_over_time} shows days
in 2017 when multi-ISP dependent disruption events occurred. Of the
333 instances, 88 (26\%)
occurred on a single day during Hurricane Irma (Sep 11). Figure~\ref{fig:corrfails_over_irma} shows
multi-ISP events during Hurricane Irma by state and by the number of individual ISPs
affected during each multi-ISP event. We observe 20 multi-ISP events
in Florida on Sep 10, when Irma made landfall~\cite{irma-noaa}. As Irma moves northwards, we see
multi-ISP events in Georgia and South Carolina as well. Other days with
many such events include Oct 30 with 19 events across six
states in the Northeastern U.S.~(Maine, New Hampshire, Vermont,
Connecticut, Massachusetts, Rhode Island); there were recorded power
outages during this time as a result of a severe
storm~\cite{oct-30-2017-northeast-bombogenesis-weathercom,
  oct-30-2017-northeast-bombogenesis-nws, oct-30-2017-northeast-bombogenesis-wapo}. On Oct 22,
there were 4 multi-ISP events in Oklahoma and 2 in Arkansas; there are
corresponding reports
of power outages during these times as
well~\cite{oct-22-2017-okar-tornado-newson6}.
