
\section{Motivation: Find dependent outage events}
\label{sec:bg}
% \subsection{Dropouts are supersets of outages and reassignment}

%Internet outages are rare events: thus, 

Detecting Internet outages requires
continuous measurement of a large fraction of the address space. Active
probing systems are capable of such measurements: the Thunderping~\cite{pingin} and
Trinocular~\cite{trinocular} systems have been detecting Internet outages for
over five years. % These systems can, in theory, monitor the entire Internet.

\subsection{Uncertainty in interpreting active probe responses}

%An IP address that is being pinged can be either responsive or
%unresponsive. Some addresses never respond to ICMP packets, either
%because the address has not been assigned to a host or due to ICMP
%filtering~\cite{v4-census-imc16} along the path to the host. For such
%always-unresponsive addresses, inferring the assigned host's Internet
%connectivity with ICMP packets is impossible.  For an address that
%responds to pings, it is possible to use ping responses to reason
%about the connectivity of a host to which the address has been
%assigned.  
%
%A ping response from an address indicates that the host's Internet connection is working. 
%If a previously responsive address ceases to respond to pings
%suggests that the host's Internet connection might no longer be
%working. 
While consistent lack of response to ICMP packets from an IP address
(e.g. due to ICMP filtering~\cite{v4-census-imc16}) makes it impossible
to infer the status of that address's connectivity, a change in its responding
behavior enables opportunities for inference.  We define the event
where a pinged address transitions from a state where it is responding to probes to a state where it ceases to respond as a \emph{dropout}.  

However, dropouts are not necessarily the result of Internet failure
events. For example, a single user might decide to turn off their home
router. Such uncertainty can be addressed
in various ways.
% could result in an active probing scheme detecting a dropout when the home router's address ceases to respond. 
% Additionally, recent work
% has shown that dropouts can also occur as a result of \emph{dynamic
% address reassignment}~\cite{addrchange-reasons}, when the host's ISP
% assigns it a new address and if the old address (the one being pinged)
% is no loner assigned to any host, or is assigned to a host that is
% configured to not respond to pings. 

\subsection{Trinocular assumes addresses within the same /24 share fate}

% Trinocular~\cite{trinocular} and Thunderping~\cite{pingin} take
% orthogonal approaches to address the tradeoff between probing traffic
% and the ability to detect Internet failure events when they occur. We describe each tool in detail next.

% The ``Trinocular'' tool detects outages affecting /24 address
% blocks~\cite{trinocular}.  It assumes that addresses in a /24 share fate.

Trinocular pings addresses in ~4M /24 address blocks and
uses the responses to detect Internet outages affecting entire blocks. Using historical
data from the ISI census~\cite{census-survey}, it models the responsiveness of
blocks and finds subsets of addresses within each block that are
likely to respond to pings. The system pings a few of these addresses
from each block at random and probes them in 11-minute
rounds. Trinocular then employs Bayesian inference to reason about
responses from blocks. When a block's responsiveness is lower than
expected, Trinocular probes the block at a faster rate and eventually
detects an outage when the follow-up probes also suggest the block's
lack of Internet connectivity.

Trinocular assumes that dynamic addressing and individual user actions
will typically not result
in an entire /24 address block becoming unresponsive. However, since
Trinocular detects outages at the /24-granularity, it can miss outages
that do not span an entire /24 address-block. 

A key difference of this work from Trinocular is that we do not assume
that Internet failures span entire /24-address blocks; instead, we
look for failures of addresses related by geography and
ISP.  While we share Trinocular's intuiton that dependent events will
affect related addresses, Trinocular's notion of relatedness is solely
based on belonging to the same /24. With the IPv4 address space
breaking up, we hypothesize that addresses in disparate /24s might be
affected by a correlated failure event. Further, a power outage may
affect multiple addresses but these addresses could belong to several ISPs.

\subsection{Goal: Address the ambiguity in Thunderping}

% To evaluate if outages indeed span an entire /24, 
In this paper, our goal is to use observed dropouts discovered by the
Thunderping technique to infer Internet failure events. 

Thunderping pings
sampled addresses from multiple ISPs in geographic areas in the United
States. Originally designed to evaluate how weather affects Internet
outages, the system uses Planetlab vantage points to ping 100 IPv4
addresses from multiple ISPs in U.S.~counties with active
weather alerts. Each address is pinged from multiple Planetlab vantage
points (at least 3) every 11 minutes, and addresses in a county are
pinged six hours before, during, and after a weather alert for that
county. 

% Since Thunderping probes an address every 11-minutes
% from each vantage point, 
% once a vantage point lists an address as unresponsive, we ensure that
% all vantage points that are pinging the address also
% confirm its unresponsiveness. 

We analyze Thunderping's ping responses to detect dropout events for each probed
address using Schulman and Spring's technique~\cite{pingin}. When an address that is responsive stops responding to pings
from all vantage points that are currently probing it, we detect a
dropout event for that address. Once we determine that an address has
continued to remain unresponsive for an entire probing window of
11-minutes from all vantage points, we detect a dropout and
adjust the start-time of the dropout to be the time at which the first
vantage point observed the address to be unresponsive. Thunderping
continues to probe an address after it has become unresponsive, allowing us to estimate how long the unresponsive period lasted.

Since Thunderping
detects dropouts for individual IP addresses, these dropouts could
be due to Internet failure events, dynamic addressing, or even due to 
individual users' actions. We address this uncertainty in the rest of
the paper.


% Address why we need Binomial instead of Bayesian. Such a large
% aggregate that Bayesian would only allow entire aggregate
% failures. But Binomial uses correlation in time to detect dropouts
% that happen close together in time. So it's all about the time. And
% even with the coarse granularity of Thunderping, dropouts are rare
% enough that it allows detection of correlated failures often.

% Our position shoud be a motivation for a broader scheme. Need to ping
% more addresses. It's worth it. That should be the goal of this paper.

% When using the Thunderping technique to detect Internet outages affecting
% multiple users, there is another challenge: the Thunderping technique
% will include events where users voluntarily power down their home
% Internet equipment. Since Thunderping probes individual IP addresses, if a user
% powers down their home router, the address assigned to the router will
% cease to respond to pings and this will be detected as a dropout event.
