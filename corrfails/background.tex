
\subsection{Prior techniques focus upon larger disruptions}

Prior techniques that detect edge Internet disruptions typically detect disruptions that affect a group of
addresses \emph{collectively}. Like us, they also leverage the
\textit{dependence} among the per-IP address ``disruptions'' that
these larger disruptions cause. However, they differ from our techique in that they look for dependence in large aggregates
(that is, so many addresses are affected at the same time that there
must be an evident anomaly) or limit their resolution to small address
blocks, looking only for outages that cause dependent
disruptions for \textit{all} the addresses in a monitored block.
Thus, these techniques may miss observing smaller residential
failures. 

For example, Trinocular looks only for outages affecting
/24 address blocks~\cite{trinocular}. Using
historical data from the ISI census~\cite{census-survey}, it models
the responsiveness of blocks and finds addresses within each block
that are likely to respond to pings. The system pings a few of these
addresses from each block at random in 11-minute
rounds. Trinocular then employs Bayesian inference to reason about
responses from blocks. When a block's responsiveness is lower than
expected, Trinocular probes the block at a faster rate and eventually
detects an outage when the follow-up probes also suggest the block's
lack of Internet connectivity. Since Trinocular will not identify an
outage if a single address in a block responds to probing, Trinocular
potentially neglects outages affecting /24 blocks only partially,
including larger outages affecting multiple /24 blocks. 

Other systems have also investigated disruptions affecting entire blocks of
addresses. Recently, Richter \etal used CDN logs to
detect disruptions affecting /24 address
blocks~\cite{advancing-outage-art}. Hubble detects prefix-level
unreachability problems~\cite{hubble}. The IODA system looks for the most impactful outages
only, those causing an extensive loss of connectivity for a
geographical area or Autonomous System
(AS)~\cite{dainotti-imc11, ioda-project-page}. 

Disco~\cite{disco} shares some features with our
work: they also detect simultaneous disconnects of multiple RIPE Atlas
probes within an ISP or geographic region to infer outages. However,
there are two major differences between the Thunderping and RIPE Atlas
datasets. At any given point in time, the Thunderping dataset
typically consists of pings sent to roughly
50,000 addresses in relatively small geographical areas with active severe weather
alerts. The Disco dataset consists of 10,000 RIPE Atlas probes
distributed around the world; this sparse distribution may prevent the
detection of smaller outages localized to one area (like a U.S. state). The second difference is that unlike
Thunderping ping data whose timestamps are only accurate to minutes,
the timestamps available in the RIPE Atlas datasets are accurate to
seconds, permitting the use of Kleinberg's burst detection to detect
bursts in probe disconnects. Discussions with the authors of Disco
suggested that Kleinberg's burst detection model would not be
appropriate for the Thunderping data, although a more detailed
evaluation of the binomial test against Kleinberg's burst
detection in the Thunderping data is future work.

\subsection{The Thunderping dataset yields per-address disruptions}

The key insight behind our technique is that simultaneous disruptions of
multiple individual IPv4 addresses could occur due to a common
underlying cause. We therefore require per-IP
address disruptions.

Such data is present in the Thunderping dataset~\cite{pingin}. Thunderping pings
sampled IPv4 addresses from multiple ISPs in geographic areas in the United
States. Originally designed to evaluate how weather affects Internet
outages, the system uses Planetlab vantage points to ping 100 IPv4
addresses from multiple ISPs in each U.S.~county with active
weather alerts. Each address is pinged from multiple Planetlab vantage
points (at least 3) every 11 minutes, and addresses in a county are
pinged six hours before, during, and after a weather alert. 

Here, we analyze a dataset of Thunderping's ping responses to detect
disruptions for each probed address using Schulman and Spring's
technique~\cite{pingin}. When an address that is responsive stops
responding to pings from all vantage points that are currently probing
it, we detect a disruption for that address. Since a disruption is
detected only when all vantage points declare unreachability, the
minimum duration of a disruption is 11 minutes (at the end of 11
minutes each vantage point has pinged the address at least
once). Thunderping continues to probe an address after it has become
unresponsive, allowing us to estimate how long the unresponsive period
lasted.

While per-IP address disruptions allow the detection of small
disruptions, all per-address disruptions are not necessarily the
result of Internet connectivity outages. For example, an individual
user may decide to turn off their home router. In the rest of this
chapter, we show how to detect dependent disruption events using
per-address disruptions.