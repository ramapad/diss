\section{Detecting correlated dropouts}
\label{sec:method}

Thunderping pings thousands of addresses simultaneously. When
several of them dropout at the same time, there could be a common
underlying cause, in such a case, the dropouts are statistically
\emph{dependent}.
%. The intuition behind our approach is that simultaneous dropouts could be \emph{dependent} upon each other (\ie sharing the same cause).
%
In addition, outages are likely to affect addresses that are related to
each other by geography or ISP, an insight that we share with
Disco~\cite{disco}. In this work, we call groups of related addresses, such as all addresses within the
same U.S.~state or ISP, \emph{address aggregates} and we look for
dependent dropouts within such aggregates. Specifically, we
focus upon one particular class of address aggregates: the state-ASN
aggregate, which includes all addresses that belong to a particular
ASN in a particular U.S.~state.
%
%However, when thousands of addresses are being pinged, multiple
%addresses can fail by random chance. 
We quantify the probability that addresses in an aggregate can fail by
random chance using the Binomial test, and identify events that are
highly unlikely to have occurred at the same time by chance.

\subsection{Finding dependent events in an address aggregate}

When the number of observed addresses ($N$) within an address
aggregate is large, there is some chance that multiple of them
dropout simultaneously but independently (by random chance).
In this section, we describe how we use Binomial hypothesis testing to find events that
are highly unlikely to have occurred independently. Suppose $P_d$ is
the probability that an address from the aggregate drops out in a
given time window. Given
$N$ addresses that can potentially dropout, the Binomial distribution gives the probability that
$K$ of them dropout \emph{independently} as:

\[
{N\choose k} \cdot P_d^k(1-P_d)^{N-k}
\]

Given $N$ and $P_d$, it is possible to calculate $D_{min}$, the minimum number of
dropouts that need to occur, such that the probability of
$D_{min}$ or more dropouts occurring independently is small (less
than 1\%). If we observe $D_{min}$ or more dropouts, we reason that
these dropouts are \emph{not} independent, that is they share a common underlying cause.

\subsection{Analyzing $D_{min}$}

% $D_{min}$ determines the minimum dropouts that need to occur in an
% aggregate. 

Here, we investigate $D_{min}$ values for different combinations of
$N$ and $P_d$.

\subsubsection*{Filtering addresses with high loss}

Due to a variety of causes, such as high response
latency~\cite{timeouts} and ICMP rate-limiting and filtering, we find
that there exist addresses for which Thunderping experiences unusually high
ping loss rates. These addresses dropout very
frequently. 

Figure~\ref{fig:lrs_per_pingedaddr} shows the distribution of ping
loss rates for IP addresses during times when the addresses were not
experiencing a dropout. 2\% of addresses have loss rates exceeding
10\%. We filter these addresses since they can inflate the probability
of dropout.

\begin{figure}[t]
\centering
\includegraphics[width=.5\linewidth]{corrfails/figs/lrs_per_pingedaddr_jan17todec17_cdf}
\caption{
\label{fig:lrs_per_pingedaddr}
Distribution of ping loss rates per IP address during times when Thunderping believed an
address was \emph{not} experiencing a dropout. While most addresses
have low loss rates, 2\% of addresses had loss rates exceeding 10\%.}
\end{figure}

\begin{figure}[t]
\centering
\includegraphics[width=.5\linewidth]{corrfails/figs/iggy_ips_frac_per_as_jan17todec17}
\caption{
\label{fig:iggy_ips_frac_per_as}
Fraction of addresses per ISP with ping loss rates exceeding 10\%
during non-dropout periods, for the 15 ISPs with most pinged
addresses. Addresses in some ISPs such as Viasat, Verizon Wireless, and Pavlov
Media, are more likely to have high loss rates. Even for these ISPs,
there remain other addresses with lower loss rates, so high loss-rate does not seem
to be an entire-ISP property.}
\end{figure}


\subsubsection*{How many addresses can potentially fail in an
  aggregate?} % ($N$)?}

% \begin{figure}[t]
% \centering
% \includegraphics[width=\linewidth]{figs/upips_per_timebin_660_jan17todec17_cdf}
% \caption{
% \label{fig:upips_per_timebin}
% Distribution of all addresses pinged by Thunderping that can potentially fail in each 11 minute
% time bin}
% \end{figure}

\begin{figure*}[t]
  \begin{subfigure}[t]{0.47\linewidth}
    \centering
    \includegraphics[width=\linewidth]{corrfails/figs/upips_per_timebin_660_jan17todec17_cdf} 
    \caption{
      \label{fig:upips_per_timebin}
      Distribution of all addresses pinged by Thunderping that can potentially fail in each 11 minute
      time bin
    }
  \end{subfigure}
  \hfill
  \begin{subfigure}[t]{0.47\linewidth}
    \centering
   \includegraphics[width=.45\textwidth]{corrfails/figs/dropout_rate_stateasnwise_660_cdf} 
 \caption{
   \label{fig:p_dropout}
   Distribution of the probability of dropout ($P_d$) for various state-ASN address aggregates
 }
\end{subfigure}
\caption{Determining candidate $N$ and $P_d$ based upon Thunderping data.
  \label{fig:binomial_params}
}
\end{figure*}


Since we use data from Thunderping, we first determine $N$, the
maximum number
of addresses that can potentially fail in each address
aggregate. Figure~\ref{fig:upips_per_timebin} shows the distribution
of addresses pinged by Thunderping in each 11 minute time bin
in 2017. The
median number is around 50K addresses across all U.S. states and
ISPs. Since many weather events tend to be active at any given point of time,
these addresses are likely to be distributed among tens of state-ASN
aggregates. Thus, in each aggregate, there can be anywhere between 0
to 20K addresses, depending upon the size of the state and ISP, and
the number of counties in the state where the weather alert is
active. In 2017, the maximum addresses that could potentially fail in
any state-ASN aggregate was 15,863. 

\subsubsection*{How does the probability of dropout ($P_d$) vary across
  state-asn aggregates?}

We calculate the probability that an address from an aggregate drops
out in a given window of time, $P_d$, using the Thunderping
dataset as follows:

\begin{equation}
P_d= \frac{dropouts}{timebins}
\label{eq:p_dropout}
\end{equation}

We use 11-minute long time-bins since Thunderping pings an address
once every 11 minutes from each vantage point. Since dropouts are
detected when every vantage point confirms that its pings have become
unresponsive, a dropout may take up to 11 minutes to get detected. For
each aggregate, the formula calculates the probability that an address
drops out in a 11-minute time bin.

We only consider state-ASN aggregates where we were able to obtain a
statistically significant value for $P_d$. For statistical
significance, we adhere to the following rule of
thumb~\cite[Chapter~6]{biostats-book}: we accept a state-asn aggregate with $t$ timebins
and estimated probability of dropout $P_d$ only if 

\begin{equation}
	tP_d(1-P_d) \geq 10
	\label{eq:stat_sig}
\end{equation}

% \begin{figure}[t]
% \centering
% \includegraphics[width=\linewidth]{figs/dropout_rate_660_cdf}
% \caption{
% \label{fig:p_dropout}
% Distribution of $P_d$ for different address aggregates. }
% \end{figure}

% \begin{figure}[t]
% \centering
% \includegraphics[width=\linewidth]{figs/dropout_rate_stateasnwise_660_cdf}
% \caption{
% \label{fig:p_dropout}
% Distribution of the probability of dropout ($P_d$) for various state-ASN address aggregates.}
% \end{figure}

Figure~\ref{fig:p_dropout} shows how $P_d$ is distributed across
state-ASN aggregates. There is extensive 
variation: addresses in some of these aggregates dropout only once every
year, whereas in other aggregates they dropout more often than once per
day.\footnote{Since dropouts are a superset of outages and dynamic
  reassignment, frequent dropouts are not necessarily indicative of
  poor Internet connectivity.} % However, when considering entire U.S. states (which contain
% several ISPs), $P_d$ exhibits lesser variation. 

% $D_{min}$ values depend not only upon $P_d$, but also upon the number
% of addresses ($N$) that can potentially fail in an aggregate in a
% timebin. 

\subsubsection*{$D_{min}$ for different combinations of $N$ and $P_d$}

For different values of $N$ and $P_d$, we list $D_{min}$ values in
Table~\ref{tbl:binomial_thresh}. 

\begin{table}[th]
%  \scriptsize
  \centering
  \hspace{-0.04in}
  \begin{tabular}{c|c|c|c|c|}
$N$ & \multicolumn{4}{c|}{Probability of dropout ($P_d$)} \\
    & \textbf{1/hour} & \textbf{1/day} & \textbf{1/week} &
    \textbf{1/month} \\
    \hline
10 & 6 & 2 & 2 & 1\\
50 & 17 & 3 & 2 & 2\\
100 & 29 & 4 & 2 & 2\\
500 & 113 & 10 & 4 & 2\\
1000 & 213 & 16 & 5 & 3\\
5000 & 982 & 54 & 13 & 5\\
10000 & 1925 & 98 & 20 & 8\\
50000 & 9369 & 429 & 73 & 23\\
    \end{tabular}
  \caption{\label{tbl:binomial_thresh} $D_{min}$ values
    for different
    values of number of addresses that can potentially fail ($N$) and
    the probability of dropout ($p_d$). There is less than 1\%
    probability according to the Binomial hypothesis test that
    $D_{min}$ or more addresses fail for each $N$ and $P_d$.
  }
\end{table}

These results show that the approach will work well for some ASNs (i.e., be
sensitive) but not for others who have such high $P_d$ that we would need to see a very large number of dropouts to suggest that something happened.


% The minimum number of dropouts
%     such that their likelihood of occurrence is small 

% Depending upon 
% Since outages are usually
% rare events, $P_o$ in a small window of time (like minutes) is
% low. Consequently, even a relatively small number of failures


% When that probability is very, very low then we can argue that those X addresses must have failed in a dependent manner, likely due to a common cause.

% Say that there is some probability, p that an outage can occur in any minute. Probability that ‘k’ outages happen independently when we ping ‘n’ addresses is given by the Binomial distribution:
% nCk*(p)^k * (1-p)^n-k

% The main problem is arriving at the probability that an outage occur in any minute. Currently, I divide #outages/total-up-minutes for some aggregate of addresses (say all Comcast addresses in a given US state). Perhaps I can do better. 

% After finding correlated failures, my goal is to make some progress towards the potential cause of each correlated failure. Failures where multiple addresses from multiple ISPs are affected would be consistent with a power outage, for example.



