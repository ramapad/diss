
\chapter{Future Work}

\label{cpt:future}

Here, I identify directions for future work in measuring residential
reliability using probing-based techniques.

% \subsection{Building a global model of dynamic address change}
\section{Tracking devices across IP addresses using IDs on a global
scale}

%Though the results from RIPE Atlas are promising, they 

In Chapter~\ref{cpt:addr_change}, I showed how to use IDs from complementary datasets
to (i) analyze dynamic addressing patterns and (ii) to confirm
outages. However, the RIPE Atlas dataset consisted of ten thousand probes, a small fraction of residential users around the world. While the CDN dataset was obtained from an order of magnitude more devices than RIPE Atlas, it
could still only offer confirmation for 1\% of Thunderping's detected outages.

With more sources of IDs, it may be
possible to model the likelihood of
address change. Such models can help prevent false inferences about outages and
their durations. For ISPs that change periodically and/or synchronously, the model can
predict when probe-loss is more likely due to
address changes than outages. For ISPs that change addresses upon most
outages, the model can inform in which ISPs outage duration detection
is particularly error-prone. For other ISPs which change addresses
mostly upon longer outages, the model can be used to estimate the
likelihood that an inferred outage ended falsely.

Orthogonally, if every outage detected by a probing-based technique could be
confirmed through a complementary dataset that provides IDs, false positives due to dynamic addressing can be entirely mitigated. Additionally, the
analysis of outage recovery durations for all of these outages will be
possible. 



% A less lofty goal is to simply analyze more ISPs.

The key to tracking IP address(es) assigned to a home router over time
is to associate some uniquely identifying feature (the ID) that remains
constant across the home router's address
changes. Chapter~\ref{cpt:addr_change} showed two examples of IDs: in the
case of RIPE Atlas probes', the probe ID remained unchanged and in the
case of the CDN software, the installation ID remained unchanged.%  I
% presented two applications of such IDs: (i) it is possible to analyze dynamic
% address assignment practices (ii) it is possible to confirm a detected
% outage if the ID associated with the address was the same before and
% after the outage.

A challenge, however, is to obtain sources of IDs that can scale
across the Internet. IP address changes can
be tracked over time if there exists some uniquely identifying feature
that remains constant across the device's address change. There are several potential datasets which have this
property:

\subsection{Dynamic DNS services}

Websites such as dyn.com~\cite{dyn} provide dynamic DNS. Dynamic
DNS is a service that allows users with a dynamic IP address to host
web services, by providing DNS services that can be easily updated to
reflect changes in users' IP addresses. Users of Dynamic DNS Services
run a daemon provided by the dynamic DNS provider, which is responsible for
determining the publicly visible IP address, and updating the A
record(s) for the user's domain(s). 

IP address changes can be tracked using the domain names
registered with dynamic DNS services. Since the domain name of a user
maps to her current IP address, we can use the domain name as a
fingerprint, and detect changes in IP addresses for each domain name
over time, by periodically obtaining the 'A' record associated with
each domain name. 

\paragraph{Geographic correlation of dynamic behavior}

\begin{figure}[tb]
% \centering
\begin{center}
\includegraphics[height=1.5in]{figs/did_dname_get_renum}
\end{center}
\caption{\label{fig:addr_change_per_ctry}
IP address renumbering in dynamic DNS domains over a week: Black
represents dynamic DNS domains which experienced at least one address
change, while grey represents domains whose addresses remained the
same. Renumbering behavior appears to be correlated with geographic
location.}
\end{figure}

As a proof of concept, I report on a preliminary result from this
approach: corroborating the geographic relationships in Figure~\ref{fig:conts_all_durs} 
while extending to countries not well represented by RIPE.
I
obtained 3000 dynamic DNS domains from three different dynamic DNS
services: 2000 from afraid.org~\cite{afraid}, 600 from dyn~\cite{dyn} and 400 from
noip.com~\cite{noip} and fetched the 'A' records from their respective
nameservers once every hour. I collected this data for a week, and
then inspected how many of these domains experienced at least one
address change during this time. Figure~\ref{fig:addr_change_per_ctry}
shows the number of domains that had at least one 
address change and the domains that had none. The y-axis is in log-scale. 
Address changes in Asian and Latin American countries appear
more prevalent, with more than a third of all domains in these
countries seeing at least one address-change. On the other hand,
Northern European countries observe fewer than 6\% of their domain names
experiencing an address change. Address changes are uncommon in
North America: only 3\% of domain names in the US and 6\% of domain
names in Canada observed an address change.

\subsection{Open DNS resolvers}

Since 2010, various studies have reported on the existence of more
than 15 million 'open' DNS resolvers on the
Internet~\cite{openresolver, schomp2014clientsidedns, kuhrer2014exit,
  kuhrer2015going}. These DNS resolvers are 'open' because they will resolve a DNS query sent from arbitrary IP
addresses on the Internet. Previous studies have found that more than
three-quarters of open DNS resolvers are likely to be
residential~\cite{schomp2014dnsvul, schomp2014clientsidedns}. I
identify two potential approaches to fingerprint these open DNS
resolvers and track address changes.

\paragraph{DNS caches}
Open DNS resolvers often cache previous
lookups~\cite{schomp2014dnsvul}. These caches can
be used to fingerprint open DNS resolvers, allowing us to track when
their IP addresses change. 

% First, we will need to find open DNS resolvers on the Internet. I propose to register a domain and deploy an Authoritative DNS server
% for it. Then I intend to perform a one-time scan of the entire IPv4
% address space by sending a DNS request for a subdomain within the
% domain we control to all the IPv4 addresses on the Internet. Each DNS
% request I send to a target IP address will embed the target address
% into the request, similar to the approach used by Dagon et
% al.~\cite{dagon2008corrupted}. The Open DNS resolvers will route the
% request to our Authoritative DNS server.  At the authoritative DNS
% server, I will note the target IP address to which this request was
% sent and generate a unique fingerprint for the device at this address,
% and embed this fingerprint in my response. When these responses
% reach the open DNS resolvers, each will now contain its unique
% fingerprint in its cache.

% Next, I will periodically inspect the caches of known open DNS resolvers.
% I will issue periodic DNS requests for the subdomain we
% control (with the target IP address embedded in the request) to all
% the addresses that contacted our Authoritative DNS server. If we
% obtain the fingerprint that we had previously issued to that address,
% we know that the device continues to be assigned that address. If we
% find that an address is no longer returning the expected cache
% fingerprint, we know that the address has changed. I then propose to
% issue DNS requests to related addresses (as described in
% Section~\ref{sec:last_mile}) with the \emph{old} target IP address
% embedded in the request. If the device is present on any of those
% addresses, then we will obtain the expected fingerprint. Upon finding
% the device at a new address, we will update
% our local mapping and note that the fingerprint is now available at
% this new address.

\paragraph{Anomalous Open DNS Resolvers}

Of the 30 million Open DNS Resolvers on the Internet, around 17
million are \emph{anomalous}~\cite{anomalousdns}, i.e.,
instead of sending DNS responses with a source port of 53, they
respond with a non-standard source port. Kaizer et al. ~\cite{anomalousdns} found that
these devices are primarily residential ADSL modems. Not only do these
devices use a non-standard source port, DNS requests can be made to
these devices in such a way that the source ports are assigned
\emph{sequentially}. We can use this sequential
assignment of source ports to fingerprint anomalous open DNS resolvers.

% The first part of our approach here is similar to my approach with
% the DNS caches: I will find open DNS resolvers that are anomalous. After registering a domain and deploying an Authoritative DNS server
% for it, I will perform a one-time scan of the entire IPv4
% address space by sending a DNS request for a subdomain within the
% domain we control to all the IPv4 addresses on the Internet. Each DNS
% request I send to a target IP address will embed the target address
% into the request as before. However, instead of embedding responses with
% unique fingerprints from the authoritative DNS server, we simply
% monitor the source ports that issue DNS responses from each
% address. If it's a non-standard port, we flag the device as an
% anomalous open DNS resolver.

% Next, I will periodically inspect the source ports used by anomalous
% open DNS resolver responses. Since we know which
%   addresses the anomalous open DNS resolvers are located at, I
%   periodically issue DNS queries to these addresses. As long as the
%   source port for successive requests to an address continues to be
%   sequential, I can state with high confidence that the address has
%   not changed. The source ports for these devices typically vary
%   between 10,000 to 30,000; thus there is only a small likelihood that
%   another device coincidentally happens to have the next value in
%   sequence. If we find that a response doesn't arrive, or that one
%   arrives but the source port is not sequential, then we know that the
%   device's address has been reassigned. As in the DNS cache approach,
%   I will then look for the expected source port in DNS responses from
%   requests sent to related addresses to find the device again.


\section{Classifying IP addresses}

Probing-based techniques that seek to detect residential Internet
outages need a list of addresses classified as residential. More
broadly, a classification of the IP address space into residential,
enterprise, campus etc., can benefit any system that uses IP addresses
as a proxy for measurement, including IP address based host-reputation
systems~\cite{fail2ban,spamhaus}. Recent work has also shown
that ISPs are increasingly likely to deploy Carrier Grade NATs (CGNs),
where tens of residential Internet connections are multiplexed over a single
public IPv4 address~\cite{cgn-imc16}.
%  and approaches to count
% participating users in peer-to-peer
% systems~\cite{p2pfilesharing,p2pavail,sen2004analyzing}.

In this dissertation, I relied upon classifications of addresses as
residential using reverse DNS based schemes from prior
work~\cite{pingin}. Many ISPs include hints about an address's
intended use in
the reverse DNS entry of that address. Recent research has further improved address
classification with reverse DNS names~\cite{youndo-rdns}. However,
it is not mandatory for ISPs to provide meaningful reverse DNS
names. Some large ISPs, such as AT\&T do not provide reverse DNS names
for most of their addresses, resulting in their addresses' under-representation in
Thunderping data as seen in Chapter~\ref{cpt:weather}.

An orthogonal approach to address classification is to use datasets with some uniquely identifying feature (an ID) that can be used to track IP addresses over time. By analyzing how many IDs are associated with an IP address simultaneously and over time, I show in preliminary work that it is possible to infer how the ISP is using the address~\cite{shared-addrs-apnic-blog, shared-addrs-aims}. An address that is observed with multiple devices over time, though with relatively few devices at any instant, is likely a dynamic residential address.  An address that remains associated with a single ID over months is either statically assigned or is a residential address with a linktype that uses DHCP. Addresses associated with many IDs simultaneously could be CGN addresses or university/enterprise proxies.



\section{Identifying outage causes can help orthogonal reliability analyses}

In this dissertation, I covered one possible reliability analysis:
examining how challenging conditions like weather affects Internet
reliability. Another potential analysis is the head-to-head comparison
of one ISP's reliability against another. Such comparisons can aid
users in their choice of ISP and can help ISPs gauge their
competition. 

When comparing reliability across ISPs, the reliability metric
should ideally only consider outages that each ISP was responsible
for. If a user voluntarily chooses to power
off their home Internet equipment, the user has an Internet outage but
this outage should not lower the user's ISP's Internet reliability. Similarly, a power
outage in an area should contribute towards lowering the reliability
of that area, but should not lower the reliability of the
ISPs whose addresses were affected. 

For conducting comparisons of ISPs, we need to
classify outages by detected cause. Chapter~\ref{cpt:corrfails} showed
the potential of using simultaneous outages of related addresses to
find addresses that failed due to a common underlying causes. When
addresses from multiple ISPs fail together in the same geographic
region, the cause is potentially a power outage. When addresses from
only a single ISP have been observed to fail, the cause is potentially
a network outage. % However, showing that \emph{no} other ISP had been
% affected by an outage requires widespread probing of addresses. By
% triggering additional probing of related addresses on demand, and by
% augmenting pings with traceroutes, it's possible to uncover additional
% context about detected outages. This context can help with classifying outages.

% Additionally, the set of detected outages that should be considered in
% reliability metrics can vary depending upon the application. Here are
% three potential applications with different requirements:

% \begin{itemize}

% \item{Suppose the goal is to measure the effectiveness of broadcasting
%     critical information (such as severe weather alerts or Amber
%     alerts) over the Internet. An Internet reliability metric, such as
%     the rate of Internet outages over time, offers a sense of how many
%     users in each ISP or geographic region can be reached through such
%     a broadcast. For this application, all Internet
%     outages---including outages due to users turning off their home
%     Internet equipment---should be represented in the metric. }

% \item {When comparing Internet reliability across geographic regions,
%     perhaps to identify areas with particularly poor connectivity, outages due to
%     user behavior should not be considered in the reliability
%     metric. Grover et al. report that users sometimes voluntarily
%     power off their home Internet
%     equipment~\cite{grover2013peeking}. Probing-based techniques would
%   detect such instances as Internet outages since a previously
%   responsive address ceases to respond to probes. Without accounting
%   for such outages, we may overestimate the outage rate in a region.}
 
% \item {When comparing reliability across ISPs, the reliability metric
% should ideally only consider outages that each ISP was responsible
% for. Doing so ensures that ISPs offering services in challenged areas
% do not have their reliability lowered by events such as power outages
% or user behavior.}

% \end{itemize}



% \subsection{Exploring the tradeoff between outage detection and ICMP blocking}



% \subsection{Measuring reliability in IPv6 networks}

