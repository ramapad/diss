
\section{Conclusion and Discussion}

% RIPE Atlas showed that some networks have stable addresses

% NS corroborated the results. Also showed that detected outages from
% probing-based techniques are fine.

% So for link-types, good to go. For other linktypes, need the
% complementary dataset.

In this chapter, I showed that dynamic address reassignment can confuse
probing-based techniques and lead them to make false inferences about
outages. Next, I conducted a measurement study with colleagues to infer and analyze patterns of address changes using
an existing set of logs from 3,038 globally distributed RIPE Atlas probes that saw address changes in
2015. We found several factors in play.
Dynamic address durations vary by geography, with addresses from North
American ISPs persisting for weeks and addresses from
many German ISPs assigned for a day. Dynamic addresses change as a
result of network and power outages in most ISPs. In some ISPs, an
outage of any duration results in an address change, while in others, the
likelihood of address change increases with outage duration. Using
this study, I was able to identify which networks have stable
addresses, where dynamic reassignment is uncommon. I also showed using a complementary dataset that it is sometimes possible
to confirm probing-based techniques' detected outages even in
networks with frequent dynamic reassignment. % The results from this
% investigation corroborated the results from the RIPE Atlas measurement
% study, showing that cable addresses tend to be stable and that DSL
% addresses change frequently. 

% Two of our findings seem at odds with the address assignment practice
% specified in the DHCP standard~\cite{rfc2131}, which states as a
% design goal that an address assigned to a client should persist as
% long as the client continues to renew its lease, even across client
% and DHCP server reboots. First, our dataset included periodically
% reassigned addresses for many European and nearby Asian probes; these
% reassignments potentially terminated sessions that were active during
% the time of renumbering. Second, many address changes seem to result
% from reboots and reconnect events. These observations---that some
% addresses change daily and may change under the control of a
% user---have implications both for researchers who might use addresses
% as a means of counting or tracking individual users and for operators
% that might blacklist addresses for misbehavior. We provide a list of
% ISPs that renumbered periodically and their renumbering parameters in
% Table~\ref{tbl:periodic_asns}; the maximum duration these ISPs are
% likely to assign an address to a CPE can be estimated accurately with
% high probability. We also provide a list of ISPs that renumber
% consistently upon reboot and reconnect events in
% Table~\ref{tbl:outages}; malicious users from these ISPs can evade
% blacklists by simply rebooting their device.

