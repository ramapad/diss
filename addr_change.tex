
\chapter{Mitigating false inferences due to dynamic addressing}

\label{sec:addr_change}

In this section, I describe how dynamic addressing can lead
probing-based outage detection
techniques to make false inferences about outages and describe
techniques to obtain a list of stable addresses for probing.

Academia and industry often rely on a simplifying assumption that IP addresses 
uniquely identify end-hosts~\cite{p2pfilesharing,p2pavail,sen2004analyzing,sekar2006multi,anomalousdns,kuhrer2015going,xie2005worm,jung2004empirical,fabian2007botnet,stone2009your,andriesse2015reliable,fail2ban,spamhaus,cbl,sorbs}.
This assumption allows researchers to track end host
behavior over time~\cite{anomalousdns, kuhrer2015going, pingin}, or to count participating users in peer-to-peer
systems~\cite{p2pfilesharing,p2pavail,sen2004analyzing}. Many organizations create blacklists of suspicious IP
addresses based on previously observed malicious traffic associated
with those addresses~\cite{fail2ban,spamhaus,cbl,sorbs}. 

When probing-based remote outage detection techniques send probes to an address, they expect that the
address continues to be assigned to the same end-host for the entirety
of the probing duration. Depending upon how a dynamic address gets
reassigned, these techniques can make false inferences about outages in two ways:

\begin{itemize}

\item{\emph{Detecting false outages} Probing-based remote outage detection techniques detect outages
    when a previously responsive address stops responding to
    probes. However, If a dynamic address being probed is
withdrawn from its host and is not assigned to any other host, active probes to the address will no longer
elicit responses. These techniques will infer false
probe-loss, leading them to infer false outages.}

\item{\emph{Detecting false outage duration} These techniques detect outage
    duration by continuing to probe an unresponsive address. When the
    address starts responding to probes again, the outage is inferred
    to end. If a user device  with a
    public dynamic address has an outage and at some point during the outage,
    the dynamic address is reassigned to some other user device
    which responds to probes, probing-based remote outage detection techniques would infer that the outage ended incorrectly.
% For ISPs that use DHCP
%     for address assignment, we would expect dynamically
%     assigned addresses to stick around on the end-host until an
%     outage occurs. However, upon the occurrence of the outage, if the
%     outage is long enough, they can get reassigned to another host,
%     especially, if the outage is longer than the DHCP lease
%     duration. Here, RODWAP techniques can detect the outage itself but
%     can perhaps not detect outages that are long
}

% TODO: CITE http://www.umiacs.umd.edu/~tdumitra/courses/ENEE757/Fall15/papers/Stone-Gross09.pdf
\end{itemize}

My approach to mitigating these false inferences is to analyze how
frequently and for what reasons dynamic addresses are
reassigned. I will use the results of these analyses to build a model
of how likely an inference about an outage using a probing-based
remote outage detection technique is
a false inference caused by dynamic addressing. For example,
preliminary work with colleagues has revealed that some European ISPs change addresses upon
very small outages and are particularly likely to change addresses at certain
times of the day~\cite{addrchange-reasons}. These results will inform
my model to not attempt detection of 
outage duration for these ISPs, and to discard outages
detected at times that are particularly likely to have dynamic address
changes. The model will ultimately yield stable addresses who either
do not undergo dynamic reassignment for months at a time, or who get
reassigned but in a predictable manner. Thunderping limits itself to
probing addresses in the U.S. where dynamic reassignment is
uncommon~\cite{addrchange-reasons}; stable addresses from the
model can help us detect outages in new areas.

% In the rest of this section, we provide background about 

\section{Dynamic addressing background}

An IP address can be used to uniquely identify the end-host it is assigned to
until the end-host's address changes for some reason. The duration of
time that a dynamic IP address continues to be assigned to the same
CPE (Customer Premises Equipment) device depends upon various causes that can induce the assigned IP
address to change. Here, I present techniques used for
assigning dynamic addresses and the events and
agents involved in dynamic address changes.

ISPs often use the Dynamic Host Configuration Protocol
(DHCP)~\cite{rfc2131} for IP address assignment. DHCP issues an IP address to a host for a lease
duration configured by the ISP. The host will try to renew the lease
before it expires, typically half-way into the lease. However,
whether the same IP address is renewed, or a different one is
assigned, depends upon ISP policy.  We speculate that the
typical behavior of ISPs using DHCP is to renew the lease of the
currently assigned IP address, since one of the stated design goals
in the DHCP specification is that a DHCP client should be assigned the same address
in response to each request, whenever possible. Thus, we typically
only expect an ISP using DHCP, to change the address of a CPE, if
something happens to prevent the CPE from renewing its lease (like an outage).
% Further, on
% reboot, the previously assigned address may be reassigned, or
% alternately a new address may be issued, again, as dicated by ISP
% policy.

In some networks, end-hosts connect to an ISP using
point-to-point links. For these networks, the Point-to-Point Protocol
(PPP) first configures and establishes the point-to-point link~\cite{rfc1661}. Next,
a Network Control Protocol (NCP) like the Internet Protocol Control
Protocol (IPCP) configures IP addresses~\cite{rfc1332}. The PPP specification
notes that the link will remain configured for communication until the
link is actively closed down through network administrator
intervention or when an inactivity timer expires.

\subsubsection{Potential dynamic address change causes} 

Next, we identify the reasons dynamic addresses assigned using
the above techniques could change. We classify the following categories of address change:

\begin{itemize}
\item{\textbf{Changes after outages}} If the client is
  disconnected or loses power long enough to fail to renew a DHCP lease,
  its address may be assigned to another; when it returns,
  it may then get a new address. We call such changes
  \emph{outage-caused address changes}.

\item{\textbf{Changes after reboot/reconnect}} While we
  expect addresses assigned through traditional DHCP to change only when the
  outage duration is long enough to prevent lease renewal, addresses
  assigned through PPP can change upon outages of any duration. Any reboot or
  network reconnect event could cause the client
  to forget its prior address and request a new one, or the
  state associated with a connection may be lost.
  We call such
  address changes \emph{reboot-caused address changes}. 
  
  % We make no
  % distinction between very short power outages and reboots.

\item{\textbf{Administrative address changes}} A purpose of
  dynamic address assignment is to allow reconfiguration of
  the network; it is possible that a reconfiguration of the
  DHCP server will force a change to the subnet on which the
  client lies.  We expect such reassignment to be rare.
  
\item{\textbf{Periodic address changes}} We observe that
  some ISPs limit the session length associated with an
  address, causing a reassignment after a fixed duration,
  typically one day to one week depending on the ISP.

\end{itemize}

\subsection{Building a global model of dynamic address change}

Conceptually, so long as there is some uniquely identifying feature
that remains constant across a device's address change,
it can be possible to track IP address changes over time. Several studies have used this broad
method~\cite{udmap, census-survey, zmap-dhcp,
maier2009dominant, dhcp-gatech, peering-shroud, dhcp-dimes}. 
UDmap~\cite{udmap} studied
dynamic address properties using Hotmail user login
traces where the user's login serves as the identifying
feature. Casado et al.~\cite{peering-shroud} tracked clients using HTTP cookies when
clients access a CDN. Other studies~\cite{zmap-dhcp, census-survey} used continuous
responsiveness of an address itself as the identifying feature, assuming
that an address that responds continuously belongs to the same user and that
when an address stops responding to pings, it has been
reassigned. 

However, these studies report conflicting results about the frequency
of address changes. While UDmap reported that over 30\% of IP addresses have
inter-user durations of 1--3 days~\cite{udmap},  Heidemann et
al. reported that 90\% of IP addresses were occupied for less than a
day~\cite{census-survey}.  Maier et al.~\cite{maier2009dominant} reported that a
major European ISP had per-user median durations of just 20 minutes during
their study in 2009 whereas our work in 2015 did not observe this
duration~\cite{addrchange-reasons}. These differences are likely due
to the different biases associated with each study: Maier et
al.~\cite{maier2009dominant} studied one European ISP in an area,
UDmap studied only Hotmail users, while studies that use continuous
responsiveness of addresses over time~\cite{zmap-dhcp, census-survey} could
potentially confuse the occurrence of an outage with an address change.

Considering these conflicting results,
I first analyze the feasibility of building a dynamic addressing
model by describing preliminary results from a novel dataset~\cite{addrchange-reasons}. The
results show that a global model is indeed feasible, but that it will
require multiple, independent, diverse datasets that track address changes across
the world. % Then I describe proposed
% work to gather other complementary datasets.

% However, most of these studies do not account for the cause of the address change,
% studying dynamic address durations in isolation. We show in this
% proposal that studying the causes of dynamic address change is vital
% especially for RODWAP techniques. The occurrence of outages can cause
% dynamic address reassignment.

\subsection{Preliminary results towards a global model}

The RIPE NCC's Atlas project deploys small devices, called probes, that
conduct measurements from globally distributed
networks~\cite{atlas}. The RIPE Atlas dataset offers measurements that allow us to
determine when an IP address change occurred and what the addresses
were before and after the change. In addition, the dataset includes many
measurements that provide context about what was happening around the
time of the address change. I was able to use these measurements to
detect when RIPE Atlas probes rebooted and were not sending pings
(indicating a power outage) and when their pings were not getting
responses (indicating a network outage). In a study with colleagues of active RIPE
Atlas probes in 2015, we found 3,038 RIPE Atlas probes with address
changes hosted across 929 ISPs and 156 countries~\cite{addrchange-reasons}.

\subsubsection{Some ISPs change addresses periodically}


ISPs can assign dynamic addresses for as long as they wish.
In DHCP, long leases simplify administration, while short
leases can be more efficient in reclaiming unused addresses.
DHCP leases, however, are meant to be renewable by devices
that are still active.  In this section, we look at periodic
address reassignment: instances where a device changes
address periodically, despite actively using the address.

If ISPs intentionally renumber after specific durations, we would
expect those address durations to be prominent in a distribution
of all address durations belonging to that ISP. We initially
considered studying distributions
of raw address durations, similar to the analyses by
Maier et al.~\cite{maier2009dominant} and
Moura et al.~\cite{zmap-dhcp}, but found that short address-durations
were overrepresented. When trying to reason about the expected duration that an address will
continue to be assigned to the CPE, we would like to know the fraction
of total time that each duration accounted for. This latter notion is more useful to find whether an
ISP is using periodic durations consistently, since the modes at
intervals on the scale of days will be more visible. 

To capture this notion we
define a metric, the \emph{total time fraction}. For a given probe and an address duration $d$,
we define the total time fraction for $d$ as the fraction of time spent by the probe in durations of length $d$.
We compute the total time fraction for a given probe and a duration
$d$ by obtaining the total address
time for the probe, and computing the fraction of the total
address time that was accounted for by address durations of 
length $d$. For a probe $p$, if $n(d)$ is the number of times the probe had an address duration
$d$ and $D$ is an array containing all address durations that were assigned to
the probe, the total time fraction for the address duration $d$ is
given by:

$f^p_d =  d \times n(d) / \Sigma(D)$

We use a similar procedure for computing the total time fraction
considering all probes in an ISP, country, or continent. We believe that  
%the distribution of 
the total time fraction offers a better representation of 
the
probability that an address was assigned for a certain
duration than a simple inspection of the address durations. 

\subsubsection*{North American addresses are assigned longer than other addresses}

\begin{figure}[tb]
  % \centering
  \begin{center}
    \includegraphics[width=3in]{figs/conts_a_all_ip_durs_connlogs_wtd_cdf}
  \end{center}
  \caption{\label{fig:conts_all_durs} 
    %% Dynamic address-durations
    %%     weighted by address-durations by continent.
    Cumulative distribution of total time fraction by continent. 
    Modes (vertical segments
    in the CDF) indicate periodic renumbering.  Addresses in North America
    are relatively long lived and free of periodic renumbering.}
\end{figure}

We begin by inspecting how address durations vary across
continents.  We expected that address scarcity might affect
address durations, leading to longer durations in North
America and shorter durations in Asia.
We use RIPE
Atlas's probe database 
to find the country to which each probe belongs. Next, we aggregate the
address durations of probes by their respective countries and
subsequently, to their continents.
Figure~\ref{fig:conts_all_durs} shows the cumulative distribution of
the total time fraction for each
continent, i.e., the y-axis shows the fraction of total address duration accounted for by durations less than the x-axis value. 
%% Figure~\ref{fig:conts_all_durs} shows a CDF of address durations, weighted as described above, for each
%% continent.
The number in
parentheses in the legend for each continent shows the total
 address duration for that continent in years ($\Sigma(D)$).

In Europe, Asia, Africa, and South America, address durations exhibit well-defined modes,
mostly at intervals that are multiples of 24 hours. The most common mode is
exactly at 24 hours: the total time fraction for European addresses at
24 hours is 0.16, African addresses is also 0.16, and Asian addresses is 0.07.
One week address durations are also common in Europe, with the total
time fraction at 1 week equaling 0.08.
South American addresses exhibit multiple modes: their total time
fraction is 0.11 at
12 hours, 0.07 at 28 hours, 0.09 at 48 hours, and 0.03 at 192 hours (8
days). 

The curves for North America and Oceania do not have well-defined modes,
suggesting that ISPs in these continents do not periodically change
addresses. Further, North American probes typically retain their dynamic
addresses for much longer durations than other continents; North
American addresses spent more
than half of the total time in address durations longer than 50
days. This suggests that IP addresses can be used as end-host
identifiers in North America for several weeks.

% \subsubsection*{Periodic renumbering is common in some parts of the world}

% \begin{figure}[tb]
%   % \centering
%   \begin{center}
%     \includegraphics[width=3in]{figs/DE_asns_a_all_ip_durs_connlogs_wtd_cdf}
%   \end{center}
%   \caption{\label{fig:DE_asns_all_durs}
% %
% 	%% Dynamic address-durations weighted by address-durations for ASes in Germany.
%     Cumulative distribution of total time fractions for ASes in Germany.
%         Many
%       German ISPs appear to change addresses every 24 hours. However,
%       some ISPs have more stable addresses.
%   %%@@rama: quantify
%   }
% \end{figure}


% Next, we investigate how the periodic renumbering behavior of ISPs
% correlates with the country in which they operate. Germany has more
% than a hundred RIPE Atlas probes deployed across several ISPs, thus we
% study their address durations in Figure~\ref{fig:DE_asns_all_durs} for
% ISPs with probes that contributed at least 3 years of total time. Many
% ISPs in Germany change addresses every 24 hours: 77\% of the duration
% in DTAG (AS 3320), 76\% in Telefonica1 (AS 6805), 74\% in Telefonica2
% (AS 13184), and 29\% in Vodafone (AS 3209), is 24 hours. We observe
% that the 'other' ISPs also have a mode at 24 hours, suggesting that
% German ISPs are particularly likely to renumber every 24
% hours. However, this behavior is not universal: Kabel Deutschland (AS
% 31334) and Kabel BW (AS29562) do not exhibit a mode at 24 hours;
% instead, more than 90\% of their total address duration was spent in
% durations longer than two weeks. These results suggest that periodic renumbering behavior can exhibit
% some geographic correlation, but is likely
% largely caused by ISP policy. 

% We found 20 ISPs in the RIPE Atlas dataset that we label as \emph{periodic} because
% these ISPs renumber many of their customers after they have held their
% address for a specific duration. This time limit varies across
% ISPs. Of 2272 dynamically assigned probes in the dataset, 193 (8.5\%)
% change addresses periodically with a period of 24 hours, and 123
% (5.4\%) do so with a period of one week. Periodic renumbering occurred
% most commonly in central European countries like Germany, Austria,
% Poland, and Croatia. Some ISPs in Russia, Kazakhstan, Mauritius, and
% South America, also periodically renumber.

% Private communication with a large European ISP confirmed that the ISP renumbers every 24
% hours, since the ISP considers this scheme to be more 'privacy secure' although
% there is no government regulation that forces this feature. The ISP
% also reported that it uses PPPoE instead of DHCP for its DSL
% lines (which accounted for the vast majority of its customers). Since
% periodic behavior would be atypical of DHCP but consistent with PPP
% techniques for address assignment, we speculate that periodic
% renumbering is a property of ISPs that use PPP.

\subsubsection*{Periodic address changes are some times synchronized}

% \begin{figure}[th]
%  \centering
%     \includegraphics[width=3in]{figs/weekly_3215_a_periodicrenums_per_h24_connlogs_bar}
%   \caption{\label{fig:3215_renums_per_h24}Periodic address changes
%     in Orange appear more evenly distributed among the hours of the
%     day.}
% \end{figure}

\begin{figure}[th]
  \centering
    \includegraphics[width=3in]{figs/daily_3320_a_periodicrenums_per_h24_connlogs_bar}
  \caption{\label{fig:3320_renums_per_h24}Periodic address changes
    are more likely in some hours for Deutsche Telekom.}
\end{figure}

We imagine two broad strategies for daily renumbering:
either leaving each customer on an independent, free-running
clock that resets after 24 hours, or synchronizing all
address changes to an off-peak time when few would be
interrupted.  Both seem reasonable strategies: independent
clocks seem simple to implement, synchronized address
changes seem more likely to shuffle addresses since many
addresses are made available during the synchronized
interval.  Probing-based outage detection techniques would benefit from knowing which strategy
is being used: if an ISP is known to change addresses at specific
times of the day, we can account for this behavior in the model.

 We expect
that plotting the time of day at which addresses change for
each ISP will expose whether the renumbering is
synchronized. For the German ISP, Deutsche Telekom AG (DTAG), Figure~\ref{fig:3320_renums_per_h24} shows the hour of
the day when an address change occurred after the address had been
assigned for 24 hours; DTAG assigns periodic durations more often during some
hours of the day. In private correspondence with a large European ISP,
we learned that many CPE devices come with an option to choose the
time at which they should disconnect and reconnect to receive a new
address, as a privacy feature. Figure~\ref{fig:3320_renums_per_h24}
supports this deployment scenario, observing almost three quarters of
all periodic address changes between hours 24 to 6 (in GMT). However,
some CPEs do not have this feature because a quarter of the periodic
address changes happen at other hours of the day.

\subsubsection{Some ISPs are more likely to change addresses upon
  outages}

Here, we investigated how outages occurring at the CPE (customer
premises equipment), due to loss of power or network connectivity
affect the likelihood of address changes. % We quantify how frequently
% and for which probes an outage event at the CPE device appears to
% cause the reassignment of its IP address.
% We quantify how the relationship between outage 


% \subsubsection*{Renumbering behavior upon outages varies across ISPs}

% For each individual probe, we considered the conditional
% probability of an address change given a detected
% outage. $P(ac|nw)$
% represents the conditional probability that an address change occurred
% given a network outage and $P(ac|pw)$ represents the same for a power outage. We estimated this probability using the
% fraction of outages occurring contemporaneously with an address change (out of the
% total number of outages).  We show the distribution of
% these probabilities by probe to estimate whether the group
% of probes (by geography or ISP) is dominated by those that
% always or seldom change addresses on an outage.

% \begin{figure}[tb]
%   % \centering
%   \begin{center}
%     \includegraphics[width=3in]{figs/top_asns_frac_norenums_over_totalnos_cdf}
%   \end{center}
%   \caption{\label{fig:top_asns_frac_norenums_over_totalnos}
% Distribution of $P(ac|nw)$ per probe for the ASes with the most probes
% that had at least one address change. Probes in DTAG, Orange, and BT, are far more likely to change addresses upon a
%     network outage than probes in Verizon and LGI.}
% \end{figure}


% We find that the likelihood of address change upon an outage event
% differs across ASes. Figure~\ref{fig:top_asns_frac_norenums_over_totalnos} shows
% the CDF of $P(ac|nw)$ for the five ASes
% that host the most probes with at least one address change and at least three
% network outage events. We find that probes in ASes that periodically renumber---Orange, DTAG, and BT---have high $P(ac|nw)$ compared to probes from ASes that
% do not periodically renumber, LGI and Verizon. Around half of the probes in both Orange and
% DTAG had  $P(ac|nw)$  equal to 1: every network outage was accompanied
% by an address change! $P(ac|pw)$ was also similar
% for these ISPs. We found 10 more ASes whose probes were particularly
% likely to renumber upon outages: all of them are in Europe and 7 of them
% also periodically renumber. Private communication with a large European
% ISP whose probes consistently had an address change upon outage confirmed that they use PPPoE and Radius to assign addresses for
% their DSL lines. We expect that this property can be used as evidence in inferring a device's link
% type.

% \subsubsection*{For some ISPs, most outages result in address changes}

\begin{figure}[t]
  \includegraphics[width=1.5in]{figs/6830_combined_merged_bar}~~~
  \includegraphics[width=1.5in]{figs/3215_combined_merged_bar}

  \caption{\label{fig:outagedurs} The likelihood of an address change (renumbering)
    given network or power outages of different durations in LGI (left)
    and Orange (right).  The top graph is a histogram; the
    complete bar represents the number of outages observed
    across all probes in that AS.  The lightly-shaded bar
    extends for those outages that also saw an address
    change.  The lower graph shows the same data as a
    percentage.  Although relatively few outages
    lasted longer than a day, the majority of these were
    coincident with an address change in both ISPs. However,
    Orange (right) changed addresses even on the shortest
    outages.}
\end{figure}

Dynamic addresses assigned using DHCP should typically retain
their addresses as long as they continue to renew their lease half-way
into the lease duration as the standard
recommends~\cite{rfc2131}. However, an outage could
prevent them from renewing their lease. Depending upon the address
churn at the time, the address they had previously been assigned may
be reassigned to another device.  In this way, an outage
longer than half a lease duration could potentially cause an address
change. To investigate the effect of outage duration on the likelihood
of address change, we analyzed the conditional probability of an
address change given the occurrence of network or power outages of different durations
for probes from LGI (AS 6830) and
Orange (AS 3215) in Figure~\ref{fig:outagedurs}. 

The behavior upon outages for the two ISPs is strikingly
different. LGI's behavior appears consistent with what we would 
expect for dynamic addresses assigned using DHCP: fewer than
3\% of outages of up to an hour resulted in an address
change.  More than 25\% of outage
durations that lasted at least twelve hours resulted in an address
change. This behavior is consistent with a DHCP lease duration on the
order of a few hours.  Not every outage longer than twelve
hours resulted in an address change, consistent with DHCP 
behavior when a client returns after an expired lease and the
previously assigned address
is still available.

For Orange, we found that even very short outages resulted in
address changes. 91\% of outages that lasted less than five
minutes resulted in an address change, and for every outage duration
longer than five minutes and shorter than three hours, more than 75\% occurred with an
address change. For outages between three hours to three days
long, the percentage of address changes was closer to 50\%, suggesting
the presence of some CPE devices that do not renumber upon every outage. However, as the
outage duration increases beyond 3 days, almost every outage results
in an address change.

% Private communication with a large European ISP
% confirmed that this behavior is expected for PPPoE based DSL lines in
% that ISP: any reboot/reconnect event will result in the assignment of a new
% address from the ISP's dynamic address pool. Since outages of such short durations can result in an
% address change, a simple reboot of the CPE (resulting in a power
% outage), or unplugging and replugging the network cable (resulting in a network outage), can change the dynamic address assigned to the end-user.%  That
% end-users can change their dynamically assigned address 
% has implications for researchers and operators who use IP
% addresses to identify end-hosts, particularly when IP addresses are
% being used to blacklist malicious actors.

\subsubsection*{Preliminary results suggest that building a global
  model is feasible}

Preliminary results offer promise that modeling the likelihood of
address change can help prevent false inferences about outages and
their durations. For ISPs that change periodically and/or synchronously, the model can
predict when probe-loss is more likely due to
address changes than outages. For ISPs that change addresses upon most
outages, the model can inform in which ISPs outage duration detection
is particularly error-prone. For other ISPs which change addresses
mostly upon longer outages, the model can be used to estimate the
likelihood that an inferred outage ended falsely.

Though the results from RIPE Atlas are promising, they are potentially
biased toward technical users like most measurement infrastructure,
and biased toward European deployment. Building the global model of dynamic address change will
require multiple, independent, diverse datasets that track address changes across
the world.

\subsection{Proposed work to gather complementary datasets}

IP address changes can
be tracked over time if there exists some uniquely identifying feature
that remains constant across the device's address change. I investigate the use of datasets which have this
property to study dynamic addresses:

% \subsubsection{Download manager logs}

% A large CDN's download manager installed on users’ desktop
% and laptops records log lines when events such as a file download
% occur. Each such log line contains a unique installation ID, the
% user’s current public IPv4 address, and the timestamp. 

\subsubsection{Dynamic DNS services}

Websites such as dyn.com~\cite{dyn} provide dynamic DNS. Dynamic
DNS is a service that allows users with a dynamic IP address to host
web services, by providing DNS services that can be easily updated to
reflect changes in users' IP addresses. Users of Dynamic DNS Services
run a daemon provided by the dynamic DNS provider, which is responsible for
determining the publicly visible IP address, and updating the A
record(s) for the user's domain(s). 

I propose to track IP address changes using domain names
registered with dynamic DNS services. Since the domain name of a user
maps to her current IP address, we can use the domain name as a
fingerprint, and detect changes in IP addresses for each domain name
over time, by periodically obtaining the 'A' record associated with
each domain name. 

\paragraph{Geographic correlation of dynamic behavior}

\begin{figure}[tb]
% \centering
\begin{center}
\includegraphics[height=1.5in]{figs/did_dname_get_renum}
\end{center}
\caption{\label{fig:addr_change_per_ctry}
IP address renumbering in dynamic DNS domains over a week: Black
represents dynamic DNS domains which experienced at least one address
change, while grey represents domains whose addresses remained the
same. Renumbering behavior appears to be correlated with geographic
location.}
\end{figure}

As a proof of concept, I report on a preliminary result from this
approach: corroborating the geographic relationships in Figure~\ref{fig:conts_all_durs} 
while extending to countries not well represented by RIPE.
I
obtained 3000 dynamic DNS domains from three different dynamic DNS
services: 2000 from afraid.org~\cite{afraid}, 600 from dyn~\cite{dyn} and 400 from
noip.com~\cite{noip} and fetched the 'A' records from their respective
nameservers once every hour. I collected this data for a week, and
then inspected how many of these domains experienced at least one
address change during this time. Figure~\ref{fig:addr_change_per_ctry}
shows the number of domains that had at least one 
address change and the domains that had none. The y-axis is in log-scale. 
Address changes in Asian and Latin American countries appear
more prevalent, with more than a third of all domains in these
countries seeing at least one address-change. On the other hand,
Northern European countries observe fewer than 6\% of their domain names
experiencing an address change. Address changes are uncommon in
North America: only 3\% of domain names in the US and 6\% of domain
names in Canada observed an address change.

The results from the dynamic DNS dataset are preliminary in
scale and based on a short measurement to show potential.
Further, our study of RIPE Atlas data showed us that the cause of
address changes is important. I intend to couple our
outage-detection tool to probe addresses corresponding to
the dynamic DNS domains while fetching their A records.  We
can thus identify outages that occur near the reassignment,
allowing us to infer if an address-change was caused by an outage and
feed results into the model. Further, if the dynamic DNS result
indicates that a probed address had recently been reassigned, then the
detected false positive outage can be filtered.
 
\subsubsection{Open DNS resolvers}

Since 2010, various studies have reported on the existence of more
than 15 million 'open' DNS resolvers on the
Internet~\cite{openresolver, schomp2014clientsidedns, kuhrer2014exit,
  kuhrer2015going}. These DNS resolvers are 'open' because they will resolve a DNS query sent from arbitrary IP
addresses on the Internet. Previous studies have found that more than
three-quarters of open DNS resolvers are likely to be
residential~\cite{schomp2014dnsvul, schomp2014clientsidedns}. I
propose two potential approaches to fingerprint these open DNS
resolvers and track address changes.

\paragraph{DNS caches}
Open DNS resolvers often cache previous
lookups~\cite{schomp2014dnsvul}. My insight is that these caches can
be used to fingerprint open DNS resolvers, allowing us to track when
their IP addresses change. I plan to do this in two phases.

First, I will find open DNS resolvers on the Internet. I propose to register a domain and deploy an Authoritative DNS server
for it. Then I intend to perform a one-time scan of the entire IPv4
address space by sending a DNS request for a subdomain within the
domain we control to all the IPv4 addresses on the Internet. Each DNS
request I send to a target IP address will embed the target address
into the request, similar to the approach used by Dagon et
al.~\cite{dagon2008corrupted}. The Open DNS resolvers will route the
request to our Authoritative DNS server.  At the authoritative DNS
server, I will note the target IP address to which this request was
sent and generate a unique fingerprint for the device at this address,
and embed this fingerprint in my response. When these responses
reach the open DNS resolvers, each will now contain its unique
fingerprint in its cache.

Next, I will periodically inspect the caches of known open DNS resolvers.
I will issue periodic DNS requests for the subdomain we
control (with the target IP address embedded in the request) to all
the addresses that contacted our Authoritative DNS server. If we
obtain the fingerprint that we had previously issued to that address,
we know that the device continues to be assigned that address. If we
find that an address is no longer returning the expected cache
fingerprint, we know that the address has changed. I then propose to
issue DNS requests to related addresses (as described in
Section~\ref{sec:last_mile}) with the \emph{old} target IP address
embedded in the request. If the device is present on any of those
addresses, then we will obtain the expected fingerprint. Upon finding
the device at a new address, we will update
our local mapping and note that the fingerprint is now available at
this new address.

\paragraph{Anomalous Open DNS Resolvers}

Of the 30 million Open DNS Resolvers on the Internet, around 17
million are \emph{anomalous}~\cite{anomalousdns}, i.e.,
instead of sending DNS responses with a source port of 53, they
respond with a non-standard source port. Kaizer et al. ~\cite{anomalousdns} found that
these devices are primarily residential ADSL modems. Not only do these
devices use a non-standard source port, DNS requests can be made to
these devices in such a way that the source ports are assigned
\emph{sequentially}. My insight is that we can use this sequential
assignment of source ports to fingerprint anomalous open DNS resolvers.

The first part of our approach here is similar to my approach with
the DNS caches: I will find open DNS resolvers that are anomalous. After registering a domain and deploying an Authoritative DNS server
for it, I will perform a one-time scan of the entire IPv4
address space by sending a DNS request for a subdomain within the
domain we control to all the IPv4 addresses on the Internet. Each DNS
request I send to a target IP address will embed the target address
into the request as before. However, instead of embedding responses with
unique fingerprints from the authoritative DNS server, we simply
monitor the source ports that issue DNS responses from each
address. If it's a non-standard port, we flag the device as an
anomalous open DNS resolver.

Next, I will periodically inspect the source ports used by anomalous
open DNS resolver responses. Since we know which
  addresses the anomalous open DNS resolvers are located at, I
  periodically issue DNS queries to these addresses. As long as the
  source port for successive requests to an address continues to be
  sequential, I can state with high confidence that the address has
  not changed. The source ports for these devices typically vary
  between 10,000 to 30,000; thus there is only a small likelihood that
  another device coincidentally happens to have the next value in
  sequence. If we find that a response doesn't arrive, or that one
  arrives but the source port is not sequential, then we know that the
  device's address has been reassigned. As in the DNS cache approach,
  I will then look for the expected source port in DNS responses from
  requests sent to related addresses to find the device again.


\subsection{Confirming that detected outages are accurate}

After mitigating false positive outages, I propose the use of datasets
from RIPE Atlas probes~\cite{atlas} as ground truth to confirm that the remaining
outages are indeed true positives. In previous work, I had inferred outages occurring on
RIPE Atlas probes by looking for gaps when probes did not perform 
measurements that they were scheduled to~\cite{addrchange-reasons}. By
probing IP addresses at which RIPE Atlas probes are also deployed, I
will compare outages we detect against outages inferred from RIPE
Atlas datasets and validate whether our detected outages are accurate.